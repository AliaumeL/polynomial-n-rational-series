%! TeX program = xelatex
%! lang = en-US
\section{Beyond Polynomials}
\label{sec:deciding}

\subsection{$\Nat$-polyregular functions}

The goal of this section is to go beyond polynomials and characterize all
$\Rel$-polyregular functions that are $\Nat$-polyregular, under the assumption
that they are \kl{commutative}.

\begin{definition}
    \label{k-combinatorial:def}
    Let $k \in \Nat$, and $f \colon \Sigma^* \to \Rel$
    be a \kl{$\Rel$-polyregular function}. The function $f$ is 
    \intro{$(k,\Nat)$-combinatorial} if there exists $\omega \in \Nat$,
    such that
    for all
    $\alpha_0, \dots, \alpha_k \in \Sigma^*$,
    for all $u_1, \dots, u_k \in \Sigma^*$,
    there exists a \kl{$\Nat$-rational polynomial} $P$
    satisfying for all $X_1, \dots, X_k \geq \omega$:
    \begin{equation*}
        f
        \left(
            \alpha_0 \prod_{i = 1}^k u_i^{\omega X_i} \alpha_i
        \right)
        = 
        P(X_1, \dots, X_k) \quad .
    \end{equation*}
\end{definition}

\begin{theorem}
    \label{decidable-n-poly:thm}
    Let $k \in \Nat$, and 
    let $f \in \ZPoly[k]$ be a \kl{commutative} \kl{$\Rel$-polyregular function}.
    The following are equivalent:
    \begin{enumerate}
        \item $f$ is \kl{$(k,\Nat)$-combinatorial},
        \item $f \in \NPoly$,
    \end{enumerate}
    Furthermore, the properties are decidable,
    and one can effectively compute a representation of $f$.
\end{theorem}

To prove the theorem, let us first provide a nice characterization of
\kl{commutative} \kl{$\Rel$-polyregular functions} in terms of polynomials.
This is actually a refinement of \cref{n-poly-combinatorics:lem} in the
commutative case. Informally, we state that \kl{commutative}
\kl{$\Rel$-polyregular functions} are obtained by applying different
\kl{$\Rel$-rational polynomials} based on the modulos...

\AP Given $\omega \in \Nat$, let us write $\phi \colon \Nat \to \set{0, \dots,
\omega} \times \set{0, \dots, \omega - 1}$ defined as $\phi(n) = (t, r)$ where
$t = \min(n, \omega)$, and $r$ is the rest of the Euclidian division of $n$ by
$\omega$.
We let $\phi^\dagger \colon \Nat^k \to$.


\begin{lemma}
    \label{decompose-polynomial:lem}
    Let $f \colon \Sigma^* \to \Rel$ be a \kl{commutative}
    \kl{$\Rel$-polyregular function}. There exists a computable
    $\omega \in \Nat$,
    and computable 
    polynomials $\seqof{P_{\rho}}{\rho \in \Delta_\omega}$,
    such that for all $w \in \Sigma^*$,
    \begin{equation*}
        f\left(a_1^{X_1} \cdots a_n ^{X_n}\right) = P_{\phi(X_1, \dots, X_n)} (X_1 / \omega, \dots, X_n / \omega)
        \quad .
    \end{equation*}
\end{lemma}
\begin{proof}
    equivalent partitions of the word, but commutativity?.q
    Lemma 5.37 phd gaëtan.
\end{proof}

As a consequence, we conclude 
\begin{proof}[Proof of \cref{decidable-n-poly:thm}]
    We use \cref{decompose-polynomial:lem}
    to effectively compute
    a polynomial representation of $f$. Remark that
    $f$ is a \kl{$\Nat$-polyregular function} if and only if
    all these polynomials are \kl{$\Nat$-rational polynomials},
    which is decidable thanks to
    \cref{decide-correct-poly:lem,corrected-version:thm}.
\end{proof}



\subsection{Star-free $\Nat$-polyregular functions}
\label{star-free:sec}


Let us now prove that the above characterizations of \kl{commutative}
\kl{$\Nat$-polyregular functions} can be combined with the recent advances in
the study of \kl{$\Rel$-polyregular functions} \cite{LOPEZ23b} to prove the
following theorem.

\begin{theorem}
    \label{zsf-npoly-nsf:thm}
    Let $\Sigma$ be a finite alphabet and $k \in \Nat$.
    Then,
    \begin{equation*}
        \ZCommut \cap \ZSF[k] \cap \NPoly
        = \ZCommut \cap \NSF[k]
        \quad .
    \end{equation*}
\end{theorem}

Note that this proves in the commutative case a conjecture that was 
formulated by Thesis Gaetan, and that is non trivial, because 

\begin{conjecture}
    \label{nsf-zsf:conj}
    In the general case, the following equation holds for all $k \in \Nat$:
    \begin{equation*}
        \ZSF[k] \cap \NPoly
        = 
        \NSF[k]
        \quad .
    \end{equation*}
\end{conjecture}

As a corollary of \cref{zsf-npoly-nsf:thm} and the effective decision
procedures of $\ZSF$ obtained in \cite{LOPEZ23b}, we conclude that $\NSF$ is
decidable inside $\NPoly$ under the assumption of commutativity.

\textbf{TODO: add the growth rate here, we are actually minimising it}.
\begin{corollary}
    One can decide if a \kl{commutative} \kl{$\Rel$-polyregular function}
    can be realized by a \kl{star-free $\Nat$-polyregular function},
    in which case the conversion is effective.
\end{corollary}
\begin{proof}
    From \cite[Theorem V.13]{LOPEZ23b}, it is decidable whether
    $f \in \ZSF[k]$ and from \cref{decidable-n-poly:thm}
    it is decidable whether $f \in \NPoly$.
    Furthermore, one can enumerate functions in $\NSF[k]$,
    and check equivalence to effectively represent $f$ as a
    \kl{star-free $\Nat$-polyregular function}.
\end{proof}

The key ingredient of this section is the use of a semantic characterization of
\kl{star-free $\Rel$-polyregular functions} among \kl{$\Rel$-rational series}
that generalizes the aperiodicity of languages to functions by the means of
polynomial behaviors.

\begin{definition}[Ultimately polynomial]
    \label{ultimately-polynomial:def}
    Let $\Sigma$ be a finite alphabet. 
    A function $f \colon \Sigma^* \to \Rel$
    is \intro{ultimately polynomial}
    whenever there exists $N_0 \in \Nat$ such that
    for all $n \in \Nat$
    and for all words $\alpha_0, w_1, \alpha_1, \cdots, \alpha_{n-1}, w_n, \alpha_n
    \in \Sigma^*$, there exists a polynomial $P \in \Rel[X_1, \dots, X_n]$
    such that
    \begin{equation*}
        f\left(
            \alpha_0 \prod_{i = 1}^{n} w_i^{X_i} \alpha_i
        \right)
        = 
        P(X_1, \dots, X_n)
        \quad 
        \forall X_1, \dots, X_n \geq N_0
        \quad .
    \end{equation*}
\end{definition}


Let us recall from \cite{LOPEZ23b} that this definition correctly
generalizes the indicator functions of regular languages.

\begin{example}
    A language $L$ is aperiodic if and only if 
    the indicator function is ultimately polynomial.
\end{example}
\begin{proof}
    TODO.
\end{proof}

\begin{proof}[Proof of \cref{zsf-npoly-nsf:thm}]
    We proceed by induction on the size of the alphabet $\card{\Sigma}$.

    \textbf{Base Case.} When $\Sigma$ is empty, $\Sigma^*$ contains
    only the empty word $\varepsilon$, and because $f$ is a
    \kl{$\Nat$-polyregular function}, $f(\varepsilon) \in \Nat$.
    Note that $f$ is therefore a constant function, equal to a natural
    number, hence belongs to $\NSF$.

    
    \textbf{Induction.}
    Because $f$ is \kl{ultimately polynomial},
    there exists $N_0 \in \Nat$, 
    and $P \in \Rel[\seqof{X_a}{a \in \Sigma}]$ 
    such that
    \begin{equation*}
        f\left( \prod_{a \in \Sigma} a^{X_a} \right)
        = 
        P(\seqof{X_a}{a \in \Sigma})
        \quad
        \text{ when }
        \forall a \in \Sigma, X_a \geq N_0
        \quad .
    \end{equation*}
    Furthermore, leveraging \cref{decompose-polynomial:lem},
    we obtain $\omega \in \Nat$
    and a family of polynomials 
    $\seqof{P_{\seqof{n_a}{a \in \Sigma}}}{0 \leq n_a < \omega}$
    such that.
    Because each of those polynomials coincide with $P$ over an infinite
    subgrid of $\Nat$,
    we conclude that all polynomials are in fact equal to a single one.
    we conclude that the polynomials 
    $P_{\seqof{n_a}{a \in \Sigma}}$ are all equal.

    Now, we conclude in particular that the function 
    computing $P$ is \kl{$\Nat$-polyregular}, because $f$ is,
    hence is computable by a \kl{star-free $\Nat$-polyregular function},
    because of \cref{nsf-polynomials:thm}.
    When fixing values of some indeterminate below $N_0$,
    we obtain a new function $g$, that remains commutative,
    remains ultimately polynomial, and remains \kl{$\Nat$-polyregular}.
    By induction hypothesis, this function $g$ is also 
    \kl{star-free $\Nat$-polyregular}, and we conclude
    that 
    $f = P + \sum g$ which remains star free.
\end{proof}


\subsection{Non-negative Residual Transducers}

The goal of the upcoming section is to provide a notion of residual transducer
for \kl{commutative} \kl{$\Nat$-polyregular functions}.

\begin{fact}
    For every $K \in \Nat$, and every $n \in \Nat$,
    $\Nat^n$ is a \kl{well-quasi-ordering}
    when endowed with the quasi-ordering $\vec{u} \leq_K \vec{v}$
    if and only if $\vec{u} = \vec{v}$ or
    $\vec{u}$ is smaller than $\vec{v}$ for the product ordering and
    $\sum_{i = 1}^n  (\vec{v} - \vec{u})_i \geq K$.
\end{fact}
\begin{proof}
    First, $\Nat^n$ with the product ordering is a \kl{well-quasi-ordering}.
    Therefore, given a sequence $\seqof{\vec{u}_i}{i \in \Nat}$,
    one can extract an infinite monotone subsequence for
    the product ordering.
    Either there exists $i < j$ such that $\vec{u}_i = \vec{u}_j$
    in the subsequence, in which case $\vec{u}_i \leq \vec{u}_j$
    and the original sequence was good. Or for all $i < j$,
    we have $\vec{u}_i$ that is strictly smaller than 
    $\vec{u}_j$ for the product ordering. As a consequence,
    $\vec{u}_0 \leq \vec{u}_K$.
\end{proof}


\begin{fact}
    Let $P \in \Nat[X,\vec{Y}]$, and $K,L \in \Nat$,
    then $P(X+L, \vec{Y}) - P(X+K,\vec{Y}) \in \Nat[X, \vec{Y}]$
    whenever $L \geq K$.
\end{fact}
\begin{proof}
    We can write
    $P = \sum_{i = 0}^n P_i(\vec{Y}) X^i$, with $P_i \in \Nat[\vec{Y}]$.
    Now,
    $P(X+L,\vec{Y}) - P(X+K,\vec{Y})$
    is precisely
    $\sum_{i = 1}^n P_i(\vec{Y}) (L-K) \left[ \sum_{j + l + 1= i} (X+L)^j (X+K)^l \right]$,
    remaining with positive coefficients.
\end{proof}

\begin{fact}
    Let $P \in \CorrectPoly[X,\vec{Y}]$, and $K,L \in \Nat$,
    then $P(X+L, \vec{Y}) - P(X+K,\vec{Y}) \in \CorrectPoly$
    whenever $L \geq K$.
\end{fact}
\begin{proof}
    We can write
    $P = \sum_{i = 0}^n P_i(\vec{Y}) X^i$, with $P_i \in \Nat[\vec{Y}]$.
    Now,
    $P(X+L,\vec{Y}) - P(X+K,\vec{Y})$
    is precisely
    $\sum_{i = 1}^n P_i(\vec{Y}) (L-K) \left[ \sum_{j + l + 1= i} (X+L)^j (X+K)^l \right]$.
    Now, $P_i(\vec{Y})$
    remaining with positive coefficients.
\end{proof}

\begin{fact}
    Let $P \in \CorrectPoly[X_1, \dots, X_n]$. 
    For all
    $\vec{u} \leq \vec{v} \in \Nat^n$,
    $Q(X_1, \dots, X_n)
    \defined 
    P(X_1 + v_1, \dots, X_n + v_n)
    -
    P(X_1 + u_1, \dots, X_n + u_n)$
    belongs to $\CorrectPoly[X_1, \dots, X_n]$.
\end{fact}
\begin{proof}
    Let us consider $K \in \Nat$ such that for all partial valuations
    $\nu \colon \vec{X} \topartial \Nat$,
    $\translate{K}(\restr{P}{\nu}) \in \Nat[\vec{X}]$.

    Remark that
    \begin{align*}
        Q
        &= 
        P(X_1 + v_1, \dots, X_n + v_n)
        -
        P(X_1 + u_1, \dots, X_n + u_n) \\
        &=
        P(X_1 + v_1, X_2 + v_2 , \dots, X_n + v_n) - P(X_1 + u_1, X_2 + v_2, \dots, X_n + v_n) \\
        &+
        P(X_1 + u_1, X_2 + v_2, \dots, X_n + v_n) - P(X_1 + u_1, X_2 + u_2, \dots, X_n + v_n) \\
        &+ \dots
    \end{align*}

    Let $\nu \colon \vec{X} \topartial \Nat$
    be a partial valuation, let us show that
    $\translate{K}(\restr{Q}{\nu}) \in \Nat[\vec{X}]$.

    \begin{align*}
        \translate{K}(\restr{Q}{\nu})
        &= 
        \translate{K}(\restr{P(X_1 + v_1, \dots, X_n + v_n)}{\nu})
        -
        \translate{K}(\restr{P(X_1 + u_1, \dots, X_n + u_n)}{\nu})
    \end{align*}
\end{proof}

\begin{lemma}
    Let $k \in \Nat$,
    $f \in \NPoly[k]$,
    and $\seqof{u_i}{i \in \Nat}$ be a sequence of
    elements in $\Nat^n$. There exists $i < j$
    such that
    $f(u_j \cdot) - f(u_i \cdot) \in \NPoly[k-1]$.
\end{lemma}

\begin{defintion}
    Residual $\Nat$-transducer.
\end{definition}

\begin{lemma}
    Computable 
\end{lemma}

\begin{theorem}
    If and only if this one is counter-free.
\end{theorem}

Note that it is not needed for decidability, as we are able to use directly the
results on $\ZSF$, but this provides an independent proof, and highlights the
versatility of this notion of "residual transducer".


