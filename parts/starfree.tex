%! TeX program = xelatex
%! lang = en-US
\section{Beyond Polynomials}
\label{beyond-polynomials:sec}
\label{star-free:sec}

The goal of this section is to go beyond polynomials and consider the case of
\kl{commutative} \kl{$\Rel$-polyregular functions}. The hypothesis of
\kl{commutativity} reduces $\ZPoly$ to a combination of a finite number of
polynomials, arranged by counting modulo some numbers. This characterization
will be proven in \cref{decompose-polynomial:lem}, and will be the key
ingredient in the decision procedures of $\NPoly$ inside $\ZPoly$, and $\NSF$
inside $\NPoly$. Note that \cref{decompose-polynomial:lem} is actually a
refinement of \cref{n-poly-combinatorics:lem} under the extra assumption of
\kl{commutativity}. 

\AP To actually state the lemma, we will need some notations that will
formalize the intuitive idea of counting tuples of integers modulo some
constant. Let $\omega \in \Nat$, we define the set $\intro*\ModuloTypes[\omega]
\defined \set{0, \dots, \omega^2} \times \set{0, \dots, \omega - 1}$ of
\intro{types modulo $\omega$}. To a number $n \in \Nat$, we associate its
\reintro{$\omega$-type} written $\intro*\moduloType[\omega](n)$ which is
defined as the pair $(t, r)$ where $t = \min (n, \omega^2)$ and $r$ is the rest
of the Euclidian division of $n$ by $\omega$. This notion of type is lifted to
vectors in $\Nat^p$ for any $p \in \Nat$ by pointwise application.


\begin{lemma}
    \label{decompose-polynomial:lem}
    Let $f \colon \Sigma^* \to \Rel$ be a \kl{commutative}
    \kl{$\Rel$-polyregular function},
    where we fix the alphabet $\Sigma = \set{a_1, \dots, a_n}$.
    There exists a computable
    $\omega \in \Nat$,
    and computable 
    polynomials $\seqof{P_{t}}{t \in \ModuloTypes[\omega]^n}$
    such that for all $w \in \Sigma^*$,
    \begin{equation*}
        f\left(a_1^{X_1} \cdots a_n ^{X_n}\right) 
        = P_{\moduloType[\omega](X_1, \dots, X_n)}
        \left(
            \floor{X_1/\omega}, \dots, \floor{X_n/\omega}
        \right)
        \quad .
    \end{equation*}
\end{lemma}

In the spirit of previous characterizations of subclasses of $\ZPoly$ in terms
of \emph{polynomial pumping arguments}
\cite{doueneau2021pebble,doueneau2022hiding,LOPEZ23b}, we will provide a
semantic property in the form of \cref{k-combinatorial:def}, that is conjecture
to characterize $\NPoly$ inside $\ZPoly$, even in the non-commutative setting.

\begin{definition}
    \label{k-combinatorial:def}
    Let $k \in \Nat$, and $f \colon \Sigma^* \to \Rel$
    be a \kl{$\Rel$-polyregular function}. The function $f$ is 
    \intro{$(k,\Nat)$-combinatorial} if there exists $\omega \in \Nat$,
    such that
    for all
    $\alpha_0, \dots, \alpha_k \in \Sigma^*$,
    for all $u_1, \dots, u_k \in \Sigma^*$,
    there exists a polynomial $P \in \CorrectPoly$,  
    satisfying for all $X_1, \dots, X_k \geq \omega$:
    \begin{equation*}
        f
        \left(
            \alpha_0 \prod_{i = 1}^k u_i^{\omega X_i} \alpha_i
        \right)
        = 
        P(X_1, \dots, X_k) \quad .
    \end{equation*}
    We say that $f$ is \reintro{combinatorial}
    if it is \reintro{$(k,\Nat)$-combinatorial} for all $k \in \Nat$.
\end{definition}

\begin{theorem}
    \label{decidable-n-poly:thm}
    Let $k \in \Nat$, and 
    let $f \in \ZPoly[k]$ be a \kl{commutative} \kl{$\Rel$-polyregular function}.
    The following are equivalent:
    \begin{enumerate}
        \item \label{f-combinatorial:item} $f$ is \kl{$(k,\Nat)$-combinatorial},
        \item \label{f-npoly-combi:item} $f \in \NPoly[k]$,
    \end{enumerate}
    Furthermore, the properties are decidable,
    and conversions effective.
\end{theorem}

Let us now prove that the above characterizations of \kl{commutative}
\kl{$\Nat$-polyregular functions} can be combined with the recent advances in
the study of \kl{$\Rel$-polyregular functions} \cite{LOPEZ23b} to decide
aperiodicity. The key ingredient of this study is the use of a semantic
characterization of \kl{star-free $\Rel$-polyregular functions} among
\kl{$\Rel$-rational series} that generalizes the aperiodicity of languages to
functions by the means of polynomial behaviors (see
\cref{aperidic-ultimately-polynomial:ex}).

\begin{definition}[Ultimately polynomial]
    \label{ultimately-polynomial:def}
    Let $\Sigma$ be a finite alphabet. 
    A function $f \colon \Sigma^* \to \Rel$
    is \intro{ultimately polynomial}
    whenever there exists $N_0 \in \Nat$ such that
    for all $n \in \Nat$
    and for all words $\alpha_0, w_1, \alpha_1, \cdots, \alpha_{n-1}, w_n, \alpha_n
    \in \Sigma^*$, there exists a polynomial $P \in \Rel[X_1, \dots, X_n]$
    such that
    \begin{equation*}
        f\left(
            \alpha_0 \prod_{i = 1}^{n} w_i^{X_i} \alpha_i
        \right)
        = 
        P(X_1, \dots, X_n)
        \quad 
        \forall X_1, \dots, X_n \geq N_0
        \quad .
    \end{equation*}
\end{definition}

\begin{example}
    \label{aperidic-ultimately-polynomial:ex}
    A language $L$ is aperiodic if and only if 
    $\ind{L}$ is \kl{ultimately polynomial}.
\end{example}

\begin{theorem}
    \label{zsf-npoly-nsf:thm}
    Let $\Sigma$ be a finite alphabet, 
    and $f \colon \Sigma^* \to \Rel$ be a \kl{commutative}
    \kl{$\Nat$-polyregular function}.
    Then, the following are equivalent:
    \begin{enumerate}
        \item $f$ is \kl{ultimately polynomial},
        \item $f \in \ZSF$,
        \item $f \in \NSF$.
    \end{enumerate}
    Furthermore, membership is decidable and conversions are effective.
\end{theorem}


