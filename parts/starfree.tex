%! TeX program = xelatex
%! lang = en-US
\section{Beyond Polynomials}
\label{beyond-polynomials:sec}
\label{star-free:sec}

In this section, we leverage the decidability results of \cref{polynomials:sec}
to decide $\NPoly$ inside $\ZPoly$ and $\NSF$ inside $\NPoly$, both under the
extra assumption of \kl{commutativity}. The hypothesis of \kl{commutativity}
reduces $\ZPoly$ to a combination of a finite number of polynomials, arranged
by counting modulo some numbers. This characterization will be proven in
\cref{decompose-polynomial:lem}, and will be the key ingredient in the decision
procedures of $\NPoly$ inside $\ZPoly$, and $\NSF$ inside $\NPoly$. Note that
\cref{decompose-polynomial:lem} refines
\cref{n-poly-combinatorics:lem} for \kl{commutative} functions.

\AP To actually state the lemma, we will need some notations that will
formalize the intuitive idea of counting tuples of integers modulo some
constant. Let $\omega \in \Nat$, we define the set $\intro*\ModuloTypes[\omega]
\defined \set{0, \dots, \omega^2} \times \set{0, \dots, \omega - 1}$ of
\intro{types modulo $\omega$}. To a number $n \in \Nat$, we associate its
\reintro{$\omega$-type} written $\intro*\moduloType[\omega](n)$ which is
defined as the pair $(t, r)$ where $t = \min (n, \omega^2)$ and $r$ is the  remainder
of the Euclidean division of $n$ by $\omega$. This notion of type is lifted to
vectors in $\Nat^p$ for any $p \in \Nat$ by pointwise application.


\begin{lemma}
    \label{decompose-polynomial:lem}
    Let $f \colon \Sigma^* \to \Rel$ be a \kl{commutative}
    \kl{$\Rel$-polyregular function},
    where we fix the alphabet $\Sigma = \set{a_1, \dots, a_n}$.
    There exists a computable
    $\omega \in \Nat$,
    and computable 
    polynomials $\seqof{P_{t}}{t \in \ModuloTypes[\omega]^n}$
    such that for all $w \in \Sigma^*$,
    \begin{equation*}
        f\left(a_1^{X_1} \cdots a_n ^{X_n}\right) 
        = P_{\moduloType[\omega](X_1, \dots, X_n)}
        \left(
            \floor{X_1/\omega}, \dots, \floor{X_n/\omega}
        \right)
        \quad .
    \end{equation*}
    \proofref{decompose-polynomial:lem}
\end{lemma}

In the spirit of previous characterizations of subclasses of $\ZPoly$ in terms
of \emph{polynomial pumping arguments} \cite{DOUE21,DOUE22,CDTL23}, we will
provide a semantic property in the form of \cref{k-combinatorial:def}, that is
conjectured to characterize $\NPoly$ inside $\ZPoly$ in the general setting.

\begin{definition}
    \label{k-combinatorial:def}
    Let $k \in \Nat$, and $f \colon \Sigma^* \to \Rel$
    be a \kl{$\Rel$-polyregular function}. The function $f$ is 
    \intro{$(k,\Nat)$-combinatorial} if there exists $\omega \in \Nat$,
    such that
    for all
    $\alpha_0, \dots, \alpha_k \in \Sigma^*$,
    for all $u_1, \dots, u_k \in \Sigma^*$,
    there exists a polynomial $P \in \CorrectPoly$,  
    satisfying for all $x_1, \dots, x_k \geq \omega$:
    \begin{equation*}
        f
        \left(
            \alpha_0 \prod_{i = 1}^k u_i^{\omega x_i} \alpha_i
        \right)
        = 
        P(x_1, \dots, x_k) \quad .
    \end{equation*}
\end{definition}

\begin{theorem}
    \label{decidable-n-poly:thm}
    Let $k \in \Nat$, and 
    let $f \in \ZPoly[k]$ be a \kl{commutative} \kl{$\Rel$-polyregular function}.
    The following are equivalent:
    \begin{enumerate}
        \item \label{f-combinatorial:item} $f$ is \kl{$(k,\Nat)$-combinatorial},
        \item \label{f-npoly-combi:item} $f \in \NPoly[k]$,
    \end{enumerate}
    Furthermore, the properties are decidable,
    and conversions effective.
\end{theorem}
\begin{proof}
    Let us prove that
    \cref{f-npoly-combi:item}
    implies \cref{f-combinatorial:item}. 
    We compute the $\omega$ by applying 
    \cref{n-poly-combinatorics:lem}, and then
    notice that the polynomial $P$ obtained is
    by definition a \kl{$\Nat$-rational polynomial}.

    For the converse implication, we use \cref{decompose-polynomial:lem} to
    effectively compute a polynomial representation of $f$. 

    \textbf{TODO: this is because they coincide en enough points}.
    Using the
    assumption, all of these polynomials are actually \kl{$\Nat$-rational
    polynomials}. Let us now construct a new way to compute $f$, by first
    computing the \kl{$\omega$-type} of the input (which is possible in
    $\MSO$), and then computing the suitable \kl{$\Nat$-rational polynomial},
    which is possible in $\NPoly$.
\end{proof}

Let us now prove that the above characterizations of \kl{commutative}
\kl{$\Nat$-polyregular functions} can be combined with the recent advances in
the study of \kl{$\Rel$-polyregular functions} \cite{CDTL23} to decide
aperiodicity. The key ingredient of this study is the use of a semantic
characterization of \kl{star-free $\Rel$-polyregular functions} among
\kl{$\Rel$-rational series} that generalizes the aperiodicity of languages to
functions by the means of polynomial behaviors (see
\cref{aperidic-ultimately-polynomial:ex}). Note that up until now, we only had
to consider polynomials with coefficients in $\Rel$. However, there are some
polynomials with coefficients in $\Rat$ that can be \kl{represented} by a
\kl{$\Nat$-polyregular function}. For instance, the polynomial $P(X) \defined
X(X+1)/2$. This is why, in the upcoming definition, we have to consider
polynomials with coefficients in $\Rat$.

\begin{definition}[Ultimately polynomial]
    \label{ultimately-polynomial:def}
    Let $\Sigma$ be a finite alphabet. 
    A function $f \colon \Sigma^* \to \Rel$
    is \intro{ultimately polynomial}
    when there exists $N_0 \in \Nat$ such that
    for all $n \in \Nat$
    and for all words $\alpha_0, w_1, \alpha_1, \cdots, \alpha_{n-1}, w_n, \alpha_n
    \in \Sigma^*$, there exists a polynomial $P \in \Rat[X_1, \dots, X_n]$
    such that:
    \begin{equation*}
        f\left(
            \alpha_0 \prod_{i = 1}^{n} w_i^{x_i} \alpha_i
        \right)
        = 
        P(x_1, \dots, x_n)
        \quad 
        \forall x_1, \dots, x_n \geq N_0
        \quad .
    \end{equation*}
\end{definition}

Let us recall that a language $L$ is \intro(language){aperiodic} whenever it is
recognized by an \kl(monoid){aperiodic} finite monoid, which turns out to be
equivalent to the following characterisation among regular languages: for all
$u,v,w$, there exists $n \in \Nat$, $u v^n w \in L$ if and only if $u v^{n+1} w
\in L$. As a consequence of this characterization, a language $L$ is aperiodic
if and only if its indicator function $\ind{L}$ is \kl{ultimately polynomial}.
This was already observed in \cite[\textbf{statement?}]{CDTL23}, and reproven
in \cite[\textbf{statement?}]{DOUE23}, but we include a proof hereafter for
readability and didactic purposes.

\begin{example}
    \label{aperidic-ultimately-polynomial:ex}
    A language $L$ is aperiodic if and only if 
    $\ind{L}$ is \kl{ultimately polynomial}.
\end{example}
\begin{proof}
    Assume that $\ind{L}$ is \kl{ultimately polynomial}. Let $u,v,w \in
    \Sigma^*$. There exists a polynomial $P \in \Rat[X]$ and a number $N_0 \in
    \Nat$, such that $\ind{L}(u v^n w) = P(n)$ for all $n \geq N_0$. Because
    $\ind{L}$ has co-domain included in $\set{0,1}$, so does $P$, which proves
    that $P$ is a constant polynomial. Then, $\ind{L}(u v^{n+1} w) = P(n+1) =
    P(n) = \ind{L}(uv^n w)$ for all $n \geq N_0$. We have proven that $L$
    is aperiodic.

    Conversely, assume that $L$ is aperiodic, let $n \in \Nat$ be a number, and
    let $\alpha_0$, $w_1$, $\alpha_1$, $\ldots$, $\alpha_{n-1}$, $w_n$, $\alpha_n$
    be words in $\Sigma^*$. Because $L$ is \kl(language){aperodic}, the language
    $L$ is recognized by an \kl(monoid){aperiodic} finite monoid $M$, via a
    morphism $\mu \colon \Sigma^* \to M$ and an accepting part $F \subseteq M$.
    Because $M$ is a \kl(monoid){aperiodic}, there exists $N_0$
    such that for all $m \in M$, $m^{N_0+1} = m^{N_0}$. As a consequence,
    for all $1 \leq i \leq n$,
    whenever $x_i \geq N_0$, we have $\mu(w_i)^{x_i} = \mu(w_i)^{N_0}$.
    Therefore, for all $x_1, \dots, x_n \geq N_0$, we have:
    \begin{align*}
        \ind{L}\left(\alpha_0 \prod_{i= 1}^n w_i^{x_i} \alpha_i \right) 
        &= 
        \ind{P}\left(\mu(\alpha_0) \cdot \prod_{i= 1}^n \mu(w_i)^{x_i} \cdot \mu(\alpha_i)\right)
        & \text{ because $M$ recognizes $L$}
        \\
        &= 
        \ind{P}\left(\mu(\alpha_0) \cdot \prod_{i= 1}^n \mu(w_i)^{N_0} \cdot \mu(\alpha_i)\right)
        & \text{ by aperiodicity }
        \\
        &\defined P(x_1, \dots, x_n) & \text{ constant polynomial }
    \end{align*}
    Which proves that $\ind{L}$ is \kl{ultimately polynomial}.
\end{proof}

\AP The decidability of aperiodicity for \kl{$\Rel$-polyregular functions}, that is
the membership of $\ZSF$ inside $\ZPoly$,
relies on the construction of a canonical object called the \emph{residual
transducer}, the latter being essentially based on differences between
functions, crucially leveraging negative outputs \cite{CDTL23}. Although the
proof method does not carry from $\Rel$-output functions to $\Nat$-output
functions, it was conjectured that the semantic property of being
\kl{ultimately polynomial} would also characterize $\NSF$ inside $\NPoly$, a
conjecture that is restated hereafter.

\begin{conjecture}[{\cite[Conjecture 7.61]{DOUE23}}]
    \label{zsf-nsf:conjecture}
    Let $k \in \Nat$.
    A function $f \in \NPoly[k]$
    belongs to $\NSF[k]$ if and only if
    it is \kl{ultimately polynomial}.
    In particular,
    $\NSF[k] = \ZSF[k] \cap \NPoly$.
\end{conjecture}

We answer
\cref{zsf-nsf:conjecture} 
positively in the commutative case, by
leveraging the semantic characterizations respectively of $\ZSF$ inside
$\ZPoly$ (\kl{ultimately polynomial}) and $\NPoly$ inside $\ZPoly$
(\kl{$(k,\Nat)$-combinatorial}), which is possible thanks to the decomposition
obtained in \cref{decompose-polynomial:lem}. Decidability then follows
immediately from the effective conversions, and the previous decidability
result for $\ZSF$.

\begin{theorem}
    \label{zsf-npoly-nsf:thm}
    Let $\Sigma$ be a finite alphabet, 
    and $f \colon \Sigma^* \to \Rel$ be a \kl{commutative}
    \kl{$\Nat$-polyregular function}.
    Then, the following are equivalent:
    \begin{enumerate}
        \item $f$ is \kl{ultimately polynomial},
        \item $f \in \ZSF$,
        \item $f \in \NSF$.
    \end{enumerate}
    Furthermore, membership is decidable and conversions are effective.
    \proofref{zsf-npoly-nsf:thm}
\end{theorem}


