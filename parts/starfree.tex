%! TeX program = xelatex
%! lang = en-US
\section{Beyond Polynomials}
\label{sec:deciding}

The goal of this section is to go beyond polynomials and characterize all
$\Rel$-polyregular functions that are $\Nat$-polyregular, under the assumption
that they are \kl{commutative}.

\begin{theorem}
    \label{decidable-n-poly:thm}
    Given a \kl{commutative}
    \kl{$\Rel$-polyregular function} $f \colon \Sigma^* \to \Rel$,
    one can decide if $f$ can be computed by a \kl{$\Nat$-polyregular function},
    and effectively compute a representation of $f$.
\end{theorem}

To prove the theorem, let us first provide a nice characterization of
\kl{commutative} \kl{$\Rel$-polyregular functions} in terms of polynomials.
This is actually a refinement of \cref{n-poly-combinatorics:lem} in the
commutative case.

\begin{lemma}
    \label{decompose-polynomial:lem}
    Let $f \colon \Sigma^* \to \Rel$ be a \kl{commutative}
    \kl{$\Rel$-polyregular function}. There exists $\omega \in \Nat$,
    and polynomials $\seqof{P_{\seqof{n_a}{a \in \Sigma}}}{ 0 \leq n_a < \omega}$,
    such that for all $w \in \Sigma^*$,
    \begin{equation*}
        f(w) = P_{\seqof{\card[a]{w} [\omega]}{a \in \Sigma}}
        (\seqof{\card[a]{w} / \omega}{a \in \Sigma}) \quad .
    \end{equation*}
    Which informally means that for every choice of rest modulo $\omega$
    for every variable, there exists a polynomial describing the
    behavior of $f$ when adding multiples of $\omega$ letters at a time.

    Furthermore, the polynomials are computable from $f$.
\end{lemma}
\begin{proof}
    equivalent partitions of the word, but commutativity?.q
    Lemma 5.37 phd gaëtan.
\end{proof}

As a consequence, we conclude 
\begin{proof}[Proof of \cref{decidable-n-poly:thm}]
    We use \cref{decompose-polynomial:lem}
    to effectively compute
    a polynomial representation of $f$. Remark that
    $f$ is a \kl{$\Nat$-polyregular function} if and only if
    all these polynomials are \kl{$\Nat$-rational polynomials},
    which is decidable thanks to
    \cref{decide-correct-poly:lem,corrected-version:thm}.
\end{proof}


\begin{conjecture}
    $\CorrectPoly$ is precisely composed of polynomials that
    have non-negative coefficients in a \emph{binomial basis},
    i.e., where polynomials are obtained
    as sums and products of $\binom{X_i}{p_i}$.
\end{conjecture}


\subsection{Star-free $\Nat$-polyregular functions}
\label{star-free:sec}


Let us now prove that the above characterizations of
commutative
$\Nat$-polyregular functions can be combined with the recent
advances in the study of $\Rel$-polyregular functions 
\cite{LOPEZ23b}
to prove the following theorem.

\begin{theorem}
    \label{zsf-npoly-nsf:thm}
    Let $\Sigma$ be a finite alphabet, and $f \colon \Sigma^* \to \Nat$
    be both a \kl{star-free $\Rel$-polyregular function}
    and a \kl{commutative} \kl{$\Nat$-polyregular function}.
    Then, $f$ is a \kl{commutative} \kl{star-free $\Nat$-polyregular function}.
    That is, the following equality holds:
    \begin{equation*}
        \ZCommut \cap \ZSF[k] \cap \NPoly
        = \ZCommut \cap \ZSF \cap \NPoly[k]
        = \ZCommut \cap \NSF
        \quad .
    \end{equation*}
\end{theorem}

Note that this proves in the commutative case a conjecture that was 
formulated by Thesis Gaetan, and that is non trivial, because 

\begin{conjecture}
    \label{nsf-zsf:conj}
    In the general case, the following equation holds:
    \begin{equation*}
        \ZSF \cap \NPoly
        = 
        \NSF
        \quad .
    \end{equation*}
\end{conjecture}

As a corollary of \cref{zsf-npoly-nsf:thm} and the effective decision
procedures of $\ZSF$ obtained in \cite{LOPEZ23b}, we conclude that $\NSF$ is
decidable inside $\NPoly$ under the assumption of commutativity.

\begin{corollary}
    One can decide if a \kl{commutative} \kl{ $\Rel$-polyregular function}
    can be realized by a \kl{star-free $\Nat$-polyregular function},
    in which case the conversion is effective.
\end{corollary}

The key ingredient of this section is the use of a semantic characterization of
\kl{star-free $\Rel$-polyregular functions} among \kl{$\Rel$-rational series}
that generalizes the aperiodicity of languages to functions by the means of
polynomial behaviors.

\begin{definition}[Ultimately polynomial]
    \label{ultimately-polynomial:def}
    Let $\Sigma$ be a finite alphabet. 
    A function $f \colon \Sigma^* \to \Rel$
    is \intro{ultimately polynomial}
    whenever there exists $N_0 \in \Nat$ such that
    for all $n \in \Nat$
    and for all words $\alpha_0, w_1, \alpha_1, \cdots, \alpha_{n-1}, w_n, \alpha_n
    \in \Sigma^*$, there exists a polynomial $P \in \Rel[X_1, \dots, X_n]$
    such that
    \begin{equation*}
        f\left(
            \alpha_0 \prod_{i = 1}^{n} w_i^{X_i} \alpha_i
        \right)
        = 
        P(X_1, \dots, X_n)
        \quad 
        \forall X_1, \dots, X_n \geq N_0
        \quad .
    \end{equation*}
\end{definition}


Let us recall from \cite{LOPEZ23b} that this definition correctly
generalizes the indicator functions of regular languages.

\begin{example}
    A language $L$ is aperiodic if and only if 
    the indicator function is ultimately polynomial.
\end{example}
\begin{proof}
    TODO.
\end{proof}

\begin{proof}[Proof of \cref{zsf-npoly-nsf:thm}]
    We proceed by induction on the size of the alphabet $\card{\Sigma}$.

    \textbf{Base Case.} When $\Sigma$ is empty, $\Sigma^*$ contains
    only the empty word $\varepsilon$, and because $f$ is a
    \kl{$\Nat$-polyregular function}, $f(\varepsilon) \in \Nat$.
    Note that $f$ is therefore a constant function, equal to a natural
    number, hence belongs to $\NSF$.

    
    \textbf{Induction.}
    Because $f$ is \kl{ultimately polynomial},
    there exists $N_0 \in \Nat$, 
    and $P \in \Rel[\seqof{X_a}{a \in \Sigma}]$ 
    such that
    \begin{equation*}
        f\left( \prod_{a \in \Sigma} a^{X_a} \right)
        = 
        P(\seqof{X_a}{a \in \Sigma})
        \quad
        \text{ when }
        \forall a \in \Sigma, X_a \geq N_0
        \quad .
    \end{equation*}
    Furthermore, leveraging \cref{decompose-polynomial:lem},
    we obtain $\omega \in \Nat$
    and a family of polynomials 
    $\seqof{P_{\seqof{n_a}{a \in \Sigma}}}{0 \leq n_a < \omega}$
    such that.
    Because each of those polynomials coincide with $P$ over an infinite
    subgrid of $\Nat$,
    we conclude that all polynomials are in fact equal to a single one.
    we conclude that the polynomials 
    $P_{\seqof{n_a}{a \in \Sigma}}$ are all equal.

    Now, we conclude in particular that the function 
    computing $P$ is \kl{$\Nat$-polyregular}, because $f$ is,
    hence is computable by a \kl{star-free $\Nat$-polyregular function},
    because of \cref{nsf-polynomials:thm}.
    When fixing values of some indeterminate below $N_0$,
    we obtain a new function $g$, that remains commutative,
    remains ultimately polynomial, and remains \kl{$\Nat$-polyregular}.
    By induction hypothesis, this function $g$ is also 
    \kl{star-free $\Nat$-polyregular}, and we conclude
    that 
    $f = P + \sum g$ which remains star free.
\end{proof}

