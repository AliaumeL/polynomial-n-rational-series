%! TeX program = xelatex
%! lang = en-US
\section{$\Nat$-rational Polynomials}
\label{polynomials:sec}


\AP In this section, we will completely characterize which polynomials in
$\Rat[\vec{X}]$ are \kl{represented} by \kl{$\Nat$-rational series} (resp.
\kl{$\Rel$-rational series}). To that end, we start by characterizing these
classes for polynomials in $\Rel[\vec{X}]$. We say that a polynomial $P \in
\Rel[X_1, \dots, X_n]$ is an \intro{$\Nat$-rational polynomial} if it is
\kl{represented} by a \kl{$\Nat$-rational series}. It is an easy check that
polynomials with coefficients in $\Nat$ are \kl{$\Nat$-rational polynomials}
(\cref{n-poly-n-poly:example}). However, \cref{negative-but-npoly:ex} provides
a polynomial with negative coefficients that is an \kl{$\Nat$-rational
polynomial}. The problem of characterizing \kl{$\Nat$-rational polynomials} was
claimed to be solved in \cite{KARH77}, using the \cref{karh:def} to
characterize \kl{$\Nat$-rational polynomials}, as restated in \cref{karh:thm}.

\begin{lemma}[restate=n-poly-n-poly:example,label=n-poly-n-poly:example]
    \proofref{n-poly-n-poly:example}
    Let $P \in \Nat[\vec{X}]$. Then, $P$
    is an \kl{$\Nat$-rational polynomial}.
\end{lemma}

\begin{example}[restate=negative-but-npoly:ex,label=negative-but-npoly:ex]
    \proofref{negative-but-npoly:ex}
    The polynomials $X$, $X^2 + 3$,
    and $X^2 - 2X + 2$
    are \kl{$\Nat$-rational polynomials},
    but $- X$ is 
    not an \kl{$\Nat$-rational polynomial}.
\end{example}



\begin{definition}[{\cite[Section 3, page 3]{KARH77}}]
    \label{karh:def}
    The class $\intro*\CoveredPoly[\vec{X}]$
    is the class of polynomials $P \in \Rel[\vec{X}]$
    that are \kl{non-negative}
    and such that every \kl{maximal monomial} is \kl{non-negative}.
    When the indeterminates are clear from the context, we write
    this class $\reintro*\CoveredPoly$.
\end{definition}

\begin{faketheorem}[{\cite[Theorem 3.3, page 4]{KARH77}}] 
    \label{karh:thm}
    Let $P \in \Rel[\vec{X}]$ be a polynomial. Then,
    $P$ is an \kl{$\Nat$-rational polynomial}
    if and only if 
    $P \in \CoveredPoly$.
\end{faketheorem}

Before giving a counterexample to the above statement, let us first exhibit in
\cref{non-neg-not-nrat:ex} some
\kl{non-negative} polynomial that is not an \kl{$\Nat$-rational polynomial}.
While the example will not be in $\CoveredPoly$, it illustrates the key
difference between \kl{non-negative} polynomials and \kl{$\Nat$-rational
polynomials}. In order to derive this example, we will need the following
fundamental result about the pre-image of regular languages by \kl{polyregular
functions}.\footnote{ In this particular case, one could have considered more
generally \kl{$\Nat$-rational series}, and replaced regular languages over a
unary alphabet by semi-linear sets. } Before that, let us remark that if a
polynomial $P$ is \kl{represented} by a \kl{$\Nat$-rational series}, then it is
in fact \kl{represented} by a \kl{$\Nat$-polyregular function} thanks to
\cref{polyregular-polynomial-growth:lemma}.

\begin{theorem}[{\cite[Theorem 1.7]{BOJA18}}]
    \label{pre-image-regular:fact}
    The pre-image of a regular language by a (string-to-string) \kl{polyregular function}
    is a regular language.
\end{theorem}

\begin{example}
    \label{non-neg-not-nrat:ex}
    Let $P(X, Y) \defined (X - Y)^2$.
    Then $P$ is \kl{non-negative}, but is
    not an \kl{$\Nat$-rational polynomial}.
    Indeed, assume by contradiction that
    $f \in \NPoly$ \kl{represents} $P$ over the alphabet $\Sigma \defined \set{a,b}$.
    Then, $f^{-1}(\set{0})$ is a regular language
    (\cref{pre-image-regular:fact}),
    but $f^{-1}(\set{0}) = \setof{ w \in \Sigma }{ \card[a]{w} = \card[b]{w} }$
    is not.
\end{example}

Please note that the same argument cannot be leveraged for proving that $P$ is
not representd by a \kl{$\Rel$-rational series}: \cref{pre-image-regular:fact}
only holds for
\emph{string-to-string} functions, and is applied to the specific case where
the output alphabet is $\set{1}$, i.e., where the output of the function
belongs to $\set{1}^*$ which is isomorphic to $\Nat$.


\AP
Let us now design a counterexample to \cref{karh:thm} by suitably tweaking
\cref{non-neg-not-nrat:ex} to ensure that the polynomial not only is
\kl{non-negative}, but also belongs to $\CoveredPoly$.
\label{def:bad-polynomial}
We define $\intro*\BadPoly(X,Y,Z) \defined Z (X + Y)^2 + 2 (X - Y)^2$.

\begin{lemma}
    \label{thm:counter-example}
    The polynomial $\BadPoly$ belongs to $\CoveredPoly$,
    but is not an \kl{$\Nat$-rational polynomial}.
    As a corollary, \cite[Theorem 3.3]{KARH77}, restated
    in \cref{karh:thm}, is false
    when allowing at least $3$ indeterminates.
\end{lemma}
\begin{proof}
    It is clear that $\BadPoly$ is \kl{non-negative}. We can expand
    the expression of $\BadPoly$ to 
    obtain
    $\BadPoly = ZX^2 + ZY^2 + 2ZXY + 2X^2 -4XY + 2Y^2$.
    The \kl{maximal monomials} of $P$
    are $ZX^2$, $ZY^2$, and $2ZXY$, all of which are
    \kl{non-negative}.

    Assume by contradiction that $\BadPoly$ is an \kl{$\Nat$-rational polynomial}.
    Let $\Sigma \defined \set{a,b,c}$ be a finite alphabet.
    There exists a \kl{commutative}
    \kl{$\Nat$-polyregular function} $f \colon \Sigma^* \to \Nat$
    such that for all $w \in \Sigma^*$,
    $\BadPoly(\card[a]{w}, \card[b]{w}, \card[c]{w}) = f(w)$.
    Remark that for all $x,y,z \geq 0$, $\BadPoly(x,y,z) = 0$
    if and only if $z(x+y)^2 = -2 (x-y)^2$. Hence,
    $\BadPoly(x,y,z)=0$ if and only if $z = 0$ and $x = y$, or 
    $z \neq 0$, and $x = y = 0$.
    Now, let us consider the language $L \defined \setof{w}{ f(w) = 0}$. By the
    above computation, we conclude that $L = \setof{ w \in \set{a,b}^* }{
    \card[a]{w} = \card[b]{w} } \cup \set{ c }^*$.
    Because $L \cap \set{a,b}^*$ is not a regular language,
    we
    conclude that $L$ is not a regular language.
    However, $L = f^{-1}(\set{0})$ is a regular language
    (\cref{pre-image-regular:fact}). 
\end{proof}
