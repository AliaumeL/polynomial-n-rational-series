%! TeX program = xelatex
%! lang = en-US
\section{$\Nat$-rational Polynomials}
\label{polynomials:sec}

The goal of this section is to provide semantic characterizations of
\kl{$\Nat$-rational polynomials}, decide if a polynomial $P$ is an
\kl{$\Nat$-rational polynomial}, and if so, explicitly compute an
\kl{$\Nat$-polyregular function} $f$ that \kl{represents} $P$. It is an easy
check that polynomials with coefficients in $\Nat$ are \kl{$\Nat$-rational
polynomials} (\cref{n-poly-n-poly:example}). However,
\cref{negative-but-npoly:ex} provides a polynomial with negative coefficients
that is an \kl{$\Nat$-rational polynomial}. As a consequence, the class of
interest strictly contains $\Nat[\vec{X}]$.

\begin{lemma}
    \label{n-poly-n-poly:example}
    Let $P \in \Nat[\vec{X}]$. Then, $P$
    is an \kl{$\Nat$-rational polynomial}.
    \proofref{n-poly-n-poly:example}
\end{lemma}

\begin{example}
    \label{negative-but-npoly:ex}
    Let $P \defined X^2 -2X + 2$. Then $P$
    is an \kl{$\Nat$-rational polynomial}.
    \proofref{negative-but-npoly:ex}
\end{example}

Let us now introduce the characterization of \kl{$\Nat$-rational polynomials}
given by \cite{KARH77}, which is stated using a class
$\CoveredPoly[\vec{X}]$ of polynomials with multiple indeterminate.

\begin{definition}[{\cite[Section 3, page 3]{KARH77}}]
    Let $\vec{X}$ be a finite tuple of indeterminate.
    The class $\intro*\CoveredPoly[\vec{X}]$
    is the class of polynomials $P \in \Rel[\vec{X}]$
    that are \kl{non-negative}
    and such that every \kl{maximal monomial} is \kl{non-negative}.
    When the indeterminate are clear from the context, we write
    this class $\reintro*\CoveredPoly$.
\end{definition}

\begin{faketheorem}[{\cite[Theorem 3.3, page 4]{KARH77}}] 
    \label{karh:thm}
    Let $\Sigma$ be a finite alphabet
    and $P$ be a polynomial. Then,
    $P$ is an \kl{$\Nat$-rational polynomial}
    if and only if 
    $P \in \CoveredPoly$.
\end{faketheorem}


\subsection{The counter example}
\label{sec:c-example}

Let us now provide a counter example to \cref{karh:thm}. The counter example
will use three indeterminates, and we will prove in \cref{sec:proof} that
\cref{karh:thm} holds on polynomial with at most two indeterminate. In
particular, the examples appearing in \cite{KARH77} are not invalidated, as
they all use at most two indeterminates. Furthermore, the converse implication,
that is, that \kl{$\Nat$-rational polynomials} belong to $\CoveredPoly$ is
correct, and can even be strengthened as we will see in
\cref{corrected-version:thm}.

\begin{definition}
    \label{def:bad-polynomial}
    We define $\intro*\BadPoly(X,Y,Z) \defined Z (X + Y)^2 + 2 (X - Y)^2$.
\end{definition}

\begin{lemma}
    \label{thm:counter-example}
    The polynomial $\BadPoly$ belongs to $\CoveredPoly$,
    but is not an \kl{$\Nat$-rational polynomial}.
\end{lemma}
\begin{proof}
    It is clear that $\BadPoly$ is \kl{non-negative}. We can develop
    the expression of $\BadPoly$ to 
    obtain
    $\BadPoly = ZX^2 + ZY^2 + 2ZXY + 2X^2 -4XY + 2Y^2$.
    The \kl{maximal monomials} of $P$
    are $ZX^2$, $ZY^2$, and $2ZXY$, all of which are
    \kl{non-negative}.

    Assume by contradiction that $\BadPoly$ is an \kl{$\Nat$-rational polynomial}.
    Let $\Sigma \defined \set{a,b,c}$ be a finite alphabet.
    There exists a \kl{commutative}
    \kl{$\Nat$-polyregular function} $f \colon \Sigma^* \to \Nat$
    such that for all $w \in \Sigma^*$,
    $\BadPoly(\card[a]{w}, \card[b]{w}, \card[c]{w}) = f(w)$.

    Remark that $\BadPoly(X,Y,Z) = 0$
    if and only if $Z(X+Y)^2 = -2 (X-Y)^2$. Hence,
    $\BadPoly(X,Y,Z)=0$ if and only if $Z = 0$ and $X = Y$, or 
    $Z \neq 0$, and $X = Y = 0$.

    Now, let us consider the language $L \defined \setof{w}{ f(w) = 0}$. By the
    above computation, we conclude that $L = \setof{ w \in \set{a,b}^* }{
    \card[a]{w} = \card[b]{w} } \cup \set{ c }^*$.
    Because $L \cap \set{a,b}^*$ not a regular language,
    we
    conclude that $L$ is not a regular language.
    However, $L = f^{-1}(\set{0})$ is a regular language
    (\cref{pre-image-regular:fact}). 
\end{proof}

\begin{corollary}
    The result stated in \cite[Theorem 3.3]{KARH77}, restated
    in \cref{karh:thm}, is false
    when allowing at least $3$ indeterminates.
\end{corollary}
