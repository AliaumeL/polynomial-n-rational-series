%! TeX program = xelatex
%! lang = en-US
% MAIN MATTER
\section{Introduction}
\label{introduction:sec}

\AP Given a semiring $\mathbb{S}$, and a finite alphabet $\Sigma$, the class of
\kl{(noncommutative) $\mathbb{S}$-rational series} is defined as functions from
$\Sigma^*$ to $\mathbb{S}$ that are computed by \kl{$\mathbb{S}$-weighted
automata} \cite{BERE10}. This computational model is a generalization of the
classical notion of non-deterministic finite automata to the weighted setting,
where transitions are labeled with elements of $\mathbb{S}$. The semantics of
\kl{$\mathbb{S}$-weighted automata} on a given word $w$ is defined by the sum
over all accepting runs reading $w$, of the product of the weights of the
transitions taken along this run. In this paper, we are interested in the case
where $\mathbb{S}$ equals $\Nat$ or $\Rel$, hence, in $\Nat$-rational series
($\intro*\NRat$) and $\Rel$-rational series ($\intro*\ZRat$). It is clear that
$\NRat$ is a proper subclass of $\ZRat$, and a longstanding open problem is to
provide an algorithm that decides whether a given $\ZRat$ is in $\NRat$
\cite{KARH77}. 

\begin{problem}
    \label{n-in-z-rat:problem}
    Input: A $\ZRat$ $f$. Output: Is $f$ in $\NRat$?
\end{problem}

\Cref{n-in-z-rat:problem} recently received attention in the context of
\kl{polyregular functions}, a computational model that aims to
generalize the theory of regular languages to the setting of string-to-string
functions \cite{BOJA18}. In the case of regular languages, \emph{star-free
languages} form a robust subclass of regular languages described equivalently
in terms of first order logic \cite{MNPA71}, counter-free automata
\cite{MNPA71}, or aperiodic monoids \cite{SCHU65}. Analogously, there exists a
\emph{star-free} fragment of \kl{polyregular functions} called \kl{star-free
polyregular functions} \cite{BOJA18}. One open question in this area is to
decide whether a given \kl{polyregular function} is \kl{star-free}.

\begin{problem}
    \label{sf-polyregular:problem}
    Input: A \kl{polyregular function} $f$. Output: Is $f$ \kl{star-free}?
\end{problem}

\AP In order to approach decision problems on \kl{polyregular functions},
restricting the output alphabet to a single letter has proven to be a fruitful
method \cite{DOUE21,DOUE22}. Because words over a unary alphabet are
canonically identified with natural numbers, unary output \kl{polyregular
functions} are often called \kl{$\Nat$-polyregular functions} ($\NPoly$), and
their \emph{star-free} counterpart \kl{star-free $\Nat$-polyregular functions}
($\NSF$). Coincidentally, \kl{polyregular functions} with unary output forms a
subclass of \kl{$\Nat$-rational series}, namely the class of
\kl{$\Nat$-rational series} of \emph{polynomial growth}, i.e. the output of the
function is bounded by a polynomial in the size of the input. In \cite{CDTL23},
the authors introduced the class of \kl{$\Rel$-polyregular functions}
($\ZPoly$) as a subclass of \kl{$\Rel$-rational series} that generalizes
\kl{$\Nat$-polyregular functions} by allowing negative outputs, and showed that
membership in the \emph{star-free} subclass $\ZSF$ inside $\ZPoly$ is decidable
\cite[Theorem V.8]{CDTL23}. Although this could not be immediately leveraged to
decide $\NSF$ inside $\NPoly$, it was conjectured that $\NPoly \cap \ZSF =
\NSF$ \cite[Conjecture 7.61]{DOUE23}. It was believed that understanding the
membership problem of $\NPoly$ inside $\ZPoly$, that is, a restricted version
of \cref{n-in-z-rat:problem}, would be a key step towards proving $\NPoly \cap
\ZSF = \NSF$, which itself would give hope in designing an algorithm for
\cref{sf-polyregular:problem}. We illustrate in
\cref{previously-known-inclusions:fig} the known inclusions and related open
problems between the discussed classes of functions.

\begin{figure}
    \centering
    \includestandalone[height=5cm]{tikz/class-inclusions}
    \caption{
        Decidability and inclusions of classes of functions,
        arranged along two axes. The first one is the complexity
        of the output alphabet ($\Rel$, $\Nat$, $\Sigma$). The second
        one is the allowed computational power
        (\kl{star-free polyregular functions}, \kl{polyregular functions}, 
        \kl{rational series}).
        Arrows denote strict inclusions,
        and effectiveness (both in terms of decidability and of effective
        representation) is represented by thick double arrows. Inclusions that are
        suspected to be effective are represented using a dashed arrow together with a
        question mark.
    }
    \label{previously-known-inclusions:fig}
\end{figure}


\subparagraph*{Contributions.} In this paper, we work under the extra
assumption of \emph{commutativity}, that is, assuming that the function is
invariant under the permutation of its input. In this setting, we 
prove that $\NPoly \cap \ZSF = \NSF$
\cite[Conjecture 7.61]{DOUE23} and design an algorithm that decides whether a
function in $\ZPoly$ is in $\NPoly$ \cite[Open question 5.55]{DOUE23}.
As a consequence, the
upper left square of \cref{previously-known-inclusions:fig} has all of its
arrows decidable and with effective conversion procedures under this extra
assumption. Because \kl{$\Rel$-rational series} with \emph{polynomial growth}
are exactly \kl{$\Rel$-polyregular functions} \cite{CDTL23}, this can be seen
as decision procedure for \cref{n-in-z-rat:problem} under the extra assumption
of \emph{commutativity} and \emph{polynomial growth}. Similarly, our results
provide an algorithm for \cref{sf-polyregular:problem} under the extra
assumption of \emph{commutativity} and \emph{unary output alphabet}.

As an intermediate step, we provide a complete and decidable characterization
of polynomials in $\Rat[\vec{X}]$ that can be computed using $\NRat$ (resp.
$\ZRat)$. These characterizations uncover a fatal flaw in the proof of a former
characterization of such polynomials \cite[Theorem 3.3, page 4]{KARH77}. We
also prove that this previous results holds for polynomials with at most two
indeterminates (\cref{lem:correct-covered-2}), which may explain why
it was not detected earlier. Furthermore, these characterizations provide
effective descriptions of polynomials that can be expressed in $\ZRat$ as those
obtained using integer combinations of products of \emph{binomial coefficients}
(called \kl{integer binomial polynomials}, defined page \kpageref{integer
binomial polynomial}) and similarly for $\NRat$ by introducing the notion of
\kl{strongly natural binomial polynomials} (defined page \kpageref{strongly
natural binomial polynomial}), which we believe has its own interest. Finally,
these characterizations demonstrate that polynomials expressible by $\ZRat$
(resp. $\NRat$) are exactly those expressible by $\ZSF$ (resp. $\NSF$), that
is, polynomials are inherently \emph{star free} functions.

\subparagraph*{Outline of the paper.} In \cref{preliminaries:sec}, we provide a
combinatorial definition of \kl{$\Nat$-polyregular functions} (resp.
\kl{$\Rel$-polyregular functions}), show that one can decide if a function $f
\in \ZPoly$ is \kl{commutative} (\cref{decidable-commutative-poly:lemma}). In
\cref{polynomials:sec}, we provide a counterexample to the flawed result of
\cite[Theorem 3.3, page 4]{KARH77} (\cref{thm:counter-example}), and correct it
by providing effective characterizations of polynomials computed by $\ZRat$
(\cref{integer-binomial-polynomials:cor}) and $\NRat$
(\cref{decide-rat-poly-npoly:cor}). Finally, in \cref{beyond-polynomials:sec},
we answer positively to \cite[Open question 5.55]{DOUE23}
(\cref{decidable-n-poly:thm}) and to \cite[Conjecture 7.61]{DOUE23}
(\cref{zsf-npoly-nsf:thm}), both under the extra assumption of
\emph{commutativity}.
