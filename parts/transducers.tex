%! TeX program = xelatex
%! TeX root = ../pnrs.tex
%! lang = en-US
\section{Beyond commutativity}
\label{beyond-commutative:sec}

\AP The goal of this section is to provide tools that do not require the
assumption of \kl{commutativity}. To that end, we introduce the notion of
\kl{$k$-residual transducer}, that generalizes the notion of residual
transducer introduced for \kl{$\Rel$-polyregular functions} in \cite{LOPEZ23b}.
This notion of residual transducer has its own interest, as it shifts the
attention from (compatible) equivalence relations of finite index that are the
classical tool in automata theory, towards (compatible) order relations that
are \kl{well-quasi-ordered}, which is the order counterpart of having finite
index. Let us recall that a sequence $\seqof{u_i}{i \in \Nat}$ of elements in a
quasi-ordered set $(X, \leq)$ is \intro{good} whenever there exist $i < j$ such
that $u_i \leq u_j$. The set $X$ is a \intro{well-quasi-ordering} when every
infinite sequence is \reintro{good}. A sequence is \intro{bad} when it is not
\reintro{good}. The notion of \reintro{good sequences} can also be applied to
binary relations that are not orderings, and a binary relation $R$ for which
every infinite sequence is \reintro{good} is said to be \intro{well}
\cite{MELL98}.

As a model of computation, we will use the following definition of transducers,
that is based on the marble transducers \cite{EHB99}, and follows the general
pattern of transducers introduced in \cite{LOPEZ23b}. Note that we only
consider $\Nat$-valued functions, but could consider other commutative
semigroups.

\begin{definition}
    Let $\mathcal{H}$ be a family of functions
    from $\Sigma^*$ to $\Nat$.
    An \intro{$\mathcal{H}$-transducer} $\aTransd$ is
    a tuple $(Q, q_0, \delta, \lambda, F)$ where
    \begin{itemize}
        \item $Q$ is a finite set of states,
        \item $q_0 \in Q$ is an initial state,
        \item $\delta \colon Q \times \Sigma \to Q$
            is a deterministic transition function,
        \item $\lambda \colon Q \times \Sigma \to \mathcal{H}$
            is a correction function,
        \item $F \colon Q \to \Nat$ is an output function.
    \end{itemize}
\end{definition}

Given the syntax of an \kl{$\mathcal{H}$-transducer}, let us now 
introduce its semantics, which is given by a simple recursion
scheme. Intuitively, the function computed by a transducer is


\begin{definition}
    Let $\aTransd \defined (Q,q_0, \delta, \lambda, F)$ be an \kl{$\mathcal{H}$-transducer}.
    The function
    $\aTransd \colon Q \times \Sigma^* \to \Nat$
    is defined inductively  as follows:
    \begin{itemize}
        \item $\aTransd(q, \varepsilon) \defined F(q)$;
        \item $\aTransd(q, a u) \defined \aTransd(\delta(q,a), u)
            + \lambda(q,a)(u)$.
    \end{itemize}
    The function \intro{computed} by a transducer $\aTransd$
    is $w \mapsto \aTransd(q_0, w)$.
\end{definition}


\begin{example}
    \label{non-canonical-transd:ex}
    Let $\BadExOk \colon \set{a}^* \to \Nat$
    be defined as
    $\BadExOk(\varepsilon) = 1$,
    $\BadExOk(a) = 0$,
    and $\BadExOk(a^n) = n - 1$ for all $n \geq 2$.
    The function $\BadExOk$ is \kl{computed} by the two \kl{$\NPoly[0]$-transducers}
    depicted in \cref{non-canonical-transd:fig}.
\end{example}

\begin{figure}
    \centering
    \begin{subfigure}[b]{0.49\linewidth}
        \begin{tikzpicture}[
            etat/.style={minimum size=2em}
            ]
            \node[etat,state,initial,
                accepting by arrow,
                accepting text={$1$},
                accepting where=below,
                ] (A) at (0,0) {$\varepsilon$};
            \node[etat,state,
                accepting by arrow,
                accepting text={$0$},
                accepting where=below,
                ] (B) at (2,0) {$a$};

            \draw[->] (A) to node[midway, below] {$a \mid 0$} (B);

            \draw[->] (B) to[bend right=45] node[midway, above] {$a \mid 1 + \ind{\card{w} \geq 1}$} (A);
        \end{tikzpicture}
        \caption{This automaton has a \kl{counter}.}
    \end{subfigure}
    \begin{subfigure}[b]{0.49\linewidth}
        \begin{tikzpicture}[
            etat/.style={minimum size=2em}
            ]
            \node[etat,state,initial,
                accepting by arrow,
                accepting text={$1$},
                accepting where=below,
                ] (A) at (0,0) {$\varepsilon$};
            \node[etat,state,
                accepting by arrow,
                accepting text={$0$},
                accepting where=below,
                ] (B) at (2,0) {$a$};

            \draw[->] (A) to node[midway, below] {$a \mid 0$} (B);

            \draw[->] (B) edge[loop right] node[midway, right]
                {$a \mid 1 + \ind{\card{w} = 0}$} (B);
        \end{tikzpicture}
        \caption{This automaton is \kl{counter-free}.}
    \end{subfigure}
    \caption{Two \kl{$\NPoly[0]$-transducers}
    \kl{computing} the function $\BadExOk$ of \cref{non-canonical-transd:ex}.}
    \label{non-canonical-transd:fig}
\end{figure}

\AP It is already known that this notion of transducers exactly captures $\NSF$
and $\NPoly$, and generalizes to outputs into other commutative semigroups than
$\Nat$. To make the above statement formal, let us recall that a
\intro{counter} in an automaton is a pair $u,w$ of words together with an
integer $n > 1$, such that $\delta(q_0, uw^n) = \delta(q_0, u)$ and
$\delta(q_0, u w^k) \neq \delta(q_0, u)$ for all $k < n$; where we used the
extension of $\delta \colon Q \times \Sigma \to Q$ to words by defining
$\delta(q, \varepsilon) = q$, and $\delta(q,au) = \delta(\delta(q,a), u)$. An
automaton is \reintro{counter-free} if it contains no \reintro{counters}. This
notion can be lifted to \kl{$\mathcal{H}$-transducers} by considering counters
of their underlying automata. 


\begin{lemma}[{\cite[Theorems 5.15 and 7.10]{gaetanphd}}]
    \label{transducer-nsf-npoly:lemma}
    Let $f \colon \Sigma^* \to \Nat$, and let $k \in \Nat$.
    The following are equivalent:
    \begin{itemize}
        \item $f \in \NPoly[k]$ (respectively $\NSF[k]$),
        \item $f$ is \kl{computed} by an \kl{$\NPoly[k-1]$-transducer}
            (respectively by an \kl{$\NSF[k-1]$-transducer} that is 
            \kl{counter-free}).
    \end{itemize}
\end{lemma}

\AP The core concept of this section is that of a \intro{residual} of a
function $f \colon \Sigma^* \to \Rel$,  defined by $\intro*\app{f}{u} \defined
w \mapsto f(uw)$. The collection of \reintro{residuals} of a function $f$ is
denoted $\intro*\Res(f)$ and is defined as the set of $\reintro*\app{f}{u}$
where $u$ ranges over words in $\Sigma^*$. Let $k \in \Nat$. We define $f
\intro*\zpolyequiv[k] g$ if and only if $(g - f) \in \ZPoly[k-1]$, and $f
\intro*\npolyleq[k] g$ if and only if $(g - f) \in \NPoly[k-1]$. Given a
function $f \colon \Sigma^* \to \Rel$, we are going to study both $(\Res(f),
\reintro*\zpolyequiv[k])$, and $(\Res(f), \reintro*\npolyleq[k])$. However, to
simplify notations, instead of writing $\app{f}{u} \reintro*\npolyleq[k]
\app{f}{v}$ when $u,v \in \Sigma^*$, we will use the convenient notation $u
\intro*\resleq{f}{k} v$, and directly consider the space $(\Sigma^*,
\reintro*\resleq{f}{k})$. 


\begin{remark}
    \label{good-residual-ordering:fact}
    Let $k \in \Nat$, and let $f \colon \Sigma^* \to \Nat$. Then,
    $(\resleq{f}{k})$ is a quasi-ordering, satisfying the following
    extra properties:
    \begin{enumerate}
        \item For all $u,v,w \in \Sigma^*$, $u \resleq{f}{k} v$
            implies $uw \resleq{f}{k} vw$,
        \item If $u \resleq{f}{k} v$ and $v \resleq{f}{k} u$,
            then $\app{f}{u} = \app{f}{v}$.
    \end{enumerate}
\end{remark}


\AP
Given a function $f$, our goal is to leverage $\resleq{f}{k}$ to build a
canonical \kl{$\NPoly[k-1]$-transducer} that \kl{computes} $f$. The naïve
approach is to consider as states some representatives for the $\resleq{f}{k}$
minimal elements of $\Sigma^*$ (assuming there are finitely many of them), all
computing the same function (\cref{good-residual-ordering:fact}), and define
transitions by letting $\delta(u, a)$ be \emph{one} state $v$ such that $v
\resleq{f}{k} ua$. This naïve approach does not yield a canonical model, as
illustrated by the two distinct \kl{$\NPoly[0]$-transducers} of
\cref{non-canonical-transd:fig} computing the function $\BadExOk$ of
\cref{non-canonical-transd:ex}, and having as states minimal elements for $\resleq{\BadExOk}{0}$.
To obtain a
canonical notion, we enforce that the set of states is a \intro{downwards
closed} subset of $\Sigma^*$ for the prefix ordering $(\prefleq)$, that is, if
$u \in \Sigma^*$ is a state, then every prefix of $u$ must also be a state.

\begin{definition}
    \label{residual-transducer:def}
    Let $f \colon \Sigma^* \to \Nat$ and $k \in \Nat$.
    A transducer $\aTransd \defined (Q, q_0, \delta, \lambda, F)$
    is a \intro{$k$-residual transducer}
    of $f$ 
    when
    it is a \kl{$\NPoly[k-1]$-transducer}
    satisfying the following properties:
    \begin{enumerate}
        \item $\aTransd$ \kl{computes} $f$;
        \item $Q \subseteq \Sigma^*$ is a \kl{downwards closed}
            for $\prefleq$;
        \item $q_0 = \varepsilon$;
        \item every state $q \in Q$ is accessible from $q_0$;
        \item For all $u, a \in Q$,
            $\delta(u,a)$ is the $\prefleq$-maximal $v \in Q$
            such that $v \prefleq ua$, and $v \resleq{f}{k} ua$.
        \item For all $u,a \in Q$,
            $\lambda(u,a) = \app{f}{ua} - \app{f}{\delta(u,a)}$.
    \end{enumerate}
\end{definition}


Because of the restriction on prefixes, the \kl{$k$-residual transducer} is
actually unique when it exists. We now introduce \cref{residual:algo} to
compute a \kl{$k$-residual transducer} given a function $f$. Notice that this
algorithm requires the ability to test if a function belongs to $\NPoly[k]$,
which is only known to be feasible for \kl{commutative} \kl{polyregular
functions}. However, the termination of this algorithm also proves the
existence of the \kl{$k$-residual transducer}.


\begin{algorithm}
    $Q \defined \{ \varepsilon \}$;
    $O \defined \setof{ a }{ a \in \Sigma}$;
    $\delta \defined \emptyset$;
    $\lambda \defined \emptyset$;
    $F \defined \emptyset$;

    \While{$O \neq \emptyset$}{
        choose $ua \in O$;

        $O \defined O \setminus \set{ ua }$;

        \eIf{$\exists v \in Q, v = \max_{\prefleq} \setof{w \in Q}{w \prefleq ua \wedge w \resleq{f}{k} ua}$}{
            $\delta(u, a) \defined v$;
            $\lambda(u, a) \defined \app{f}{ua} - \app{f}{v}$;
        }{
            $Q \defined Q \uplus \set{ ua }$;
            $\delta(u,a) \defined ua$;
            $\lambda(u,a) \defined 0$;
            $O \defined O \cup \setof{ uab }{b \in \Sigma}$;
        }
    }
    \For{$u \in Q$}{
        $F(u) \defined f(u)$;
    }
    \Return{$(Q, \varepsilon, \delta, \lambda, F)$};
    \caption{Computing a $k$-residual transducer given a function $f$.}
    \label{residual:algo}
\end{algorithm}

\begin{lemma}
    \label{correct-residual:lemma}
    If \cref{residual:algo} terminates on 
    an input $f \colon \Sigma^* \to \Nat$, $k \in \Nat$,
    then it computes the \kl{$k$-residual transducer} of $f$.
\end{lemma}



\begin{lemma}
    \label{wqo-implies-termination:lemma}
    Let $f \colon \Sigma^* \to \Nat$, and $k \in \Nat$ be such that
    every infinite, $\prefleq$-increasing sequence is \kl{good}
    in $(\Sigma^*, \resleq{f}{k})$
    (or equivalently, such that the relation $(\prefleq \Rightarrow \resleq{f}{k})$
    is \kl{well}).
    Then, \cref{residual:algo} terminates on the input $(f,k)$.
\end{lemma}

The following \cref{n-poly-k-implies-wqo:lemma} is the core argument of the
section, and states that for a function $f \in \NPoly[k]$, not only $(\prefleq
\Rightarrow \resleq{f}{k})$ is \kl{well}, but the quasi-ordering $(\resleq{f}{k})$ is a
\kl{well-quasi-ordering}. This has to be compared with the property of being
\kl{$(k,\Nat)$-combinatorial}, which states that ultimately, the function is
increasing in every direction.

\begin{lemma}
    \label{n-poly-k-implies-wqo:lemma}
    Let $k \in \Nat$, and let $f \in \NPoly[k]$.
    Then, $(\Sigma^* \resleq{f}{k})$ is a \kl{well-quasi-ordering}.
\end{lemma}

\begin{theorem}
    \label{non-commutative-npoly:thm}
    Let $f \in \ZPoly$ be a non-negative function, 
    and $k \in \Nat$,
    the following are equivalent:
    \begin{enumerate}
        \item \label{n-poly-1-transd:item} $f$ is \kl{computed}
            by a \kl{$\NPoly[k-1]$-transducer};
        \item \label{n-poly-k:item} $f \in \NPoly[k]$;
        \item \label{n-poly-wqo:item} $(\Sigma^*, \resleq{f}{k})$ is a
            \kl{well-quasi-ordering};
        \item \label{n-poly-well:item} every $\prefleq$-increasing sequence
            of $\Sigma^*$  is a \kl{good sequence}
            for $\resleq{f}{k}$;
        \item \label{n-poly-residual:item} $f$ is \kl{computed} by a
            \kl{$k$-residual transducer}.
    \end{enumerate}
    If $f$ is assumed to be \kl{commutative}, then all the above
    properties are decidable, and the conversions effective.
\end{theorem}
\begin{proof}
    \cref{n-poly-1-transd:item} implies \cref{n-poly-k:item} by
    \cref{transducer-nsf-npoly:lemma}. Then,
    \cref{n-poly-k:item} implies \cref{n-poly-wqo:item} by
    \cref{n-poly-k-implies-wqo:lemma}.
    The implication \cref{n-poly-wqo:item} $\Rightarrow$ \cref{n-poly-well:item}
    is obvious.
    Then, \cref{wqo-implies-termination:lemma} proves
    that \cref{n-poly-well:item} implies \cref{n-poly-residual:item}.
    Finally, because a \kl{$k$-residual transducer} is a \kl{$\NPoly[k-1]$-transducer},
    \cref{n-poly-residual:item} implies \cref{n-poly-1-transd:item}.
\end{proof}
