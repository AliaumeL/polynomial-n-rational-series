%! TeX program = xelatex
%! TeX root = ../pnrs.tex
%! lang = en-US
\section{Beyond commutativity}
\label{beyond-commutative:sec}

The goal of this section is to provide tools that do not require
\kl{commutative} \kl{$\Nat$-polyregular functions}. To that end, we introduce
the notion of \kl{$k$-residual transducer}, that generalizes the notion of
residual transducer introduced for \kl{$\Rel$-polyregular functions} in
\cite{LOPEZ23b}. We will first introduce the object, provide an algorithm (with
an oracle function) that computes said object, and finally, check that in
simple cases, the aperiodicity of a function can be read on its
\kl{$k$-residual transducer}. The main drawbacks of the generalization are the
following: one needs to decide whether a \kl{$\Rel$-polyregular function} is in
$\NPoly$, which is only known in the \kl{commutative} case, and the
characterization in terms of counter free also assumes \kl{commutativity}.
However, we believe that the introduction of this object is relevant.


The goal of the upcoming section is to provide a notion of residual transducer
for \kl{commutative} \kl{$\Nat$-polyregular functions}. The notion of
transducer is based on the \kl{marble transducers} \cite{EHB99}.

\begin{definition}
    Let $\mathcal{H}$ be a family of functions
    from $\Sigma^*$ to $\Nat$.
    An \intro{$\mathcal{H}$-transducer} $\aTransd$ is
    a tuple $(Q, q_0, \delta, \lambda, F)$ where
    \begin{itemize}
        \item $Q$ is a finite set of states,
        \item $q_0 \in Q$ is an initial state,
        \item $\delta \colon Q \times \Sigma \to Q$
            is a deterministic transition function,
        \item $\lambda \colon Q \times \Sigma \to \mathcal{H}$
            is a correction function,
        \item $F \colon Q \to \Nat$ is an output function.
    \end{itemize}
\end{definition}

Given the syntax of an \kl{$\mathcal{H}$ transducer}, let us now 
introduce its semantics, which is given by a simple recursion
scheme.
\begin{definition}
    Let $\aTransd \defined (Q,q_0, \delta, \lambda, F)$ be an \kl{$\mathcal{H}$-transducer}.
    The function
    $\aTransd \colon Q \times \Sigma^* \to \Nat$
    is defined inductively  as follows:
    \begin{itemize}
        \item $\aTransd(q, \varepsilon) \defined F(q)$;
        \item $\aTransd(q, a u) \defined \aTransd(\delta(q,a), u)
            + \lambda(q,a)(u)$.
    \end{itemize}
    The function \intro{computed} by a transducer $\aTransd$
    is $w \mapsto \aTransd(q_0, w)$.
\end{definition}

\AP As in the case of regular automata, we generalize the function $\delta
\colon Q \times \Sigma \to Q$ to words by defining $\delta(q, \varepsilon) =
q$, and $\delta(q,au) = \delta(\delta(q,a), u)$. Similarly, we write
$\lambda(q,\varepsilon) \defined 0$, and $\lambda(q, au) \defined \app{\lambda(q,a)}{u}
+ \lambda(\delta(q,a), u)$. Following these equations, we obtain that
$\aTransd(q_0, uv) = \aTransd(\delta(q_0, u), v) + \lambda(q_0, u)(v)$.
Furthermore, let us notice that if $\mathcal{H}$ is closed under finite sums
and shifts, then $\lambda(q, u) \in \mathcal{H}$ for all $(q,u) \in Q \times
\Sigma^*$.

\begin{example}
    \label{non-canonical-transd:ex}
    Let $f \colon \set{a}^* \to \Nat$
    be defined as follows
    \begin{equation*}
        f(a^n) \defined
        \begin{cases}
            1 & \text{ if } n = 0 \\
            0 & \text{ if } n = 1 \\
            n & \text{ otherwise }
        \end{cases}
        \quad .
    \end{equation*}
    The function is computed by the two automata
    in \cref{non-canonical-transd:fig}.
\end{example}

\begin{figure}
    \centering
    \begin{subfigure}[b]{0.9\linewidth}
        \begin{tikzpicture}[
            etat/.style={minimum size=2em}
            ]
            \node[etat,state,initial,
                accepting by arrow,
                accepting text={$1$},
                accepting where=below,
                ] (A) at (0,0) {$\varepsilon$};
            \node[etat,state,
                accepting by arrow,
                accepting text={$0$},
                accepting where=below,
                ] (B) at (2,0) {$a$};

            \draw[->] (A) to node[midway, below] {$a \mid 0$} (B);

            \draw[->] (B) to[bend right=45] node[midway, above] {$a \mid 1 + \ind{\card{w} \geq 1}$} (A);
        \end{tikzpicture}
        \caption{This automaton has a \kl{counter}.}
    \end{subfigure}
    \begin{subfigure}[b]{0.9\linewidth}
        \begin{tikzpicture}[
            etat/.style={minimum size=2em}
            ]
            \node[etat,state,initial,
                accepting by arrow,
                accepting text={$1$},
                accepting where=below,
                ] (A) at (0,0) {$\varepsilon$};
            \node[etat,state,
                accepting by arrow,
                accepting text={$0$},
                accepting where=below,
                ] (B) at (2,0) {$a$};

            \draw[->] (A) to node[midway, below] {$a \mid 0$} (B);

            \draw[->] (B) edge[loop right] node[midway, right]
                {$a \mid 1 + \ind{\card{w} = 0}$} (B);
        \end{tikzpicture}
        \caption{This automaton is \kl{counter-free}.}
    \end{subfigure}
    \caption{Two \kl{$\NPoly[0]$-transducers}
    computing \cref{non-canonical-transd:ex}.}
    \label{non-canonical-transd:fig}
\end{figure}

\AP Let us recall that a \intro{counter} in an automaton is a pair $u,w$ of
words together with an integer $n > 1$, such that $\delta(q_0, uw^n) =
\delta(q_0, u)$ and $\delta(q_0, u w^k) \neq \delta(q_0, u)$ for all $k < n$.
An automaton is \reintro{counter-free} if it contains no \reintro{counters}.
This notion can be applied to \kl{$\mathcal{H}$-transducers} by
considering counters of their underlying automata.

\begin{theorem}[Cite gaëtan]
    Let $f \colon \Sigma^* \to \Nat$, and let $k \in \Nat$.
    The following are equivalent:
    \begin{itemize}
        \item $f \in \NPoly[k]$,
        \item $f$ is computed by a \kl{$\NPoly[k-1]$-transducer}.
    \end{itemize}
    \begin{itemize}
        \item $f \in \NSF[k]$,
        \item $f$ is computed by a \kl{$\NSF[k-1]$-transducer}
            that is counter-free.
    \end{itemize}
\end{theorem}

\AP Let us now introduce the notion of \intro{residual} of a function $f
\colon \Sigma^* \to \Rel$, by defining $\app{f}{u} \defined w \mapsto f(uw)$.
The collection of \reintro{residuals} of a function $f$ is denoted $\Res(f)$
and is defined as the set of $\app{f}{u}$ where $u$ ranges over words in
$\Sigma^*$.

\AP Let $k \in \Nat$. The residual transducer in the case of $\Rel$ output was
built using an equivalence relation defined as $f \equiv_k g$ if and only if $g
- f \in \ZPoly[k-1]$. However, in the case of $\Nat$ output, the relation
cannot be symmetric, and furthermore, we are going to order words rather than
residuals. Namely, given $k \in \Nat$, and $f \colon \Sigma^* \to \Nat$, we
define $u \resleq{f}{k} v$ if and only if $\app{f}{v} - \app{f}{u} \in
\NPoly[k-1]$.

\begin{fact}
    \label{good-residual-ordering:fact}
    Let $k \in \Nat$, and let $f \colon \Sigma^* \to \Nat$. Then,
    $(\resleq{f}{k})$ is a quasi-ordering, satisfying the following
    extra properties:
    \begin{enumerate}
        \item For all $u,v,w \in \Sigma^*$, $u \resleq{f}{k} v$
            implies $uw \resleq{f}{k} vw$,
        \item If $u \resleq{f}{k} v$ and $v \resleq{f}{k} u$,
            then $\app{f}{u} = \app{f}{v}$.
    \end{enumerate}
\end{fact}

There is one major issue arising when defining the notion of residual
transducer based on the quasi-ordering $\resleq{f}{k}$. The naïve approach is
to consider as states some representatives for the $\resleq{f}{k}$ minimal
elements of $\Sigma^*$, all computing the same function
(\cref{good-residual-ordering:fact}), and define transitions by letting
$\delta(u, a)$ be \emph{one} state $v$ such that $v \resleq{f}{k} ua$. This
naïve approach does not yield a canonical model.
For instance, the function
of \cref{non-canonical-transd:ex} can be computed in two ways if we only
consider $\resleq{f}{k}$ minimal elements, as witnessed in
\cref{non-canonical-transd:fig}. Even worse, the function $f$ belongs to
$\NSF[0]$, but one of the two presented transducers has a \kl{counter}.

This is why in \cref{residual-transducer:def}, we introduce the prefix ordering
$\prefleq$ to further constrain how the model computes.

\begin{definition}
    \label{residual-transducer:def}
    Let $f \colon \Sigma^* \to \Nat$ and $k \in \Nat$.
    A transducer $\aTransd \defined (Q, q_0, \delta, \lambda, F)$
    is a \intro{k-residual transducer}
    of $f$ 
    when
    it is a \kl{$\NPoly[k-1]$-transducer}
    satisfying the following properties:
    \begin{enumerate}
        \item $\aTransd$ \kl{computes} $f$;
        \item $Q \subseteq \Sigma^*$ is a \kl{downwards closed}
            for $\prefleq$;
        \item $q_0 = \varepsilon$;
        \item every state $q \in Q$ is accessible from $q_0$;
        \item For all $u, a \in Q$,
            $\delta(u,a)$ is the $\prefleq$-maximal $v \in Q$
            such that $v \prefleq ua$, and $v \resleq{f}{k} ua$.
        \item For all $u,a \in Q$,
            $\lambda(u,a) = \app{f}{ua} - \app{f}{\delta(u,a)}$.
    \end{enumerate}
\end{definition}

Because of the restriction on prefixes, the \kl{$k$-residual transducer}
is actually unique when it exists.

\begin{fact}
    \label{unique-res-transducer:fact}
    Let $f \colon \Sigma^* \to \Nat$ and $k \in \Nat$.
    Then $f$ has at most one \kl{$k$-residual transducer}.
\end{fact}
\begin{proof}
    Let $\aTransd_1$ and $\aTransd_2$ be two
    \kl{$k$-residual transducers} for $f$.
    The two initial states must be $\varepsilon$.
    Let us prove by induction on $u \in \Sigma^*$ that
    $\delta_1(\varepsilon, u) = \delta_2(\varepsilon, u)$
    and that $Q_1$ equals $Q_2$ over prefixes of $u$.
    This will prove that 
    $Q_1 = Q_2$, hence that $\aTransd_1 = \aTransd_2$.

    Let $u \in \Sigma^* \cap Q_1 \cap Q_2$ and $a \in \Sigma$, $v_1 \in Q_1$ be
    defined as $v \defined \delta_1(u,a)$, and $v_2 \defined \delta_2(u,a)$.
    Remark that by induction hypothesis, for all $v \prefle u$, $v \in Q_1 \cap
    Q_2$. If $\delta_1(u,a) = Q_1$, it means that for all $v \in Q_2$ such that
    $v \prefle ua$, we have $\neg( v \resleq{f}{k} ua )$. The only possible
    transition in $\aTransd_2$ is therefore $\delta_2(u,a) = ua$, and $ua \in
    Q_2$. Similarly, if $\delta_1(u,a) \prefleq u$, then $\delta_2(u,a) =
    \delta_1(u,a)$ by definition of $\delta_2$ as a maximum.
\end{proof}

We now introduce \cref{residual:algo} to compute a \kl{$k$-residual transducer}
given a function $f$. Notice that this algorithm requires the ability to test
if a function belongs to $\NPoly[k]$, which is only known to be feasible for
\kl{commutative} polyregular functions. However, the termination of this
algorithm also proves the existence of the \kl{$k$-residual transducer}.


\begin{algorithm}
    $Q \defined \{ \varepsilon \}$;
    $O \defined \setof{ a }{ a \in \Sigma}$;
    $\delta \defined \emptyset$;
    $\lambda \defined \emptyset$;
    $F \defined \emptyset$;

    \While{$O \neq \emptyset$}{
        choose $ua \in O$;

        $O \defined O \setminus \set{ ua }$;

        \eIf{$\exists v \in Q, v = \max_{\prefleq} \setof{w \in Q}{w \prefleq ua \wedge w \resleq{f}{k} ua}$}{
            $\delta(u, a) \defined v$;

            $\lambda(u, a) \defined \app{f}{ua} - \app{f}{v}$;
        }{
            $Q \defined Q \uplus \set{ ua }$;

            $\delta(u,a) \defined ua$;

            $\lambda(u,a) \defined 0$;

            $O \defined O \cup \setof{ uab }{b \in \Sigma}$;
        }
    }
    \For{$u \in Q$}{
        $F(u) \defined f(u)$;
    }
    \Return{$(Q, \varepsilon, \delta, \lambda, F)$};
    \caption{Computing a $k$-residual transducer given a function $f$.}
    \label{residual:algo}
\end{algorithm}

\begin{fact}
    \label{q-o-prefix-cool:fact}
    Let $f \colon \Sigma^* \to \Nat$ and $k \in \Nat$.
    At each step of the \texttt{while loop}
    of \cref{residual:algo}, the sets
    $Q$ and $O$ are such that
    \begin{enumerate}
        \item $Q \cup O$ is a \kl{downwards closed} subset of 
            $\Sigma^*$ for $\prefleq$;
        \item elements in $O$ are pairwise incomparable
            for $\prefleq$, and are maximal
            for $\prefleq$ inside $Q \cup O$.
    \end{enumerate}
\end{fact}

\begin{lemma}
    \label{correct-residual:lemma}
    If \cref{residual:algo} terminates on 
    an input $f \colon \Sigma^* \to \Nat$, $k \in \Nat$,
    then it computes the \kl{$k$-residual transducer} of $f$.
\end{lemma}
\begin{proof}
    Because of \cref{q-o-prefix-cool:fact},
    we already know that $q_0 = \varepsilon$,
    $Q$ is a \kl{downwards closed} subset of $\Sigma^*$
    for $\prefleq$, 
    that every state of $Q$ is accessible from $q_0$.
    Notice that at every step,
    $\lambda(u,a)$ is defined as
    $\app{f}{ua} - \app{f}{\delta(u,a)}$.
    Finally, since $Q \cup O$ is a \kl{downwards closed} subset of $\Sigma^*$
    at every step,
    we have that at step $i$,
    for all $ua \in O_i$,
    $\setof{w \in Q}{w \prefleq ua} = \setof{w \in Q_i}{ w \prefleq ua}$,
    which proves that the maximum considered in the algorithm
    is indeed computing correctly.
\end{proof}


\AP Let us recall that a sequence $\seqof{u_i}{i \in \Nat}$ of elements in a
quasi-ordered set $(X, \leq)$ is \kl{good} whenever there exist $i < j$ such
that $u_i \leq u_j$. The set $X$ is a \kl{well-quasi-ordering} when
every infinite sequence is \kl{good}. A sequence is \kl{bad}
when it is not \kl{good}.

\begin{lemma}
    \label{wqo-implies-termination:lemma}
    Let $f \colon \Sigma^* \to \Nat$, and $k \in \Nat$ be such that
    every infinite, $\prefleq$-increasing sequence is \kl{good}
    in $(\Sigma^*, \resleq{f}{k})$.
    Then, \cref{residual:algo} terminates on the input $(f,k)$.
\end{lemma}

\begin{lemma}
    \label{n-poly-k-implies-wqo:lemma}
    Let $k \in \Nat$, and let $f \in \NPoly[k]$.
    Then, $(\Sigma^* \resleq{f}{k})$ is a \kl{well-quasi-ordering}.
\end{lemma}

\begin{theorem}
    \label{non-commutative-npoly:thm}
    Let $f \in \ZPoly$ and $k \in \Nat$,
    the following are equivalent:
    \begin{enumerate}
        \item $f$ is \kl{computed}
            by a \kl{$\NPoly[k-1]$-transducer};
        \item $f \in \NPoly[k]$;
        \item $(\Sigma^*, \resleq{f}{k})$ is a
            \kl{well-quasi-ordering};
        \item every $\prefleq$-increasing sequence
            of $\Sigma^*$  is a \kl{good sequence}
            for $\resleq{f}{k}$;
        \item $f$ is \kl{computed} by a
            \kl{$k$-residual transducer}.
    \end{enumerate}
    If $f$ is asumed to be \kl{commutative}, then all of the above
    properties are decidable, and the conversions effective.
\end{theorem}
\begin{proof}
    The first two items are known to be equivalent since 
    the phd of gaëtan.
    We have proven that transducers imply
    wqo, and that wqo implies $k$-residual transducer.
\end{proof}


\begin{lemma}
    \label{aperiodic-iff-residual:lem}
    Let $L$ be a regular language.
    The following are equivalent:
    \begin{enumerate}
        \item $L$ is aperiodic;
        \item The \kl{$0$-residual transducer} of $\ind{L}$
            is \kl{counter-free}.
    \end{enumerate}
\end{lemma}
\begin{proof}
    If the $0$-residual transducer is counter free,
    then $L$ is aperiodic because the minimal automata is also counter-free.

    % TODO
    \textbf{TODO Claim:}
    on every branch of the automata, the words have distinct residuals.

    Now, assume that $L$ is aperiodic, and let $(q,w^n)$ be a counter with $n
    \geq 1$. We know that $qw \equiv_L q$ because the language $L$ is
    aperiodic. Let us write $t \defined \delta(q,w)$, we know that $t \equiv_L
    q$. Assume by contradiction that $t$ and $q$ are incomparable for the
    prefix relation. Let us split $w = w_1 w_2$ where $w_1$ is the shortest
    prefix of $w$ such that $s_0 \defined \delta(q,w_1)$ is an ancestor of $q$
    and of $t$ for the prefix relation, it must exist because $\delta(q,w_1
    w_2) = t$.

    Now, consider $s_1 \defined \delta(t, w_1)$. Assume by contradiction that
    $s_0$ is not comparable with $s_1$ for the prefix relation. Then, consider
    the smallest $v$ for the prefix relation such that $\delta(t, v)$ is a
    strict prefix of $s_0$. It must exist, because otherwise the two are always
    comparable. Since $t \equiv_L q$, we know that $tv \equiv_L qv$, but then,
    this contradicts the minimality of $u$.

    We have proven that $s_0$ and $s_1$ are comparable, hence they
    are equal.
    Finally, we have proven that $\delta(q, w_1) = s_0$,
    $\delta(s_0, w_2) = t$, and $\delta(s_0, w_2) = \delta(t, w_1w_2) = q$ which is absurd.

    As a consequence $t$ and $q$ were comparable for the prefix relation,
    hence equal, and therefore $\delta(q, w) = q$.
\end{proof}

\begin{lemma}
    Let $\NSF[0] = \NPoly \cap \NSF[0]$,
    and elements of $\NSF[0]$
    have a \kl{counter-free} \kl{$0$-residual transducer}.
\end{lemma}
\begin{proof}
    TODO.
\end{proof}


\textbf{TODO:} one may think that we can ask that $u \resleq{f}{k} u w^n$
implies $u \resleq{f}{k} uw$, but it is false, as the following example will
demonstrate. This is why the definition of counter is more involved in our
case.

\begin{example}
    Let us consider $f(w) \defined (\card{w} - 3)^2$, 
    which is a \kl{star-free $\Nat$-polyregular function}.
    Then,
    $\app{f}{a} - f$ is not in $\NPoly$,
    but
    $\app{f}{a^10} - f$ is in $\NPoly$,
    hence there is a non-trivial counter.
\end{example}
\begin{proof}
    Remark that $\app{f}{a} = (\card{w} - 2)^2$,
    hence that $\app{f}{a}- f = 2\card{w} - 5$.
    Similarly, $\app{f}{a^10} - f = 20\card{w} + k$ where $k \geq 0$.

    Finally,
    let $n > 1$ be such that
    $\app{f}{a^n} - f$ belong to $\NPoly$.
    In this specific example,
    $\app{f}{a^n} = (\card{w} + n -3)^2$,
    and $\app{f}{a^n} - f = (2 \card{w} + n - 6)n$.
    The latter belongs to $\NPoly$ if and only if
    $n \geq 6$. Let $n \geq 6$, 
    remark that
    $\app{f}{a^n} - \app{f}{a^{n-1}}
    = 2 \card{w} + 2n - 7$. Because $n = n_0 + 6$ with $n_0 \geq 0$,
    we can rewrite the latter 
    as $2 \card{w} + 2n_0 + 5$, which belongs to $\NPoly$.
\end{proof}

\begin{lemma}
    \label{sf-no-periods-on-sequences:lemma}
    Let $f \colon \Sigma^* \to \Nat$ be a \kl{star-free $\Nat$-polyregular
    function}. Then,
    for all $u, w \in \Sigma^*$,
    there exists $n \in \Nat$ such that
    $u w^n \resleq{f}{k} u w^{n+1}$.
\end{lemma}
\begin{proof}[Proof sketch.]
    Let $t$ be a first order type,
    the map $g_t \colon X \mapsto \vcount{t}(uw^X)$ is a
    \kl{commutative} \kl{star-free $\Nat$-polyregular function},
    hence ultimately a polynomial, and therefore,
    translated enough, it is a $\Nat[X]$ polynomial.
    But this means that $g_t(X+1) - g_t(X) \in \Nat[X]$ for a large
    enough $X$, which is precisely what suffices.
    Now, we can do this for every first order type $t$
    with fewer variables and lower quantifier rank than those
    needed to define $f$, and conclude.
\end{proof}

Even though computability is not guaranteed, we have the following proof
scheme to decide aperiodicity of \kl{$\Nat$-polyregular functions}.
\begin{lemma}
    Let $f \in \NPoly[k]$ be \kl{ultimately polynomial},
    then every \kl{$k$-residual transducer} computing 
    $f$ has no counters, and has labels
    that are themselves \kl{ultimately polynomial}
    \kl{$\Nat$-polyregular functions}.
\end{lemma}
\begin{proof}

    We prove the result by induction on $k$. By definition
    of a residual transducer, the labels are \kl{ultimately polynomial}.
    Now, assume that there exists a counter
    in some \kl{$k$-residual transducer}
    $\aTransd \defined (Q, q_0, \delta, \lambda, F)$ of $f$. That is, there exists $\alpha \in \Sigma^*$,
    $w \in \Sigma^+$, and $n > 1$, such that
    $\delta(q_0,\alpha w^n) = \delta(q_0, \alpha)$.
    We want to prove that
    $\delta(q_0, \alpha w) = \delta(q_0, \alpha)$.
    Because $f$ is \kl{ultimately polynomial},
    it is in particular in $\ZSF[k]$, and therefore
    $g \defined \app{f}{\alpha w} - \app{f}{\alpha}$
    belongs to $\ZSF[k-1]$.
    It remains to be proven that this difference is a function in
    $\NPoly$.

    To that end, notice that the map $f(\alpha w^X \mathcal{M})$ is a
    \kl{commutative} \kl{star-free $\Nat$-polyregular function}. For a large
    enough $N_0$, we conclude that $f(\alpha w^{N_0+1} \mathcal{M}) - f(\alpha
    w^{N_0} \mathcal{M})$ belongs to $\NPoly$, hence that $\alpha w^{N_0}
    \resleq{f}{k-1} \alpha w^{N_0 + 1}$. Does that prove that $\alpha
    \resleq{f}{k-1} \alpha w$? Not at all and we cannot conclude.
\end{proof}

\begin{theorem}
    The following are equivalent for $f \in \NPoly[k]$.
    \begin{itemize}
        \item The function $f \in \NSF[k]$,
        \item $f$ is \kl{ultimately polynomial}.
        \item Every \kl{$k$-residual transducer} of $f$ has no counters,
        \item $f$ is \kl{computed} by some
            \kl{$\NSF[k-1]$-transducer}.
    \end{itemize}
\end{theorem}
\begin{proof}
    The only implication to prove is the first implies the last.
    We know that the automaton is counter free, so the languages of the states
    are star-free. Now, we just write 
    \begin{equation*}
        TODO
    \end{equation*}
    and we have finished the proof.
\end{proof}

Remark that the theorem is effective because being \kl{ultimately polynomial}
is an effective property! However, effective conversions are not possible as of
now, because the $k$-residual transducer algorithm uses an oracle to decide if
a function is in $\NPoly$.


\begin{lemma}
    \label{trivial-for-unary:lemma}
    Let $f \in \NPoly$ having unary input.
    In particular, $f$ is commutative.
    Then, $f \in \NSF$ if and only if 
    the \kl{residual transducer} of $f$
    is \kl{counter-free}.
\end{lemma}
\begin{proof}
    If there is a counter... well we will find it ^^
\end{proof}

\begin{conjecture}
    If $f$ is commutative, and the \kl{residual transducer}
    has a counter, then it has a single letter counter.
\end{conjecture}

\begin{conjecture}
    If $f$ is commutative, then 
    $f \in \NSF$ if and only if the \kl{residual transducer}
    of $f$ is \kl{counter-free}.
\end{conjecture}

\section{Trash polynomials}

\begin{definition}
    For every polynomial $P \in \Rel[X \vec{Y}]$, we define 
    the partial derivative with respect to $X$ as follows:
    \begin{equation*}
        \partial_X P \defined P(X + 1, \vec{Y}) - P(X, \vec{Y}) \quad .
    \end{equation*}
\end{definition}

\begin{fact}
    Let $P \in \CorrectPoly[X_1, \dots, X_n]$, then
    for all $1 \leq i \leq n$,
    $\partial_{X_i} P \in \CorrectPoly$.
\end{fact}
\begin{proof}
    We prove the result by induction on the number of variables.
    Let $\nu \colon \vec{X} \topartial \Nat$.
    Two cases, either $X_i \in \dom(\nu)$,
    in which case, we remark that
    we can split $\nu$ into two valuations,
    and that coefficients are good.

    Otherwise, evaluation commutes with partial,
    and we conclude by induction hypothesis.
\end{proof}

\begin{remark}
    The converse does not hold,
    for instance $P(X) \defined X - 1$ has all its derivatives
    that are in $\CorrectPoly$, but is not itself in $\CorrectPoly$.
\end{remark}

\begin{fact}
    Let $P \in \Nat[X,\vec{Y}]$, and $K,L \in \Nat$,
    then $P(X+L, \vec{Y}) - P(X+K,\vec{Y}) \in \Nat[X, \vec{Y}]$
    whenever $L \geq K$.
\end{fact}
\begin{proof}
    We can write
    $P = \sum_{i = 0}^n P_i(\vec{Y}) X^i$, with $P_i \in \Nat[\vec{Y}]$.
    Now,
    $P(X+L,\vec{Y}) - P(X+K,\vec{Y})$
    is precisely
    $\sum_{i = 1}^n P_i(\vec{Y}) (L-K) \left[ \sum_{j + l + 1= i} (X+L)^j (X+K)^l \right]$,
    remaining with positive coefficients.
\end{proof}

\begin{fact}
    Let $P \in \CorrectPoly[X,\vec{Y}]$, and $K,L \in \Nat$,
    then $P(X+L, \vec{Y}) - P(X+K,\vec{Y}) \in \CorrectPoly$
    whenever $L \geq K$.
\end{fact}
\begin{proof}
    We can write
    $P = \sum_{i = 0}^n P_i(\vec{Y}) X^i$, with $P_i \in \Nat[\vec{Y}]$.
    Now,
    $P(X+L,\vec{Y}) - P(X+K,\vec{Y})$
    is precisely
    $\sum_{i = 1}^n P_i(\vec{Y}) (L-K) \left[ \sum_{j + l + 1= i} (X+L)^j (X+K)^l \right]$.
    Now, $P_i(\vec{Y})$
    remaining with positive coefficients.
\end{proof}
\begin{fact}
    For every $K \in \Nat$, and every $n \in \Nat$,
    $\Nat^n$ is a \kl{well-quasi-ordering}
    when endowed with the quasi-ordering $\vec{u} \leq_K \vec{v}$
    if and only if $\vec{u} = \vec{v}$ or
    $\vec{u}$ is smaller than $\vec{v}$ for the product ordering and
    $\sum_{i = 1}^n  (\vec{v} - \vec{u})_i \geq K$.
\end{fact}
\begin{proof}
    First, $\Nat^n$ with the product ordering is a \kl{well-quasi-ordering}.
    Therefore, given a sequence $\seqof{\vec{u}_i}{i \in \Nat}$,
    one can extract an infinite monotone subsequence for
    the product ordering.
    Either there exists $i < j$ such that $\vec{u}_i = \vec{u}_j$
    in the subsequence, in which case $\vec{u}_i \leq \vec{u}_j$
    and the original sequence was good. Or for all $i < j$,
    we have $\vec{u}_i$ that is strictly smaller than 
    $\vec{u}_j$ for the product ordering. As a consequence,
    $\vec{u}_0 \leq \vec{u}_K$.
\end{proof}



\begin{fact}
    Let $P \in \CorrectPoly[X_1, \dots, X_n]$. 
    For all
    $\vec{u} \leq \vec{v} \in \Nat^n$,
    $Q(X_1, \dots, X_n)
    \defined 
    P(X_1 + v_1, \dots, X_n + v_n)
    -
    P(X_1 + u_1, \dots, X_n + u_n)$
    belongs to $\CorrectPoly[X_1, \dots, X_n]$.
\end{fact}
\begin{proof}
    Let us consider $K \in \Nat$ such that for all partial valuations
    $\nu \colon \vec{X} \topartial \Nat$,
    $\translate{K}(\restr{P}{\nu}) \in \Nat[\vec{X}]$.

    Remark that
    \begin{align*}
        Q
        &= 
        P(X_1 + v_1, \dots, X_n + v_n)
        -
        P(X_1 + u_1, \dots, X_n + u_n) \\
        &=
        P(X_1 + v_1, X_2 + v_2 , \dots, X_n + v_n) - P(X_1 + u_1, X_2 + v_2, \dots, X_n + v_n) \\
        &+
        P(X_1 + u_1, X_2 + v_2, \dots, X_n + v_n) - P(X_1 + u_1, X_2 + u_2, \dots, X_n + v_n) \\
        &+ \dots
    \end{align*}

    Let $\nu \colon \vec{X} \topartial \Nat$
    be a partial valuation, let us show that
    $\translate{K}(\restr{Q}{\nu}) \in \Nat[\vec{X}]$.

    \begin{align*}
        \translate{K}(\restr{Q}{\nu})
        &= 
        \translate{K}(\restr{P(X_1 + v_1, \dots, X_n + v_n)}{\nu})
        -
        \translate{K}(\restr{P(X_1 + u_1, \dots, X_n + u_n)}{\nu})
    \end{align*}
\end{proof}

\begin{definition}
    Residual $\Nat$-transducer.
\end{definition}

\begin{lemma}
    If $P \in \CorrectPoly$, then it is computable
    by a $\Nat$-residual transducer.
\end{lemma}

\begin{theorem}
    The following are equivalent for \kl{commutative}.
    \begin{itemize}
        \item $f$ is in $\NSF$,
        \item The residual automata of $f$ has no counters
    \end{itemize}
\end{theorem}


\begin{fact}[Folklore about regular languages]
    \label{regular:fact}
    The language $\setof{ w \in \Sigma}{ \card[a]{w} = \card[b]{w}}$
    is not regular, whenever $a,b \in \Sigma$.
    Regular languages are closed under intersection.
\end{fact}
