%! TeX program = xelatex
%! TeX root = ../pnrs.tex
%! lang = en-US
\section{Beyond commutativity}
\label{beyond-commutative:sec}

\AP The goal of this section is to provide tools that do not require the
assumption of \kl{commutativity}. To that end, we introduce the notion of
\kl{$k$-residual transducer}, that generalizes the notion of residual
transducer introduced for \kl{$\Rel$-polyregular functions} in \cite{CDTL23}.
This notion of residual transducer has its own interest, as it shifts the
attention from (compatible) equivalence relations of finite index that are the
classical tool in automata theory, towards (compatible) order relations that
are \kl{well-quasi-ordered}, which is the order counterpart of having finite
index. Let us recall that a sequence $\seqof{u_i}{i \in \Nat}$ of elements in a
quasi-ordered set $(X, \leq)$ is \intro{good} whenever there exist $i < j$ such
that $u_i \leq u_j$. The set $X$ is a \intro{well-quasi-ordering} when every
infinite sequence is \reintro{good}. A sequence is \intro{bad} when it is not
\reintro{good}. The notion of \reintro{good sequences} can also be applied to
binary relations that are not orderings, and a binary relation $R$ for which
every infinite sequence is \reintro{good} is said to be \intro{well}
\cite{MELL98}.

\AP The core concept of this section is that of a \intro{residual} of a
function $f \colon \Sigma^* \to \Rel$,  defined by $\intro*\app{f}{u} \defined
w \mapsto f(uw)$. The collection of \reintro{residuals} of a function $f$ is
denoted $\intro*\Res(f)$ and is defined as the set of $\reintro*\app{f}{u}$
where $u$ ranges over words in $\Sigma^*$. Given $k \in \Nat$, we define $f
\intro*\zpolyequiv[k] g$ if and only if $(g - f) \in \ZPoly[k-1]$, and $f
\intro*\npolyleq[k] g$ if and only if $(g - f) \in \NPoly[k-1]$. The spaces of
interest for a function $f \in \ZPoly[k]$ will be $(\Res(f),
\reintro*\zpolyequiv[k])$, and $(\Res(f), \reintro*\npolyleq[k])$. However, to
simplify notations, instead of writing $\app{f}{u} \reintro*\npolyleq[k]
\app{f}{v}$ when $u,v \in \Sigma^*$, we will use the convenient notation $u
\intro*\resleq{f}{k} v$, and directly consider the space $(\Sigma^*,
\reintro*\resleq{f}{k})$. 


\AP
Given a function $f$, our goal is to leverage $\resleq{f}{k}$ to build a
canonical \kl{$\NPoly[k-1]$-transducer} that \kl{computes} $f$. The naïve
approach is to consider as states some representatives for the $\resleq{f}{k}$
minimal elements of $\Sigma^*$ (assuming there are finitely many of them), all
computing the same function (\cref{good-residual-ordering:fact}), and define
transitions by letting $\delta(u, a)$ be \emph{one} state $v$ such that $v
\resleq{f}{k} ua$. This naïve approach does not yield a canonical model, as
illustrated by the two distinct \kl{$\NPoly[0]$-transducers} of
\cref{non-canonical-transd:fig} computing the function $\BadExOk$ of
\cref{non-canonical-transd:ex}, and having as states minimal elements for $\resleq{\BadExOk}{0}$.
To obtain a
canonical notion, we enforce that the set of states is a \intro{downwards
closed} subset of $\Sigma^*$ for the prefix ordering $(\prefleq)$, that is, if
$u \in \Sigma^*$ is a state, then every prefix of $u$ must also be a state.

\begin{definition}
    \label{residual-transducer:def}
    Let $f \colon \Sigma^* \to \Nat$ and $k \in \Nat$.
    A transducer $\aTransd \defined (Q, q_0, \delta, \lambda, F)$
    is a \intro{$k$-residual transducer}
    of $f$ 
    when
    it is a \kl{$\NPoly[k-1]$-transducer}
    satisfying the following properties:
    \begin{enumerate}
        \item $\aTransd$ \kl{computes} $f$;
        \item $Q \subseteq \Sigma^*$ is a \kl{downwards closed}
            for $\prefleq$;
        \item $q_0 = \varepsilon$;
        \item every state $q \in Q$ is accessible from $q_0$;
        \item For all $u, a \in Q$,
            $\delta(u,a)$ is the $\prefleq$-maximal $v \in Q$
            such that $v \prefleq ua$, and $v \resleq{f}{k} ua$.
        \item For all $u,a \in Q$,
            $\lambda(u,a) = \app{f}{ua} - \app{f}{\delta(u,a)}$.
    \end{enumerate}
\end{definition}


Because of the restriction on prefixes, the \kl{$k$-residual transducer} is
actually unique when it exists. We now introduce \cref{residual:algo} to
compute a \kl{$k$-residual transducer} given a function $f$. Notice that this
algorithm requires the ability to test if a function belongs to $\NPoly[k]$,
which is only known to be feasible for \kl{commutative} \kl{polyregular
functions}. However, the termination of this algorithm also proves the
existence of the \kl{$k$-residual transducer}.


\begin{algorithm}
    $Q \defined \{ \varepsilon \}$;
    $O \defined \setof{ a }{ a \in \Sigma}$;
    $\delta \defined \emptyset$;
    $\lambda \defined \emptyset$;
    $F \defined \emptyset$;

    \While{$O \neq \emptyset$}{
        choose $ua \in O$;

        $O \defined O \setminus \set{ ua }$;

        \eIf{$\exists v \in Q, v = \max_{\prefleq} \setof{w \in Q}{w \prefleq ua \wedge w \resleq{f}{k} ua}$}{
            $\delta(u, a) \defined v$;
            $\lambda(u, a) \defined \app{f}{ua} - \app{f}{v}$;
        }{
            $Q \defined Q \uplus \set{ ua }$;
            $\delta(u,a) \defined ua$;
            $\lambda(u,a) \defined 0$;
            $O \defined O \cup \setof{ uab }{b \in \Sigma}$;
        }
    }
    \For{$u \in Q$}{
        $F(u) \defined f(u)$;
    }
    \Return{$(Q, \varepsilon, \delta, \lambda, F)$};
    \caption{Computing a $k$-residual transducer given a function $f$.}
    \label{residual:algo}
\end{algorithm}



The following \cref{n-poly-k-implies-wqo:lemma} is the core argument of the
section, and states that for a function $f \in \NPoly[k]$, not only $(\prefleq
\Rightarrow \resleq{f}{k})$ is \kl{well}, but the quasi-ordering $(\resleq{f}{k})$ is a
\kl{well-quasi-ordering}. This has to be compared with the property of being
\kl{$(k,\Nat)$-combinatorial}, which states that ultimately, the function is
increasing in every direction.

\begin{theorem}
    \label{non-commutative-npoly:thm}
    Let $f \in \ZPoly$ be a non-negative function, 
    and $k \in \Nat$,
    the following are equivalent:
    \begin{enumerate}
        \item \label{n-poly-1-transd:item} $f$ is \kl{computed}
            by a \kl{$\NPoly[k-1]$-transducer};
        \item \label{n-poly-k:item} $f \in \NPoly[k]$;
        \item \label{n-poly-wqo:item} $(\Sigma^*, \resleq{f}{k})$ is a
            \kl{well-quasi-ordering};
        \item \label{n-poly-well:item} every $\prefleq$-increasing sequence
            of $\Sigma^*$  is a \kl{good sequence}
            for $\resleq{f}{k}$;
        \item \label{n-poly-residual:item} $f$ is \kl{computed} by a
            \kl{$k$-residual transducer}.
    \end{enumerate}
    If $f$ is assumed to be \kl{commutative}, then all the above
    properties are decidable, and the conversions effective.
\end{theorem}

\AP
We can prove that for functions in $\NPoly[0]$, aperiodicity is equivalent to
the absence of counters in the \kl{residual transducer} of the function (see
\cref{aperiodic-iff-residual:lem}), however this result does not generalize to
higher growth rates, as witnessed by \cref{non-aperiodic-residual-transd:ex}.
In the example, no \kl{$\NPoly[0]$-transducer} with the minimal number of
states computing the function is \kl{counter-free}, which suggests that a more
sophisticated object is needed to observe aperiodicity. In the spirit of the
characterization of $\NPoly$ based the $(\resleq{f}{k})$ quasi-ordering
(\cref{non-commutative-npoly:thm}), we introduce the notion of \intro{aperiodic
ordering} of $\Sigma^*$ as follows: $(\Sigma^*, \leq)$ is \reintro[aperiodic
ordering]{aperiodic} whenever for all $u, w \in \Sigma^*$, there exists $N_0
\in \Nat$, such that the sequence $\seqof{uw^n}{n \geq N_0}$ is increasing.
This leads to the following \cref{sf-no-periods-on-sequences:conj}, of which
the easy implication is already known to hold
(\cref{sf-no-periods-on-sequences:lemma}), and requires the introduction of the
\emph{star-free variant} $(\intro*\resleqsf{f}{k})$ of $(\resleq{f}{k})$,
defined by $u \reintro*\resleqsf{f}{k} v$ whenever $(\app{f}{v} - \app{f}{u})
\in \NSF[k-1]$.

\begin{lemma}
    \label{aperiodic-iff-residual:lem}
    Let $f \in \NPoly[0]$. Then,
    $f \in \NSF$ if and only if 
    $f \in \ZSF$, if and only if
    $f$ is \kl{ultimately polynomial}, if and only if 
    the \kl{$0$-residual transducer} of $f$ is \kl{counter-free}.
    \proofref{aperiodic-iff-residual:lem}
\end{lemma}

\begin{example}
    \label{non-aperiodic-residual-transd:ex}
    Let us define
    $\BadExKo(\varepsilon) = 1$,
    $\BadExKo(a) = 0$,
    $\BadExKo(a^2) = 1$,
    and $\BadExKo(a^n) = n - 3$ for all $n \geq 3$.
    The \kl{$0$-residual transducer} of $\BadExKo$ has a \kl{counter} and two states.
    Furthermore,
    every \kl{$\NPoly[0]$-transducer} with two states contains a \kl{counter}.
\end{example}


\begin{lemma}
    \label{sf-no-periods-on-sequences:lemma}
    Let $k \in \Nat$, and $f \in \SF[k]$. Then,
    $(\Sigma^*, \resleqsf{f}{k})$ is an
    \kl[aperiodic ordering]{aperiodic} \kl{well-quasi-ordering}.
    \proofref{sf-no-periods-on-sequences:lemma}
\end{lemma}

\begin{conjecture}
    \label{sf-no-periods-on-sequences:conj}
    For all $k \in \Nat$ and $f \colon \Sigma^* \to \Rel$,
    $f \in \NSF[k]$ if and only if $(\Sigma^*, \resleqsf{f}{k})$ is an \kl[aperiodic ordering]{aperiodic}
    \kl{well-quasi-ordering}.
\end{conjecture}
