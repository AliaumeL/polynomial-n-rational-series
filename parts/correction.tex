%! TeX program = xelatex
%! lang = en-US
\section{The Corrected Theorem}
\label{sec:proof}

The counter example provided by \cref{def:bad-polynomial} relies on the fact
that $\CoveredPoly$ is not stable under fixing indeterminates, while
\kl{$\Nat$-polyregular functions} are stable under the residual operation
$u^{-1} \colon f \mapsto f(u \cdot)$. In this section, we prove that closing
$\CorrectPoly$ under assignments of variable is enough to recover from
\cref{karh:thm}.

\AP We use the following notation to fix the value of some indeterminate, if
$P(X,Y)$ is a polynomial in $\Rel[X,Y]$, then $\intro*\restr{P(X,Y)}{X = 1}$ is
the polynomial $P(1,Y) \in \Rel[Y]$. More generally, if $\nu$ is a partial
function from $\vec{X}$ to $\Nat$, written $\nu \colon \vec{X} \topartial
\Nat$, the restriction $\restr{P(\vec{X})}{\nu}$ is the polynomial with
indeterminates $\vec{Y} \defined \vec{X} - \dom(\nu)$ obtained by fixing the
variables of the domain of $\nu$.


\begin{definition}
    Let $\vec{X}$ be a finite tuple of indeterminates.
    The class $\CorrectPoly[\vec{X}]$ is the collection of
    polynomials $P \in \Rel[\vec{X}]$ such that
    $P$ is 
    such that, for every partial function $\nu \colon \vec{X} \topartial \Nat$,
    every \kl{maximal monomial} of
    $\restr{P}{\nu}$ is \kl{non-negative}.
\end{definition}

First, let us remark that $\CorrectPoly \subseteq \CoveredPoly$, because
polynomials in $\CorrectPoly$ are \kl{non-negative}. Let us also check that the
counter example provided in \cref{thm:counter-example} is not in
$\CorrectPoly$. For that, notice that for $\BadPoly$ introduced in
\cref{def:bad-polynomial}, $\BadPoly(X,Y,1) = 3X^2 + 3Y^2 - 2XY$, which has a negative
coefficient for a \kl{maximal monomial}, namely $-2XY$. 

\begin{remark}
    \label{inductive-correct-poly:remark}
    Let $P \in \Rel[X,\vec{Y}]$ be a polynomial.
    The following are equivalent:
    \begin{enumerate}
        \item $P \in \CorrectPoly$,
        \item The \kl{maximal monomials} of 
            $P$ are \kl{non-negative}
            and 
            for all $n \in \Nat$,
            $\restr{P}{X = n} \in \CorrectPoly$.
    \end{enumerate}
    Furthermore, if $P \in \Rel[]$ is a polynomial without
    indeterminate, then $P \in \CorrectPoly$ if and only
    if its \kl{maximal monomial} (the constant term)
    is \kl{non-negative}.
\end{remark}
\begin{proof}
    By induction on the number of indeterminate,
    leveraging the fact that if there exists at least one indeterminate
    $X$, i.e., that $\vec{X} = X \vec{Y}$, then for all $n \in \Nat$
    and partial function $\nu \colon \vec{Y} \topartial \Nat$,
    we have 
    \begin{equation*}
        \restr{\left(\restr{P}{X = n}\right)}{\nu} = \restr{P}{ \nu[X = n]}
        \quad
        ,
    \end{equation*}
    where $\nu[X = n]$ is the partial function from
    $\vec{X}$ to $\Nat$ that coincides with $\nu$ on $\vec{Y}$
    and maps $X$ to $n$.
\end{proof}

Let us now prove that \kl{$\Nat$-rational polynomials} are in $\CorrectPoly$.
The following \cref{n-poly-combinatorics:lem} is a slight refinement over the
classical combinatorial analysis of \kl{$\Rel$-polyregular functions}
\cite[Lemma 4.16]{LOPEZ23b} and \kl{$\Nat$-polyregular functions} that takes
into account \emph{all} \kl{maximal monomials} instead of a specific one
\cite[Lemma 5.37]{gaetanphd}.

\begin{lemma}
    \label{n-poly-combinatorics:lem}
    Let $f$ be a  \kl{$\Nat$-polyregular} function. 
    There exists $\omega \in \Nat$
    such that for all $p \in \Nat$,
    for all $\alpha_0, \dots, \alpha_p \in \Sigma^*$,
    for all $u_1, \dots, u_p \in \Sigma^*$,
    there exists a polynomial $P \in \Rel[X_1, \dots, X_p]$
    whose \kl{maximal monomials} are \kl{non-negative},
    and such that for all $X_1, \dots, X_p \geq \omega$:
    \begin{equation*}
        f\left(
            \alpha_0 \prod_{i = 1}^p (u_i^{\omega \times X_i} \alpha_i)
        \right)
        = P(X_1, \dots, X_p) \quad .
    \end{equation*}
\end{lemma}
\begin{proof}
    Let $\omega$ be an idempotent power for the finite monoid $M$,
    and
    $w \defined \alpha_0 \prod_{i = 1}^p (u_i^{\omega \times X_i} \alpha_i)$.
    Let $k \in \Nat$, and $\pi \colon M^k \topartial \Nat$ be such that
    $f = \pi^\dagger$. That is, 
    $f(w)$ is the sum over all factorizations of $w$
    into $k$ words $(w_1, \dots, w_k) \in \Sigma^*$
    of the value $\pi(\mu(w_1), \dots, \mu(w_k))$.

    Let us define an equivalence relation $\equiv$ over words of
    $\Sigma^*$ as follows: two words $v_1, v_2$ are equivalent if they are
    equal when normalizing them with the rules $u_i^{\omega} u_i^{\omega} \to
    u_i^{\omega}$ for all $1 \leq i \leq p$. That is, we disregard
    repetitions of the factors $u_i^{\omega}$ in the words.
    We lift the equivalence relation $\equiv$ to $k$-tuples of
    words by pointwise application.

    Remark that if $(v_1, \dots, v_k) \equiv (w_1, \dots, w_k)$ then
    $\pi(\mu(v_1), \dots, \mu(v_k)) = \pi(\mu(w_1), \dots, \mu(w_k))$ because
    $\mu(u_i^\omega) = \mu(u_i)^\omega$ is an idempotent element of the monoid
    $M$. In particular, $f(w)$ can be computed as the sum over equivalence
    classes for $\equiv$, of the value of $\pi$ on one representative,
    which belongs to $\Nat$.

    Therefore, it suffices to prove that the number of partitions of $w$ in a
    single equivalence class for $\equiv$ is a polynomial $P(X_1, \dots, X_p)$
    with \kl{positive coefficients} for its \kl{maximal monomials}.

    Notice that to produce all elements in a given equivalence class
    means adding the factors $u_i^\omega$ where $1 \leq i \leq p$
    back to the normalized version
    of the equivalence class. For all elements in an equivalence class,
    one has to add the same amount of $u_i^\omega$ for $1 \leq i \leq p$,
    and therefore the number of possible choices
    is either constant, or 
    is a product of binomials of the form $\binom{X - l}{s}$
    which has the desired shape.
\end{proof}


\begin{lemma}
    \label{n-rat-correct:lem}
    Let $P \in \Rel[X_1, \dots, X_p]$ be \kl{represented}
    by a \kl{$\Nat$-rational series}. Then,
    $P \in \CorrectPoly$.
\end{lemma}
\begin{proof}
    Let $f$ be the \kl{commutative}
    \kl{$\Nat$-rational series}
    with domain $\Sigma \defined \set{a_1, \dots, a_p}$
    that \kl{represents} $P$. Because $f$ has \kl{polynomial growth},
    it is in fact a \kl{$\Nat$-polyregular function}
    (\cref{polynomial-rational-polyreg:fact}).
    Using \cref{n-poly-combinatorics:lem},
    there exists a number $\omega \in \Nat$
    and polynomial $Q$ with \kl{positive coefficients} for its
    \kl{maximal monomials} such that
    for all $X_1, \dots, X_p \geq \omega$:
    \begin{equation*}
        f\left(
            \prod_{i = 1}^p (a_i)^{\omega \times X_i}
        \right)
        = Q(X_1, \dots, X_p)
        = P(\omega X_1, \dots, \omega X_p) 
        \quad .
    \end{equation*}
    We conclude that $P(\omega X_1, \dots, \omega X_p) = Q(X_1, \dots, X_p)$
    as polynomials,
    and in particular that
    the \kl{maximal monomials} of 
    $P$ have \kl{positive coefficients}.

    Remark that for every partial valuation $\nu \colon \vec{X} \topartial \Nat$,
    the polynomial $\restr{P}{\nu}$ continues to be represented
    by a \kl{$\Nat$-polyregular function}, namely
    $f$ partially applied to a word. As a consequence,
    the \kl{maximal monomials} of
    $\restr{P}{\nu}$ also have \kl{positive coefficients}, and 
    we have proven that $P \in \CorrectPoly$.
\end{proof}



Let us also assert that polynomials that have all coefficients in $\Nat$ are
\kl{$\Nat$-rational polynomials}. In particular, \kl{non-negative}
\kl{monomials} are \kl{$\Nat$-rational polynomials}, which will be a base case
in an upcoming induction proving \cref{lem:correct-to-n-rat}.

\begin{lemma}
    \label{n-poly-n-poly:example}
    Let $P \in \Nat[\vec{X}]$. Then, $P$
    is a \kl{$\Nat$-rational polynomial}.
\end{lemma}
\begin{proof}
    We prove it by remarking that $\NPoly$
    is stable under non-negative linear combinations
    and contains the functions ...
\end{proof}


\AP The core of the upcoming \cref{lem:correct-to-n-rat} leverages a notion of
\kl{discrete derivative} to perform an induction on the \kl{maximal monomials},
and suitable notions of \kl{translations} of the domain of polynomials. Namely,
we write $\translate{K}$ for the \intro{translation function} that maps a
polynomial $P \in \Rel[X_1, \dots, X_n]$ to the polynomial $P(X_1 - K, \dots,
X_n - K)$.

\begin{fact}
    \label{translation-invariance:fact}
    The following sets are stable under \kl{translations}
    $\translate{K}$,
    where $K \in \Nat$ is non-negative:
    \begin{itemize}
        \item $\CorrectPoly$,
        \item \kl{$\Nat$-rational polynomials},
        \item Polynomials \kl{represented}
            by \kl{star-free $\Nat$-polyregular functions}.
    \end{itemize}
\end{fact}

The idea behind the \kl{discrete derivative} introduced in
\cref{discrete-derivative:def} is to consider the map $P \mapsto P(X) -
P(X-K)$, which will be formulated using the \kl{translation function} in the
case of multiple variables.

\begin{definition}
    \label{discrete-derivative:def}
    Let $\vec{X}$ be a tuple of indeterminates,
    $K \in \Nat$,
    and 
    $P \in \Rel[\vec{X}]$ be a polynomial.
    \begin{equation*}
        \Diff{K}{P} \defined 
        P - \translate{-K}(P) \quad .
    \end{equation*}
\end{definition}

Let us now prove that for a single indeterminate $X$, polynomials $P \in
\CorrectPoly$ are \kl{$\Nat$-rational polynomials}. This will illustrate how
the different ingredients piece together. Given a polynomial $P \in
\CorrectPoly$, the goal is to build a \kl{$\Nat$-polyregular function} $f$ that
\kl{represents} $P$. Note that in this case, the function has unary input and
unary output. We prove the result on the degree $\deg(P)$ of the polynomial
$P$, where the base case is trivial because when $P$ is a constant function,
this constant must be in $\Nat$. For the induction step, we can assume that $P
= \alpha X^n + Q$, where $\deg(Q) < \deg(P)$, $n \geq 1$, and $\alpha > 0$. On
the one hand, $\alpha X^n$ is a \kl{$\Nat$-rational polynomial}, but on the
other hand, $Q$ may not belong to $\CorrectPoly$. However, there exists $K \in
\Nat$ such that $\Diff{K}{X^n} + Q$ belongs to $\CorrectPoly$, for instance by
considering $K$ to be the sum of absolute values of coefficients of $P$.
Furthermore, $\Diff{K}{X^n} + Q$ has degree smaller than $\deg(P)$, hence is a
\kl{$\Nat$-rational polynomial}. Now, $P = \alpha (X - K)^n + \Diff{K}{X^n} +
Q$, and therefore, $P(X+K) = X^n + (\Diff{K}{X^n} + Q)(X+K)$ is a
\kl{$\Nat$-rational polynomial}, \kl{represented} by a \kl{$\Nat$-polyregular
function} $f_1$. Now, for values $k$ of $X$ that are smaller than $K$, we can
compute $P(k) = \restr{P}{X=k}$, which belongs to $\Nat$. It is possible to
build a \kl{$\Nat$-polyregular function} $f_2 \colon \Nat \to \Nat$ that
outputs $P(k)$ if the input is at most $K$, and $f_1(k)$ otherwise.


In the actual statement of \cref{corrected-version:thm}, the representation of
polynomials in $\CorrectPoly$ is strengthened from \kl{$\Nat$-polyregular
functions} to \kl{star-free $\Nat$-polyregular functions}. This has to be
compared with the characterization of \kl{star-free $\Rel$-polyregular
functions} in terms of polynomials \cite[Theorem??]{LOPEZ23b},
that we will re-introduce in \cref{star-free:sec}.


When multiple indeterminate are involved, a bit of care is needed. The key
ingredient is that one has a fine control over the \kl{maximal monomials} of
the \kl{discrete derivatives}, even when fixing some indeterminates.
As one would expect, the \kl{discrete derivatives} are linear operations on
polynomials, that commutes with the partial application operators
$\restr{\cdot}{\nu}$, and the \kl{translation operators} $\translate{K}$.

\begin{fact}
    \label{discrete-deriv-linear:fact}
    For all tuple $\vec{X}$ of indeterminates,
    for all $K \in \Nat$, $L \in \Rel$, for all partial function
    $\nu \colon \vec{X} \topartial \Nat$:
    \begin{enumerate}
        \item $\translate{K} \colon \Rel[\vec{X}] \to \Rel[\vec{X}]$
            is a linear operator,
        \item $\Diff{K}{ \cdot } \colon \Rel[\vec{X}] \to \Rel[\vec{X}]$
            is a linear operator,
        \item $\restr{\cdot}{\nu} \colon \Rel[\vec{X}] \to \Rel[\vec{X}]$
            is a linear operator,
        \item $\Diff{K}{ \cdot } \circ \restr{\cdot}{\nu}
            = \restr{\cdot}{\nu} \circ \Diff{K}{\cdot}$.
        \item $\Diff{K}{ \cdot } \circ \translate{L}
            = \translate{K} \circ \Diff{L}{\cdot}$,
        \item $\restr{\cdot}{\nu} \circ \translate{L}
            = \translate{L} \circ \restr{\cdot}{\translate{L}(\nu)}$.
    \end{enumerate}
\end{fact}

\begin{fact}
    \label{translation-maximal:fact}
    For all $P \in \Rel[\vec{X}]$, and $K \in \Nat$,
    $\MaximalMonomials(P) = \MaximalMonomials(\translate{K}(P))$.
\end{fact}

\AP Let us now introduce the ordering over which the induction of
\cref{lem:correct-to-n-rat} is built. Recall that $\MaximalMonomials(P)$ is the
set of \kl{maximal monomials} of $P$, hence belongs to $\Pfin(\Monomials)$. We
order \kl{monomials} with the \kl{divisibility ordering}, making $\Monomials$ a
\kl{well-quasi-ordering} that is isomorphic to $\Nat^k$ with the product
ordering \cite[see e.g.][Dickson’s Lemma]{SCSC12}. We endow $\Pfin(\Monomials)$
with the \intro{Hoare ordering}, that is, $S_1 \hoareleq S_2$ whenever for all
\kl{monomials} $M_1 \in S_1$, there exists a monomial $M_2 \in S_2$, such that
$M_1$ \kl{divides} $M_2$. The set $(\Pfin(\Monomials), \hoareleq)$ remains a
\kl{well-quasi-ordering} \cite[see e.g.][Hoare quasi-ordering]{SCSC12}.

The following two lemmas show how the \kl{discrete differentiation} operator
interacts with \kl{maximal monomials} with respect to the \kl{Hoare ordering}:
it extracts \emph{submaximal} monomials from a given polynomial.

\begin{lemma}
    \label{derivation-simplifies:lemma}
    For all $P \in \Rel[\vec{X}]$ that are non-constant,
    for all $K \in \Nat$,
    $\MaximalMonomials(\Diff{K}{P}) \hoarele
    \MaximalMonomials(P)$.
\end{lemma}
\begin{proof}
    By linearity of $\Diff{K}{P}$ (\cref{discrete-deriv-linear:fact}), 
    it suffices to prove the result
    for \kl{monomials} $S \in \Monomials$.
\end{proof}

\begin{fact}
    Let $M,T$ be two \kl{monomials}, such that
    $T$ \kl{strictly divides} $M$.
    Then,
    $T$ \kl{divides} some \kl{maximal monomial}
    of $\Diff{K}{M}$.
\end{fact}

\begin{lemma}
    \label{derivation-covers:lemma}
    For all $P,Q \in \Rel[\vec{X}]$ that are non-constant,
    for all $K \in \Nat$,
    if $\MaximalMonomials(P) \hoarele \MaximalMonomials(Q)$,
    then
    $\MaximalMonomials(P) \hoareleq \MaximalMonomials(\Diff{K}{Q})$.
\end{lemma}


\begin{lemma}
    \label{derivation-stabilises-correct:lem}
    Let $P \in \CorrectPoly$,
    $P_1$ be the sum of \kl{maximal monomials} of $P$,
    and $P_2 \defined P - P_1$ be the sum of
    non-maximal monomials of $P$.
    There exists a computable $K$,
    such that
    $(\Diff{K}{P_1} + P_2) \in \CorrectPoly$.
\end{lemma}

\begin{lemma}
    \label{derivation-translation:lem}
    Let $P \in \Rel[\vec{X}]$, the following are equivalent
    \begin{enumerate}
        \item $P \in \CorrectPoly$,
        \item $P$ is \kl{non-negative}
            and there exists a computable $K \in \Nat$
            such that $\translate{K}(P) \in \Nat[\vec{X}]$.
    \end{enumerate}
\end{lemma}
\begin{proof}
    \textbf{TODO: do the proof}.
    To prove \cref{corrected-5:item} $\implies$ \cref{corrected-1:item}, it
    suffices to remark that for every polynomial $P$, for every partial
    valuation $\nu \colon \vec{X} \topartial \Nat$, for every $K \in \Nat$, the
    \kl{maximal monomials} of $\restr{\translate{K}(P)}{\nu}$ are exactly the
    \kl{maximal monomials} of $\restr{P}{\nu}$, the former being
    \kl{non-negative} if $\translate{K}(P) \in \Nat[\vec{X}]$.

    Conversely, the implication \cref{corrected-1:item} $\implies$
    \cref{corrected-5:item} follows immediately from the fact that polynomials
    in $\CorrectPoly$ are \kl{non-negative}, and on the following induction.
    Let $P \in \CorrectPoly$, if $P$ is constant, then $\translate{0}(P) \in \Nat[\vec{X}]$.
    Otherwise, if $P = P_1 + P_2$ where $P_1$ is the sum of the \kl{maximal
    monomials} of $P$, there exists $K \in \Nat$ such that $Q \defined
    \Diff{K}{P_1}+ P_2$ belongs to $\CorrectPoly$,
    and by induction hypothesis there exists $L \in
    \Nat$ such that $\translate{L}{Q} \in \Nat[\vec{X}]$. As a consequence,
    $\translate{K+L}(P) = \translate{L}(P_1) + \translate{K}(\translate{L}(Q))$
    is a sum of two polynomials in $\Nat[\vec{X}]$, and we have concluded.
\end{proof}

\begin{lemma}
    \label{lem:correct-to-n-rat}
    Let $\vec{X}$ be a tuple of indeterminates,
    and let $P \in \Rel[\vec{X}]$.
    If $P \in \CorrectPoly$, then $P$ is \kl{represented}
    by a \kl{star-free $\Nat$-polyregular function},
    which can be explicitly constructed from $P$.
\end{lemma}
\begin{proof}
    We prove the result by induction on $\MaximalMonomials(P)$. 
    In the proof, we write $\vec{X}$ for the indeterminates appearing in $P$,
    that is, we assume without loss of generality that all indeterminates are used.

    \textbf{Base case:} If the (unique) \kl{maximal monomial} of $P$ is a
    constant term. Since $P \in \CorrectPoly$, $P = n \in \Nat$, and therefore
    $P$ is \kl{represented} by a constant \kl{star-free $\Nat$-polyregular
    function}.

    \textbf{Induction:} Assume that $P$ is not a constant polynomial, and let
    us write $P = P_1 + P_2$ where $P_1$ is the sum of the \kl{maximal
    monomials} of $P$. We compute a bound $K$ such that $Q \defined
    (\Diff{K}{P_1} + P_2) \in \CorrectPoly$ using
    \cref{derivation-stabilises-correct:lem}. Thanks to
    \cref{derivation-simplifies:lemma}, we also know that $\MaximalMonomials(Q)
    \hoarele \MaximalMonomials(P)$. By induction hypothesis, $Q$ is
    \kl{represented} by a \kl{star-free $\Nat$-polyregular function}
    which is effectively computable.

    Let us now remark that $P_1 \in \Nat[\vec{X}]$, and is therefore
    (effectively) \kl{represented} by a \kl{star-free $\Nat$-polyregular function}
    (\cref{n-poly-n-poly:example}).
    Furthermore, $\translate{K}(Q)$ is also (effectively)
    \kl{represented} by a \kl{star-free $\Nat$-polyregular function}
    (see \cref{translation-invariance:fact}).
    As a consequence, $\translate{K}(P)$ is (effectively) \kl{represented} 
    by a function $f_\Delta$,
    since it is obtained as the following sum:
    \begin{equation*}
        \translate{K}(P) 
        = P_1 + \translate{K}(Q) \quad .
    \end{equation*}

    For all partial valuations $\nu \colon \vec{X} \topartial \set{0, \dots, K}$
    fixing at least one indeterminate, $\MaximalMonomials(\restr{P}{\nu})
    \hoarele \MaximalMonomials{P}$. Because $\restr{P}{\nu} \in \CorrectPoly$,
    one can use the induction hypothesis to compute a \kl{star-free
    $\Nat$-polyregular function} $f_\nu$ that \kl{represents} $\restr{P}{\nu}$.


    Let us assume that the alphabet over which the (\kl{commutative}) functions
    $f_\Delta$ and $f_\nu$ are defined is $\set{a_1, \dots, a_n}$, with $a_i$
    representing the indeterminate $X_i$ of the polynomials. Now, let us define
    by case analysis the following \kl{commutative} \kl{star-free
    $\Nat$-polyregular function}, defined on words $w$ of the form $w \defined
    a_1^{X_1} \cdots a_n^{X_n}$, with $X_1, \dots, X_n \geq 0$.

    \begin{equation*}
        f(w) \defined
        \begin{cases}
            f_\nu(w) & \text{ if } \exists i \in \set{1, \dots, n}, X_i \leq K \\
            f_\Delta(a_1^{X_1 - K} \cdots a_n^{X_n - K}) & \text{ otherwise }
        \end{cases}
        \quad .
    \end{equation*}
    It is clear that
    $f$ is a \kl{commutative} \kl{star-free $\Nat$-polyregular function},
    and that
    $f$ \kl{represents} $P$.
\end{proof}

While \cref{lem:correct-to-n-rat} provides effective conversion, it does not
explicitly state that the nembership is decidable. However, it is not too
difficult to remark that $\CorrectPoly$, the proof scheme can be followed to
obtain decidability. For the sake of completeness, we provide a self-contained
proof hereafter, that furthermore 

\begin{lemma}
    \label{decidability-correct:lem}
    Membership in $\CorrectPoly$ is decidable for
    polynomials in $\Rel[\vec{X}]$.
\end{lemma}
\begin{proof}
    Let $P \in \Rel[\vec{X}]$. We prove by induction on
    $\MaximalMonomials(P)$ that whether $P \in \CorrectPoly$
    is decidable.

    If $P$ is a constant polynomial, then
    $P \in \CorrectPoly$ if and only if $P = n \in \Nat$
    which is decidable.

    Otherwise, $P$ is not a constant polynomial, and let us write $P = P_1 +
    P_2$ where $P_1$ is the sum of the \kl{maximal monomials} of $P$. If $P_1
    \not \in \Nat[\vec{X}]$, then $P \not \in \CorrectPoly$, and this is
    decidable. Hence, we can assume that $P_1 \in \Nat[\vec{X}]$. Using
    \cref{derivation-stabilises-correct:lem}, there exists a computable bound
    $K$, such that if $P \in \CorrectPoly$, then t $Q \defined (\Diff{K}{P_1} +
    P_2) \in \CorrectPoly$. Leveraging, \cref{derivation-simplifies:lemma}, we
    also know that $\MaximalMonomials(Q) \hoarele \MaximalMonomials(P)$, and
    therefore that we can decide whether $Q \in \CorrectPoly$. Notice that if
    $Q \not \in \CorrectPoly$, then $P \not \in \CorrectPoly$, hence we can
    assume that $Q \in \CorrectPoly$.

    Similarly, for partial valuations
    $\nu \colon \vec{X} \topartial \set{0, \dots, K}$
    fixing at least one indeterminate, $\MaximalMonomials(\restr{P}{\nu})
    \hoarele \MaximalMonomials{P}$. 
    We can check for all these partial valuations that
    $\restr{P}{\nu} \in \CorrectPoly$ by induction hypothesis,
    and if one test is not satisfied, then $P \not \in \CorrectPoly$.

    Let us now conclude that if all the above checks answered correctly,
    then $P \in \CorrectPoly$. Let $\nu \colon \vec{X} \topartial \Nat$
    be a partial valuation. Then, 
    one of the three following cases arise:
    \begin{itemize}
        \item 
            Either,
        $\nu$ fixes \emph{some} indeterminate $X_i$ to a value $n_i < K$.
        In that case, let us define $\delta$ as the partial function that coincides
        with $\nu$ on its domain
        $\dom(\delta) = \dom(\nu) \setminus \set{ X_i }$.
        Then, $\restr{P}{\nu} = \restr{\restr{P}{X_i = n_i}}{\delta}$.
        As we previously checked that $\restr{P}{X_i = n_i} \in \CorrectPoly$,
        we conclude in particular that 
        $\restr{P}{\nu}$ has \kl{non-negative} \kl{maximal monomials}.

        \item Or, for all $X_i$ in the (potentially empty) domain of $\nu$, $\nu(X_i)
            \geq K$. In that case, let us write $\delta \defined
            \translate{-K}(\nu)$, that is the partial map $\delta \colon
            \vec{X} \topartial \Nat$ having the same domain as $\nu$, and such
            that for all $X_i \in \dom(\nu)$, $\delta(X_i) \defined \nu(X_i) -
            K$. Recall that $Q = \Diff{K}{P_1} + P_2$, and therefore that
            $\translate{K}(P) = P_1 + \translate{K}(Q)$. By linearity,
            $\restr{\translate{K}(P)}{\delta} = \restr{P_1}{\delta} +
            \restr{\translate{K}(Q)}{\delta}$, and we conclude that the
            \kl{maximal monomials} of $\restr{\translate{K}(P)}{\delta}$ are
            \kl{non-negative}.
            Since $\restr{\translate{K}(P)}{\delta} = \translate{K}{\restr{P}{\nu}}$,
            and since
            $\translate{K}$ leaves \kl{maximal monomials} invariant,
            we conclude that
            the \kl{maximal monomials} of $\restr{P}{\nu}$
            are the same as those of $\restr{\translate{K}(P)}{\delta}$,
            and are therefore \kl{non-negative}.
            \qedhere
    \end{itemize}
\end{proof}

We are now ready to state the corrected and generalized version of
\cref{karh:thm}, which is the main technical contribution of the paper.

\begin{theorem}
    \label{corrected-version:thm}
    Let $P \in \Rel[\vec{X}]$ be a polynomial.
    The following are equivalent:
    \begin{enumerate}
        \item \label{corrected-1:item} $P \in \CorrectPoly$,
        \item \label{corrected-2:item} $P$ is \kl{represented} by a \kl{$\Nat$-rational series},
        \item \label{corrected-3:item} $P$ is \kl{represented} by a \kl{$\Nat$-polyregular function},
        \item \label{corrected-4:item} $P$ is \kl{represented} by a \kl{star-free $\Nat$-polyregular function},
        \item \label{corrected-5:item} $P$ is \kl{non-negative}
            and there exists $K \in \Nat$ such that $\translate{K}(P) \in \Nat[\vec{X}]$.
    \end{enumerate}
    Furthermore, the membership is decidable, and effective conversion
    procedures exist between all the representations.
\end{theorem}
\begin{proof}
    The implications 
    \cref{corrected-4:item} $\implies$
    \cref{corrected-3:item} $\implies$
    \cref{corrected-2:item} are obvious.
    \cref{lem:correct-to-n-rat} proves
    \cref{corrected-1:item} $\implies$ \cref{corrected-4:item},
    while \cref{n-rat-correct:lem}
    proves 
    \cref{corrected-2:item} $\implies$ \cref{corrected-1:item}.
    Note that the lemmas provide effective conversion procedures,
    and that \cref{lem:correct-to-n-rat} also provides a decision
    procedure.

    The equivalence between
    \cref{corrected-5:item} and \cref{corrected-1:item}
    is precisely \cref{derivation-translation:lem}.
\end{proof}

Let us now state, for the sake of a completeness, the equality of
$\CorrectPoly$ and $\CoveredPoly$ in the case of two indeterminates. This is
relevant because it shows that examples given in \cite{KARH77} are actually
correct, even though based on an invalid result, and it may be the case for
other works based on \cite{KARH77}.

\begin{lemma}
    \label{lem:correct-covered-2}
    $\CorrectPoly[X,Y] = \CoveredPoly[X,Y]$.
\end{lemma}
\begin{proof}
    It is clear that $\CorrectPoly[X,Y] \subseteq \CoveredPoly[X,Y]$,
    by considering the empty valuation $\nu \colon \set{X,Y} \topartial \Nat$.
    For the converse inclusion, let us consider $P(X,Y)$
    that is \kl{non-negative}, such that the \kl{maximal monomials}
    have \kl{positive coefficients}.
   

    If we fix none of the variables, then the \kl{maximal monomials}
    have \kl{positive coefficients} by assumption. If we fix one of the
    variables, we can assume without loss of generality that we 
    fix $X = k$ for some $k \in \Nat$.
    Then $P(k,Y)$ is a \kl{non-negative} \emph{univariate} polynomial, 
    and therefore must have a positive leading coefficient
    (which is the unique \kl{maximal monomial} in this case)
    or be constant equal to 0. In both cases, the \kl{maximal monomials}
    have \kl{positive coefficients}.
    The same reasoning applies \emph{a fortiori} in the case where
    we fix the two indeterminate, leading to a constant polynomial.
\end{proof}

