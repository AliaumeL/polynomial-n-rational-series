%! TeX program = xelatex
%! lang = en-US
\subsection{The Corrected Theorem}
\label{sec:proof}

The counter example provided by \cref{def:bad-polynomial} relies on the fact
that $\CoveredPoly$ is not stable under fixing indeterminates, while
\kl{$\Nat$-polyregular functions} are. In this section, we prove that closing
$\CorrectPoly$ under assignments of variable is enough to recover from
\cref{karh:thm}.

\AP We use the following notation to fix the value of some indeterminate, if
$P(X,Y)$ is a polynomial in $\Rel[X,Y]$, then $\intro*\restr{P(X,Y)}{X = 1}$ is
the polynomial $P(1,Y) \in \Rel[Y]$. More generally, if $\nu$ is a partial
function from $\vec{X}$ to $\Nat$, written $\nu \colon \vec{X} \topartial
\Nat$, the restriction $\restr{P(\vec{X})}{\nu}$ is the polynomial with
indeterminates $\vec{Y} \defined \vec{X} - \dom(\nu)$ obtained by fixing the
variables of the domain of $\nu$.


\begin{definition}
    Let $\vec{X}$ be a finite tuple of indeterminates.
    The class $\intro*\CorrectPoly[\vec{X}]$ is the collection of
    polynomials $P \in \Rel[\vec{X}]$ such that
    $P$ is 
    such that, for every partial function $\nu \colon \vec{X} \topartial \Nat$,
    every \kl{maximal monomial} of
    $\restr{P}{\nu}$ is \kl{non-negative}.
\end{definition}

First, let us remark that $\CorrectPoly \subseteq \CoveredPoly$, because
polynomials in $\CorrectPoly$ are \kl{non-negative}. Let us also check that the
counter example provided in \cref{thm:counter-example} is not in
$\CorrectPoly$. For that, notice that for $\BadPoly$ introduced in
\cref{def:bad-polynomial}, $\BadPoly(X,Y,1) = 3X^2 + 3Y^2 - 2XY$, which has a negative
coefficient for a \kl{maximal monomial}, namely $-2XY$. 

Let us now prove that \kl{$\Nat$-rational polynomials} are in $\CorrectPoly$.
This follows from the correct implication in the statement of \cref{karh:thm},
but we provide a self-contained proof using a refinement of the
classical combinatorial arguments for $\ZPoly$ \cite[Lemma 4.16]{LOPEZ23b} and
$\NPoly$ \cite[Lemma 5.37]{gaetanphd}.

\begin{lemma}
    \label{n-poly-combinatorics:lem}
    Let $f$ be a  \kl{$\Nat$-polyregular} function. 
    There exists a computable $\omega \in \Nat$
    such that for all $p \in \Nat$,
    for all $\alpha_0, \dots, \alpha_p \in \Sigma^*$,
    for all $u_1, \dots, u_p \in \Sigma^*$,
    there exists a polynomial $P \in \Rel[X_1, \dots, X_p]$
    whose \kl{maximal monomials} are \kl{non-negative},
    and such that for all $X_1, \dots, X_p \geq \omega$:
    \begin{equation*}
        f\left(
            \alpha_0 \prod_{i = 1}^p u_i^{\omega X_i} \alpha_i
        \right)
        = P(X_1, \dots, X_p) \quad .
    \end{equation*}
\end{lemma}


\begin{corollary}
    \label{n-rat-correct:lem}
    Let $P \in \Rel[X_1, \dots, X_p]$ be \kl{represented}
    by a \kl{$\Nat$-rational series}. Then,
    $P \in \CorrectPoly$.
\end{corollary}

\AP The core of the upcoming \cref{lem:correct-to-n-rat} leverages a notion of
\intro{discrete derivative} to perform an induction on the \kl{maximal monomials}.
This notion of derivation is built by translating the domain of the polynomial.
To that end, let us write $\intro*\translate{K}$ for the \intro{translation function}
that maps a polynomial $P \in \Rel[X_1, \dots, X_n]$ to the polynomial $P(X_1 +
K, \dots, X_n + K)$.

\begin{definition}
    \label{discrete-derivative:def}
    Let $\vec{X}$ be a tuple of indeterminates,
    $K \in \Nat$,
    and 
    $P \in \Rel[\vec{X}]$ be a polynomial.
    \begin{equation*}
        \intro*\Diff{K}{P} \defined 
        \translate{K}(P) - P
        \quad .
    \end{equation*}
\end{definition}

As one would expect, the \kl{discrete derivatives} are linear operations on
polynomials, that commutes with \kl{translation operators}. However, the
\kl{translation operators} do not perfectly commute with partial applications
$\restr{\cdot}{\nu}$.

\begin{remark}
    \label{discrete-deriv-linear:fact}
    For all tuples $\vec{X}$ of indeterminates,
    for all $K \in \Nat$, $L \in \Rel$, for all partial functions
    $\nu \colon \vec{X} \topartial \Nat$:
    \begin{enumerate}
        \item The maps $\translate{K} \colon \Rel[\vec{X}] \to \Rel[\vec{X}]$,
        $\Diff{K}{ \cdot } \colon \Rel[\vec{X}] \to \Rel[\vec{X}]$,
        and
        $\restr{\cdot}{\nu} \colon \Rel[\vec{X}] \to \Rel[\vec{X}]$
            are linear operators,
        \item $\Diff{K}{ \cdot } \circ \translate{L}
            = \translate{L} \circ \Diff{K}{\cdot}$,
        \item $\restr{\cdot}{\nu} \circ \translate{L}
            = \translate{L} \circ \restr{\cdot}{\translate{L}(\nu)}$.
    \end{enumerate}
\end{remark}

In the actual statement of \cref{corrected-version:thm}, the representation of
polynomials in $\CorrectPoly$ is strengthened from \kl{$\Nat$-polyregular
functions} to \kl{star-free $\Nat$-polyregular functions}. This is not
surprising, given the conjecture that \kl{ultimately polynomial}
\kl{$\Nat$-polyregular functions} (re-introduced in \cref{star-free:sec}) were
conjectured to be star free (\cref{zsf-nsf:conjecture}), and that polynomials
are always \kl{ultimately polynomial}.

\AP When multiple indeterminate are involved, the notion of degree must be
adapted, and to ensure that the decision procedure is effective, one must
obtain explicit bounds on the translations that are needed. To that end, let us
now introduce the ordering over which the induction of
\cref{lem:correct-to-n-rat} is built. Recall that $\MaximalMonomials(P)$ is the
set of \kl{maximal monomials} of $P$, hence belongs to $\Pfin(\Monomials)$. We
order \kl{monomials} with the \kl{divisibility ordering}, making $\Monomials$ a
\kl{well-quasi-ordering} that is isomorphic to $\Nat^k$ with the product
ordering \cite[Dickson’s Lemma]{SCSC12}. We endow $\Pfin(\Monomials)$ with the
\intro{Hoare ordering}, that is, $S_1 \hoareleq S_2$ whenever for all
\kl{monomials} $M_1 \in S_1$, there exists a monomial $M_2 \in S_2$, such that
$M_1$ \kl{divides} $M_2$. The set $(\Pfin(\Monomials), \hoareleq)$ remains a
\kl{well-quasi-ordering} \cite[Hoare quasi-ordering]{SCSC12}.


Let us now illustrate how the \kl{discrete differentiation} operator interacts
with \kl{maximal monomials} with respect to the \kl{Hoare ordering}: it
extracts \emph{submaximal} monomials from a given polynomial.

\begin{fact}
    \label{derivation-monomials:fact}
    Let $K > 0$,
    and let $M,T$ be two \kl{monomials}, such that
    $T$ \kl{strictly divides} $M$.
    Then,
    every \kl{monomial} of $\Diff{K}{M}$ \kl{strictly divides} $M$,
    and 
    $T$ \kl{divides} some \kl{maximal monomial}
    of $\Diff{K}{M}$, which has a coefficient
    that is a multiple of $K$.
\end{fact}

\begin{lemma}
    \label{derivation-simplifies:lemma}
    For all $P \in \Rel[\vec{X}]$ that are non-constant,
    for all $K \in \Nat$,
    $\MaximalMonomials(\Diff{K}{P}) \hoarele
    \MaximalMonomials(P)$.
\end{lemma}

The following \cref{derivation-stabilises-correct:lem} is the main
combinatorial argument of this section. It leverages the positivity of the
\kl{maximal monomials} to compute a shift large enough to make lower degree
coefficients positive. This lemma will be lifted to a given polynomial $P$ by
noticing that if $K \in \Nat$, and $P = P_1 + P_2$, then $\translate{K}(P) =
P_1 + \Diff{K}{P_1} + \translate{K}(P_2)$.

\begin{lemma}
    \label{derivation-stabilises-correct:lem}
    Let $P \in \CorrectPoly$,
    $P_1$ be the sum of \kl{maximal monomials} of $P$,
    and $P_2 \defined P - P_1$ be the sum of
    non-maximal monomials of $P$.
    There exists a computable $K$,
    such that
    $Q \defined (\Diff{K}{P_1} + \translate{K}(P_2)) \in \CorrectPoly$.
\end{lemma}


\begin{lemma}
    \label{lem:correct-to-n-rat}
    Let $\vec{X}$ be a tuple of indeterminates,
    and let $P \in \Rel[\vec{X}]$.
    If $P \in \CorrectPoly$, then $P$ is \kl{represented}
    by a \kl{star-free $\Nat$-polyregular function},
    which can be explicitly constructed from $P$.
\end{lemma}
\begin{proof}
    We prove the result by induction on $\MaximalMonomials(P)$. 
    In the proof, we write $\vec{X}$ for the indeterminates appearing in $P$,
    that is, we assume without loss of generality that all indeterminates are used.

    \textbf{Base case:} If the (unique) \kl{maximal monomial} of $P$ is a
    constant term. Since $P \in \CorrectPoly$, $P = n \in \Nat$, and therefore
    $P$ is \kl{represented} by a constant \kl{star-free $\Nat$-polyregular
    function}.

    \textbf{Induction:} Assume that $P$ is not a constant polynomial, and let
    us write $P = P_1 + P_2$ where $P_1$ is the sum of the \kl{maximal
    monomials} of $P$. We compute a bound $K$ such that $Q \defined
    (\Diff{K}{P_1} + \translate{K}(P_2)) \in \CorrectPoly$ using
    \cref{derivation-stabilises-correct:lem}. Thanks to
    \cref{derivation-simplifies:lemma}, we also know that $\MaximalMonomials(Q)
    \hoarele \MaximalMonomials(P)$. By induction hypothesis, $Q$ is
    \kl{represented} by a \kl{star-free $\Nat$-polyregular function} which is
    effectively computable.

    Let us now remark that $P_1 \in \Nat[\vec{X}]$, and is therefore
    (effectively) \kl{represented} by a \kl{star-free $\Nat$-polyregular
    function} (\cref{n-poly-n-poly:example}). As a consequence,
    $\translate{K}(P) = P_1 + Q$ is (effectively) \kl{represented} by a
    function $f_\Delta$.

    For all partial valuations $\nu \colon \vec{X} \topartial \set{0, \dots,
    K}$ fixing at least one indeterminate,  we know that
    $\MaximalMonomials(\restr{P}{\nu}) \hoarele \MaximalMonomials{P}$. Because
    $\restr{P}{\nu} \in \CorrectPoly$, one can use the induction hypothesis to
    compute a \kl{star-free $\Nat$-polyregular function} $f_\nu$ that
    \kl{represents} $\restr{P}{\nu}$.


    Let us assume that the alphabet over which the (\kl{commutative}) functions
    $f_\Delta$ and $f_\nu$ are defined is $\set{a_1, \dots, a_n}$, with $a_i$
    representing the indeterminate $X_i$ of the polynomials. Now, let us define
    by case analysis the following \kl{commutative} \kl{star-free
    $\Nat$-polyregular function}, defined on words $w$ of the form $w \defined
    a_1^{X_1} \cdots a_n^{X_n}$, with $X_1, \dots, X_n \geq 0$.

    \begin{equation*}
        f(w) \defined
        \begin{cases}
            f_\nu(w) & \text{ if } \exists i \in \set{1, \dots, n}, X_i \leq K \\
            f_\Delta(a_1^{X_1 - K} \cdots a_n^{X_n - K}) & \text{ otherwise }
        \end{cases}
        \quad .
    \end{equation*}
    It is clear that
    $f$ is a \kl{commutative} \kl{star-free $\Nat$-polyregular function},
    and that
    $f$ \kl{represents} $P$.
\end{proof}


While \cref{lem:correct-to-n-rat} provides effective conversion, it does not
explicitly state that the membership is decidable to keep the proof clearer. A
similar proof scheme can be followed to conclude that membership is decidable,
and even show that elements in $\CorrectPoly$ are, up to suitable translations,
polynomials in $\Nat[\vec{X}]$. Beware that partial applications are still
needed in this characterization, as \cref{bad-poly-translate:ex} illustrates.

\begin{lemma}
    \label{derivation-translation:lem}
    Let $P \in \Rel[\vec{X}]$, 
    there exists a computable $K \in \Nat$
    such that the following are equivalent:
    \begin{enumerate}
        \item \label{d-t-correct:item} $P \in \CorrectPoly$,
        \item \label{d-t-transl:item}
            for 
            all partial functions $\nu \colon \vec{X} \topartial \Nat$,
            $\translate{K}(\restr{P}{\nu}) \in \Nat[\vec{X}]$,
        \item \label{d-t-transl-fin:item}
            for all partial functions
            $\nu \colon \vec{X} \topartial \set{0, \dots, K}$,
            $\translate{K}(\restr{P}{\nu}) \in \Nat[\vec{X}]$.
    \end{enumerate}
    Furthermore, the membership is decidable.
\end{lemma}


\begin{example}
    \label{bad-poly-translate:ex}
    The polynomial $\BadPoly$ is not a 
    \kl{$\Nat$-rational polynomial},
    but is \kl{non-negative} and satisfies
    $\translate{10}(\BadPoly) \in \Nat[\vec{X}]$.
\end{example}

We are now ready to state the corrected and generalized version of
\cref{karh:thm}, which is the main technical contribution of the paper.

\begin{theorem}
    \label{corrected-version:thm}
    Let $P \in \Rel[\vec{X}]$ be a polynomial.
    The following are equivalent:
    \begin{enumerate}
        \item \label{corrected-1:item} $P \in \CorrectPoly$,
        \item \label{corrected-2:item} $P$ is \kl{represented} by a \kl{$\Nat$-rational series},
        \item \label{corrected-3:item} $P$ is \kl{represented} by a \kl{$\Nat$-polyregular function},
        \item \label{corrected-4:item} $P$ is \kl{represented} by a \kl{star-free $\Nat$-polyregular function},
    \end{enumerate}
    Furthermore, the membership is decidable, and effective conversion
    procedures exist between all the representations.
\end{theorem}
\begin{proof}
    The implications 
    \cref{corrected-4:item} $\implies$
    \cref{corrected-3:item} $\implies$
    \cref{corrected-2:item} are obvious.
    \cref{lem:correct-to-n-rat} proves
    \cref{corrected-1:item} $\implies$ \cref{corrected-4:item},
    while \cref{n-rat-correct:lem}
    proves 
    \cref{corrected-2:item} $\implies$ \cref{corrected-1:item}.
    Note that the lemmas provide effective conversion procedures,
    and that \cref{lem:correct-to-n-rat} also provides a decision
    procedure.
\end{proof}

Let us now state, for the sake of a completeness, the equality of
$\CorrectPoly$ and $\CoveredPoly$ in the case of two indeterminates. This is
relevant because it shows that examples given in \cite{KARH77} are actually
correct, even though based on an invalid result, and it may be the case for
other works based on \cite{KARH77}.

\begin{lemma}
    \label{lem:correct-covered-2}
    $\CorrectPoly[X,Y] = \CoveredPoly[X,Y]$.
\end{lemma}
\begin{proof}
    It is clear that $\CorrectPoly[X,Y] \subseteq \CoveredPoly[X,Y]$,
    by considering the empty valuation $\nu \colon \set{X,Y} \topartial \Nat$.
    For the converse inclusion, let us consider $P(X,Y)$
    that is \kl{non-negative}, such that the \kl{maximal monomials}
    are  \kl{non-negative}.
   

    If we fix none of the variables, then the \kl{maximal monomials}
    are \kl{non-negative} by assumption. If we fix one of the
    variables, we can assume without loss of generality that we 
    fix $X = k$ for some $k \in \Nat$.
    Then $P(k,Y)$ is a \kl{non-negative} \emph{univariate} polynomial, 
    and therefore must have a positive leading coefficient
    (which is the unique \kl{maximal monomial} in this case)
    or be constant equal to 0. In both cases, the \kl{maximal monomials}
    have positive coefficient.
    The same reasoning applies \emph{a fortiori} in the case where
    we fix the two indeterminate, leading to a constant polynomial.
\end{proof}

