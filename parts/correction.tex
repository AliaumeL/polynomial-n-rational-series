%! TeX program = xelatex
%! lang = en-US
\subsection{The Corrected Theorem}
\label{sec:proof}

\AP
The counter example provided by \cref{def:bad-polynomial} relies on the fact
that $\CoveredPoly$ is not stable under fixing indeterminates, while
\kl{$\Nat$-polyregular functions} are. 
Indeed, the polynomial $\BadPoly$ introduced in
\cref{def:bad-polynomial} satisfies $\BadPoly(X,Y,1) = 3X^2 + 3Y^2 - 2XY$, which has a negative
coefficient for a \kl{maximal monomial}, namely $-2XY$.
In this section, we prove that closing
$\CoveredPoly$ under variable assignments is enough to recover from
\cref{karh:thm}.
We use the following notation to fix the value of some indeterminate, if
$P(X,Y)$ is a polynomial in $\Rel[X,Y]$, then $\intro*\restr{P(X,Y)}{X = 1}$ is
the polynomial $P(1,Y) \in \Rel[Y]$. More generally, if $\nu$ is a partial
function from $\vec{X}$ to $\Nat$, written $\nu \colon \vec{X} \topartial
\Nat$, the restriction $\restr{P(\vec{X})}{\nu}$ is the polynomial with
indeterminates $\vec{Y} \defined \vec{X} - \dom(\nu)$ obtained by fixing the
variables of the domain of $\nu$.


\begin{definition}
    Let $\vec{X}$ be a finite tuple of indeterminates.
    The class $\intro*\CorrectPoly[\vec{X}]$ is the collection of
    polynomials $P \in \Rel[\vec{X}]$ such that,
    for every partial function $\nu \colon \vec{X} \topartial \Nat$,
    every \kl{maximal monomial} of
    $\restr{P}{\nu}$ is \kl{non-negative}.
\end{definition}

First, let us remark that $\CorrectPoly \subseteq \CoveredPoly$, because
polynomials in $\CorrectPoly$ are \kl{non-negative}. We also remarked at the
beginning of this section that our counter example provided in
\cref{thm:counter-example} is not in $\CorrectPoly$. 

\AP Let us now prove that \kl{$\Nat$-rational polynomials} are in
$\CorrectPoly$. This follows from the correct implication in the statement of
\cref{karh:thm}, but we provide a self-contained proof using a refinement of
the classical combinatorial arguments for $\ZPoly$ \cite[Lemma 4.16]{CDTL23}
and $\NPoly$ \cite[Lemma 5.37]{DOUE23}. We believe that the formulation of such
pumping arguments can be made clearer through some extra notations, with the
added property of highlighting the central role of \kl{commutative} polyregular
functions. Let us define that a function $q$ is a \intro{(noncommutative output)
polynomial function} from $\Nat^p$ to $\Sigma^*$ whenever there exists words
$\alpha_0, \dots, \alpha_p \in \Sigma^*$, and words $u_1, \dots, u_p \in
\Sigma^*$, such that $q(X_1, \dots, X_p) = \alpha_0 \prod_{i = 1}^p u_i^{X_i}
\alpha_i$. That is, $q$ is syntactically defined by a non-commutative
polynomial function over the semi-ring $\Sigma^*$. Almost by definition,
\kl{polynomial functions} are \kl{commutative} \kl{polyregular functions}.

\AP There is still a minor detail that remains to be smoothened before proving
our combinatorial \cref{n-poly-combinatorics:lem}, which is that some
polyregular functions can exhibit a limited form of division. For instance, the
function computing the sum of the lengths of the suffixes of a word $w$
actually computes $\card{w} \times (\card{w} - 1) / 2$. This will be at odds
with our restriction to polynomials with coefficients in $\Rel$. However, we
argue that up to a suitable change of basis for the ring of polynomials, all
coefficients will actually be in $\Nat$. This relatively simple trick works by
considering as ``basic polynomials" the \intro{binomial monomial} of the form
$\binom{X - \ell}{k}$, where $\ell$ and $k$ are natural numbers. Beware that
$\binom{X - \ell}{k}$ is defined to be $0$ when $X \leq \ell$, and is therefore
a ``true polynomial" only when $X \geq \ell$. This is why we introduce the
notion of \intro{natural binomial polynomials} as $\Nat$-linear combinations of
products of \kl{binomial monomials}.

\begin{lemma}
    \label{n-poly-combinatorics:lem}
    Let $f$ be an \kl{$\Nat$-polyregular} function. 
    There exists a computable $\omega \in \Nat_{\geq 1}$
    such that for all \kl{polynomial functions}
    $q \colon \Nat^p \to \Sigma^*$,
    there exists a computable \kl{natural binomial polynomial} $P$
    such that:
    \begin{equation*}
        f \circ q(\omega X_1, \dots, \omega X_p)
        =
        P
        \quad 
        \text{ over } (\Nat_{\geq 1})^p
    \end{equation*}
\end{lemma}
\begin{proof}
    Let us fix a finite monoid $M$, a morphism $\mu \colon \Sigma^* \to M$, and 
    let $\omega$ be an idempotent power for the finite monoid $M$,
    i.e., a number such that for all $x \in M$,
    $x^{\omega+1} = x^\omega$.
    We will prove by induction on $k \in \Nat_{\geq 1}$
    that for all
    $\pi \colon M^k \to \Nat$,
    the lemma holds for the function 
    $f = \pi^\dagger$, i.e., such that
    $f(w)$ is the sum over all factorizations of $w$
    into $k$ words $(w_1, \dots, w_k) \in \Sigma^*$
    of the value $\pi(\mu(w_1), \dots, \mu(w_k))$.
    Since every \kl{$\Nat$-polyregular function}
    is obtained via some choice of $M, \mu, \pi$ (\cref{nat-rel-poly:def})
    we derive the desired result.

    In all cases, we will use a \kl{polynomial function} $q \colon \Nat^p \to
    \Sigma^*$ defined by $q(X_1, \dots, X_p) \defined \alpha_0 \prod_{i = 1}^p
    (u_i^{\omega \times X_i} \alpha_i)$ using some words $\alpha_i$ (for $0 \leq i
    \leq p$) and $u_i$ (for $1 \leq i \leq p)$ in $\Sigma^*$. That is, we will
    implicitly multiply all variables by $\omega$ in order to make the
    equations more readable. Note that when using an induction hypothesis, we
    will therefore have to check that the polynomial $q$ will have this
    specific form.

    \textbf{Base case: $k = 1$.}
    Let us show that $f \circ q$ is
    actually a constant (positive) function (and in particular a \kl{natural binomial polynomial}), as for
    all $n_1, \dots, n_p \geq 1$,
    \begin{align*}
        f(q(n_1, \dots, n_p))
        &\eqdef \pi\left(\mu(q(n_1, \dots, n_p))\right) \\
        &\eqdef \pi\left(\mu(\alpha_0 \prod_{i = 1}^p u_i^{\omega n_i} \alpha_i)\right)
        \\
        &= \pi\left(\mu(\alpha_0) \prod_{i = 1}^p \mu(u_i)^{\omega n_i} \mu(\alpha_i)\right) 
        & \text{ morphism }
        \\
        &= \pi\left(\mu(\alpha_0) \prod_{i = 1}^p \mu(u_i) \mu(\alpha_i)\right)
        & \text{ idempotent and } n_i \geq 1
        &= c \geq 0
    \end{align*}

    \textbf{Induction hypothesis.}
    The proof will essentially follow the same schema as the base case,
    with extra care needed to properly handle the summation of polynomials
    that arises when $k \geq 2$.

    We will use the fact that given $n_1, \dots, n_p \geq 0$,
    there exists a bijection between
    partitions of $q(\omega n_1, \dots, \omega n_p)$
    into $k+1$ words, and 
    triples composed of
    a position  $1 \leq \ell \leq p$,
    a ratio     $0 \leq b \leq n_\ell$,
    a remainder $0 \leq r < \omega \times \card{u_\ell}$,
    and a partition of
    $u_\ell^\omega\mid_{[-r:]} u_\ell^{\omega \times (n_\ell - b)}
    \alpha_{\ell+1} \cdots \alpha_{p-1} u_p^{n_p} \alpha_{p}$
    into $k$ words,
    where we have used the notation 
    $v\mid_{[-r:]}$ to denote the word obtained from $v$ 
    by keeping only the last $r$ letters, in a sort of Pythonic syntax.
    That is, we look at where the first ``cut" is made.

    Now, given $(\ell, b, r)$ respecting the inequalities 
    mentioned above,
    one can define the \kl{polynomial functions}
    $s_{\ell,b,r} \colon \Nat^{\ell - 1} \to \Sigma^*$
    and
    $q_{\ell,b,r} \colon \Nat^{p - \ell + 1} \to \Sigma^*$
    so that
    $q(X_1, \dots, X_p) = s_{\ell, b, r}(X_1, \dots, X_{\ell-1}) 
    q_{\ell, b, r}(X_{\ell}, \dots, X_p)$
    under the assumption that $X_1 \geq b$:
    \begin{equation*}
        q_{\ell,b,r}(X_{1}, \dots, X_{p - \ell + 1})
        \defined
        u_\ell^\omega\mid_{[-r:]} u_\ell^{\omega \times (X_1 - b)}
        \alpha_{\ell+1} \cdots \alpha_{p-1} u_p^{X_{p - \ell + 1}} \alpha_{p}
        \quad .
    \end{equation*}
    Similarly, let us define 
    $s_{\ell, b, r}(X_1, \dots, X_{\ell - 1})$
    as 
    $\alpha_0 u_1^{\omega X_1} \alpha_2 \dots \alpha_{\ell - 1} 
    u_\ell^{\omega b} u_{\ell}^\omega \mid_{[1:r]}$,
    Remark that $q_{\ell,b,r}$ has all of its indeterminates
    multiplied by $\omega$, so that the induction hypothesis 
    can be used.

    The key remark is that $\mu(s_{\ell,b,r}(X_1, \dots, X_{\ell - 1}))$ is a
    constant function from $\Nat^{\ell-1}$ to $M$, and that its value is also
    constant when $b$ varies among numbers greater than $0$. This holds with
    the exact same proof as the base case of this induction. Let us call
    $m_{\ell,r,1}$ and $m_{\ell,r,0}$ the two elements of the monoid that are
    obtained by $\mu(s_{\ell,b,r}(X_1, \dots, X_{\ell-1}))$ respectively when
    $b = 0$ and when $b > 0$. Finally, let us define $\pi_{\ell,r,0} \colon M^k
    \to \Nat$ via $\pi_{\ell,r,0}(m_2, \dots, m_{k+1}) \defined
    \pi(m_{\ell,r,0}, m_1, \dots, m_{k+1})$ and similarly for $\pi_{\ell,r,1}
    \colon M^k \to \Nat$.

    With all of this preliminary work done, we are now ready to conclude by
    simply unrolling the definition of 
    a function
    $f$ defined using a \kl{production function}
    $\pi \colon M^{k+1} \to \Nat$.
    In the following equations, we omit the precise
    range for the sum over $(\ell, b, r)$ to save space.
    Similarly, we will use 
    $q_{\ell,r}(X_1, \dots, X_{p-\ell+1})
    \defined q_{\ell,b,r}(X_1+b, X_2, \dots, X_{p - \ell + 1})$
    at the end of the computation.
    \begin{align*}
        &f(q(n_1, \dots, n_p))\\
        &\eqdef 
        \sum_{v_1 \dots v_{k+1} = q(n_1, \dots, n_p)}
        \pi(\mu(v_1), \dots, \mu(v_{k+1})) \\
        &= \sum_{(\ell, b, r)}
           \sum_{v_2 \dots v_{k+1} = q_{\ell,b,r}(n_\ell, \dots, n_p)}
           \pi(\mu(s_{\ell,b,r}(n_1, \dots, n_{\ell-1})), \mu(v_2), \dots, \mu(v_{k+1}))
        \\
        &= 
           \sum_{(\ell, 0, r)}
           \sum_{v_2 \dots v_{k+1} = q_{\ell,0,r}(n_\ell, \dots, n_p)}
           \pi(m_{\ell,r,0}, \mu(v_2), \dots, \mu(v_{k+1}))
           \\
        &+ 
           \sum_{(\ell, b > 0, r)}
           \sum_{v_2 \dots v_{k+1} = q_{\ell,b,r}(n_\ell, \dots, n_p)}
           \pi(m_{\ell,r,1}, \mu(v_2), \dots, \mu(v_{k+1}))
        \\
        &= 
           \sum_{(\ell, 0, r)}
           \sum_{v_2 \dots v_{k+1} = q_{\ell,0,r}(n_\ell, \dots, n_p)}
           \pi_{\ell,r,0}(\mu(v_2), \dots, \mu(v_{k+1}))
           \\
        &+ 
           \sum_{(\ell, b > 0, r)}
           \sum_{v_2 \dots v_{k+1} = q_{\ell,b,r}(n_\ell, \dots, n_p)}
           \pi_{\ell,r,1}(\mu(v_2), \dots, \mu(v_{k+1}))
        \\
        &= 
           \sum_{1 \leq \ell \leq p}
           \sum_{0 \leq r < \omega \times \card{u_\ell}}
           (\pi_{\ell,r,0})^\dagger (q_{\ell, 0, r}(n_\ell, \dots, n_p))
           \\
        &+ 
           \sum_{1 \leq \ell \leq p}
           \sum_{0 \leq r < \omega \times \card{u_\ell}}
           \sum_{0 < b < n_\ell}
           (\pi_{\ell,r,1})^{\dagger}(q_{\ell,r}(n_\ell - b, \dots, n_p))
    \end{align*}
    Now, using the induction hypothesis,
    we have \kl{natural binomial polynomials} $P_{\ell,r,0}$ and $P_{\ell,r,1}$
    satisfying 
    \begin{equation*}
       (\pi_{\ell,r,1})^{\dagger}(q_{\ell,r}(n_\ell - b, \dots, n_p))
       = P_{\ell,r,1}(\omega (n_\ell - b), \dots, \omega n_p)
       \quad 
       \forall n_\ell - b, \dots, n_p \geq 1
       \quad .
   \end{equation*}
   And similarly
    \begin{equation*}
        (\pi_{\ell,r,0})^{\dagger}(q_{\ell,r}(n_1, \dots, n_{\ell-1})
        = P_{\ell,r,0}(\omega n_1, \dots, \omega n_{\ell-1})
       \quad 
       \forall n_1, \dots, n_{\ell - 1} \geq 1
       \quad .
   \end{equation*}

   It is an easy and folklore fact that \kl{natural binomial polynomials} are
   stable under the operation $\Sigma_Y \colon Q(X_1,\dots,X_n,Y) \mapsto \sum_{0
   < b < Y} Q(X_1, \dots, X_n, b)$, and by pre-composition with 
   a multiplication by a constant $\omega \geq 1$.

   This allows us to conclude by defining 
   \begin{equation*}
       P(X_1, \dots, X_p) \defined
       \sum_{1 \leq \ell \leq p} \sum_{0 \leq r < \omega \card{u_\ell}}
       P_{l,r,0}(\omega X_{\ell}, \dots, \omega X_p) + \Sigma_Y P_{l,r,1}(\omega Y,
       \omega X_{\ell+1}, \dots, \omega X_p)
       \quad .
   \end{equation*}
   The latter being a \kl{natural binomial polynomial} satisfying
   the desired equality.
\end{proof}

Let us highlight that the multiplicative factor $\omega$ is necessary in
\cref{n-poly-combinatorics:lem}. Indeed, the function $f \colon \set{a}^* \to
\Nat$ defined as $0$ when the input is of odd length and $1$ when the input is
of even length is \kl{$\Nat$-polyregular}, but $f(a^X)$ is not a polynomial.
Let us remark that the \kl{maximal monomials} of the polynomial $\binom{X -
p}{k}$ have coefficient $1/k!$, and are therefore \kl{non-negative}.

\begin{corollary}
    \label{n-rat-correct:lem}
    Let $P \in \Rel[X_1, \dots, X_p]$ be an \kl{$\Nat$-rational polynomial}.
    Then,
    $P \in \CorrectPoly$.
    \proofref{n-rat-correct:lem}
\end{corollary}
\begin{proof}
    Let $f$ be the \kl{commutative}
    \kl{$\Nat$-polyregular function}
    with domain defined as $\Sigma \defined \set{a_1, \dots, a_p}$
    that \kl{represents} $P$. 
    Using \cref{n-poly-combinatorics:lem},
    there exists a number $\omega \in \Nat_{\geq 1}$
    and \kl{natural binomial polynomial} $Q$
    such that
    for all $n_1, \dots, n_p \geq 1$:
    \begin{equation*}
        f\left(
            \prod_{i = 1}^p a_i^{\omega n_i}
        \right)
        = Q(n_1, \dots, n_p)
        = P(\omega n_1, \dots, \omega n_p) 
        \quad .
    \end{equation*}
    We conclude that $P(\omega X_1, \dots, \omega X_p) = Q(X_1, \dots, X_p)$
    as polynomials,
    and in particular that
    the \kl{maximal monomials} of 
    $P$ are \kl{non-negative}.

    Remark that for every partial valuation $\nu \colon \vec{X} \topartial \Nat$,
    the polynomial $\restr{P}{\nu}$ continues to be represented
    by a \kl{$\Nat$-polyregular function}, namely
    $f$ partially applied to a word $u$ and using a restricted input alphabet. As a consequence,
    the \kl{maximal monomials} of
    $\restr{P}{\nu}$ are also \kl{non-negative}, 
    and
    we have proven that $P \in \CorrectPoly$.
\end{proof}

\AP
We have proven that every \kl{$\Nat$-rational polynomial} is in $\CorrectPoly$,
and the rest of this section is devoted to proving the converse implication.
The core of the upcoming \cref{lem:correct-to-n-rat} leverages a notion of
\intro{discrete derivative} to perform an induction on the \kl{maximal monomials}.
This notion of derivation is built by translating the domain of the polynomial.
To that end, let us write $\intro*\translate{K}$ for the \intro{translation function}
that maps a polynomial $P \in \Rel[X_1, \dots, X_n]$ to the polynomial $P(X_1 +
K, \dots, X_n + K)$.

\begin{definition}
    \label{discrete-derivative:def}
    Let
    $K \in \Nat$,
    and 
    $P \in \Rel[\vec{X}]$ be a polynomial,
    then 
    %\begin{equation*}
    $
        \intro*\Diff{K}{P} \defined 
        \translate{K}(P) - P
    $.
    %    \quad .
    %\end{equation*}
\end{definition}

\AP The key lemma of this section is \cref{lem:correct-to-n-rat}, which is
proved by induction on the \kl{maximal monomials} of a polynomial $P$. The core
combinatorial argument allowing us to perform the induction is
\cref{derivation-stabilises-correct:lem}, which is proved by a straightforward
binomial expansion. Using this lemma, the induction is performed as follows:
after decomposing a polynomial $P$ into a sum of \kl{maximal monomials} $P_1$
and the rest $P_2$, notice that for all $K \in \Nat$, $\translate{K}(P) = P_1 +
\Diff{K}{P_1} + \translate{K}(P_2)$. Then, because $P \in \CorrectPoly$, $P_1$
is a \kl{$\Nat$-rational polynomial}. Furthermore, using a large enough value
of $K$, the polynomial $\Diff{K}{P_1} + \translate{K}(P_2)$ has lower degree
than $P$, and belongs to $\CorrectPoly$. As a consequence, we conclude that
$\translate{K}(P)$ is an \kl{$\Nat$-rational polynomial}. To conclude that $P$
itself is a \kl{$\Nat$-rational polynomial}, it suffices to apply the induction
hypothesis to $\restr{P}{\nu}$ where $\nu$ is a partial function from $\vec{X}$
to $\set{0, \dots, K}$ fixing at least one variable. It is then just a matter
of combining these functions to obtain a \kl{$\Nat$-polyregular function} that
\kl{represents} $P$. Notice that in \cref{lem:correct-to-n-rat}, polynomials
can be represented using \kl{star-free $\Nat$-polyregular functions}, which is
a stronger statement than the one in \cref{karh:thm}, but is in line with the
idea that polynomials are not periodic functions (this will be made precise in
\cref{zsf-nsf:conjecture}).

As one would expect, the \kl{discrete derivatives} are linear operations on
polynomials, and they commute with \kl{translation operators}. However, the
\kl{translation operators} do not perfectly commute with partial applications
$\restr{\cdot}{\nu}$.

\begin{fact}
    \label{discrete-deriv-linear:fact}
    For all tuples $\vec{X}$ of indeterminates,
    for all $K \in \Nat$, $L \in \Rel$, for all partial functions
    $\nu \colon \vec{X} \topartial \Nat$:
    \begin{enumerate}
        \item The maps $\translate{K} \colon \Rel[\vec{X}] \to \Rel[\vec{X}]$,
        $\Diff{K}{ \cdot } \colon \Rel[\vec{X}] \to \Rel[\vec{X}]$,
        and
        $\restr{\cdot}{\nu} \colon \Rel[\vec{X}] \to \Rel[\vec{X}]$
            are linear operators,
        \item $\Diff{K}{ \cdot } \circ \translate{L}
            = \translate{L} \circ \Diff{K}{\cdot}$,
        \item $\restr{\cdot}{\nu} \circ \translate{L}
            = \translate{L} \circ \restr{\cdot}{\translate{L}(\nu)}$.
    \end{enumerate}
\end{fact}

\AP Let us now introduce the ordering over which the induction of
\cref{lem:correct-to-n-rat} is built. Recall that $\MaximalMonomials(P)$ is the
set of \kl{maximal monomials} of $P$, hence belongs to $\Pfin(\Monomials)$. We
order \kl{monomials} with the \kl{divisibility ordering}, making $\Monomials$ a
\kl{well-quasi-ordering} that is isomorphic (once quotiented to be a partial
ordering) to $\Nat^k$ with the product ordering \cite[Dickson’s Lemma]{SCSC12}.
We endow $\Pfin(\Monomials)$ with the \intro{Hoare ordering}, that is, $S_1
\hoareleq S_2$ whenever for all \kl{monomials} $M_1 \in S_1$, there exists a
monomial $M_2 \in S_2$, such that $M_1$ \kl{divides} $M_2$. The set
$(\Pfin(\Monomials), \hoareleq)$ remains a \kl{well-quasi-ordering}, and is in
particular well-founded \cite[Hoare quasi-ordering]{SCSC12}.

\begin{lemma}
    \label{all-positive-derivative:lem}
    Let $P \in \Nat[\vec{X}]$ that is non-constant, and $K \in \Nat$,
    then $\Diff{K}{P} \in \Nat[\vec{X}]$ and all of its
    coefficients are (positive) multiples of $K$.
    Furthermore, every monomial that strictly divides some monomial of $P$
    appears in $\Diff{K}{P}$.
\end{lemma}
\begin{proof}
    We prove the result for monomials, as it extends
    to $\Nat$-linear combinations by linearity.
    Let $P = \prod_{i = 1}^k X_i^{\alpha_i}$ be a monomial.
    Notice that $\translate{K}(P) = \prod_{i = 1}^n (X_i + K)^{\alpha_i}$,
    and using a binomial expansion
    we list all the possible divisors of $P$,
    all of which with coefficients that are positive integers and multiples of $K$ except the coefficient
    of the maximal monomial (equal to $P$ itself) which is $1$.
    As a consequence, $\translate{K}(P) - P$ is simply
    obtained by removing this maximal monomial, which concludes the proof.
\end{proof}

\begin{lemma}
    \label{derivation-stabilises-correct:lem}
    Let $P \in \CorrectPoly$,
    $P_1$ be the sum of \kl{maximal monomials} of $P$,
    and $P_2 \defined P - P_1$ be the sum of
    non-maximal monomials of $P$.
    There exists a computable number $K \in \Nat$,
    such that
    $Q \defined (\Diff{K}{P_1} + \translate{K}(P_2)) \in \Nat[\vec{X}]$.
\end{lemma}
\begin{proof}
    Let us first tackle the specific case where $P$ is a constant polynomial.
    In this case, $P_1 = P$ and $P_2 = 0$.
    Furthermore,
    $\translate{K}(P_1) = P_1$ and $\Diff{K}{P_1} = 0$ for all $K \in \Nat$.
    We conclude that $\Diff{K}{P_1} + \translate{K}(P_2) = 0$
    for all $K \in \Nat$, hence belongs to $\Nat[\vec{X}]$. Selecting $K = 0$
    we conclude.

    Assume now that $P$ is not a constant polynomial. We will use
    \cref{all-positive-derivative:lem} on a well-selected value of $K$. Let us
    write $\alpha$ to be the maximal absolute value of a coefficient in $P$.
    Let $D$ be the number of unitary monomials that divide some monomial
    appearing in $P$. We can now define $K \defined D \times \alpha$,
    and let
    $Q \defined (\Diff{K}{P_1} + \translate{K}(P_2))$.
    Remark that $\Diff{K}{P_1}$ is already in $\Nat[\vec{X}]$,
    and the constant term of $\translate{K}(P_2)$ is also
    in $\Nat$.
    For any other monomial of $P_2$, by the maximality of $P_1$,
    it strictly divides some monomial of $P_1$,
    equals some monomial of $\Diff{K}{P_1}$. Because every monomial
    of $\Diff{K}{P_1}$ has a coefficient that is a multiple of $K = \alpha \times D$, we can
    match every monomial of $P_2$ to $\alpha$ copies of a unitary
    monomial appearing in $\Diff{K}{P_1}$, and because $\alpha$ is large enough
    we conclude that 
    $Q \in \Nat[\vec{X}]$.
\end{proof}

\begin{lemma}
    \label{lem:correct-to-n-rat}
    Let $\vec{X}$ be a tuple of indeterminates,
    and let $P \in \Rel[\vec{X}]$.
    If $P \in \CorrectPoly$, then $P$ is \kl{represented}
    by a \kl{star-free $\Nat$-polyregular function},
    which is computable given $P$.
\end{lemma}
\begin{proof}
    We prove the result by induction on the number of indeterminates of $P$.
    In the proof, we write $\vec{X}$ for the indeterminates appearing in $P$,
    that is, we assume without loss of generality that all indeterminates are used.

    \textbf{Base case:} If the (unique) \kl{maximal monomial} of $P$ is a
    constant term. Since $P \in \CorrectPoly$, $P = n \in \Nat$, and therefore
    $P$ is \kl{represented} by a constant \kl{star-free $\Nat$-polyregular
    function}.

    \textbf{Induction:} Assume that $P$ is not a constant polynomial, and let
    us write $P = P_1 + P_2$ where $P_1$ is the sum of the \kl{maximal
    monomials} of $P$. We compute a bound $K$ such that $Q \defined
    (\Diff{K}{P_1} + \translate{K}(P_2)) \in \Nat[\vec{X}]$ using
    \cref{derivation-stabilises-correct:lem}. In particular, $Q$ is
    \kl{represented} by a \kl{star-free $\Nat$-polyregular function} using
    \cref{n-poly-n-poly:example}, the latter being effectively computable from
    $Q$. Let us now remark that $P_1 \in \Nat[\vec{X}]$, and is therefore
    (effectively) \kl{represented} by a \kl{star-free $\Nat$-polyregular
    function} (using again \cref{n-poly-n-poly:example}). As a consequence,
    $\translate{K}(P) = P_1 + Q$ is (effectively) \kl{represented} by a
    function $f_\Delta$.

    For all partial valuations $\nu \colon \vec{X} \topartial \set{0, \dots,
    K}$ fixing at least one indeterminate, one can use the induction hypothesis
    to compute a \kl{star-free $\Nat$-polyregular function} $f_\nu$ that
    \kl{represents} $\restr{P}{\nu}$. This is possible because we assumed that
    all indeterminates in $\vec{X}$ are used in $P$.


    Let us assume that the alphabet over which the (\kl{commutative}) functions
    $f_\Delta$ and $f_\nu$ are defined is $\set{a_1, \dots, a_n}$, with $a_i$
    representing the indeterminate $X_i$ of the polynomials. Now, let us define
    by case analysis the following \kl{commutative} \kl{star-free
    $\Nat$-polyregular function}, defined on words $w$ of the form $w \defined
    a_1^{x_1} \cdots a_n^{x_n}$, with $x_1, \dots, x_n \geq 0$.

    \begin{equation*}
        f(w) \defined
        \begin{cases}
            f_{[X_i \mapsto x_i]}(w) & \text{ if } \exists i \in \set{1, \dots, n}, x_i \leq K \\
            f_\Delta(a_1^{x_1 - K} \cdots a_n^{x_n - K}) & \text{ otherwise }
        \end{cases}
        \quad .
    \end{equation*}
    Remark that
    $f$ is a \kl{commutative} \kl{star-free $\Nat$-polyregular function}
    that
    \kl{represents} $P$.
\end{proof}


While \cref{lem:correct-to-n-rat} provides an effective conversion procedure,
it does not explicitly state that the membership is decidable to keep the proof
clearer. A similar proof scheme can be followed to conclude that membership is
decidable, and even show that elements in $\CorrectPoly$ are, up to suitable
translations, polynomials in $\Nat[\vec{X}]$. Beware that partial applications
are still needed in this characterization, as \cref{bad-poly-translate:ex}
illustrates. Effectiveness was the last missing piece of our main result
\cref{corrected-version:thm}, which corrects and generalizes \cref{karh:thm}.

\begin{lemma}
    \label{derivation-translation:lem}
    Let $\vec{X}$ be a tuple of indeterminates,
    and let $P \in \Rel[\vec{X}]$.
    There exists a computable number $K \in \Nat$
    such that the following are equivalent:
    \begin{enumerate}
        \item \label{d-t-correct:item} $P \in \CorrectPoly$,
        \item \label{d-t-transl:item}
            for 
            all partial functions $\nu \colon \vec{X} \topartial \Nat$,
            $\translate{K}(\restr{P}{\nu}) \in \Nat[\vec{X}]$,
        \item \label{d-t-transl-fin:item}
            for all partial functions
            $\nu \colon \vec{X} \topartial \set{0, \dots, K}$,
            $\translate{K}(\restr{P}{\nu}) \in \Nat[\vec{X}]$.
    \end{enumerate}
    In particular, the above properties are decidable.
    \proofref{derivation-translation:lem}
\end{lemma}

\begin{example}
    \label{bad-poly-translate:ex}
    The polynomial $\BadPoly$ is not a 
    \kl{$\Nat$-rational polynomial},
    but is \kl{non-negative} and satisfies
    $\translate{10}(\BadPoly) \in \Nat[\vec{X}]$.
\end{example}


\begin{theorem}
    \label{corrected-version:thm}
    Let $\vec{X}$ be a tuple of indeterminates,
    and let $P \in \Rel[\vec{X}]$.
    The following are equivalent:
    \begin{enumerate}
        \item \label{corrected-0:item} $P$ is a \kl{natural binomial polynomial},
        \item \label{corrected-1:item} $P \in \CorrectPoly$,
        \item \label{corrected-2:item} $P$ is \kl{represented} by a \kl{$\Nat$-rational series},
        \item \label{corrected-3:item} $P$ is \kl{represented} by a \kl{$\Nat$-polyregular function},
        \item \label{corrected-4:item} $P$ is \kl{represented} by a \kl{star-free $\Nat$-polyregular function},
    \end{enumerate}
    Furthermore, the properties are decidable, and conversions effective.
\end{theorem}
\begin{proof}
    The implications 
    \cref{corrected-4:item} $\implies$
    \cref{corrected-3:item} $\implies$
    \cref{corrected-2:item} are obvious.
    \cref{lem:correct-to-n-rat} proves
    \cref{corrected-1:item} $\implies$ \cref{corrected-4:item},
    while \cref{n-rat-correct:lem}
    proves 
    \cref{corrected-2:item} $\implies$ \cref{corrected-1:item}.
    Note that the lemmas provide effective conversion procedures,
    and that
    \cref{derivation-translation:lem}
    also provides a decision
    procedure.
\end{proof}

For completeness, let us remark that the counterexample of
\cref{thm:counter-example} uses three indeterminates, and this is not a
coincidence: in the particular case of at most two indeterminates, the classes
$\CorrectPoly$ and $\CoveredPoly$ coincide. In particular, the examples
appearing in \cite{KARH77} are not invalidated, as they all use at most two
indeterminates. Note that the equivalence is clear for the univariate case,
where being non-negative and having non-negative maximal coefficient clearly
imply being an \kl{$\Nat$-rational polynomial}.

\begin{lemma}
    \label{lem:correct-covered-2}
    $\CorrectPoly[X,Y] = \CoveredPoly[X,Y]$.
\end{lemma}
\begin{proof}
    It is clear that $\CorrectPoly[X,Y] \subseteq \CoveredPoly[X,Y]$,
    by considering the empty valuation $\nu \colon \set{X,Y} \topartial \Nat$.
    For the converse inclusion, let us consider $P(X,Y)$
    that is \kl{non-negative}, such that the \kl{maximal monomials}
    are  \kl{non-negative}.
   

    If we fix none of the variables, then the \kl{maximal monomials}
    are \kl{non-negative} by assumption. If we fix one of the
    variables, we can assume without loss of generality that we 
    fix $X = k$ for some $k \in \Nat$.
    Then $P(k,Y)$ is a \kl{non-negative} \emph{univariate} polynomial, 
    and therefore must have a positive leading coefficient
    (which is the unique \kl{maximal monomial} in this case)
    or be constant equal to 0. In both cases, the \kl{maximal monomials}
    have positive coefficient.
    The same reasoning applies \emph{a fortiori} in the case where
    we fix the two indeterminates, leading to a constant polynomial.
\end{proof}


\begin{corollary}
    \label{decide-rat-poly-npoly:cor}
    Let $P \in \Rat[\vec{X}]$ be a polynomial with \emph{rational}
    coefficients and let $\alpha$ be the smallest number in $\Nat_{\geq 1}$
    such that $\alpha P \in \Rel[\vec{X}]$. Then, the following are equivalent:
    \begin{enumerate}
        \item $P$ is a \kl{$\Nat$-rational polynomial},
        \item $P$ is \kl{represented} by a \kl{star-free $\Nat$-polyregular function},
        \item $\alpha P \in \CorrectPoly$,
            and for all $\vec{x} \in \Nat^{\vec{X}}$,
            $P(\vec{x})$ is a multiple of $\alpha$.
    \end{enumerate}
    In particular, the properties are decidable.
\end{corollary}
\begin{proof}
    If $P$ is a \kl{$\Nat$-rational polynomial}, then ...
    If $\alpha P \in \CorrectPoly$, then  it is an \kl{$\Nat$-rational polynomial}.
    But in particular, there exists a polyregular function $f$ that represents $\alpha P$
    with outputs in $\set{ a }$.
    Now, let us post-compose this function by the one that
    maps $a^{k \alpha + r}$ to $a^{k}$, which is a \kl{polyregular function}.
    We conclude that $P$ is \kl{represented} by an \kl{$\Nat$-polyregular function}.
\end{proof}

\AP As a consequence, we can also fully characterize polynomials that can be
computed using \kl{$\Rel$-rational series}, as they are precisely those
computable by a difference of two \kl{$\Nat$-rational polynomials}. This drives
us to introduce the class of \intro{integer binomial polynomials}, which are
obtained by $\Rel$-linear combinations of \kl{binomial monomials} of the form
$\binom{X - p}{k}$ for some $k,p \in \Nat$. 

\begin{corollary}
    \label{integer-binomial-polynomials:cor}
    Let $P \in \Rat[\vec{X}]$. Then, $P$ is a \kl{$\Rel$-rational polynomial}
    if and only if $P$ is a \kl{integer binomial polynomial}.
\end{corollary}


