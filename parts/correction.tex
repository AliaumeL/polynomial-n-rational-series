%! TeX program = xelatex
%! lang = en-US
\subsection{The Corrected Theorem}
\label{sec:proof}

\AP
The counter example provided by \cref{def:bad-polynomial} relies on the fact
that $\CoveredPoly$ is not stable under fixing indeterminates, while
\kl{$\Nat$-polyregular functions} are. In this section, we prove that closing
$\CoveredPoly$ under variable assignments is enough to recover from
\cref{karh:thm}.
We use the following notation to fix the value of some indeterminate, if
$P(X,Y)$ is a polynomial in $\Rel[X,Y]$, then $\intro*\restr{P(X,Y)}{X = 1}$ is
the polynomial $P(1,Y) \in \Rel[Y]$. More generally, if $\nu$ is a partial
function from $\vec{X}$ to $\Nat$, written $\nu \colon \vec{X} \topartial
\Nat$, the restriction $\restr{P(\vec{X})}{\nu}$ is the polynomial with
indeterminates $\vec{Y} \defined \vec{X} - \dom(\nu)$ obtained by fixing the
variables of the domain of $\nu$.


\begin{definition}
    Let $\vec{X}$ be a finite tuple of indeterminates.
    The class $\intro*\CorrectPoly[\vec{X}]$ is the collection of
    polynomials $P \in \Rel[\vec{X}]$ such that,
    for every partial function $\nu \colon \vec{X} \topartial \Nat$,
    every \kl{maximal monomial} of
    $\restr{P}{\nu}$ is \kl{non-negative}.
\end{definition}

First, let us remark that $\CorrectPoly \subseteq \CoveredPoly$, because
polynomials in $\CorrectPoly$ are \kl{non-negative}. Let us also check that the
counter example provided in \cref{thm:counter-example} is not in
$\CorrectPoly$. For that, notice that for $\BadPoly$ introduced in
\cref{def:bad-polynomial}, $\BadPoly(X,Y,1) = 3X^2 + 3Y^2 - 2XY$, which has a negative
coefficient for a \kl{maximal monomial}, namely $-2XY$. 

Let us now prove that \kl{$\Nat$-rational polynomials} are in $\CorrectPoly$.
This follows from the correct implication in the statement of \cref{karh:thm},
but we provide a self-contained proof using a refinement of the classical
combinatorial arguments for $\ZPoly$ \cite[Lemma 4.16]{CDTL23} and $\NPoly$
\cite[Lemma 5.37]{DOUE23}. The proof is based on a combinatorial
characterization of $\NPoly$ reminiscent of the \emph{finite counting automata}
introduced by \cite{SCHU62}.

\begin{lemma}
    \label{n-poly-combinatorics:lem}
    Let $f$ be an \kl{$\Nat$-polyregular} function. 
    There exists a computable $\omega \in \Nat_{\geq 1}$
    such that for all $p \in \Nat$,
    for all $\alpha_0, \dots, \alpha_p \in \Sigma^*$,
    for all $u_1, \dots, u_p \in \Sigma^*$,
    there exists a polynomial $P \in \Rel[X_1, \dots, X_p]$
    whose \kl{maximal monomials} are \kl{non-negative},
    and such that for all $X_1, \dots, X_p \geq \omega$:
    \begin{equation*}
        f\left(
            \alpha_0 \prod_{i = 1}^p u_i^{\omega X_i} \alpha_i
        \right)
        = P(X_1, \dots, X_p) \quad .
    \end{equation*}
\end{lemma}
\begin{proof}
    \textbf{TODO: reprove}

    We leverage the characterization of \kl{$\Nat$-polyregular functions}
    given in \cref{nat-rel-poly:def}.
    Let $\omega$ be an idempotent power for the finite monoid $M$,
    and
    $w \defined \alpha_0 \prod_{i = 1}^p (u_i^{\omega \times X_i} \alpha_i)$.
    Let $k \in \Nat$, and $\pi \colon M^k \topartial \Nat$ be such that
    $f = \pi^\dagger$. That is, 
    $f(w)$ is the sum over all factorizations of $w$
    into $k$ words $(w_1, \dots, w_k) \in \Sigma^*$
    of the value $\pi(\mu(w_1), \dots, \mu(w_k))$.

    Let us define an equivalence relation $\equiv$ over words of
    $\Sigma^*$ as follows: two words $v_1, v_2$ are equivalent if they are
    equal when normalizing them with the rules $u_i^{\omega} u_i^{\omega} \to
    u_i^{\omega}$ for all $1 \leq i \leq p$. That is, we disregard
    repetitions of the factors $u_i^{\omega}$ in the words.
    We lift the equivalence relation $\equiv$ to $k$-tuples of
    words by pointwise application.

    Remark that if $(v_1, \dots, v_k) \equiv (w_1, \dots, w_k)$ then their
    production are the same, i.e., $\pi(\mu(v_1), \dots, \mu(v_k)) =
    \pi(\mu(w_1), \dots, \mu(w_k))$. This holds because $\mu(u_i^\omega) =
    \mu(u_i)^\omega$ is an idempotent element of the monoid $M$. In particular,
    $f(w)$ can be computed as the sum over equivalence classes for $\equiv$, of
    the value of $\pi$ on one representative, which belongs to $\Nat$.

    Therefore, it suffices to prove that the number of partitions of $w$ in a
    single equivalence class for $\equiv$ is a polynomial $P(X_1, \dots, X_p)$
    with \kl{non-negative} \kl{maximal monomials}.

    Notice that to produce all elements in a given equivalence class
    means adding the factors $u_i^\omega$ where $1 \leq i \leq p$
    back to the normalized version
    of the equivalence class. For all elements in an equivalence class,
    one has to add the same amount of $u_i^\omega$ for $1 \leq i \leq p$,
    and therefore the number of possible choices
    is either constant, or 
    is a product of binomials of the form $\binom{X - l}{s}$
    which almost has the desired shape.

    Indeed, the binomial 
    $\binom{X - l}{s}$ is equal to
    $(X-l) \cdots (X-l-s+1) / s!$.
    As a consequence,
    $\binom{s! X - l}{s} \in \Rel[X]$,
    and considering $\omega' \defined \omega \times s!$ (for some large enough $s$)
    allows us to conclude.
\end{proof}


\begin{corollary}
    \label{n-rat-correct:lem}
    Let $P \in \Rel[X_1, \dots, X_p]$ be an \kl{$\Nat$-rational polynomial}.
    Then,
    $P \in \CorrectPoly$.
    \proofref{n-rat-correct:lem}
\end{corollary}
\begin{proof}
    Let $f$ be the \kl{commutative}
    \kl{$\Nat$-rational series}
    with domain defined as $\Sigma \defined \set{a_1, \dots, a_p}$
    that \kl{represents} $P$. Because $f$ has \kl{polynomial growth},
    it is in fact a \kl{$\Nat$-polyregular function}
    (\cite[Theorem 5.22]{DOUE23}).
    Using \cref{n-poly-combinatorics:lem},
    there exists a number $\omega \in \Nat_{\geq 1}$
    and polynomial $Q$ with \kl{non-negative}
    \kl{maximal monomials} such that
    for all $X_1, \dots, X_p \geq \omega$:
    \begin{equation*}
        f\left(
            \prod_{i = 1}^p a_i^{\omega X_i}
        \right)
        = Q(X_1, \dots, X_p)
        = P(\omega X_1, \dots, \omega X_p) 
        \quad .
    \end{equation*}
    We conclude that $P(\omega X_1, \dots, \omega X_p) = Q(X_1, \dots, X_p)$
    as polynomials,
    and in particular that
    the \kl{maximal monomials} of 
    $P$ are \kl{non-negative}.

    Remark that for every partial valuation $\nu \colon \vec{X} \topartial \Nat$,
    the polynomial $\restr{P}{\nu}$ continues to be represented
    by a \kl{$\Nat$-polyregular function}, namely
    $f$ partially applied to a word. As a consequence,
    the \kl{maximal monomials} of
    $\restr{P}{\nu}$ are also \kl{non-negative}, 
    and
    we have proven that $P \in \CorrectPoly$.
\end{proof}

\AP The core of the upcoming \cref{lem:correct-to-n-rat} leverages a notion of
\intro{discrete derivative} to perform an induction on the \kl{maximal monomials}.
This notion of derivation is built by translating the domain of the polynomial.
To that end, let us write $\intro*\translate{K}$ for the \intro{translation function}
that maps a polynomial $P \in \Rel[X_1, \dots, X_n]$ to the polynomial $P(X_1 +
K, \dots, X_n + K)$.

\begin{definition}
    \label{discrete-derivative:def}
    Let
    $K \in \Nat$,
    and 
    $P \in \Rel[\vec{X}]$ be a polynomial,
    then 
    %\begin{equation*}
    $
        \intro*\Diff{K}{P} \defined 
        \translate{K}(P) - P
    $.
    %    \quad .
    %\end{equation*}
\end{definition}

\AP The key lemma of this section is \cref{lem:correct-to-n-rat}, which is
proved by induction on the \kl{maximal monomials} of a polynomial $P$. The core
combinatorial argument allowing us to perform the induction is
\cref{derivation-stabilises-correct:lem}, which is proved by a straightforward
binomial expansion. Using this lemma, the induction is performed as follows:
after decomposing a polynomial $P$ into a sum of \kl{maximal monomials} $P_1$
and the rest $P_2$, notice that for all $K \in \Nat$, $\translate{K}(P) = P_1 +
\Diff{K}{P_1} + \translate{K}(P_2)$. Then, because $P \in \CorrectPoly$, $P_1$
is a \kl{$\Nat$-rational polynomial}. Furthermore, using a large enough value
of $K$, the polynomial $\Diff{K}{P_1} + \translate{K}(P_2)$ has lower degree
than $P$, and belongs to $\CorrectPoly$. As a consequence, we conclude that
$\translate{K}(P)$ is an \kl{$\Nat$-rational polynomial}. To conclude that $P$
itself is a \kl{$\Nat$-rational polynomial}, it suffices to apply the induction
hypothesis to $\restr{P}{\nu}$ where $\nu$ is a partial function from $\vec{X}$
to $\set{0, \dots, K}$ fixing at least one variable. It is then just a matter
of combining these functions to obtain a \kl{$\Nat$-polyregular function} that
\kl{represents} $P$. Notice that in \cref{lem:correct-to-n-rat}, polynomials
can be represented using \kl{star-free $\Nat$-polyregular functions}, which is
a stronger statement than the one in \cref{karh:thm}, but is in line with the
idea that polynomials are not periodic functions (this will be made precise in
\cref{zsf-nsf:conjecture}).

\begin{lemma}
    \label{derivation-stabilises-correct:lem}
    Let $P \in \CorrectPoly$,
    $P_1$ be the sum of \kl{maximal monomials} of $P$,
    and $P_2 \defined P - P_1$ be the sum of
    non-maximal monomials of $P$.
    There exists a computable number $K \in \Nat$,
    such that
    $Q \defined (\Diff{K}{P_1} + \translate{K}(P_2)) \in \CorrectPoly$.
    \proofref{derivation-stabilises-correct:lem}
\end{lemma}


\begin{lemma}
    \label{lem:correct-to-n-rat}
    Let $\vec{X}$ be a tuple of indeterminates,
    and let $P \in \Rel[\vec{X}]$.
    If $P \in \CorrectPoly$, then $P$ is \kl{represented}
    by a \kl{star-free $\Nat$-polyregular function},
    which is computable given $P$.
\end{lemma}
\begin{proof}
    We prove the result by induction on $\MaximalMonomials(P)$. 
    In the proof, we write $\vec{X}$ for the indeterminates appearing in $P$,
    that is, we assume without loss of generality that all indeterminates are used.

    \textbf{Base case:} If the (unique) \kl{maximal monomial} of $P$ is a
    constant term. Since $P \in \CorrectPoly$, $P = n \in \Nat$, and therefore
    $P$ is \kl{represented} by a constant \kl{star-free $\Nat$-polyregular
    function}.

    \textbf{Induction:} Assume that $P$ is not a constant polynomial, and let
    us write $P = P_1 + P_2$ where $P_1$ is the sum of the \kl{maximal
    monomials} of $P$. We compute a bound $K$ such that $Q \defined
    (\Diff{K}{P_1} + \translate{K}(P_2)) \in \CorrectPoly$ using
    \cref{derivation-stabilises-correct:lem}. Thanks to
    \cref{derivation-simplifies:lemma}, we also know that $\MaximalMonomials(Q)
    \hoarele \MaximalMonomials(P)$. By induction hypothesis, $Q$ is
    \kl{represented} by a \kl{star-free $\Nat$-polyregular function} which is
    effectively computable.

    Let us now remark that $P_1 \in \Nat[\vec{X}]$, and is therefore
    (effectively) \kl{represented} by a \kl{star-free $\Nat$-polyregular
    function} (\cref{n-poly-n-poly:example}). As a consequence,
    $\translate{K}(P) = P_1 + Q$ is (effectively) \kl{represented} by a
    function $f_\Delta$.

    For all partial valuations $\nu \colon \vec{X} \topartial \set{0, \dots,
    K}$ fixing at least one indeterminate,  we know that
    $\MaximalMonomials(\restr{P}{\nu}) \hoarele \MaximalMonomials{P}$. Because
    $\restr{P}{\nu} \in \CorrectPoly$, one can use the induction hypothesis to
    compute a \kl{star-free $\Nat$-polyregular function} $f_\nu$ that
    \kl{represents} $\restr{P}{\nu}$. This is possible because we assumed 
    that all indeterminates in $\vec{X}$ are used in $P$.


    Let us assume that the alphabet over which the (\kl{commutative}) functions
    $f_\Delta$ and $f_\nu$ are defined is $\set{a_1, \dots, a_n}$, with $a_i$
    representing the indeterminate $X_i$ of the polynomials. Now, let us define
    by case analysis the following \kl{commutative} \kl{star-free
    $\Nat$-polyregular function}, defined on words $w$ of the form $w \defined
    a_1^{x_1} \cdots a_n^{x_n}$, with $x_1, \dots, x_n \geq 0$.

    \begin{equation*}
        f(w) \defined
        \begin{cases}
            f_{[X_i \mapsto x_i]}(w) & \text{ if } \exists i \in \set{1, \dots, n}, x_i \leq K \\
            f_\Delta(a_1^{x_1 - K} \cdots a_n^{x_n - K}) & \text{ otherwise }
        \end{cases}
        \quad .
    \end{equation*}
    Remark that
    $f$ is a \kl{commutative} \kl{star-free $\Nat$-polyregular function}
    that
    \kl{represents} $P$.
\end{proof}


While \cref{lem:correct-to-n-rat} provides effective conversion, it does not
explicitly state that the membership is decidable to keep the proof clearer. A
similar proof scheme can be followed to conclude that membership is decidable,
and even show that elements in $\CorrectPoly$ are, up to suitable translations,
polynomials in $\Nat[\vec{X}]$. Beware that partial applications are still
needed in this characterization, as \cref{bad-poly-translate:ex} illustrates.
Effectiveness was the last missing piece of our main result
\cref{corrected-version:thm}, which corrects and generalizes \cref{karh:thm}.

\begin{lemma}
    \label{derivation-translation:lem}
    Let $\vec{X}$ be a tuple of indeterminates,
    and let $P \in \Rel[\vec{X}]$.
    There exists a computable number $K \in \Nat$
    such that the following are equivalent:
    \begin{enumerate}
        \item \label{d-t-correct:item} $P \in \CorrectPoly$,
        \item \label{d-t-transl:item}
            for 
            all partial functions $\nu \colon \vec{X} \topartial \Nat$,
            $\translate{K}(\restr{P}{\nu}) \in \Nat[\vec{X}]$,
        \item \label{d-t-transl-fin:item}
            for all partial functions
            $\nu \colon \vec{X} \topartial \set{0, \dots, K}$,
            $\translate{K}(\restr{P}{\nu}) \in \Nat[\vec{X}]$.
    \end{enumerate}
    \proofref{derivation-translation:lem}
\end{lemma}

\begin{example}
    \label{bad-poly-translate:ex}
    The polynomial $\BadPoly$ is not a 
    \kl{$\Nat$-rational polynomial},
    but is \kl{non-negative} and satisfies
    $\translate{10}(\BadPoly) \in \Nat[\vec{X}]$.
\end{example}


\begin{theorem}
    \label{corrected-version:thm}
    Let $\vec{X}$ be a tuple of indeterminates,
    and let $P \in \Rel[\vec{X}]$.
    The following are equivalent:
    \begin{enumerate}
        \item \label{corrected-1:item} $P \in \CorrectPoly$,
        \item \label{corrected-2:item} $P$ is \kl{represented} by a \kl{$\Nat$-rational series},
        \item \label{corrected-3:item} $P$ is \kl{represented} by a \kl{$\Nat$-polyregular function},
        \item \label{corrected-4:item} $P$ is \kl{represented} by a \kl{star-free $\Nat$-polyregular function},
    \end{enumerate}
    Furthermore, the properties are decidable, and conversions effective.
\end{theorem}
\begin{proof}
    The implications 
    \cref{corrected-4:item} $\implies$
    \cref{corrected-3:item} $\implies$
    \cref{corrected-2:item} are obvious.
    \cref{lem:correct-to-n-rat} proves
    \cref{corrected-1:item} $\implies$ \cref{corrected-4:item},
    while \cref{n-rat-correct:lem}
    proves 
    \cref{corrected-2:item} $\implies$ \cref{corrected-1:item}.
    Note that the lemmas provide effective conversion procedures,
    and that
    \cref{derivation-translation:lem}
    also provides a decision
    procedure.
\end{proof}

For completeness, let us remark that the counter example of
\cref{thm:counter-example} uses three indeterminates, and this is not a
coincidence: in the particular case of at most two indeterminates, the classes
$\CorrectPoly$ and $\CoveredPoly$ coincide. In particular, the examples
appearing in \cite{KARH77} are not invalidated, as they all use at most two
indeterminates. Note that the equivalence is clear for the univariate case,
where being non-negative and having non-negative maximal coefficient clearly
imply being an \kl{$\Nat$-rational polynomial}.

\begin{lemma}
    \label{lem:correct-covered-2}
    $\CorrectPoly[X,Y] = \CoveredPoly[X,Y]$.
\end{lemma}
\begin{proof}
    It is clear that $\CorrectPoly[X,Y] \subseteq \CoveredPoly[X,Y]$,
    by considering the empty valuation $\nu \colon \set{X,Y} \topartial \Nat$.
    For the converse inclusion, let us consider $P(X,Y)$
    that is \kl{non-negative}, such that the \kl{maximal monomials}
    are  \kl{non-negative}.
   

    If we fix none of the variables, then the \kl{maximal monomials}
    are \kl{non-negative} by assumption. If we fix one of the
    variables, we can assume without loss of generality that we 
    fix $X = k$ for some $k \in \Nat$.
    Then $P(k,Y)$ is a \kl{non-negative} \emph{univariate} polynomial, 
    and therefore must have a positive leading coefficient
    (which is the unique \kl{maximal monomial} in this case)
    or be constant equal to 0. In both cases, the \kl{maximal monomials}
    have positive coefficient.
    The same reasoning applies \emph{a fortiori} in the case where
    we fix the two indeterminates, leading to a constant polynomial.
\end{proof}

