%! TeX program = xelatex
%! lang = en-US
\section{Outlook}
\label{sec:ccl}

While \kl{$k$-residual transducers} will provide canonical objects associated
to functions in $\NPoly$, their effective construction requires the ability to
decide $\NPoly$ inside $\ZPoly$, which is not currently known (except in the
commutative case). Furthermore, the correspondence between \kl{counters} in
this automaton and \kl{star-free $\Nat$-polyregular functions} is non-trivial,
and cannot be used \emph{as is} to decide membership in $\NSF$
based on this analysis of the \kl{$k$-residual transducer}.

Let us prove that the \kl{residual transducer} is enough to decide aperiodicity
in the case of functions in $\NPoly[0]$, that is, with finite image. In
particular, this generalizes the result of \cite{LOPEZ23b}, by proving that
$\NSF[0] = \NPoly[0] \cap \ZSF$. Even though the proposed model is canonical,
and allows to effectively decide aperiodicity for \kl{$\Nat$-polyregular
functions} of growth rate $0$ (that is, with bounded image), it is not clear
yet if it can be leveraged to decide aperiodicity for \kl{$\Nat$-polyregular
functions} of higher growth rates.

\begin{lemma}
    \label{aperiodic-iff-residual:lem}
    Let $f \in \NPoly[0]$, the following are equivalent:
    \begin{enumerate}
        \item $f \in \NSF$,
        \item $f \in \ZSF$.
        \item $f$ is \kl{ultimately polynomial},
        \item The \kl{$0$-residual transducer} of $f$ is \kl{counter-free},
    \end{enumerate}
\end{lemma}

\textbf{TODO:} one may think that we can ask that $u \resleq{f}{k} u w^n$
implies $u \resleq{f}{k} uw$, but it is false, as the following example will
demonstrate. This is why the definition of counter is more involved in our
case.


\begin{lemma}
    \label{sf-no-periods-on-sequences:lemma}
    Let $f \colon \Sigma^* \to \Nat$ be a \kl{star-free $\Nat$-polyregular
    function}. Then,
    for all $u, w \in \Sigma^*$,
    there exists $n \in \Nat$ such that
    $u w^n \resleq{f}{k} u w^{n+1}$.
\end{lemma}

\begin{theorem}
    The following are equivalent for \kl{commutative}.
    \begin{itemize}
        \item $f$ is in $\NSF$,
        \item The residual automata of $f$ has no counters
    \end{itemize}
\end{theorem}


\begin{itemize}
    \item Actually, we have an algorithm that allows
        us to \emph{decide} if a polynomial is $\Nat$-rational.
    \item Actually, we obtain a characterisation of
        commutative 
        $\Nat$-rational series of polynomial growth
        among $\Rel$-rational series.
    \item The question of $\Nat$-polyregular functions
        among $\Rel$-polyregular functions remains open
        because of the non-commutativity of the input.
    \item The question of $\Nat$-rational series (commutative)
        among $\Rel$-rational series remains open.
\end{itemize}

 This suggests the
following conjecture in the non-commutative case:

\begin{conjecture}
    Let $k \in \Nat$,
    and $f \in \ZPoly[k]$. Then, $f \in \NPoly[k]$
    if and only if $f$ is \kl{combinatorial}, which is decidable,
    and an effective conversion procedure exists.
\end{conjecture}

