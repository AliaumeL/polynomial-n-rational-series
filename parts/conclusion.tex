%! TeX program = xelatex
%! lang = en-US
\section{Conclusion}
\label{sec:ccl}

\begin{itemize}
    \item Actually, we have an algorithm that allows
        us to \emph{decide} if a polynomial is $\Nat$-rational.
    \item Actually, we obtain a characterisation of
        commutative 
        $\Nat$-rational series of polynomial growth
        among $\Rel$-rational series.
    \item The question of $\Nat$-polyregular functions
        among $\Rel$-polyregular functions remains open
        because of the non-commutativity of the input.
    \item The question of $\Nat$-rational series (commutative)
        among $\Rel$-rational series remains open.
\end{itemize}

 This suggests the
following conjecture in the non-commutative case:

\begin{conjecture}
    Let $k \in \Nat$,
    and $f \in \ZPoly[k]$. Then, $f \in \NPoly[k]$
    if and only if $f$ is \kl{combinatorial}, which is decidable,
    and an effective conversion procedure exists.
\end{conjecture}


\begin{center}
    \begin{tabular}{c|cc}
        \toprule
        & \textbf{Commutative} & \textbf{General} \\
        \midrule
        \textbf{Commutative}
        and \textbf{Invertible} & commutative $\ZPoly$ & $\ZPoly$ \\
        \textbf{Commutative} & commutative $\NPoly$ & $\NPoly$ \\
        \textbf{General} & ? & polyregular 
    \end{tabular}
\end{center}


\begin{conjecture}
    For commutative functions, the residual transducer 
    is helpful.
\end{conjecture}

\begin{fact}
    \label{translation-invariance:fact}
    The following sets are stable under \kl{translations}
    $\translate{K}$,
    where $K \in \Nat$ is non-negative:
    \begin{itemize}
        \item $\CorrectPoly$,
        \item \kl{$\Nat$-rational polynomials},
        \item Polynomials \kl{represented}
            by \kl{star-free $\Nat$-polyregular functions}.
    \end{itemize}
\end{fact}

\begin{conjecture}
    Let $f \in \NPoly$. The following are equivalent,
    \begin{enumerate}
        \item $f \in \ZSF$,
        \item the \kl{$k$-residual transducer} of $f$
            is \kl{counter-free},
            and its labels belong to $\ZSF$.
    \end{enumerate}
\end{conjecture}
\begin{proof}
    If it has a counter free residual transducer recursively, then
    it clearly belongs to $\ZSF$.

    % TODO: mix orderings and trees and maximal elements :(

    Now, assume that $L$ is aperiodic, and let $(q,w^n)$ be a counter with $n
    \geq 1$. We know that $qw \equiv_L q$ because the language $L$ is
    aperiodic. Let us write $t \defined \delta(q,w)$, we know that $t \equiv_L
    q$. Assume by contradiction that $t$ and $q$ are incomparable for the
    prefix relation. Let us split $w = w_1 w_2$ where $w_1$ is the shortest
    prefix of $w$ such that $s_0 \defined \delta(q,w_1)$ is an ancestor of $q$
    and of $t$ for the prefix relation, it must exist because $\delta(q,w_1
    w_2) = t$.

    Now, consider $s_1 \defined \delta(t, w_1)$. Assume by contradiction that
    $s_0$ is not comparable with $s_1$ for the prefix relation. Then, consider
    the smallest $v$ for the prefix relation such that $\delta(t, v)$ is a
    strict prefix of $s_0$. It must exist, because otherwise the two are always
    comparable. Since $t \equiv_L q$, we know that $tv \equiv_L qv$, but then,
    this contradicts the minimality of $u$.

    We have proven that $s_0$ and $s_1$ are comparable, hence they
    are equal.
    Finally, we have proven that $\delta(q, w_1) = s_0$,
    $\delta(s_0, w_2) = t$, and $\delta(s_0, w_2) = \delta(t, w_1w_2) = q$ which is absurd.

    As a consequence $t$ and $q$ were comparable for the prefix relation,
    hence equal, and therefore $\delta(q, w) = q$.
\end{proof}

We do not know if \kl{commutativity} is decidable for more powerful models of
computation, and conjecture that it is decidable for \kl{polyregular functions}
in general.

\begin{conjecture}
    Let $f \in \Poly$. One can decide if $f$ is \kl{commutative}.
\end{conjecture}
