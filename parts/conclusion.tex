%! TeX program = xelatex
%! lang = en-US
\section{Outlook}
\label{sec:ccl}


In the commutative case, the conjectured semantic characterizations of $\NSF$
and $\NPoly$ have been proven to hold. Furthermore, the existence of a
canonical transducer computing functions in $\NPoly$ opens the door to several
questions with respect to aperiodicity. We can prove that for functions in
$\NPoly[0]$, aperiodicity is equivalent to the absence of counters in the
\kl{residual transducer} of the function (see
\cref{aperiodic-iff-residual:lem}), however this result does not generalize to
higher growth rates, as witnessed by \cref{non-aperiodic-residual-transd:ex}.
In the example, no \kl{$\NPoly[0]$-transducer} with the minimal number
of states computing the function is \kl{counter-free}, which suggests that a
more sophisticated object is needed to observe aperiodicity.

\begin{lemma}
    \label{aperiodic-iff-residual:lem}
    Let $f \in \NPoly[0]$, the following are equivalent:
    \begin{enumerate}
        \item $f \in \NSF$,
        \item $f \in \ZSF$,
        \item $f$ is \kl{ultimately polynomial},
        \item The \kl{$0$-residual transducer} of $f$ is \kl{counter-free}.
    \end{enumerate}
\end{lemma}

\begin{example}
    \label{non-aperiodic-residual-transd:ex}
    Let us define
    $\BadExKo(\varepsilon) = 1$,
    $\BadExKo(a) = 0$,
    $\BadExKo(a^2) = 1$,
    and $\BadExKo(a^n) = n - 3$ for all $n \geq 3$.
    The \kl{$0$-residual transducer} of $\BadExKo$ has a \kl{counter} and two states.
    Furthermore,
    every \kl{$\NPoly[0]$-transducer} with two states contains a \kl{counter}.
\end{example}

\AP In the spirit of the characterization of $\NPoly$ based the
$(\resleq{f}{k})$ quasi-ordering (\cref{non-commutative-npoly:thm}), we
introduce the notion of \intro{aperiodic ordering} of $\Sigma^*$ as follows:
$(\Sigma^*, \leq)$ is \reintro[aperiodic ordering]{aperiodic} whenever for all
$u, w \in \Sigma^*$, there exists $N_0 \in \Nat$, such that the sequence
$\seqof{uw^n}{n \geq N_0}$ is increasing. This leads to the following
\cref{sf-no-periods-on-sequences:conj}, of which the easy implication is
already known to hold (\cref{sf-no-periods-on-sequences:lemma}), and requires
the introduction of the \emph{star-free variant} $(\intro*\resleqsf{f}{k})$ of
$(\resleq{f}{k})$, defined by $u \reintro*\resleqsf{f}{k} v$ whenever $(v - u) \in
\NSF[k-1]$.

\begin{lemma}
    \label{sf-no-periods-on-sequences:lemma}
    Let $k \in \Nat$, and $f \in \SF[k]$. Then,
    $(\Sigma^*, \resleqsf{f}{k})$ is an
    \kl[aperiodic ordering]{aperiodic} \kl{well-quasi-ordering}.
\end{lemma}

\begin{conjecture}
    \label{sf-no-periods-on-sequences:conj}
    For all $k \in \Nat$ and $f \colon \Sigma^* \to \Rel$,
    $f \in \NSF[k]$ if and only if $(\Sigma^*, \resleqsf{f}{k})$ is an \kl[aperiodic ordering]{aperiodic}
    \kl{well-quasi-ordering}.
\end{conjecture}



Let us conclude with a more general remark on the status of commutative input
in the study of polyregular functions. Notice that semantic characterizations
of subclasses of \kl{polyregular functions} (\kl{combinatorial}, \kl{ultimately
polynomial}, pumpable) arise from pre-composition with a \kl{commutative}
\kl{star-free polyregular function}, as formally stated in
\cref{pre-compose-growth-commut:lemma,pre-compose-sf-commut:lemma}.
Generalizing this idea, we conjecture that an effective encoding of $\ZPoly$ in
terms of commutative functions could be obtained, as stated in
\cref{z-poly-commutative-encoding:conj}, playing the same role as the
decomposition obtained in \cref{decompose-polynomial:lem} in the
\kl{commutative} case.

\begin{remark}
    \label{pre-compose-growth-commut:lemma}
    Let $f \in \ZRat$, and $k \in \Nat$. Then,
    $f \in \ZPoly[k]$ if and only if 
    for every \kl{commutative} \kl{star-free polyregular function} $h$
            of \kl{growth rate} $l \in \Nat$,
            $(f \circ h) \in \ZPoly[k+l-1]$.
\end{remark}


\begin{remark}
    \label{pre-compose-sf-commut:lemma}
    Let $f \in \ZPoly$. Then, $f \in \ZSF$,
    if and only if for every \kl{commutative} \kl{star-free polyregular function} $h$,
            $(f \circ h) \in \ZSF$.
\end{remark}

\begin{conjecture}
    \label{z-poly-commutative-encoding:conj}
    Let $k \in \Nat$, and let $f \in \ZPoly[k]$ (resp. $\ZSF[k]$).
    There exists a finite set $Q$, 
    parameters $l_q \in \Nat$ for all $q \in Q$,
    \kl{commutative} polyregular functions (resp. \kl{star-free polyregular})
    $(g_q)_{q \in Q}$
    from $\Nat^{l_q}$ to $\Rel$,
    a surjective polyregular function (resp. \kl{star-free polyregular})
    $\mathsf{dec} \colon \Sigma^* \tosurj \sum_{q \in Q} \Nat^{l_q}$,
    and a polyregular function (resp. \kl{star-free polyregular})
    $\mathsf{enc} \colon \sum_{q \in Q} \Nat^{l_q} \to \Sigma^*$,
    such that
    $f = (\sum_{q \in Q} g_q) \circ \mathsf{dec}$
    and $\mathsf{dec} \circ \mathsf{enc} = \mathsf{id}$.
\end{conjecture}

In the specific subclass of \emph{polyblind functions}
\cite{LENP21,DOUE22}, parts of the conjecture holds,
because they essentially behave as commutative functions \cite[Theorem
6.12]{DOUE23}. Indeed, a suitable application of Parikh's Theorem
\cite{PARI66} guarantees the existence of the functions $\mathsf{enc}$
and $\mathsf{dec}$, although they may fail to be  \kl{polyregular functions}.
Although \cref{z-poly-commutative-encoding:remark} is not yet applicable in
practice, we consider this partial result as a strong indicator in favor of the
conjecture.

\begin{remark}
    \label{z-poly-commutative-encoding:remark}
    The conjecture \cref{z-poly-commutative-encoding:conj} holds for the
    subclass of polyblind functions when dropping the assumption that
    $\mathsf{enc}$ and $\mathsf{dec}$ are \kl{polyregular functions}
    (resp. \kl{star-free polyregular functions}).
\end{remark}
