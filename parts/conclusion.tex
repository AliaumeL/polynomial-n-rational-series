%! TeX program = xelatex
%! lang = en-US
\section{Outlook}
\label{sec:ccl}

Let us conclude with a more general remark on the status of commutative input
in the study of unary output polyregular functions. Notice that semantic
characterizations of subclasses of $\ZPoly$ (\kl{combinatorial}, \kl{ultimately
polynomial}) arise from pre-composition with a \kl{commutative}
\kl{star-free polyregular function}, as formally stated in
\cref{pre-compose-growth-commut:lemma,pre-compose-sf-commut:lemma}. Following
this idea, we conjecture that $\ZPoly$ and $\NPoly$ can be described as
combinations of commutative functions, leading to
\cref{z-poly-commutative-encoding:conj}, which would be a non-commutative
version of \cref{decompose-polynomial:lem}. For the subclass of 
\emph{polyblind functions} \cite{LENP21,DOUE22}, parts of the conjecture holds,
because they essentially behave as commutative functions \cite[Theorem
6.12]{DOUE23}.

%Indeed, a suitable application of Parikh's Theorem \cite{PARI66}
%guarantees the existence of the functions $\mathsf{enc}$ and $\mathsf{dec}$,
%although they may fail to be  \kl{polyregular functions}.

\begin{lemma}
    \label{pre-compose-growth-commut:lemma}
    Let $f \in \ZRat$, and $k \in \Nat$. Then,
    $f \in \ZPoly[k]$ if and only if 
    for every \kl{commutative} \kl{star-free polyregular function} $h$
            of \kl{growth rate} $l \in \Nat$,
            $(f \circ h) \in \ZPoly[k\times l]$.
    \proofref{pre-compose-growth-commut:lemma}
\end{lemma}


\begin{lemma}
    \label{pre-compose-sf-commut:lemma}
    Let $f \in \ZPoly$. Then, $f \in \ZSF$,
    if and only if for every \kl{commutative} \kl{star-free polyregular function} $h$,
            $(f \circ h) \in \ZSF$.
    \proofref{pre-compose-sf-commut:lemma}
\end{lemma}

