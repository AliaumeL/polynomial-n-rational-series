%! TeX program = xelatex
%! lang = en-US
% MAIN MATTER
\section{Introduction}
\label{introduction:sec}

Traditional results from automata theory state the equivalence of languages
defined in terms of deterministic finite automata and regular expressions
\cite{kleene1956representation}, Monadic Second Order Logic ($\MSO$)
\cite{buchi1960weak}, or finite monoids \cite{schutzenberger1961definition}.
Among regular languages, there exists a robust subclass sharing similar
characterizations, namely the \emph{star-free} regular languages. These are
obtained via counter-free automata \cite{mcnaughton1971counter}, represented by
star-free regular expressions \cite{schutzenberger1965finite}, definable in
First Order Logic ($\FO$) \cite{perrin1986first}, and recognized by aperiodic
monoids \cite{schutzenberger1965finite}. It is decidable whether a regular
language is aperiodic, thanks to the characterizations in terms of minimal
automaton and/or syntactic monoid.

There have been recurrent attempts to generalize the notions of regularity from
languages to functions, leading to classes of various expressiveness
(sequential, rational, regular, and \kl{polyregular functions}) \cite[see,
e.g.][]{bojanczyk2019string}. In this paper, we will focus on \kl{polyregular
functions}, whose class is denoted $\Poly$, and that shares with regular
languages the characterizations in terms of logic (via $\MSO$ interpretations),
automata (via pebble automata), and adds new bridges with programming languages
via its relationships with lambda-terms and so-called for-transducers. As for
regular languages, the class $\SF$ of \intro{star-free polyregular function}
can equivalently be introduced in terms of first-order definability,
counter-free pebble transducers, and suitable restrictions of for-tranducers.


Classical problems, including the decidability of aperiodicity, become
difficult in the case of \kl{polyregular functions}, due to the lack of
canonical objects akin to the syntactic monoid or the minimal automaton.
Indeed, for sequential functions that admit a canonical object, aperiodicity is
decidable \cite{choffrut03}, and more recently, aperiodicity has been proven
decidable for rational functions
\cite{filiot2016aperiodicity,filiot2018canonical}. The problem remains open for
regular functions, although it has been proven decidable when changing the
semantics of function equality to account for the \emph{origins} in the output
\cite{bojanczyk14}.

Another way to approach the problem of deciding star-free for polyregular
functions is to not restrict the computational power, nor change the semantics,
but consider restrictions of the output alphabet. In particular, there have
been successful results proven on \emph{unary polyregular functions}, also
known as \kl{$\Nat$-polyregular functions}
\cite{doueneau2021pebble,doueneau2022hiding}. Let us note that results on
\emph{unary output functions} often generalize to the slightly more general
class of functions with \emph{commutative output}, and that algorithms leverage
the absence of combinatorial information in the output. Furthermore,
\kl{$\Nat$-polyregular functions} benefit from theorems proven in the field of
rational series \cite{berstel2011noncommutative}, where they coincide with
weighted automata of polynomial ambiguity, rational series of polynomial
growth, and a notion of quantitative $\MSO$
\cite{berstel2011noncommutative,kreutzer2013,schutzenberger1962}. Although
promising, the restriction to unary output did not yield effective procedure to
decide the aperiodicity of polyregular functions. 


Recently, aperiodicity has been shown decidable over a restricted class of
\kl{polyregular functions} called \kl{$\Rel$-polyregular functions} and denoted
by $\ZPoly$ \cite{LOPEZ23b}.  These functions generalise \kl{$\Nat$-polyregular
functions} by allowing negative outputs, and therefore enabling a limited form
of backtracking. The decidability of aperiodicity for \kl{$\Rel$-polyregular
functions} relies on the construction of a canonical object called the
\intro{residual transducer}, the latter being essentially based on differences
between functions, crucially leveraging negative outputs.
If the proof method does not carry from $\Rel$-output to $\Nat$-output
functions, it was conjectured that the semantic characterization of \kl{star
free $\Rel$-polyregular functions} ($\ZSF$) would continue to characterize
\kl{star-free $\Nat$-polyregular functions} ($\NSF$), and in particular that
$\NPoly \cap \ZSF = \NSF$.

Let us remark that, \kl{$\Nat$-polyregular functions} and
\kl{$\Rel$-polyregular functions} are proper subsets respectively of
\kl{$\Nat$-rational series} ($\ZRat$) and of \kl{$\Rel$-rational series}
($\NRat$). For these larger classes of functions, several competing notions of
aperiodicity exist \cite{droste2019aperiodic,reutenauer_series_1980,LOPEZ23b},
and decidability of the ones that extends $\ZSF$ and $\NSF$ remains open.
Furthermore, the decidability of $\NRat$ among $\ZRat$ is also a longstanding
open problem, that was to the knowledge of the authors, only solved in the case
of \emph{polynomials} by \textcite{KARH77}.

We refer the reader to \cref{previously-known-inclusions:fig} to get an
overview of the different classes of functions appearing in the paper. Arrows
denote strict inclusions, and effectiveness (both in terms of decidability and
of effective representation) is represented by thick double arrows. Inclusions
that are suspected to be effective are represented using a dashed arrow
together with a question mark.


\begin{figure}
    \includestandalone[width=\linewidth]{tikz/class-inclusions}
    \caption{
        Decidability and inclusions of classes of functions,
        arranged along two axes. The first one is the complexity
        of the output alphabet ($\Nat$, $\Rat$, $\Sigma$). The second
        one is the allowed computational power
        (star-free polyregular functions, polyregular functions, 
        rational series).
    }
    \label{previously-known-inclusions:fig}
\end{figure}


\paragraph{Contributions.} We prove that for \kl{commutative} functions, all
the membership problems involving the classes $\ZSF$, $\ZPoly$, $\NSF$, and
$\NPoly$ are all decidable (this is the lower left square of inclusions in
\cref{previously-known-inclusions:fig}), and effective conversion algorithms
are provided. Furthermore, we show that in the commutative case, the conjecture
stating that $\ZSF \cap \NPoly = \NSF$ holds. 

These results are obtained by completely understanding which \emph{polynomials}
are computable using \kl{$\Nat$-polyregular functions}, and lifting these
results from polynomials to \kl{$\Rel$-polyregular functions}. This study
invalidates the main theorem of \cite{KARH77}, for which we provide an explicit
counter-example. 



\begin{itemize}
    \item We show that commutativity is decidable.
    \item We provide a candidate semantic characterization
    \item We prove that it is correct for commutative inputs
    \item We provide a counter-example to the theorem of nanania.
    \item We show that effective procedures are possible for all arrows 
        and answer positively to the conjecture of XX
        in the case of commutative inputs.
\end{itemize}
It was conjectured in a thesis that some particular form of polynomials
would characterize star-free functions.

This paper follows the line of
research on \kl{polyregular functions} that focuses on semantic
characterizations of the asymptotic behavior, rather than the construction
of canonical objects of computation
\cite{doueneau2021pebble,doueneau2022hiding,LOPEZ23b}. The goal was to
decide arrows X,Y and Z, namely, to understand precisely the relationship
between star-free and $\Nat$ output. In the case of \kl{$\Nat$-rational
series}, a result of \citeauthor{KARH77} characterizes \kl{commutative}
\kl{$\Nat$-rational series} among \kl{$\Rel$-rational series}
\cite{KARH77}.
This conjecture was based on the previously introduced result of
\textcite[Theorem 3.3]{KARH77}, which we will disprove in \cref{sec:c-example}.
We then provide a corrected version of the theorem, and prove the conjecture of
Douéneau in the case of \emph{commutative} $\Nat$-rational series of polynomial
growth.

\cref{thm:counter-example}
\cref{corrected-version:thm}

\cref{decidable-n-poly:thm}
\cref{zsf-npoly-nsf:thm}

\hrule


\paragraph*{Outline of the paper.}
In \cref{preliminaries:sec}, we introduce the computational models
and prove that 
\begin{itemize}
    \item In a first section, we recall the preliminary definitions.
    \item In a second section, we provide a counter example to the
        theorem of XX.
    \item In a third section, we provide an alternative result.
    \item In a fourth section, we make the results effective,
        and compute everyting.
\end{itemize}
