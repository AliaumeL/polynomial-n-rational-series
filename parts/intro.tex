%! TeX program = xelatex
%! lang = en-US
% MAIN MATTER
\section{Introduction}
\label{introduction:sec}

\AP Traditional results from automata theory state the equivalence of languages
defined in terms of deterministic finite automata and regular expressions
\cite{KLEE56}, Monadic Second Order Logic ($\intro*\MSO$)
\cite{BUCH60,ELGO61,TRAK66}, or finite monoids \cite{RABI59,SCHU61}. Among regular
languages, there exists a robust subclass sharing similar characterizations,
namely the \emph{star-free} regular languages. These are obtained via
counter-free (minimal) automata \cite{MNPA71}, represented by star-free regular
expressions \cite{SCHU65}, definable in First Order Logic ($\intro*\FO$)
\cite{MNPA71}, and recognized by aperiodic monoids \cite{SCHU65}. It is
decidable whether a regular language is aperiodic, thanks to the
characterizations in terms of minimal automaton and/or syntactic monoid.

\AP Let us briefly recall one of the equivalent definitions of a
\intro{$K$-rational series} (where $K$ is a ring), and refer to \cite{BERE10}
for a comprehensive introduction. The \kl{$K$-rational series} from $\Sigma^*$
to $K$ are the functions computed by \kl{$K$-weighted automata} which are
non-deterministic finite automata with transitions labelled by elements of $K$,
and whose output on a given word is the sum over all accepting paths of the
products of the weights along the path. For instance, the function $w \mapsto
2^{\card{w}}$ is computed by an \kl{$\Nat$-weighted automaton} with a single
state, and a loop labelled by $2$, and is therefore an \kl{$\Nat$-rational
series}. We use $\intro*\NRat$ and $\intro*\ZRat$ to denote respectively
the set of \kl{$\Nat$-rational series} and the set of \kl{$\Rel$-rational
series}.

\AP \kl{Polyregular functions} with a single letter output alphabet can be seen
as having outputs in $\Nat$, and this is why they were introduced respectively
as \kl{$\Nat$-polyregular functions} ($\NPoly$, for \kl{polyregular functions}
with unary output) and \kl{star-free $\Nat$-polyregular functions} ($\NSF$ for
\kl{star-free polyregular functions} with unary output) \cite{DOUE21,DOUE22}.
From this point of view, $\NPoly$ is a subclass of a larger and well-studied
class of functions called (non-commutative) \kl{rational series}
\cite{REUT80,BERE88,BERE10}. This other model of computation generalizes the
notion of regular language to functions with outputs in a \emph{commutative}
semiring, and it was frequently rediscovered that \kl{$\Nat$-polyregular
functions} form a (strict) subclass of \kl{rational series} with outputs in
$\Nat$, called \kl{$\Nat$-rational series} ($\NRat$)
\cite{SCHU62,KRRC13,CDTL23}.

\AP In order to approach the decision problems on \kl{polyregular functions},
restricting the output alphabet to a single letter has proven to be a fruitful
approach \cite{DOUE21,DOUE22}. The key advantage of single-letter output
alphabets is that the particular order in which letters are produced by the
machine is irrelevant. This restriction greatly simplifies the definition of
\kl{polyregular functions}, and opens the door to results that would otherwise
fail because of this ordering, such as the so-called \emph{pebble minimization
theorem} of \cite{DOUE21}, which fails in the general case
\cite{BOJA22,KLEP23}. However, even with the restriction to
single letter output alphabets, the decidability of \kl{star-free polyregular
functions} among \kl{polyregular functions} remains an open problem
\cite[Conjecture 7.61]{DOUE23}.


\AP This approach through rational series bears numerous limitations. First,
$\ZPoly$ does not correspond to a restriction of \kl{polyregular functions},
because of the extra expressive power introduced by the negative outputs
\cite{CDTL23}. Furthermore, deciding the membership of $\NRat$ inside $\ZRat$
is an open problem (see e.g. \cite{KARH77}), that remains open even in the
simple case of $\NPoly$ and $\ZPoly$ \cite[Open question 5.55]{DOUE23}. It was
conjectured that, being ``computable using coefficients in $\Nat$'' and being
``star-free'' were two orthogonal properties, and therefore that the following
equation would hold: $\NSF = \NPoly \cap \ZSF$ \cite[Conjecture 7.61]{DOUE23}.
This would imply the decidability of aperiodicity for single letter output
polyregular functions since aperiodicity is decidable for $\ZSF$. We refer the
reader to \cref{previously-known-inclusions:fig} to get an overview of the
different classes of functions appearing in the paper. 

\AP The decidability of aperiodicity for \kl{$\Rel$-polyregular functions}, that is
the membership of $\ZSF$ inside $\ZPoly$,
relies on the construction of a canonical object called the \emph{residual
transducer}, the latter being essentially based on differences between
functions, crucially leveraging negative outputs \cite{CDTL23}. Although the
proof method does not carry from $\Rel$-output functions to $\Nat$-output
functions, it was conjectured that the semantic property of being
\kl{ultimately polynomial} would also characterize $\NSF$ inside $\NPoly$ (\cref{zsf-nsf:conjecture}).

\begin{conjecture}[{\cite[Conjecture 7.61]{DOUE23}}]
    \label{zsf-nsf:conjecture}
    Let $k \in \Nat$.
    A function $f \in \NPoly[k]$
    belongs to $\NSF[k]$ if and only if
    it is \kl{ultimately polynomial}.
    In particular,
    $\NSF[k] = \ZSF[k] \cap \NPoly$.
\end{conjecture}

\begin{figure}
    \centering
    \includestandalone[width=7cm]{tikz/class-inclusions}
    \caption{
        Decidability and inclusions of classes of functions,
        arranged along two axes. The first one is the complexity
        of the output alphabet ($\Rel$, $\Nat$, $\Sigma$). The second
        one is the allowed computational power
        (\kl{star-free polyregular functions}, \kl{polyregular functions}, 
        \kl{rational series}).
        Arrows denote strict inclusions,
        and effectiveness (both in terms of decidability and of effective
        representation) is represented by thick double arrows. Inclusions that are
        suspected to be effective are represented using a dashed arrow together with a
        question mark.
    }
    \label{previously-known-inclusions:fig}
\end{figure}

In \cite{CDTL23}, the authors introduced the class of \kl{$\Rel$-polyregular
functions} ($\ZPoly$) that generalize \kl{$\Nat$-polyregular functions} by
allowing negative outputs. This remains a subclass of \kl{rational series},
more precisely, of the class of \kl{$\Rel$-rational series}
($\ZRat$). For this class of functions the aperiodic variant $\ZSF$ has
been shown to be decidable \cite{CDTL23}. Let us underline that several
non-equivalent notions of aperiodicity coexist for rational series
\cite{REUT80,DRGA19,CDTL23}, and that we only refer to the one applying to
\kl{rational series} that are actually in $\ZPoly$ \cite{CDTL23}.


\AP There have been recurrent attempts to generalize the notions of regularity
and aperiodicity from languages to functions, leading to classes of various
expressiveness, such as\footnote{This list is far from exhaustive.} Mealy
Machines ($\intro*\Mealy$) \cite{MEAL55}, sequential functions
($\intro*\Sequential$) \cite{SCHU77}, rational functions ($\intro*\Rational$)
\cite{EILE74}, \intro{regular functions} ($\intro*\Regular$) \cite{ENHO01}, and
\intro{polyregular functions} ($\intro*\Poly$) \cite{ENMA02,BOJA18,BOKL19}. For
all of these models, an \emph{aperiodic} counterpart can be defined
(respectively, $\intro*\AMealy$, $\intro*\ASequential$, $\intro*\ARational$,
$\intro*\ARegular$, $\intro*\SF$), lifting the correspondence between
counter-free automata, star-free regular languages and aperiodic monoids to the
functional setting: for Mealy Machines \cite{SCHU65,MNPA71}, for sequential
functions \cite{CHOF03}, for rational functions \cite{FGL16,FGLM18}, for
regular functions \cite{BOJA14,CADA15,DJR16,DGK21}, and for polyregular
functions \cite{BDK18,BOKL19}. The inclusions between these classes of
functions are all known to be strict, and we depicted in
\cref{computational-models:fig} the status of the \emph{membership problem}
associated to these strict inclusions (that is, given a function $f$, decide if
$f$ can be computed by a function of a proper subclass). We also redirect the
reader to the survey of Muscholl and Puppis for a more comprehensive
introduction to the status of various membership problems between classes of
string-to-string functions \cite{MUSC19}.


%Note that in
%\cref{computational-models:fig}, if $\mathsf{B}$ is a proper subclass of
%$\mathsf{C}$, then the \emph{aperiodic} variant of $\mathsf{B}$ is precisely
%the intersection of $\mathsf{B}$ with the \emph{aperiodic} variant of the
%larger class $\mathsf{C}$. Because such a (strong) statement does not
%explicitly appear in the literature, a standalone proof will be given in
%\cref{orthogonal-aperiodic-power:thm}. As an example, to conclude that a
%function can be computed by an \emph{aperiodic sequential function}
%($\ASequential$) it is enough to prove that it is sequential ($\Sequential$),
%and computed by an \intro{aperiodic polyregular function} ($\SF$). 


\AP To approach the \emph{aperiodicity} membership problems the traditional
approach has been to develop so-called canonical objects, akin to the minimal
automaton or the syntactic monoid of regular languages, over which aperiodicity
is equivalent to a simple syntactic criterion, e.g. having no counters
\cite{MNPA71}. This is the proof scheme that has been implemented for
sequential functions \cite{CHOF03}, and more recently for rational functions
\cite{FGL16,FGLM18}. Unfortunately, such a canonical object remains to be
defined for regular and polyregular functions.

\begin{figure}
    \centering
    \begin{tikzpicture}[
        xscale=1.2,
        decidable/.style={
            ->, thick, color=black
        },
        unknown/.style={
            ->, dashed, color=gray
        }
        ]
        % Create nodes for the classes and their aperiodic variants
        % Mealy / sequential / rational / regular / polyregular
        \node (mealy)  at (0, 2) {$\Mealy$};
        \node (seq)    at (2, 2) {$\Sequential$};
        \node (rat)    at (4, 2) {$\Rational$};
        \node (reg)    at (6, 2) {$\Regular$};
        \node (poly)   at (8, 2) {$\Poly$};
        % Aperiodic variants
        \node (amealy) at (0,0) {$\AMealy$};
        \node (aseq)   at (2,0) {$\ASequential$};
        \node (arat)   at (4,0) {$\ARational$};
        \node (areg)   at (6,0) {$\ARegular$};
        \node (apoly)  at (8,0) {$\SF$};
        % Arrows
        % The decidability is known for inclusions
        % mealy -> sequential -> rational -> regular -> polyregular
        % similarly for the aperiodic variants
        \draw[decidable] (mealy) --
            node[midway, above] {folklore}
            (seq);
        \draw[decidable] (seq)   -- 
            node[midway, above] {\cite{CHOF03}}
            (rat);
        \draw[decidable] (rat)   -- 
            node[midway, above] {\cite{FGRS13,BGMP15}}
            (reg);
        \draw[decidable] (reg)   -- 
            node[midway, above] {\cite{BOKL19}}
            (poly);
        \draw[decidable] (amealy) --
            (aseq);
        \draw[decidable] (aseq) --
            (arat);
        \draw[decidable] (arat) --
            (areg);
        \draw[decidable] (areg) --
            (apoly);
        % Decidability of the aperidoic variants are known
        % for mealy, sequential, rational but not for regular nor polyregular.
        % we draw the arrows
        \draw[decidable] (amealy) --
            node[midway, left] {\cite{SCHU65,MNPA71}}
            (mealy);
        \draw[decidable] (aseq) -- 
            node[midway, left] {\cite{CHOF03}}
            (seq);
        \draw[decidable] (arat) -- 
            node[midway, left] {\cite{FGL16,FGLM18}}
            (rat);
        \draw[unknown] (areg) --
            node[midway, left] {$\simeq$\cite{BOJA14}}
            (reg);
        \draw[unknown] (apoly) --
            node[midway, left] {?}
            (poly);
    \end{tikzpicture}
    \caption{
        In this picture, arrows denote strict inclusions. Plain
        arrows signify that the corresponding membership problem is decidable,
        while dashed arrows signify that the corresponding membership problem
        is conjectured to be decidable. In the case of \kl{regular functions}
        ($\Regular$),
        the decidability has only been obtained under a different notion
        of equivalence of functions \cite{BOJA14}.
    }
    \label{computational-models:fig}
\end{figure}




\subparagraph*{Contributions.} This paper focuses on \kl{$\Nat$-polyregular
functions} that are furthermore \kl{commutative}, that is where the output of
the functions remains unchanged when the letters of the input word are
permuted. These functions from $\Sigma^*$ to $\Rel$ can be understood as
functions from $\Nat^{\card{\Sigma}}$ to $\Rel$, and are therefore easier to
study using classical mathematical tools. The simplest kind of functions in
this class are actual (multivariate) polynomial functions, which were studied
by Karhumäki in \cite{KARH77}. Our analysis of \kl{commutative}
\kl{$\Nat$-polyregular functions} is divided in two main parts: the first one
completely revisits the results of \cite{KARH77} on polynomials, and the second
one extends the former decidability results to arbitrary \kl{commutative}
\kl{$\Nat$-polyregular functions}.

Our first contribution is to provide a counter-example to the main theorem of
\cite{KARH77} that claimed to characterize the (multivariate) polynomials in
$\Rat[\vec{X}]$ that can be computed by some \kl{$\Nat$-rational series}
(\cref{thm:counter-example}). Our second contribution is to correct this error,
and provide (effective) characterizations of polynomials in $\Rat[\vec{X}]$
that can be computed by \kl{$\Rel$-polyregular functions}
(\cref{integer-binomial-polynomials:cor}), and by \kl{$\Nat$-polyregular
functions} (\cref{decide-rat-poly-npoly:cor}). These characterizations also
show that polynomials computable using $\NRat$ (resp. $\ZRat$) are also
computable using $\NSF$ (resp. $\ZSF$), that is: polynomials are inherently
\emph{star-free}. Furthermore, these characterizations provide effective
descriptions of polynomials that can be expressed in $\ZPoly$ as those obtained
using integer combinations of products of \emph{binomial coefficients} (called
\kl{integer binomial polynomials}, defined page \kpageref{integer binomial
polynomial}) and similarly for $\NPoly$ using the notion of \kl{strongly
natural binomial polynomials} (defined page \kpageref{strongly natural binomial
polynomial}).

Our third and last contribution
of this paper is to answer \cite[Conjecture 7.61]{DOUE23} (stating that $\NSF =
\NPoly \cap \ZSF$) positively in the \kl{commutative} case, and prove that the
membership problems between $\NSF$, $\ZSF$, $\NPoly$, and $\ZPoly$ are all
decidable, with effective conversion algorithms. This effectively shows that
all arrows in the upper left square of \cref{previously-known-inclusions:fig}
are decidable in the \kl{commutative} setting. 

\subparagraph*{Outline of the paper.} In \cref{preliminaries:sec}, we provide a
combinatorial definition of \kl{$\Nat$-polyregular functions} (resp.
$\Rel$-polyregular functions), show that one can decide if a function $f \in
\ZPoly$ is \kl{commutative} (\cref{decidable-commutative-poly:lemma}), and give
the basic definitions used on multivariate polynomials. In
\cref{polynomials:sec}, we focus on \emph{polynomials}, by introducing the
\cref{karh:thm} of \cite{KARH77}, and providing a counter-example in
\cref{thm:counter-example}. We then correct this result in
\cref{sec:n-poly-to-poly,sec:poly-to-n-poly,sec:rel-to-rat} by providing first
a characterization of polynomials in $\Rel[\vec{X}]$ that can be computed by
\kl{$\Nat$-polyregular functions} (\cref{corrected-version:thm}), and then
extending this result to $\Rat[\vec{X}]$ (\cref{decide-rat-poly-npoly:cor}).
Along the way, we obtain a characterization of polynomials in $\Rat[\vec{X}]$
that can be obtained through \kl{$\Rel$-rational series}
(\cref{integer-binomial-polynomials:cor}). Finally, in
\cref{beyond-polynomials:sec}, we leverage the results over polynomials to
decide $\NPoly$ inside $\ZPoly$ in the \kl{commutative} case
(\cref{decidable-n-poly:thm}), and prove that \cref{zsf-nsf:conjecture}
\cite[Conjecture 7.61]{DOUE23} holds for \kl{commutative} functions
(\cref{zsf-npoly-nsf:thm}).
