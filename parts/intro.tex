%! TeX program = xelatex
%! lang = en-US
% MAIN MATTER
\section{Introduction}
\label{introduction:sec}

Traditional results from automata theory state the equivalence of languages
defined in terms of deterministic finite automata and regular expressions
\cite{kleene1956representation}, Monadic Second Order Logic ($\MSO$)
\cite{buchi1960weak}, or finite monoids \cite{schutzenberger1961definition}.
Among regular languages, there exists a robust subclass sharing similar
characterizations, namely the \emph{star-free} regular languages. These are
obtained via counter-free automata \cite{mcnaughton1971counter}, represented by
star-free regular expressions \cite{schutzenberger1965finite}, definable in
First Order Logic ($\FO$) \cite{perrin1986first}, and recognized by aperiodic
monoids \cite{schutzenberger1965finite}. It is decidable whether a regular
language is aperiodic, thanks to the characterizations in terms of minimal
automaton and/or syntactic monoid.

There have been recurrent attempts to generalize the notions of regularity from
languages to functions, leading to classes of various expressiveness
(sequential, rational, regular, and \kl{polyregular functions}) \cite[see,
e.g.][]{bojanczyk2019string}. In this paper, we will focus on \kl{polyregular
functions}, that shares with regular languages the characterizations in terms
of logic (via $\MSO$ interpretations), automata (via pebble automata or SST
\cite{}), and adds new bridges with programming languages (lambda-terms)
for-transducers. Although the class of "star-free" polyregular functions is
defined and similarly characterized using first-order logic.

Classical problems, including the decidability of aperiodicity, become
difficult in the case of \kl{polyregular functions}, due to the lack of
canonical objects akin to the syntactic monoid or the minimal automaton.
Indeed, for sequential functions that admit a canonical object, aperiodicity is
decidable \cite{choffrut03}, and more recently, it aperiodicity has been proven
decidable for rational functions
\cite{filiot2016aperiodicity,filiot2018canonical}. The problem remains open for
regular functions, although it has been proven decidable when changing the
semantics of function equality to account for the \emph{origins} in the output
\cite{bojanczyk14}.

Another way to approach the problem of deciding star-free for polyregular
functions is to not restrict the computational power, nor change the semantics,
but consider restrictions of the output alphabet. In particular, there have
been successful results proven on \emph{unary polyregular functions}, also
known as \kl{$\Nat$-polyregular functions}
\cite{doueneau2021pebble,doueneau2022hiding}. Let us note that results on
\emph{unary output functions} often generalize to the slightly more general
class of functions with \emph{commutative output}, and that algorithms leverage
the absence of combinatorial information in the output. Furthermore,
\kl{$\Nat$-polyregular functions} benefit from theorems proven in the field of
rational series \cite{berstel2011noncommutative}, where they coincide with
weighted automata of polynomial ambiguity, rational series of polynomial
growth, and a notion of quantitative $\MSO$
\cite{berstel2011noncommutative,kreutzer2013,schutzenberger1962}. Although
promising, the restriction to unary output did not yield effective procedure to
decide the aperiodicity of polyregular functions.
Notice that unary outputs in $\Rel$ and $\Nat$
are therefore special kind of weighted automata,
and that questions arising from the weighted automata
theory (good first order logic?) are also connected.
\cite{droste2019aperiodic,reutenauer_series_1980}.

Recently, the introduction of $\Rel$-output polyregular functions, called
\kl{$\Rel$-polyregular functions} (and more generally, commutative invertible
outputs) allowed to solve the general problem \cite{LOPEZ23b}. Note that
$\Rel$-output generalizes $\Nat$-output, but somehow has a combinatorial
description that is farther away from traditional \kl{polyregular functions}
because of the invertible outputs. The proof relies on the construction of a
canonical object essentially based on differences between functions, which is
not applicable to unary output. Note that already for rational series, the
problem is open.


We write $\Poly$ for the class of \kl{polyregular functions},
$\SF$ for the class of \kl{star-free polyregular functions},
$\NRat$ for the class of \kl{$\Nat$-rational series},
and $\ZRat$ for the class of \kl{$\Rel$-rational series}.
Arrows denote strict inclusions, and effectiveness
(both in terms of decidability and of effective representation)
is represented by thick double arrows.
Inclusions that are suspected to be effective have a question mark.
It is also conjectured that $\NPoly \cap \ZSF = \NSF$.

\begin{figure}
    \includestandalone[width=\linewidth]{tikz/class-inclusions}
    \caption{
        Decidability and inclusions of classes of functions. 
    }
\end{figure}

It was conjectured in a thesis that some particular form of polynomials
would characterize star-free functions.This paper follows the line of
research on \kl{polyregular functions} that focuses on semantic
characterizations of the asymptotic behavior, rather than the construction
of canonical objects of computation
\cite{doueneau2021pebble,doueneau2022hiding,LOPEZ23b}. The goal was to
decide arrows X,Y and Z, namely, to understand precisely the relationship
between star-free and $\Nat$ output. In the case of \kl{$\Nat$-rational
series}, a result of \citeauthor{KARH77} characterizes \kl{commutative}
\kl{$\Nat$-rational series} among \kl{$\Rel$-rational series}
\cite{KARH77}.
This conjecture was based on the previously introduced result of
\textcite[Theorem 3.3]{KARH77}, which we will disprove in \cref{sec:c-example}.
We then provide a corrected version of the theorem, and prove the conjecture of
Douéneau in the case of \emph{commutative} $\Nat$-rational series of polynomial
growth.

\hrule

\paragraph{Contributions.} 
\begin{itemize}
    \item We provide a candidate semantic characterization
    \item We prove that it is correct for commutative inputs
    \item We provide a counter-example to the theorem of nanania.
    \item We show that effective procedures are possible for all arrows 
        and answer positively to the conjecture of XX
        in the case of commutative inputs.
\end{itemize}


\paragraph*{Outline of the paper.}
\begin{itemize}
    \item In a first section, we recall the preliminary definitions.
    \item In a second section, we provide a counter example to the
        theorem of XX.
    \item In a third section, we provide an alternative result.
    \item In a fourth section, we make the results effective,
        and compute everyting.
\end{itemize}
