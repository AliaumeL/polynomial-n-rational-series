%! TeX program = xelatex
%! lang = en-US
\section{Preliminaries}
\label{preliminaries:sec}

\AP
In the rest of this paper, $\Rel$ (resp. $\Nat$) denotes the set of integers
(resp. nonnegative integers). The capital letters $\Sigma,\Gamma$ denotes fixed
alphabets, i.e. a finite set of letters, and $\Sigma^*, \Gamma^*$ (resp.
$\Sigma^+, \Gamma^+$) are the set of words (resp. non-empty words) over
$\Sigma, \Gamma$. The empty word is written $\varepsilon \in \Sigma^*$. When $w
\in \Sigma^*$ and $a \in \Sigma$, we let $\card{w} \in \Nat$ be the length of
$w$, and $\card[a]{w}$ be the number of occurrences of $a$ in $w$. We assume
that the reader is familiar with the basics of automata theory, in particular
the notions of monoid morphisms, idempotents in monoids, monadic second-order
($\MSO$) logic and first-order ($\FO$) logic over finite words (see e.g.
\cite{thomas1997languages}). 

\AP
For every language $L \subseteq \Sigma^*$ we define $\ind{L} \colon \Sigma^*
\to \set{0,1}$ as the indicator function of $L$. If $f,g \colon \Sigma^* \to
\Rel$ are two functions, we define their Cauchy product as $f \cauchy g (w)
\defined \sum_{uv = w} f(u)g(w)$. Given a set $X$ of functions from $\Sigma^*$
to $\Rel$, and a subset $\mathcal{S}$ of $\Rel$, we define
$\Span{\mathbb{S}}(X)$ as the smallest subset of functions from $\Sigma^* \to
\Rel$ containing $X$ and closed under (finite) sums and multiplication with $s
\in \mathbb{S}$.

\AP
Given an $\MSO$ formula with first-order free variables $\varphi(\vec{x})$, we
define $\vcount{\varphi(\vec{x})}$ as the function from $\Sigma^*$ to $\Nat$
that maps a word $w$ to the number of assignments $\nu \colon \vec{x} \to w$,
such that $w, \nu \models \varphi(\vec{x})$.

\AP Let $f \colon \Sigma^* \to \Rel$ be a function. We define
the \intro{growth rate} of
$f$, $\growth{f} \colon \Nat \to \Nat$, as follows $\growth{f}(n) \defined
\sup_{\card{w} = n} |f(w)|$. A function $f$ has \kl{growth rate} $k \in \Nat$
if $\growth{f} = \bigO(n^k)$, and if such a $k$ exists,
we say that $f$ has \kl{polynomial growth}.

\subsection{Polyregular Functions}

\AP All functions in this paper have output in $\Nat$ or $\Rel$, hence, we
avoid the definition of \kl{polyregular functions} in full generality. The
following \cref{nat-rel-poly:def} is one of the equivalent definitions of
\cite{LOPEZ23b}, and is similar in shape to the \emph{finite counting automata}
introduced by \textcite{schutzenberger1962}. Let us recall that a monoid $M$ is
\intro(monoid){aperiodic} whenever there exists $n \in \Nat$ such that for all
$x \in M$, $x^{n+1} = x^n$.

\begin{definition}[$\Rel$-polyregular functions {\cite[see, e.g.][]{LOPEZ23b}}]
    \label{nat-rel-poly:def}
    Let $M$ be a finite monoid, $\mu \colon \Sigma^* \to M$
    be a morphism, $k \in \Nat$, and 
    $\pi \colon M^k \to \Rel$ be a production function.
    The \intro{$\Rel$-polyregular function}
    $\pi^\dagger \colon \Sigma^* \to \Rel$
    is computed as follows:
    \begin{equation*}
        \pi^\dagger (w) \defined
        \sum_{w = u_1 \cdots u_k} \pi(\mu(u_1), \dots, \mu(u_k))
        \quad .
    \end{equation*}
    When $\pi$ has co-domain $\Nat$, the function $\pi^\dagger$
    is called \intro{$\Nat$-polyregular}.
    When the monoid $M$ is \kl(monoid){aperiodic}
    then
    the function $f$ is \intro{star-free $\Rel$-polyregular}
    (resp. \intro{star-free $\Nat$-polyregular}).
\end{definition}


\begin{example}
    The map $f \colon w \mapsto \card{w} + 1$
    belongs to $\NSF$.
\end{example}
\begin{proof}
    Let us define $M \defined (\set{1}, \times)$ which is 
    a finite \kl{aperiodic monoid}, $\mu \colon \Sigma^* \to M$
    defined by $\mu(w) \defined 1$, and
    $\pi \colon M^2 \to \Nat$
    that is the constant function equal to $1$.
    We check that for all $w \in \Sigma^*$:
    \begin{equation*}
        \pi^\dagger(w)
        =
        \sum_{uv = w} 1
        =
        \card{w} + 1
        = f(w)
        \quad . 
        \qedhere
    \end{equation*}
\end{proof}


\AP Let us briefly recall in the following lemma alternative characterizations
of $\NPoly$ (resp. $\ZPoly$), some of which will be helpful in the upcoming
analysis of their \kl{commutative} counterpart, and others that provide
a nicer syntax to construct examples.

\begin{lemma}[TODO]
    \label{polynomial-rational-polyreg:fact}
    Let $f \in \NRat$ (resp. $f \in \ZRat$), then
    the following are equivalent:
    \begin{enumerate}
        \item $f \in \NPoly$ (resp. $\ZPoly$);
        \item $f$ is a \kl{polyregular function} with output
            in $\set{1}^*$,
            postcomposed with $\polysum$
            (resp. with output in $\set{-1,+1}^*$);
        \item $f$ has \kl{polynomial growth};
        \item $f$ can be constructed
            using $+$, $\cauchy$, and multiplication by
            an integer $m \in \Nat$ (resp. $m \in \Rel$)
            starting from
            indicator functions $\ind{L}$ of regular languages;
        \item \label{npoly-counting-mso:item} $f$ belongs to
            $\Span{\mathbb{S}}(\setof{\vcount{\varphi}}{\varphi(\vec{x}) \in \MSO})$
            with $\mathbb{S} = \Nat$
            (resp. $\mathbb{S} = \Rel$).
    \end{enumerate}
\end{lemma}

Similarly, we provide several known characterizations of
$\NSF$ and $\ZSF$ in the following lemma.

\begin{lemma}[TODO]
    Let $f \in \NPoly$ (resp. $f \in \ZPoly$), then
    the following are equivalent:
    \begin{enumerate}
        \item $f \in \NSF$ (resp. $\ZSF$);
        \item $f$ is a \kl{star-free polyregular function} with output
            in $\set{1}^*$,
            postcomposed with $\polysum$
            (resp. with output in $\set{-1,+1}^*$);
        \item $f$ can be constructed
            using $+$, $\cauchy$, and multiplication by
            an integer $m \in \Nat$ (resp. $m \in \Rel$)
            starting from
            indicator functions $\ind{L}$ of star-free regular languages;
        \item $f$ belongs to
            $\Span{\mathbb{S}}(\setof{\vcount{\varphi}}{\varphi(\vec{x}) \in \FO})$
            with $\mathbb{S} = \Nat$
            (resp. $\mathbb{S} = \Rel$).
    \end{enumerate}
\end{lemma}

\begin{example}
    \label{regular-language:ex}
    Let $\Sigma$ be a finite alphabet, and
    $L \subseteq \Sigma^*$. Then,
    $L$ is a regular language if and only if
    $\ind{L} \colon \Sigma^* \to \set{0,1}$ is a
    \kl{$\Nat$-polyregular function}.
    Furthermore, $L$ is a star-free regular language
    if and only if $\ind{L}$ is a
    \kl{star-free $\Nat$-polyregular function}.
\end{example}

\textbf{This is wrong, we should talk about the number of variables,
    and say that we do not know in the case of $\NSF$ if it corresponds
    to the minimal number of variables needed!}
\AP Inside $\ZPoly$, there is an infinite hierarchy of classes, defined using
the \kl{growth rate}. Namely, one defines $\ZPoly[k]$ as the functions $f \in
\ZPoly$ such that $\growth{f}(n) = \bigO(n^k)$, i.e., have \kl{growth rate}
$k$. Similarly, we let $\NPoly[k] = \NPoly \cap \ZPoly[k]$.


\subsection{Commutative Input}

\paragraph{Commutative Polyregular Functions.}
\AP 
Because this paper focuses on functions that are commutative both in the
input and the output, let us introduce precisely what it means and how it
relates to other functions from $\Nat^k \to \Nat$. For that, let $\Sigma$ be a
finite alphabet. We define $\commute \colon \Sigma^* \to \Nat^\Sigma$ for the
map that counts occurrences of every letter in the input word, namely: $
\commute[w] \defined (a \mapsto \card[a]{w})$.

\AP Let $X$ be a set. A function $f \colon \Sigma^* \to X$ is
\intro{commutative} whenever for all $u,v \in \Sigma^*$, $f(uv) = f(vu)$.
Equivalently, it is \reintro{commutative} whenever there exists a map $g \colon
\Nat^\Sigma \to X$ such that $g \circ \commute = f$.

\begin{example}
    \label{commutative-function:ex}
    The map $f \colon w \mapsto \card[a]{w} \times \card[b]{w}$ is \kl{commutative}.
\end{example}

Let us briefly state that one can decide whether a \kl{$\Rel$-polyregular
function} $f$ is \kl{commutative}, ensuring that we are working in a decidable
subclass of functions. 

\begin{lemma}
    \label{decide-commutative:lemma}
    Let $f \in \ZPoly$. One can decide if 
    $f$
    is \kl{commutative}.
\end{lemma}
\begin{proof}
    Let $\Sigma$ be a finite alphabet, endowed with a
    total ordering over its letters,
    and let $f \colon \Sigma^* \to \Rel$ be a
    \kl{$\Rel$-polyregular function}.
    The map $\polysort \colon \Sigma^* \to \Sigma^*$
    that sorts the letters in the input word $w$ according
    to the chosen ordering is a \kl{polyregular function}
    \cite[Proposition II.12]{LOPEZ23b}.
    As a consequence,
    $(f \circ \polysort) \in \ZPoly$.
    Finally, $f$ is \kl{commutative} if and only
    if $f = f \circ \polysort$, which is decidable
    in the case of $\ZPoly$
    \cite[Corollary II.24]{LOPEZ23b}, and more generally in the case 
    of $\ZRat$ \cite[Corollary 3.6 p 38]{berstel2011noncommutative}.
\end{proof}


For the sake of completeness, let us also state that \kl{commutativity} is
decidable for \kl{$\Rel$-rational series}, although using more involved
arguments, since the latter are not closed under pre-composition by a
\kl{polyregular function}.

\begin{lemma}
    \label{decidable-commutative-rat:lemma}
    Let $f \in \ZRat$. One can decide if
    $f$ is \kl{commutative}.
\end{lemma}


Let us briefly remark that the characterizations of subclasses of
\kl{polyregular functions} that are obtained in practice can be obtained by
pre-composition with a \kl{commutative} \kl{polyregular function}. While the
two following
\cref{pre-compose-growth-commut:lemma,pre-compose-sf-commut:lemma} do not
provide simpler proofs, they suggest the central position of \kl{commutative}
\kl{star-free polyregular functions} as \emph{test functions} for various
properties (growth rate, aperiodicity).

\begin{lemma}
    \label{pre-compose-growth-commut:lemma}
    Let $f \in \ZRat$, and $k \in \Nat$,
    the following are equivalent:
    \begin{enumerate}
        \item $f \in \ZPoly[k]$,
        \item For every \kl{commutative} \kl{star-free polyregular function} $h$
            of \kl{growth rate} $l \in \Nat$,
            $(f \circ h) \in \ZPoly[k+l-1]$.
    \end{enumerate}
\end{lemma}


\begin{lemma}
    \label{pre-compose-sf-commut:lemma}
    Let $f \in \ZPoly$.
    The following are equivalent:
    \begin{enumerate}
        \item $f \in \ZSF$,
        \item For every \kl{commutative} \kl{star-free polyregular function} $h$,
            $(f \circ h) \in \ZSF$.
    \end{enumerate}
\end{lemma}

\AP A function $g \colon \Nat^n \to \Rel$ is \intro{represented} by a function
$f \colon \Sigma^* \to \Rel$ if there exists a map $\eta \colon \set{1, \dots,
n} \to \Sigma$ such that $g \circ (v \mapsto \seqof{v_{\eta(k)}}{1 \leq k \leq
n}) \circ \commute = f$. A key family of functions will be \kl{represented} by
\kl{$\Rel$-polyregular functions}, namely polynomials with multiple indeterminates.

\paragraph*{Polynomials.} \AP All polynomials considered in this paper have
coefficients in $\Rel$ unless explicitly stated otherwise. A polynomial $P \in
\Rel[X_1, \dots, X_n]$ is a \intro{$\Nat$-rational polynomial} if it is
\kl{represented} by a \kl{$\Nat$-polyregular function}. The analogue notion of
\intro{$\Rel$-rational polynomials} is not of particular interest, since every
polynomial $P$ is \kl{represented} by a \kl{$\Rel$-polyregular function}.

\begin{example}
    \label{negative-not-nrat:ex}
    The polynomials $X$, and $X^2 + 3$ are \kl{$\Nat$-rational polynomials},
    but $- X$ is a \kl{$\Rel$-rational polynomial} that is 
    not a \kl{$\Nat$-rational polynomial}.
\end{example}
\begin{proof}
    The function $w \mapsto |w|$ is a \kl{$\Nat$-polyregular function}.
    Thus, 
    $P(X) \defined X$ is
    a \kl{$\Nat$-rational polynomial}. Similarly,
    $w \mapsto |w|^2 + 3$ is a \kl{$\Nat$-polyregular function},
    showing that $Q(X) \defined X^2 + 3$
    is a \kl{$\Nat$-rational polynomial}.
    Finally, 
    $T(X) \defined - X$ cannot be 
    a \kl{$\Nat$-rational polynomial} as \kl{$\Nat$-polyregular functions}
    are non-negative.
\end{proof}

\AP A polynomial $P \in \Rel[X_1, \dots, X_k]$ is \intro{non-negative} when for
all $n_1, \dots, n_k \geq 0$, $P(n_1, \dots, n_k) \geq 0$. Beware that we do
not consider negative values as input, as the numbers $n_i$ will ultimately
count the number of occurrences of a letter in a word. As an example, the
polynomial $(X - Y)^2$ is \kl{non-negative}, and so is the polynomial $X^3$,
but the polynomial $X^2 - 2X$ is not.

All \kl{$\Nat$-rational polynomials} are \kl{non-negative}, but the converse does
not hold,
as the following example illustrates. Note that the proof scheme of
this example will be at the core of \cref{thm:counter-example}, and leverages
a key property of $\NRat$ recalled hereafter.

\begin{fact}
    \label{pre-image-regular:fact}
    The pre-image of a regular language by a \kl{$\Nat$-rational series}
    is a regular language. 
\end{fact}

\begin{example}
    Let $P(X, Y) \defined (X - Y)^2$.
    Then $P$ is
    is \kl{non-negative}, but is
    not a \kl{$\Nat$-rational polynomial}.
\end{example}
\begin{proof}
    Assume by contradiction that
    $f \in \NPoly$ \kl{represents} $f$ over the alphabet $\Sigma \defined \set{a,b}$.
    Then, $f^{-1}(\set{0})$ is a regular language
    (\cref{pre-image-regular:fact}),
    but $S^{-1}(\set{0}) = \setof{ w \in \Sigma }{ \card[a]{w} = \card[b]{w} }$
    is not.
\end{proof}


\AP Let us now give some vocabulary on polynomials with multiple indeterminate
over $\Rel$. A \intro{monomial} is a product of indeterminates and integers.
For instance, $XY$ is a \kl{monomial}, $3 X$ is a \kl{monomial}, $-Y$ is a
\kl{monomial}, but $X + Y$ or $2X^2 + XY$ are not. We write $\Monomials[X_1,
\dots, X_n]$ for the set of \kl{monomials} over these indeterminates.
Every polynomial $P \in \Rel[X_1, \dots, X_n]$ decomposes uniquely
into a sum of \kl{monomials}.

\AP A \kl{monomial} $S$ \intro{divides} a \kl{monomial} $T$, when $S$ divides
$T$ seen as polynomials in $\Rat$. For instance, $2X$ \kl{divides} $XY$, $-YZ$
\kl{divides} $X^2 Y Z^3$, and $Y$ does not \kl{divide} $X$. In the
decomposition of $P \in \Rel[X_1, \dots, X_k]$, a \kl{monomial} is a
\intro{maximal monomial} if it is a maximal element for the \kl{divisibility
ordering} of \kl{monomials}. In the polynomial $P(X,Y) \defined X^2 - 2XY + X^2
+ X + Y$, the set of \reintro{maximal monomials}, written
$\MaximalMonomials(P)$, is $\set{X^2,  -2 XY,  X^2}$. A \kl{monomial} has a
\intro{non-negative coefficient} if its multiplicative constant belongs to
$\Nat$, or equivalently, if it is \kl{non-negative} as a polynomial. In the
polynomial $P(X,Y) \defined (X - Y)^2$, the \kl{non-negative} \kl{monomials}
are $X^2$ and $Y^2$.
