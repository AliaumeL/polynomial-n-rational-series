%! TeX program = xelatex
%! lang = en-US
\section{Preliminaries}
\label{preliminaries:sec}

\AP In the rest of this paper, $\Rel$ (resp. $\Nat$) denotes the set of
integers (resp. non-negative integers). The capital letters $\Sigma,\Gamma$
denotes fixed alphabets, i.e. a finite set of letters, and $\Sigma^*, \Gamma^*$
(resp. $\Sigma^+, \Gamma^+$) are the set of words (resp. non-empty words) over
$\Sigma, \Gamma$. The empty word is written $\varepsilon \in \Sigma^*$. When $w
\in \Sigma^*$ and $a \in \Sigma$, we let $\card{w} \in \Nat$ be the length of
$w$, and $\card[a]{w}$ be the number of occurrences of $a$ in $w$. We assume
that the reader is familiar with the basics of automata theory, in particular
the notions of monoid morphisms, idempotents in monoids, monadic second-order
($\MSO$) logic and first-order ($\FO$) logic over finite words (see e.g.
\cite{THOM97}). 

\AP In order to build functions in $\NPoly$, the following constructions will
be useful. For every language $L \subseteq \Sigma^*$ we define $\ind{L} \colon
\Sigma^* \to \set{0,1}$ as the indicator function of $L$. Given an $\MSO$
(resp. $\FO$) formula with first-order free variables $\varphi(\vec{x})$, we
define $\vcount{\varphi(\vec{x})}$ as the function from $\Sigma^*$ to $\Nat$
that maps a word $w$ to the number of assignments $\nu \colon \vec{x} \to w$,
such that $w, \nu \models \varphi(\vec{x})$. If $X$ is a set of functions from
$\Sigma^*$ to $\Rel$, and a subset $\mathbb{S}$ of $\Rel$, we define
$\intro*\Span{\mathbb{S}}(X)$ as the smallest subset of functions from
$\Sigma^* \to \Rel$ containing $X$ and closed under (finite) sums and
multiplication with $s \in \mathbb{S}$.

\subsection{Polyregular Functions}

\AP
% How the problem has been approached so far
To construct subclasses of
\kl{polyregular function}, an interesting parameter is the \intro{growth rate}
of a function $f$, which is given by a number $k \in \Nat$, which can be used
to create a strict hierarchy inside $\Poly$. All functions in $\Poly$ have
\intro{polynomial growth}, and we can write $\Poly[k]$ for the \kl{polyregular
functions} of \kl{growth rate} $k$. It has been shown that the parameter $k$
controls the maximal complexity of the $\MSO$ interpretation needed to
represent the function \cite{BOJA22}.

\AP Because the functions of interest in this paper have output in $\Nat$ or
$\Rel$, we will only provide the definition of \kl{polyregular functions} for
these two output semigroups, and we refer the reader to \cite{BOKL19} for the
general definition of \kl{polyregular functions}. The following
\cref{nat-rel-poly:def} is one of the equivalent definitions of \cite{CDTL23},
and is similar in shape to the \emph{finite counting automata} introduced by
\cite{SCHU62}. Let us recall that a monoid $M$ is \intro(monoid){aperiodic}
whenever for all $x \in M$, there exists $n \in \Nat$ such that $x^{n+1} =
x^n$.

\begin{definition}[$\Rel$-polyregular functions {\cite{CDTL23}}]
    \label{nat-rel-poly:def}
    Let $M$ be a finite monoid, $\mu \colon \Sigma^* \to M$
    be a morphism, $k \in \Nat$, and 
    $\pi \colon M^k \to \Rel$ be a production function.
    The \intro{$\Rel$-polyregular function}
    $\pi^\dagger \colon \Sigma^* \to \Rel$
    is computed as follows:
    \begin{equation*}
        \pi^\dagger (w) \defined
        \sum_{w = u_1 \cdots u_k} \pi(\mu(u_1), \dots, \mu(u_k))
        \quad .
    \end{equation*}
    When $\pi$ has co-domain $\Nat$, the function $\pi^\dagger$
    is called \intro{$\Nat$-polyregular}.
    When the monoid $M$ is \kl(monoid){aperiodic}
    then
    the function $f$ is \intro{star-free $\Rel$-polyregular}
    (resp. \intro{star-free $\Nat$-polyregular}).
\end{definition}


\begin{example}
    \label{size-of-word-nsf:ex}
    The map $f \colon w \mapsto \card{w} + 1$
    belongs to $\NSF$.
    \proofref{size-of-word-nsf:ex}
\end{example}


\AP Let us briefly recall in the following lemma alternative characterizations
of $\NPoly$ (resp. $\ZPoly$), some of which will be helpful in the upcoming
analysis of their \kl{commutative} counterpart, and others that provide a nicer
syntax to construct examples. These characterizations were previously known,
and can be found for instance in \cite{DOUE23,CDTL23}.

\begin{lemma}
    \label{polynomial-rational-polyreg:fact}
    Let $f \in \NRat$ (resp. $f \in \ZRat$), then
    the following are equivalent:
    \begin{enumerate}
        \item $f \in \NPoly$ (resp. $\ZPoly$);
        \item $f$ is a \kl{polyregular function} with output
            in $\set{1}^*$,
            post-composed with $\polysum$
            (resp. with output in $\set{-1,+1}^*$);
        \item $f$ has \kl{polynomial growth};
        \item \label{npoly-counting-mso:item} $f$ belongs to
            $\Span{\mathbb{S}}(\setof{\vcount{\varphi}}{\varphi(\vec{x}) \in \MSO})$
            with $\mathbb{S} = \Nat$
            (resp. $\mathbb{S} = \Rel$).
    \end{enumerate}
\end{lemma}

Similar characterizations can be obtained for $\NSF$ and $\ZSF$, which correctly
generalizes aperiodicity to functions as \cref{regular-language:ex} illustrates.

\begin{lemma}
    Let $f \in \NPoly$ (resp. $f \in \ZPoly$), then
    the following are equivalent:
    \begin{enumerate}
        \item $f \in \NSF$ (resp. $\ZSF$);
        \item $f$ is a \kl{star-free polyregular function} with output
            in $\set{1}^*$,
            post-composed with $\polysum$
            (resp. with output in $\set{-1,+1}^*$);
        \item $f$ belongs to
            $\Span{\mathbb{S}}(\setof{\vcount{\varphi}}{\varphi(\vec{x}) \in \FO})$
            with $\mathbb{S} = \Nat$
            (resp. $\mathbb{S} = \Rel$).
    \end{enumerate}
\end{lemma}

\begin{example}
    \label{regular-language:ex}
    Let $\Sigma$ be a finite alphabet, and
    $L \subseteq \Sigma^*$. Then,
    $L$ is a regular language if and only if
    $\ind{L} \colon \Sigma^* \to \set{0,1}$ is a
    \kl{$\Nat$-polyregular function}.
    Furthermore, $L$ is a star-free regular language
    if and only if $\ind{L}$ is a
    \kl{star-free $\Nat$-polyregular function}.
\end{example}


\AP Inside $\ZPoly$ there is an infinite hierarchy of classes, defined by
limiting the maximal number of free variables used to construct the function as
a sum of counting formulas (\cref{npoly-counting-mso:item}). We write
$\ZPoly[k]$ (resp. $\NPoly[k]$, $\ZSF[k]$, $\NSF[k]$) for the functions in
$\ZPoly$ (resp. $\NPoly$, $\ZSF$, $\NSF$) that can be constructed using at most
$k$ free variables. It has been proven in
\cite{CDTL23,BOJA22} that the classes $\ZPoly[k]$ (resp.
$\NPoly[k]$, $\ZSF[k]$, $\NSF[k]$) are characterized by the \kl{growth rate},
that is, $f \in \ZPoly[k]$ if and only if $f \in \ZPoly$ and $\card{f(w)} =
\bigO\left(\card{w}^k\right)$, and similarly for the other classes.

\subsection{Commutative Input}

\AP 
Let $\Sigma$ be a finite alphabet. 
We define $\commute \colon \Sigma^* \to \Nat^\Sigma$ for the
map that counts occurrences of every letter in the input word (that is,
computes the Parikh vector) namely: $ \commute[w] \defined (a \mapsto
\card[a]{w})$. Let $X$ be a set, a function $f \colon \Sigma^* \to X$ is
\intro{commutative} whenever for all $u,v \in \Sigma^*$, $f(uv) = f(vu)$.
Equivalently, it is \reintro{commutative} whenever there exists a map $g \colon
\Nat^\Sigma \to X$ such that $g \circ \commute = f$.

To ensure that we are working inside a decidable subclass of \kl{$\Rel$-rational series}
let us state that commutativity is decidable. The proof is straightforward in the case
of \kl{$\Rel$-polyregular functions}, and more involved for general \kl{$\Rel$-rational series}.

\begin{lemma}
    \label{decidable-commutative-poly:lemma}
    \label{decidable-commutative-rat:lemma}
    Let $f \in \ZRat$. One can decide if 
    $f$
    is \kl{commutative}.
\end{lemma}


\AP A function $g \colon \Nat^n \to \Rel$ is \intro{represented} by a function
$f \colon \Sigma^* \to \Rel$ if there exists a map $\eta \colon \set{1, \dots,
n} \to \Sigma$ such that $g \circ (v \mapsto \seqof{v_{\eta(k)}}{1 \leq k \leq
n}) \circ \commute = f$. This notion will be useful to formally state that a
polynomial ``is'' a \kl{polyregular function}.

\subsection{Polynomials} \AP All polynomials considered in this paper have
coefficients in $\Rel$ unless explicitly stated otherwise. A polynomial $P \in
\Rel[X_1, \dots, X_n]$ is an \intro{$\Nat$-rational polynomial} if it is
\kl{represented} by a \kl{$\Nat$-polyregular function}. The analogue notion of
\intro{$\Rel$-rational polynomials} is not of particular interest, since every
polynomial $P$ is \kl{represented} by a \kl{$\Rel$-polyregular function}.

\begin{example}
    \label{negative-not-nrat:ex}
    The polynomials $X$, and $X^2 + 3$ are \kl{$\Nat$-rational polynomials},
    but $- X$ is a \kl{$\Rel$-rational polynomial} that is 
    not an \kl{$\Nat$-rational polynomial}.
\end{example}

\AP A polynomial $P \in \Rel[X_1, \dots, X_k]$ is \intro{non-negative} when for
all non-negative integer inputs $n_1, \dots, n_k \geq 0$, the output  $P(n_1,
\dots, n_k)$ of the polynomial is non-negative. Beware that we do not consider
negative values as input, as the numbers $n_i$ will ultimately count the number
of occurrences of a letter in a word. As an example, the polynomial $(X - Y)^2$
is \kl{non-negative}, and so is the polynomial $X^3$, but the polynomial $X^2 -
2X$ is not.

All \kl{$\Nat$-rational polynomials} are \kl{non-negative}, but the converse
does not hold, as the following example illustrates. Note that the proof scheme
of this example will be at the core of the counter-example to \cite[Theoerm
3.3]{KARH77} in \cref{thm:counter-example}, and leverages a key property of
$\NRat$ recalled hereafter.

\begin{fact}
    \label{pre-image-regular:fact}
    The pre-image of a regular language by an \kl{$\Nat$-rational series}
    is a regular language. 
\end{fact}

\begin{example}
    Let $P(X, Y) \defined (X - Y)^2$.
    Then $P$ is
    is \kl{non-negative}, but is
    not an \kl{$\Nat$-rational polynomial}.
\end{example}
\begin{proof}
    Assume by contradiction that
    $f \in \NPoly$ \kl{represents} $f$ over the alphabet $\Sigma \defined \set{a,b}$.
    Then, $f^{-1}(\set{0})$ is a regular language
    (\cref{pre-image-regular:fact}),
    but $S^{-1}(\set{0}) = \setof{ w \in \Sigma }{ \card[a]{w} = \card[b]{w} }$
    is not.
\end{proof}


\AP Let us now give some vocabulary on polynomials with multiple indeterminate
over $\Rel$. A \intro{monomial} is a product of indeterminates and integers.
For instance, $XY$ is a \kl{monomial}, $3 X$ is a \kl{monomial}, $-Y$ is a
\kl{monomial}, but $X + Y$ or $2X^2 + XY$ are not. We write $\intro*\Monomials[X_1,
\dots, X_n]$ for the set of \kl{monomials} over these indeterminates.
Every polynomial $P \in \Rel[X_1, \dots, X_n]$ decomposes uniquely
into a sum of \kl{monomials}.

\AP A \kl{monomial} $S$ \intro{divides} a \kl{monomial} $T$, when $S$ divides
$T$ seen as polynomials in $\Rat$. For instance, $2X$ \kl{divides} $XY$, $-YZ$
\kl{divides} $X^2 Y Z^3$, and $Y$ does not \kl{divide} $X$. In the
decomposition of $P \in \Rel[X_1, \dots, X_k]$, a \kl{monomial} is a
\intro{maximal monomial} if it is a maximal element for the \kl{divisibility
ordering} of \kl{monomials}. In the polynomial $P(X,Y) \defined X^2 - 2XY + X^2
+ X + Y$, the set of \reintro{maximal monomials}, written
$\intro*\MaximalMonomials(P)$, is $\set{X^2,  -2 XY,  X^2}$.  In the polynomial
$P(X,Y) \defined (X - Y)^2$, the \kl{non-negative} \kl{monomials} are $X^2$ and
$Y^2$.
