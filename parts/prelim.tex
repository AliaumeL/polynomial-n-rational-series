%! TeX program = xelatex
%! lang = en-US
\section{Preliminaries}
\label{preliminaries:sec}

\AP The capital letters $\Sigma,\Gamma$ denote
fixed alphabets, i.e. finite set of letters, and $\Sigma^*, \Gamma^*$ (resp.
$\Sigma^+, \Gamma^+$) are the set of words (resp. non-empty words) over
$\Sigma, \Gamma$. The empty word is written $\varepsilon \in \Sigma^*$. When $w
\in \Sigma^*$ and $a \in \Sigma$, we let $\card{w} \in \Nat$ be the length of
$w$, and $\card[a]{w}$ be the number of occurrences of $a$ in $w$. 

\AP We assume that the reader is familiar with the basics of automata theory,
in particular the notions of monoid morphisms, idempotents in monoids, monadic
second-order ($\intro*\MSO$) logic and first-order ($\intro*\FO$) logic over finite words
(see e.g. \cite{THOM97}). As aperiodicity will be a central notion of this
paper, let us recall that a monoid $M$ is \intro(monoid){aperiodic} whenever
for all $x \in M$, there exists $n \in \Nat$ such that $x^{n+1} = x^n$. If the
monoid $M$ is finite, this $n$ can be uniformly chosen for all elements in $M$.


\AP We use the notation $\commute \colon \Sigma^* \to \Nat^\Sigma$ for the map
that counts occurrences of every letter in the input word (that is, computes
the Parikh vector) namely: $ \commute[w] \defined \seqof{a \mapsto
\card[a]{w}}{a \in \Sigma}$. Given a set $X$, a function $f \colon \Sigma^* \to
X$ is \intro{commutative} whenever for all $u \in \Sigma^*$, for all
permutations $\sigma$ of $\set{1, \dots, \card{w}}$, $f(\sigma(u)) = f(u)$.
Equivalently, it is \reintro{commutative} whenever there exists a map $g \colon
\Nat^\Sigma \to X$ such that $g \circ \commute = f$.

\AP Let $k \in \Nat$, and let $\Sigma$ be a finite alphabet. Given a function
$\eta \colon \set{1, \dots, k} \to \Sigma$, we define the $\eta^\dagger \colon
\Nat^k \to \Sigma^*$ as $\eta^\dagger(\vec{x}) \defined \eta(1)^{x_1} \dots
\eta(k)^{x_k}$. A function $f \colon \Nat^k \to X$ is \intro{represented} by a
\kl{commutative} function $g \colon \Sigma^* \to X$ if there exists a map $\eta
\colon \set{1, \dots, k} \to \Sigma$ such that $g \circ \eta^\dagger = f$. This
notion will be useful to formally state that a polynomial ``is'' a
\kl{commutative} \kl{polyregular function}. For instance, the polynomial
function $P(X,Y) = X \times Y$ is \kl{represented} by the \kl{commutative}
function $g \colon \set{a,b}^* \to \Rel$ defined by $g(w) \defined \card[a]{w}
\times \card[b]{w}$.

\subsection{Polynomials} \AP A polynomial $P \in \Rel[X_1, \dots, X_k]$ is
\intro{non-negative} when for all non-negative integer inputs $n_1, \dots, n_k
\geq 0$, the output  $P(n_1, \dots, n_k)$ of the polynomial is non-negative. In
the case of at most three indeterminates, we use variables $X,Y,Z$ instead of
$X_1, X_2, X_3$ to lighten the notation. Beware that we do not consider
negative values as input, as the numbers $n_i$ will ultimately count the number
of occurrences of a letter in a word. As an example, the polynomial $(X - Y)^2$
is \kl{non-negative}, and so is the polynomial $X^3$, but the polynomial $X^2 -
2X$ is not.

\AP A \intro{monomial} is a product of indeterminates and integers. For
instance, $XY$ is a \kl{monomial}, $3 X$ is a \kl{monomial}, $-Y$ is a
\kl{monomial}, but $X + Y$ and $2X^2 + XY$ are not. Every polynomial $P \in
\Rel[X_1, \dots, X_n]$ decomposes uniquely into a sum of \kl{monomials}. A
\kl{monomial} $S$ \intro{divides} a \kl{monomial} $T$, when $S$ divides $T$
seen as polynomials in $\Rat$. For instance, $2X$ \kl{divides} $XY$, $-YZ$
\kl{divides} $X^2 Y Z^3$, and $Y$ does not \kl{divide} $X$. In the
decomposition of $P \in \Rel[X_1, \dots, X_k]$, a \kl{monomial} is a
\intro{maximal monomial} if it is a maximal element for the \kl{divisibility
preordering} of \kl{monomials}. In the polynomial $P(X,Y) \defined X^2 - 2XY + Y^2
+ X + Y$, the set of \reintro{maximal monomials} is $\set{X^2,  -2 XY,  Y^2}$.
In the polynomial $P(X,Y) \defined (X - Y)^2$, the \kl{non-negative}
\kl{monomials} are $X^2$ and $Y^2$.

\subsection{Polyregular Functions}
\label{polyregular:sec}

\AP Because the functions of interest in this paper have output in $\Nat$ or
$\Rel$, we will only provide the definition of \intro{polyregular functions}
for these two output semigroups, and we refer the reader to \cite{BOKL19} for
the general definition of \kl{polyregular functions} and their aperiodic
counterpart, the \intro{star-free polyregular functions}. We chose in this
paper to provide a combinatorial description of polyregular functions with
commutative outputs because it will play nicely with our analysis on
polynomials. This description is very similar in shape to the \emph{finite
counting automata} introduced by \cite{SCHU62}. 

\begin{definition}[$\Rel$-polyregular functions {\cite{CDTL23}}]
    \label{nat-rel-poly:def}
    Let $d \in \Nat$. The set $\intro*\ZPoly[d]$ of polyregular
    \intro{$\Rel$-polyregular functions} of degree at most $d$,
    is the set of functions $f \colon \Sigma^* \to \Rel$
    such that
    there exists a finite monoid $M$,
    a morphism $\mu \colon \Sigma^* \to M$,
    and a function $\pi \colon M^{d+1} \to \Rel$
    satisfying for all $w \in \Sigma^*$:
    \begin{equation*}
        f (w) = \pi^\dagger (w) \defined
        \sum_{w = u_1 \cdots u_{d+1}} \pi(\mu(u_1), \dots, \mu(u_{d+1}))
        \quad .
    \end{equation*}
    We call $\pi$ the \intro{production function} of $f$.
    If the function $\pi$ has codomain $\Nat$,
    then $f$ is \intro{$\Nat$-polyregular} of degree at most $d$,
    i.e., $f \in \intro*\NPoly[d]$.
    If the monoid $M$ is \kl(monoid){aperiodic}
    then
    the function $f$ is \intro{star-free $\Rel$-polyregular}
    ($\intro*\ZSF[d]$), resp. \intro{star-free $\Nat$-polyregular} ($\intro*\NSF[d]$).
\end{definition}

\AP We complete \cref{nat-rel-poly:def} by letting $\reintro*\NPoly \defined
\bigcup_{d \in \Nat} \NPoly[d]$, and similarly for $\ZPoly$, $\NSF$, and
$\ZSF$. In order to illustrate these definitions, let us provide an example of
an \kl{$\Nat$-polyregular function} computed using a finite monoid in
\cref{size-of-word-nsf:ex}. Let us also introduce in
\cref{q-polynomial-n-poly:ex} a function that serves as an example of
\emph{division} computed by a \kl{$\Nat$-polyregular function}.

\begin{example}
    \label{size-of-word-nsf:ex}
    The map $f \colon w \mapsto \card{w} + 1$
    belongs to $\NSF[1]$.
\end{example}
\begin{proof}
    Let us define $M \defined (\set{1}, \times)$ which is 
    a finite \kl{aperiodic monoid}, $\mu \colon \Sigma^* \to M$
    defined by $\mu(w) \defined 1$, and
    $\pi \colon M^2 \to \Nat$
    that is the constant function equal to $1$.
    We check that for all $w \in \Sigma^*$:
    $
        \pi^\dagger(w)
        =
        \sum_{uv = w} 1
        =
        \card{w} + 1
        = f(w)
        $.
\end{proof}

\begin{example}
    \label{q-polynomial-n-poly:ex}
    Let $f \colon \Sigma^* \to \Nat$ that maps a word $w$
    to the number of distinct pairs of positions in $w$,
    i.e., $f(w) = \binom{\card{w}}{2} = \card{w}(\card{w} - 1) / 2$.
    Then, $f \in \NSF[2]$.
\end{example}
\begin{proof}
    Let us remark that the set $P_w$ of distinct pairs of positions
    $i < j$ in a word $w$ is in bijection with the set $D_w$ of
    decompositons of the form $w = xyz$,
    where $x$ and $y$ are non-empty, 
    via the map $(i,j) \mapsto (w_{1,i}, w_{i+1,j}, w_{j+1,\card{w}})$.
    Let us write $M \defined (\set{0,1}, \max)$
    which is a finite aperiodic monoid, and $\mu \colon \Sigma^* \to M$
    that maps the empty word $\varepsilon$ to $0$ and the other words to $1$.
    Then, let us define $\pi \colon M^3 \to \Nat$
    via $\pi(x,y,z) = x \times y$.
    We conclude because:
    \begin{align*}
        \pi^\dagger(w) 
        &\defined
        \sum_{xyz = w} \pi(\mu(x), \mu(y), \mu(z)) 
        = 
        \sum_{xyz = w \wedge x \neq \varepsilon \wedge y \neq \varepsilon} 1
        = 
        \card{D_w}
        =
        \card{P_w}
        = f(w) \quad .
        \qedhere
    \end{align*}
\end{proof}

\AP One of the appeals of $\NPoly$ and $\ZPoly$ are the numerous
characterizations of these classes in terms of logic, weighted automata, and
the larger class of \kl{polyregular functions} \cite{CDTL23,DOUE23}. In this
paper, the main focus will be the connection to \kl{weighted automata}, which is
based on the notion of \emph{growth rate}. The \intro{growth rate} of a
function $f \colon \Sigma^* \to \Rel$ is defined as the minimal $d$ such that
$\card{f(w)} = \bigO\left(\card{w}^d\right)$. If such a $d$ exists, we say that
the function $f$ has \intro{polynomial growth}. It turns out that for all $k
\in \Nat$, $\ZPoly[d]$ (resp. $\NPoly[d]$) are precisely functions in $\ZRat$
(resp. in $\NRat)$ that have growth rate at most $d$.

\begin{lemma}[{\cite[Theorem 5.22]{DOUE23}}]
    \label{polyregular-polynomial-growth:lemma}
    Let $f \in \ZRat$. The following are equivalent:
    \begin{enumerate}
        \item $f \in \ZPoly[d]$.
        \item $f$ has \kl{polynomial growth} of degree at most $d$.
    \end{enumerate}
    And similarly for $\NPoly$ and $\NRat$.
\end{lemma}

Let us introduce some compositional properties of
\kl{$\Rel$-polyregular functions} that will be used in this paper
to construct \kl{$\Rel$-polyregular functions}.
\begin{lemma}[{\cite[Theorem II.20]{CDTL23}}]
    \label{stability-polyregular:lemma}
    Let $d \geq 1$,
    $f,g \in \NPoly[d]$ (resp. $\ZPoly[d]$, $\NSF[d]$, $\ZSF[d]$),
    $L$ be a star-free language over $\Sigma^*$,
    and $h \colon \Sigma^* \to \Gamma^*$ be a \kl{polyregular function}
    (resp. a \kl{star-free polyregular function}).
    Then, the following 
    are also in $\NPoly[d]$ (resp. $\ZPoly[d]$,
    $\NSF[d]$, $\ZSF[d]$):
    $f \circ h$,
    $f + g \defined w \mapsto f(w) + g(w)$,
    $f \times g \defined w \mapsto f(w) \times g(w)$,
    $\ind{L} \times f$.
    Furthermore, the above constructions preserve \kl{commutativity}.
\end{lemma}

Let us briefly state that \kl{commutativity} is a decidable property of
\kl{$\Rel$-rational series}, hence of \kl{$\Rel$-polyregular functions}. As a
consequence, we are working inside a relatively robust and decidable subclass of
\kl{$\Rel$-rational series}.

\begin{lemma}
    \label{decidable-commutative-poly:lemma}
    \label{decidable-commutative-rat:lemma}
    Let $f \in \ZRat$. One can decide if 
    $f$
    is \kl{commutative}.
\end{lemma}
\begin{proof}
    Remark that the group of permutations of $\set{1, \dots, n}$ is generated by
    the cycle $c \defined (n,1, \dots, n-1)$ and the transposition $t \defined (1, 2)$.
    As a consequence, a function $f$ is commutative if and only if
    $f \circ c = f = f \circ t$.
    When $f$ is a \kl{rational series},
    $f \circ c$ and $f \circ t$ are both \kl{rational series} that can be
    effectively computed from $f$,\footnote{
        This can be done by guessing the second (resp. last) letter of the input word, 
        remembering the first letter in a state, and 
        then running the original automaton for $f$ on the modified input, checking at the second position 
        (resp. the end of the word)
        if the guess was correct.
    } and since equivalence
    of rational series is decidable 
    \cite[Corollary 3.6]{BERE10},
    we have obtained a decision procedure.
\end{proof}

