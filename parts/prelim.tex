%! TeX program = xelatex
%! lang = en-US
\section{Preliminaries}
\label{preliminaries:sec}

\subsection{Polyregular Functions}

All functions in this paper have output in $\Nat$ or $\Rel$,
hence, we avoid the definition of \kl{polyregular functions}
in full generality. The following \cref{nat-rel-poly:def}
is one of the equivalent definitions of \cite{LOPEZ23b},
and is similar in shape to the \emph{finite counting automata}
introduced by \textcite{schutzenberger1962}.

\begin{definition}[$\Rel$-polyregular functions {\cite[see, e.g.][]{LOPEZ23b}}]
    \label{nat-rel-poly:def}
    Let $M$ be a finite monoid, $\mu \colon \Sigma^* \to M$
    be a morphism, $k \in \Nat$, and 
    $\pi \colon M^k \to \Rel$ be a production function.
    The \intro{$\Rel$-polyregular function}
    $\pi^\dagger \colon \Sigma^* \to \Rel$
    is computed as follows:
    \begin{equation*}
        \pi^\dagger (w) \defined
        \sum_{w = u_1 \cdots u_k} \pi(\mu(u_1), \dots, \mu(u_k))
        \quad .
    \end{equation*}
    When $\pi$ has co-domain $\Nat$, the function $\pi^\dagger$
    is called \intro{$\Nat$-polyregular}.
    When the monoid $M$ is \kl{aperiodic}
    then
    the function $f$ is \intro{star-free $\Rel$-polyregular}
    (resp. \intro{star-free $\Nat$-polyregular}).
\end{definition}

Remark II.21. It follows from Remark II.6, [11, Proposition 6.1], and Theorem
II.20, that Z-rational series of polynomial growth are exactly those computable
by weigthed automata with coefficients in {0,1,−1} of polynomial ambiguity. We
are not aware of a direct proof of this correspondence.


Let us illustrate on an example that these notions actually generalise
blabla.

\begin{example}
    \label{regular-language:ex}
    Let $\Sigma$ be a finite alphabet, and
    $L \subseteq \Sigma^*$. Then,
    $L$ is a regular language if and only if
    $\ind{L} \colon \Sigma^* \to \set{0,1}$ is a
    \kl{$\Nat$-polyregular function}.
    Furthermore, $L$ is a star-free regular language
    if and only if $\ind{L}$ is a
    \kl{star-free $\Nat$-polyregular function}.
\end{example}


\AP Notice that for a \kl{$\Rel$-polyregular function} $f \colon \Sigma^* \to
\Rel$, the \intro{growth rate} of $f$, defined as the map $\growth{f} \colon
\Nat \to \Nat$ via $\growth{f}(n) \defined \sup_{\card{w} = n} |f(w)|$, is
bounded by a polynomial in $n$. This property of having a \intro{polynomial
growth} actually characterizes $\ZPoly$ (resp. $\NPoly$) inside $\ZRat$ (resp.
$\NRat$). We formally state this property in \cref{polynomial-rational-polyreg:fact}.

\begin{fact}[{\cite[Exercise 1.2 p 169]{berstel2011noncommutative}}]
    \label{polynomial-rational-polyreg:fact}
    Let $\Sigma$ be a finite alphabet, and $f \colon \Sigma^* \to \Nat$
    be a function. The following are equivalent:
    \begin{enumerate}
        \item $f$ is a \kl{$\Nat$-rational series}
            of \kl{polynomial growth}.
        \item $f$ is a \kl{$\Nat$-polyregular function}.
    \end{enumerate}
\end{fact}


\AP Inside $\ZPoly$, there is an infinite hierarchy of classes, defined using
the \kl{growth rate}. Namely, one defines $\ZPoly[k]$ as the functions $f \in
\ZPoly$ such that $\growth{f}(n) = \bigO(n^k)$. Similarly, we let $\NPoly[k] =
\NPoly \cap \ZPoly[k]$. It follows from the results of \cite{LOPEZ23b} that
this hierarchy is nice and for every $f$, there exists $k$ such that
$\growth{f}(n) = \Theta(n^k)$. This parameter also coincides with the minimal
$k$ such that $f$ is represented by a morphism $\mu \colon M^k \to \Rel$.


\begin{fact}[Folklore about regular languages]
    \label{regular:fact}
    The language $\setof{ w \in \Sigma}{ \card[a]{w} = \card[b]{w}}$
    is not regular, whenever $a,b \in \Sigma$.
    Regular languages are closed under intersection.
\end{fact}

\begin{fact}
    \label{pre-image-regular:fact}
    The pre-image of a regular language by a $\Nat$-rational series
    is a regular language. In particular,
    for all $\Nat$-rational series $S$, $S^{-1}(\set{0})$ is a regular
    language.
\end{fact}


\subsection{Commutative Input}


\AP 
\paragraph{Commutative Polyregular Functions.}
Because this paper focuses on functions that are commutative both in the
input and the output, let us introduce precisely what it means and how it
relates to other functions from $\Nat^k \to \Nat$. For that, let $\Sigma$ be a
finite alphabet. We define $\commute \colon \Sigma^* \to \Nat^\Sigma$ for the
map that counts occurrences of every letter in the input word, namely: $
\commute[w] \defined (a \mapsto \card[a]{w})$.

\AP Let $X$ be a set. A function $f \colon \Sigma^* \to X$ is
\intro{commutative} whenever for all $u,v \in \Sigma^*$, $f(uv) = f(vu)$.
Equivalently, it is \reintro{commutative} whenever there exists a map $g \colon
\Nat^\Sigma \to X$ such that $g \circ \commute = f$.

\begin{example}
    \label{commutative-function:ex}
    The map $f \colon w \mapsto \card[a]{w} \times \card[b]{w}$ is \kl{commutative}.
\end{example}

Let us briefly state that one can decide whether a \kl{$\Rel$-polyregular
function} $f$ is \kl{commutative}, ensuring that we are working in a decidable
subclass of functions. 

\begin{lemma}
    \label{decide-commutative:lemma}
    One can decide if a function $f \in \ZPoly$
    is \kl{commutative}.
\end{lemma}
\begin{proof}
    Let $\Sigma$ be a finite alphabet, endowed with a
    total ordering over its letters,
    and let $f \colon \Sigma^* \to \Rel$ be a
    \kl{$\Rel$-polyregular function}.
    The map $\polysort \colon \Sigma^* \to \Sigma^*$
    that sorts the letters in the input word $w$ according
    to the chosen ordering is a \kl{polyregular function}
    \cite[Proposition II.12]{LOPEZ23b}.
    As a consequence,
    $(f \circ \polysort) \in \ZPoly$.
    Finally, $f$ is \kl{commutative} if and only
    if $f = f \circ \polysort$, which is decidable
    in the case of $\ZPoly$
    \cite[Corollary II.24]{LOPEZ23b}, and more generally in the case 
    of $\ZRat$ \cite[Corollary 3.6 p 38]{berstel2011noncommutative}.
\end{proof}

\cref{decide-commutative:lemma} does not state decidability in the case of
\kl{$\Rel$-rational series}, and its proof relies on the stability of $\ZPoly$
under the pre-composition by \kl{polyregular functions}, which does not hold
for $\ZRat$. However, and for the sake of completeness, let us sketch a (more
involved) proof that \kl{commutativity} is decidable for this larger class,
using non-trivial results coming from topological things.

\begin{lemma}
    \label{decidable-commutative-rat:lemma}
    Let $f \in \ZRat$, it is decidable whether
    $f$ is \kl{commutative}.
\end{lemma}
\begin{proof}
    Because $f$ is a \kl{$\Rel$-rational series},
    there exists a $n \in \Nat$,
    vectors $v_1$ and $v_2$,
    and $n$ by $n$ integer matrices $\seqof{M_a}{a \in \Sigma}$,
    such that
    \begin{equation*}
        \forall w \in \Sigma^*,
        f(w) = {}^t v_1 \left(\prod_{i = 1}^{|w|} M_{w_i}\right) v_2 \quad .
    \end{equation*}
    To simplify notations, 
    let us write $M_w \defined \prod_{i = 1}^{|w|} M_{w_i}$,
    and 
    $S \defined \setof{M_w}{w \in \Sigma^+}$ be the
    semigroup generated by $\seqof{M_a}{a \in \Sigma}$.

    The function $f$ is \kl{commutative} if and only if for all $u,v \in
    \Sigma^+$, ${}^t v_1 (M_u M_v  - M_v M_u) v_2 = 0$. The above is a
    polynomial equation involving the coefficients of $M_u$ and $M_v$.
    Let us compute $\overline{S \times S}$ the Zariski closure of $S \times S$.
    The above polynomial vanishes over $S \times S$ if and only
    if it belongs to the Zariski closure, which is decidable
    \cite{HROUPOWO18}.
\end{proof}

Let us briefly remark that the characterizations of subclasses of
\kl{polyregular functions} that are obtained in practice can be obtained by
pre-composition with a \kl{commutative} \kl{regular function}.

\begin{remark}
    Let $f \in \ZRat$, and $k \in \Nat$,
    the following are equivalent:
    \begin{enumerate}
        \item $f \in \ZPoly[k]$,
        \item For every \kl{commutative} \kl{polyregular function} $h$
            of \kl{growth rate} $l \in \Nat$,
            $(f \circ h) \in \ZPoly[k+l-1]$.
    \end{enumerate}
\end{remark}


\begin{remark}
    Let $f \in \ZPoly$.
    The following are equivalent:
    \begin{enumerate}
        \item $f \in \ZSF$,
        \item For every \kl{commutative} \kl{star-free polyregular function} $h$,
            $(f \circ h) \in \ZSF$.
    \end{enumerate}
\end{remark}

\AP A function $g \colon \Nat^n \to \Rel$ is \intro{represented} by a function
$f \colon \Sigma^* \to \Rel$ if there exists a map $\eta \colon \set{1, \dots,
n} \to \Sigma$ such that $g \circ (v \mapsto \seqof{v_{\eta(k)}}{1 \leq k \leq
n}) \circ \commute = f$. A key family of functions will be \kl{represented} by
\kl{$\Rel$-polyregular functions}, namely polynomials with multiple indeterminates.

\AP \paragraph*{Polynomials.} All polynomials considered in this paper have
coefficients in $\Rel$ unless explicitly stated otherwise. A polynomial $P \in
\Rel[X_1, \dots, X_n]$ is a \intro{$\Rel$-rational polynomial} if it
is \kl{represented} by a \kl{$\Rel$-polyregular function}. Similarly, it is a
\intro{$\Nat$-rational polynomial} if it is \kl{represented} by a
\kl{$\Nat$-polyregular function}. 

\begin{example}
    \label{negative-not-nrat:ex}
    The polynomials $X$, and $X^2 + 3$ are \kl{$\Nat$-rational polynomial},
    but $- X$ is a \kl{$\Rel$-rational polynomial} that is 
    not a \kl{$\Nat$-rational polynomial}.
\end{example}
\begin{proof}
    The function $w \mapsto |w|$ is a \kl{$\Nat$-polyregular function},
    hence, $P(X) \defined X$ is
    a \kl{$\Nat$-rational polynomial}. Similarly,
    $w \mapsto |w|^2 + 3$ is a \kl{$\Nat$-polyregular function},
    showing that $Q(X) \defined X^2 + 3$
    is a \kl{$\Nat$-rational polynomial}.
    Finally, 
    $T(X) \defined - X$ cannot be 
    a \kl{$\Nat$-rational polynomial} as \kl{$\Nat$-polyregular functions}
    have non-negative output.
\end{proof}

\AP A polynomial $P \in \Rel[X_1, \dots, X_k]$ is \intro{non-negative} when for
all $n_1, \dots, n_k \geq 0$, $P(n_1, \dots, n_k) \geq 0$. Beware that we do
not consider negative values as input, as the numbers $n_i$ will ultimately
count the number of occurrences of a letter in a word. As an example, the
polynomial $(X - Y)^2$ is \kl{non-negative}, and so is the polynomial $X^3$,
but the polynomial $X^2 - 2X$ is not.

We saw in \cref{negative-not-nrat:ex} a simple criterion to check whether a
polynomial is a \kl{$\Nat$-rational polynomial}, but being \kl{non-negative} is
not enough, as the following example illustrates. Note that the proof scheme of
this example will be at the core of \cref{thm:counter-example}.

\begin{example}
    The polynomial $P(X, Y) \defined (X - Y)^2$
    is a \kl{$\Rel$-rational polynomial} that is \kl{non-negative},
    but is
    not a \kl{$\Nat$-rational polynomial}.
\end{example}
\begin{proof}
    Assume by contradiction that
    $S$ is a $\Nat$-rational series computing $P$.
    Then, $S^{-1}(\set{0})$ should be a regular language
    (\cref{pre-image-regular:fact}),
    but $S^{-1}(\set{0}) = \setof{ w \in \Sigma }{ \card[a]{w} = \card[b]{w} }$
    is not a regular language (\cref{regular:fact}).
\end{proof}


\AP Let us now give some vocabulary on polynomials with multiple indeterminate
over $\Rel$. A \intro{monomial} is a product of indeterminates and integers.
For instance, $XY$ is a \kl{monomial}, $3 X$ is a \kl{monomial}, $-Y$ is a
\kl{monomial}, but $X + Y$ or $2X^2 + XY$ are not. We write $\Monomials[X_1,
\dots, X_n]$ for the set of \kl{monomials} over these indeterminates.
Every polynomial $P \in \Rel[X_1, \dots, X_n]$ decomposes uniquely
into a sum of \kl{monomials}.

\AP A \kl{monomial} $S$ \intro{divides} a \kl{monomial} $T$, when $S$ divides
$T$ seen as polynomials in $\Rat$. For instance, $2X$ \kl{divides} $XY$, $-YZ$
\kl{divides} $X^2 Y Z^3$, and $Y$ does not \kl{divide} $X$. In the
decomposition of $P \in \Rel[X_1, \dots, X_k]$, a \kl{monomial} is a
\intro{maximal monomial} if it is a maximal element for the \kl{divisibility
ordering} of \kl{monomials}. In the polynomial $P(X,Y) \defined X^2 - 2XY + X^2
+ X + Y$, the set of \reintro{maximal monomials}, written
$\MaximalMonomials(P)$, is $\set{X^2,  -2 XY,  X^2}$. A \kl{monomial} has a
\intro{non-negative coefficient} if its multiplicative constant belongs to
$\Nat$, or equivalently, if it is \kl{non-negative} as a polynomial. In the
polynomial $P(X,Y) \defined (X - Y)^2$, the \kl{non-negative monomials} are $X^2$
and $Y^2$.




