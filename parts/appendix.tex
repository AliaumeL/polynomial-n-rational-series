%! TeX program = xelatex
%! lang = en-US
\section{Proofs of section \ref{preliminaries:sec}}

\begin{proof}[Proof of \cref{decidable-commutative-poly:lemma}, in the polyregular case]
    Let $\Sigma$ be a finite alphabet, endowed with a
    total ordering over its letters,
    and let $f \colon \Sigma^* \to \Rel$ be a
    \kl{$\Rel$-polyregular function}.
    The map $\polysort \colon \Sigma^* \to \Sigma^*$
    that sorts the letters in the input word $w$ according
    to the chosen ordering is a \kl{polyregular function}
    \cite[Proposition II.12]{CDTL23}.
    As a consequence,
    $(f \circ \polysort) \in \ZPoly$.
    Finally, $f$ is \kl{commutative} if and only
    if $f = f \circ \polysort$, which is decidable
    in the case of $\ZPoly$
    \cite[Corollary II.24]{CDTL23}, and more generally in the case 
    of $\ZRat$ \cite[Corollary 3.6 p 38]{BERE10}.
\end{proof}

\begin{proof}[Proof of \cref{decidable-commutative-rat:lemma} in the general case]
    Because $f$ is a \kl{$\Rel$-rational series},
    there exists a $n \in \Nat$,
    vectors $v_1$ and $v_2$,
    and $n$ by $n$ integer matrices $\seqof{M_a}{a \in \Sigma}$,
    such that
    \begin{equation*}
        \forall w \in \Sigma^*,
        f(w) = {}^t v_1 \left(\prod_{i = 1}^{|w|} M_{w_i}\right) v_2 \quad .
    \end{equation*}
    To simplify notations, 
    let us write $M_w \defined \prod_{i = 1}^{|w|} M_{w_i}$,
    and 
    $S \defined \setof{M_w}{w \in \Sigma^+}$ be the
    semigroup generated by $\seqof{M_a}{a \in \Sigma}$.

    The function $f$ is \kl{commutative} if and only if for all $u,v \in
    \Sigma^+$, ${}^t v_1 (M_u M_v  - M_v M_u) v_2 = 0$. The above is a
    polynomial equation involving the coefficients of $M_u$ and $M_v$.
    Let us compute $\overline{S \times S}$ the Zariski closure of $S \times S$.
    The above polynomial vanishes over $S \times S$ if and only
    if it belongs to the Zariski closure, which is decidable
    \cite[\textbf{WHERE}]{HOPW18}.
\end{proof}

\section{Proofs of section \ref{polynomials:sec}}



\begin{proofof}[negative-not-nrat:ex]
    The function $w \mapsto |w|$ is a \kl{$\Nat$-polyregular function}.
    Thus, 
    $P(X) \defined X$ is
    a \kl{$\Nat$-rational polynomial}. Similarly,
    $w \mapsto |w|^2 + 3$ is a \kl{$\Nat$-polyregular function},
    showing that $Q(X) \defined X^2 + 3$
    is a \kl{$\Nat$-rational polynomial}.

    Remark that $P = (X-1)^2 + 1$. Let $f  \colon \Sigma^* \to \Nat$
    that maps $aw \mapsto |w|^2$, and $\varepsilon \mapsto 0$.
    It is clear that $f \in \NPoly$, and that 
    $P$ is \kl{represented} by
    $\ind{|w| \geq 1} \times f + \ind{|w| = 0} + 1$.

    Finally, 
    $T(X) \defined - X$ cannot be 
    a \kl{$\Nat$-rational polynomial} as \kl{$\Nat$-polyregular functions}
    are non-negative.
\end{proofof}








\textbf{Redo the proof}.
\begin{proofof}[derivation-stabilises-correct:lem]
    If $P$ is constant, then $P_2 = 0$, and $\Diff{K}{P_1} = 0$,
    hence for $K = 0$, $Q \in \CorrectPoly$.

    Otherwise, let $\alpha$ be the maximal absolute value 
    of coefficients appearing in $P$,
    and $D$ be the maximal number of divisors for
    a monomial in $P$.
    We define $K \defined \alpha \times D + 1$.

    Let $\nu \colon \vec{X} \topartial \Nat$, we are going to prove that
    $\restr{Q}{\nu}$ has \kl{non-negative} \kl{maximal monomials}. To that end,
    let us consider $T$ such a maximal monomial, and let us write $T = \beta
    T_1$ so that $T_1$ has coefficient $1$ and $\beta \in \Rel$. Because we
    will need to normalize other monomials in this way, we define for the rest
    of the proof the function $\mathsf{unit}$ that maps a monomial to the
    monomial obtained by replacing its constant term by $1$ if the monomial is
    non-zero, and leaving the monomial unchanged otherwise.

    Let us write $\mathcal{D}_1$ to be the set of \kl{monomials} of
    $\Diff{K}{P_1}$ that are \kl{divided} by $T$. Similarly, let us write
    $\mathcal{D}_2$ for the set of monomials of $\translate{K}(P_2)$ that are
    \kl{divided} by $T$. Thanks to \cref{derivation-monomials:fact}, we know
    that every element of $\mathcal{D}_2$ divides some element in
    $\mathcal{D}_1$. We write $\mathcal{D}_2^M$ for the set of monomials $S \in
    \mathcal{D}_2$ such that $S$ \kl{divides} $M$.
    
    Now, remark that the coefficient $\beta$ is obtained as follows:
    \begin{align*}
        \beta &= \sum_{M \in \mathcal{D}_1} \restr{M / T_1}{\nu}
               + \sum_{M \in \mathcal{D}_2} \restr{M / T_1}{\nu} \\
              &\geq \sum_{M \in \mathcal{D}_1} \restr{M / T_1}{\nu}
              - \sum_{M \in \mathcal{D}_2} \left|\restr{M / T_1}{\nu}\right|
              \\
              &\geq
              \sum_{M \in \mathcal{D}_1}
              \left[
              \restr{M / T_1}{\nu}
              -
              \sum_{S \in \mathcal{D}_2^M}
              \left|\restr{S/T_1}{\nu}\right|
              \right]
              \\
              &\geq
              \sum_{M \in \mathcal{D}_1}
              \restr{\mathsf{unit}(M)/T_1}{\nu}
              \left[
              K
              - \alpha D
              \right]
              \\
              &\geq 0 \quad . \qedhere
    \end{align*}
\end{proofof}


\begin{proofof}[derivation-translation:lem]

    We will first prove the equivalence between \cref{d-t-transl:item} and
    \cref{d-t-correct:item} without keeping track of the constant $K$, and then
    notice that this $K$ is actually computable. This will simplify the reading
    process.

    We will first prove that \cref{d-t-transl:item} $\implies$
    \cref{d-t-correct:item}, for any choice of $K \in \Nat$. Indeed, let us
    consider some partial valuation $\nu \colon \vec{X} \topartial \Nat$. By
    assumption, $\translate{K}(\restr{P}{\nu}) \in \Nat[\vec{X}]$, and in
    particular conclude that \kl{maximal monomials} of
    $\translate{K}(\restr{P}{\nu})$ are exactly the \kl{maximal monomials} of
    $\restr{P}{\nu}$ thanks to \cref{translation-maximal:fact}, and are
    therefore \kl{non-negative}.

    Conversely, the implication \cref{d-t-correct:item} $\implies$
    \cref{d-t-transl:item} is proven by induction on $\MaximalMonomials(P)$. If
    $P$ is a constant polynomial, then $P = n$ for some $n \in \Nat$, and we
    conclude that $\translate{0}(P) \in \Nat[\vec{X}]$. Otherwise, $P = P_1 +
    P_2$ where $P_1$ is the sum of the \kl{maximal monomials} of $P$. Using
    \cref{derivation-stabilises-correct:lem}, there exists a computable $K \in
    \Nat$ such that $Q \defined \Diff{K}{P_1} + \translate{K}(P_2)$ belongs to
    $\CorrectPoly$. Since, $\MaximalMonomials(Q) \hoarele
    \MaximalMonomials(P)$, we conclude by induction hypothesis that there
    exists $L \in \Nat$ such that for every partial valuation $\nu \colon
    \vec{X} \topartial \Nat$, $\translate{L}(\restr{Q}{\nu}) \in
    \Nat[\vec{X}]$. Similarly, for all partial functions $\nu \colon \vec{X}
    \topartial \set{0, \dots, K}$ fixing at least one indeterminate of $P$,
    $\MaximalMonomials(\restr{P}{\nu}) \hoarele \MaximalMonomials(P)$, hence
    there exists $L_\nu \in \Nat$ such that $\translate{L_\nu}(\restr{P}{\nu})
    \in \Nat[\vec{X}]$. Let us define $L_m$ to be the maximum of $L$, and
    $L_\nu$ where $\nu$ ranges over partial functions described above.


    Let us prove that for all partial functions $\nu \colon \vec{X} \topartial
    \Nat$, the translation $\translate{L_m + K}(\restr{P}{\nu})$ belongs to
    $\Nat[\vec{X}]$. Without loss of generality, $\nu$ fixes only
    indeterminates that appear in $P$. If $\nu$ fixes at least one
    indeterminate $X_i$ to a value $\nu(X_i) = k \leq K$, then we can write
    $\nu = \mu [X_i = k]$, where $\mu$ does not fix the value of $X_i$. By
    construction, we know that $\translate{L_i}(\restr{\restr{P}{X_i =
    k}}{\mu})$ belongs to $\Nat[\vec{X}]$, and therefore that $\translate{K +
    L_m}(\restr{P}{\nu})$ does so too. The only case left is when all the
    variables appearing in $\nu$ are greater than $K$.
    In that case, $\nu = \translate{K}(\mu)$ for some partial 
    valuation $\mu$.
    In this case,
    $\translate{K}(\restr{P}{\translate{K}(\mu)}
    = \restr{\translate{K}(P)}{\mu}$
    thanks to 
    \cref{discrete-deriv-linear:fact}.
    Remark that by definition,
    $\translate{K}(P) = P_1 + Q$,
    hence that
    $\translate{K + L_m}(P) = \translate{L_m}(P_1) + \translate{L_m}(Q)$
    belongs to $\Nat[\vec{X}]$.
    Finally,
    we conclude that $\restr{\translate{K+L_m}(P)}{\mu}$
    belongs to $\Nat[\vec{X}]$
    because
    the latter is closed under partial applications.


    Now, remark that in the proof, we have computed $K$ recursively from $P$,
    and only applied partial functions using values below $K$. This not only
    shows that $K$ is computable from $P$, but also that
    \cref{d-t-transl-fin:item} and \cref{d-t-transl:item} are equivalent.

    For the effective decision procedure,
    given $P \in \Rel[\vec{X}]$,
    one applies \cref{derivation-translation:lem}
    to obtain a bound $K$, and then 
    computes $\translate{K}(\restr{P}{\nu})$ for all 
    partial valuations $\nu \colon \vec{X} \topartial \set{0, \dots, K}$.
    Then $P \in \CorrectPoly$ if and only if all those polynomials
    belong to $\Nat[\vec{X}]$, which is decidable.
\end{proofof}



\section{Proofs of section \ref{beyond-polynomials:sec}}

\begin{proofof}[decompose-polynomial:lem]
    Let us first write $f \in \ZPoly$ as a linear combination 
    $f = f_+ - f_-$ where both $f_+$ and $f_-$ are in $\NPoly$.
    We 
    compute $\omega_+$ (resp. $\omega_-$)
    by
    applying
    \cref{n-poly-combinatorics:lem} 
    to $f_+$ (resp. to $f_-$).
    Finally, let $\omega \defined \ppcm(\omega_+, \omega_-)$.

    Using this $\omega$,
    for any choice of
    $r \defined (r_1, \dots, r_n) \in \set{0, \dots, \omega - 1}$,
    there exists an \kl{integer binomial polynomial}
    $P_r \in \Rel[X_1, \dots, X_p]$,
    obtained as
    the difference of the two \kl{natural binomial polynomial} $P_r^+$ and $P_r^-$
    associated to $f_+$ and $f_-$,
    such that $X_1, \dots, X_p \geq 1$:
    \begin{equation*}
        f\left(
        a_1^{\omega X_1 + r_1} \cdots a_n ^{\omega X_n + r_n}\right) 
        = P_r(X_1, \dots, X_p) \quad .
    \end{equation*}
    We conclude by selecting different values of $r$ and
    setting some subsets of variables to be assigned to the constant $0$.
\end{proofof}






\section{Proofs of section \ref{sec:ccl}}

\begin{proofof}[pre-compose-growth-commut:lemma]
    If $f \in \ZPoly[k]$, then 
    for every \kl{polyregular function} $g \in \Poly[\ell]$,
    $f \circ g \in \ZPoly[k \times \ell]$
    \cite{CDTL23}.

    Conversely, it was proven in \cite[Theorem III.3]{CDTL23}
    that $f \in \ZPoly[k]$ if and only if
    for all $\alpha_0, \cdots, \alpha_k$,
    for all $w_1, \dots, w_k$,
    we have 
    \begin{equation}
        \label{k-pumpable:eqn}
        f\left(
            \alpha_0 \prod_{i = 1}^k w_i^{X_i} \alpha_i
        \right)
        = \bigO( (X_1 + \cdots X_k)^{k} )
        \quad .
    \end{equation}
    Now, the \kl{commutative} \kl{star-free polyregular function} 
    $h \colon \set{1, \dots, k}^* \to \Sigma^*$ that maps
    a word $u$
    to $\alpha \prod_{i = 1}^k w_i^{\card[i]{u}}$
    has \kl{growth rate} $1$.
    Hence, 
    $f \circ h \in \ZPoly[k]$, 
    and we indeed conclude that 
    \cref{k-pumpable:eqn}
    holds, i.e. that $f \in \ZPoly[k]$.
\end{proofof}

\begin{proofof}[pre-compose-sf-commut:lemma]
    If $f \in \ZSF$, then for all $h \in \SF$,
    $f \circ h \in \ZSF$, and \emph{a fortiori}
    for \kl{commutative} functions $h$.

    Conversely, assume that $f \circ h \in \ZSF$
    for all \kl{commutative} functions $h \in \SF$.
    Using \cite[Theorem V.13]{CDTL23},
    to conclude that $f \in \ZSF$,
    it suffices to prove that
    for all $\alpha_0, \cdots, \alpha_k \in \Sigma^*$,
    for all $w_1, \dots, w_k \in \Sigma^*$,
    there exists a polynomial $P \in \Rel[X_1, \dots,X_k]$
    and a constant $N_0 \in \Nat$,
    such that if $X_1, \dots, X_k \geq N_0$:
    \begin{equation}
        \label{ultimate-polynomial:eqn}
        f\left(
            \alpha_0 \prod_{i = 1}^k w_i^{X_i} \alpha_i
        \right)
        = P(X_1, \dots, X_k)
        \quad .
    \end{equation}
    Let us consider
    the \kl{commutative} \kl{star-free polyregular function}
    $h \colon \set{1, \dots, k}^* \to \Sigma^*$ that maps
    a word $u$
    to $\alpha \prod_{i = 1}^k w_i^{\card[i]{u}}$.
    We know that
    $f \circ h \in \ZSF$, hence, that 
    there exists $Q \in \Rel[X_1, \dots, X_n]$
    and $M_0 \in \Nat$,
    such that for all $X_1, \dots, X_n \geq N_0$,
    $f \circ h( \prod_{i = 1}^k \underline{i}^{X_i}) = Q(X_1, \dots, X_n)$.
    In particular,
    we can take $N_0 = M_0$, and $P = Q$ to conclude that
    \cref{ultimate-polynomial:eqn} holds, hence, that
    $f \in \ZSF$.
\end{proofof}


\begin{proofof}[z-poly-commutative-encoding:remark]
    We are going to prove that if $f \in \ZPoly[k]$ is polyblind, 
    then the \cref{z-poly-commutative-encoding:conj} holds when allowing 
    for non polyregular functions $\mathsf{enc}$ and $\mathsf{dec}$.

    Let us consider $f \in \ZPoly[k]$ that is polyblind.
    Using \cite[Theorem 6.12]{DOUE23},
    we know that there exists a family of $\MSO$ formulas
    $\seqof{\psi_{i,j}(x)}{1 \leq i,j \leq n}$,
    and coefficients $\seqof{z_i}{1 \leq i \leq n}$,
    such that:
    \begin{equation*}
        f = \sum_{i = 1}^n z_i \times \prod_{j = 1}^n \vcount{\psi_{ij}(x)}
        \quad .
    \end{equation*}

    Let us now consider a monoid $M$ that recognizes all formulas $\psi_{ij}$
    for $1 \leq i,j \leq n$ with a morphism $\mu \colon \Sigma^* \to M$. Let us
    consider the map $\mathsf{t} \colon \Sigma^* \to (M^3)^*$, that associates
    to a word $w \in \Sigma^*$ the word $\mathsf{t}(w) \defined \prod_{i =
    1}^{|w|} ( \mu(w_{< i}), \mu(w_i), \mu(w_{> i}))$. That is, to every letter
    one associates its left and right context in the monoid.


    Let us define $\mathsf{s} \colon \Sigma^* \to (\Nat)^{M^3}$
    by letting $\mathsf{s}(w)$ be the Parikh image of $\mathsf{t}$, that is,
    counting the number of occurrences of every letter.
    It is clear how to define the polyregular function $g \colon (\Nat)^{M^3} \to \Rel$
    that makes the following diagram commute, at is it even a polynomial function
    with $M^3$ indeterminates:
    \begin{center}
        \begin{tikzcd}
            (M^3)^* 
            \arrow{dr}[below,anchor=center,rotate=-39,yshift=-3pt]{\mathsf{count}}
            & \Sigma^* \arrow{l}[above]{\mathsf{t}} \arrow{r}{f} \arrow{d}[left]{\mathsf{s}} & \Rel \\
            & \Nat^{M^3} \arrow{ur}[below]{g} & \\
        \end{tikzcd}
    \end{center}
    One problem is that the map $\mathsf{s}$ is not surjective, because of
    potential dependencies between the types encountered in $w$.
    Let us however notice that one can build a context-free grammar
    recognizing exactly the image of $\mathsf{t}$. Namely,
    \begin{align*}
        S &\to S^{1,m,1} &  m \in M \\
        S^{m_r, m, m_l} &\to (m_l, m, m_r) & \exists a \in \Sigma, \mu(a) = m \\
        S^{m_r, m, m_l} &\to 
        S^{m_r, m_x, m_y m_l}
        S^{m_r m_x, m_y, m_l}
                        & m_x m_y = m
    \end{align*}

    Using Parikh's Theorem \cite{PARI66},
    we conclude that the image of $\mathsf{s}$ is a semilinear set,
    i.e., that there exists a finite set $Q$,
    numbers $\seqof{d_q}{q \in Q}$,
    vectors $\seqof{v_{i,q}}{0 \leq i \leq d_q, q \in Q}$ in $\Nat^{M^3}$,
    such that
    \begin{equation*}
        \mathsf{s}(\Sigma^*) =
        \bigcup_{q \in Q}
        \setof{
            v_0 + \sum_{i = 1}^{d_q} \lambda_i v_{i,q}
        }{
            (\lambda_1, \dots, \lambda_{d_q}) \in \Nat^{d_q}
        } \quad .
    \end{equation*}

    We can therefore define a function $\mathsf{decode} \colon \Sigma^* \to
    \sum_{q \in Q} \Nat^{d_q}$ that maps to a word $w$ to \emph{some} $q \in Q$
    and parameters $(\lambda_1, \dots, \lambda_{d_q}) \in \Nat^{d_q}$ such that
    $\mathsf{s}(w) = v_0 + \sum_{i = 1}^{d_q} \lambda_i v_{i,q}$. By taming the
    semilinear set prior to this step, we can ensure that the function is in
    fact uniquely defined, this is done by computing an equivalent semilinear
    set that is \emph{unambiguous} and \emph{proper} \cite{CHHA16}.
    Similarly, let us define $\mathsf{encode} \colon \sum_{q \in Q} \Nat^{d_q}
    \to \Sigma^*$ that maps an element $q \in Q$ with parameters $(\lambda_1,
    \dots, \lambda_{d_q}) \in \Nat^{d_q}$ to \emph{some} word $w \in \Sigma^*$
    such that $\mathsf{s}(w) = v_0 + \sum_{i = 1}^{d_q} \lambda_i v_{i,q}$.
    Notice that $\mathsf{decode} \circ \mathsf{encode} = \mathsf{id}$, by the
    unicity of the representation in $\sum_{q \in Q} \Nat^{d_q}$.
    We can define the map $\mathsf{repr} \colon \sum_{q \in Q} \Nat^{d_q} \to \Nat^{M^3}$
    as follows:
    \begin{equation*}
        \mathsf{repr}(q, \lambda_1, \dots, \lambda_{d_q}) \defined
        v_0 + \sum_{i = 1}^{d_q} \lambda_i v_{i,q}
        \quad .
    \end{equation*}
    Finally, let us define for $q \in Q$ the map $g_q \colon \Nat^{d_q} \to \Rel$ as the polynomial
    expression:
    \begin{equation*}
        g_q(\lambda_1, \dots, \lambda_{d_q}) \defined \mathsf{repr}(q, \lambda_1, \dots, \lambda_{d_q}) \quad .
    \end{equation*}

    We have proven that the following diagram commutes:
    \begin{center}
        \begin{tikzcd}
            \Sigma^* \arrow{rr}{f} 
                     \arrow{dr}{\mathsf{s}}
                     \arrow[bend right=20]{dd}[rotate=90,anchor=center,yshift=4pt]{\mathsf{decode}}
                     &        & \Rel \\
                     & \Nat^{M^3} \arrow{ur}[below]{g} & \\
            \sum_{q \in Q} \Nat^{d_q}
            \arrow[bend right=20]{uu}[right,rotate=90,anchor=center,yshift=-4pt]{\mathsf{encode}}
            \arrow{ur}[left]{\mathsf{repr}}
            \arrow[bend right=45]{uurr}[right]{\seqof{g_q}{q \in Q}}
                     & & \\
        \end{tikzcd}
    \end{center}

    The only missing part of this proof is the fact that the maps $\mathsf{encode}$ and
    $\mathsf{decode}$ can be computed as \kl{polyregular functions}. Notice
    that \emph{mutatis mutandis}, the same proof scheme applies to $f \in
    \ZSF[k]$, by assuming that the monoid is aperiodic, since the functions
    constructed (apart from $\mathsf{encode}$ and $\mathsf{decode}$) will be
    \kl{star-free}.
\end{proofof}
