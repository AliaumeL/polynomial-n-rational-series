%! TeX program = xelatex
%! lang = en-US

\section{Proofs of section \ref{polynomials:sec}}



\begin{proofof}[negative-not-nrat:ex]
    The function $w \mapsto |w|$ is a \kl{$\Nat$-polyregular function}.
    Thus, 
    $P(X) \defined X$ is
    a \kl{$\Nat$-rational polynomial}. Similarly,
    $w \mapsto |w|^2 + 3$ is a \kl{$\Nat$-polyregular function},
    showing that $Q(X) \defined X^2 + 3$
    is a \kl{$\Nat$-rational polynomial}.

    Remark that $P = (X-1)^2 + 1$. Let $f  \colon \Sigma^* \to \Nat$
    that maps $aw \mapsto |w|^2$, and $\varepsilon \mapsto 0$.
    It is clear that $f \in \NPoly$, and that 
    $P$ is \kl{represented} by
    $\ind{|w| \geq 1} \times f + \ind{|w| = 0} + 1$.

    Finally, 
    $T(X) \defined - X$ cannot be 
    a \kl{$\Nat$-rational polynomial} as \kl{$\Nat$-polyregular functions}
    are non-negative.
\end{proofof}







\begin{proofof}[derivation-translation:lem]

    We will first prove the equivalence between \cref{d-t-transl:item} and
    \cref{d-t-correct:item} without keeping track of the constant $K$, and then
    notice that this $K$ is actually computable. This will simplify the reading
    process.

    We will first prove that \cref{d-t-transl:item} $\implies$
    \cref{d-t-correct:item}, for any choice of $K \in \Nat$. Indeed, let us
    consider some partial valuation $\nu \colon \vec{X} \topartial \Nat$. By
    assumption, $\translate{K}(\restr{P}{\nu}) \in \Nat[\vec{X}]$, and in
    particular conclude that \kl{maximal monomials} of
    $\translate{K}(\restr{P}{\nu})$ are exactly the \kl{maximal monomials} of
    $\restr{P}{\nu}$, and are therefore \kl{non-negative}.

    Conversely, the implication \cref{d-t-correct:item} $\implies$
    \cref{d-t-transl:item} is proven by induction on the number of
    indeterminates of $P$. If $P$ is a constant polynomial, then $P = n$ for
    some $n \in \Nat$, and we conclude that $\translate{0}(P) \in
    \Nat[\vec{X}]$. Otherwise, $P = P_1 + P_2$ where $P_1$ is the sum of the
    \kl{maximal monomials} of $P$. Using
    \cref{derivation-stabilises-correct:lem}, there exists a computable $K \in
    \Nat$ such that $Q \defined \Diff{K}{P_1} + \translate{K}(P_2)$ belongs to
    $\Nat[\vec{X}]$. Furthermore, for all partial functions $\nu \colon \vec{X}
    \topartial \set{0, \dots, K}$ fixing at least one indeterminate of $P$, the
    induction hypothesis yields a number $L_\nu \in \Nat$ such that
    $\translate{L_\nu}(\restr{P}{\nu}) \in \Nat[\vec{X}]$. Let us define $L_m$
    to be the maximum of $L_\nu$ where $\nu$ ranges over partial
    functions described above.


    Let us prove that for all partial functions $\nu \colon \vec{X} \topartial
    \Nat$, the translation $\translate{L_m + K}(\restr{P}{\nu})$ belongs to
    $\Nat[\vec{X}]$. Without loss of generality, $\nu$ fixes only
    indeterminates that appear in $P$. If $\nu$ fixes at least one
    indeterminate $X_i$ to a value $\nu(X_i) = k \leq K$, then we can write
    $\nu = \mu [X_i = k]$, where $\mu$ does not fix the value of $X_i$. By
    construction, we know that $\translate{L_i}(\restr{\restr{P}{X_i =
    k}}{\mu})$ belongs to $\Nat[\vec{X}]$, and therefore that $\translate{K +
    L_m}(\restr{P}{\nu})$ does so too. The only case left is when all the
    variables appearing in $\nu$ are greater than $K$.
    In that case, $\nu = \translate{K}(\mu)$ for some partial 
    valuation $\mu$.
    In this case,
    $\translate{K}(\restr{P}{\translate{K}(\mu)})
    = \restr{\translate{K}(P)}{\mu}$.
    Remark that by definition,
    $\translate{K}(P) = P_1 + Q$,
    hence that
    $\translate{K + L_m}(P) = \translate{L_m}(P_1) + \translate{L_m}(Q)$
    belongs to $\Nat[\vec{X}]$.
    Finally,
    we conclude that $\restr{\translate{K+L_m}(P)}{\mu}$
    belongs to $\Nat[\vec{X}]$
    because
    the latter is closed under partial applications.


    Now, remark that in the proof, we have computed $K$ recursively from $P$,
    and only applied partial functions using values below $K$. This not only
    shows that $K$ is computable from $P$, but also that
    \cref{d-t-transl-fin:item} and \cref{d-t-transl:item} are equivalent.

    For the effective decision procedure,
    given $P \in \Rel[\vec{X}]$,
    one applies \cref{derivation-translation:lem}
    to obtain a bound $K$, and then 
    computes $\translate{K}(\restr{P}{\nu})$ for all 
    partial valuations $\nu \colon \vec{X} \topartial \set{0, \dots, K}$.
    Then $P \in \CorrectPoly$ if and only if all those polynomials
    belong to $\Nat[\vec{X}]$, which is decidable.
\end{proofof}



\section{Proofs of section \ref{beyond-polynomials:sec}}

\begin{proofof}[decompose-polynomial:lem]
    Let us first write $f \in \ZPoly$ as a linear combination 
    $f = f_+ - f_-$ where both $f_+$ and $f_-$ are in $\NPoly$.
    We 
    compute $\omega_+$ (resp. $\omega_-$)
    by
    applying
    \cref{n-poly-combinatorics:lem} 
    to $f_+$ (resp. to $f_-$).
    Finally, let $\omega \defined \ppcm(\omega_+, \omega_-)$.

    Using this $\omega$,
    for any choice of
    $r \defined (r_1, \dots, r_n) \in \set{0, \dots, \omega - 1}$,
    there exists two \kl{natural binomial functions} $F_r^+$ and $F_r^-$
    associated to $f_+$ and $f_-$,
    such that $X_1, \dots, X_p \geq 1$:
    \begin{equation*}
        f\left(
        a_1^{\omega X_1 + r_1} \cdots a_n ^{\omega X_n + r_n}\right) 
        = F_r^+(X_1, \dots, X_p) - F_r^-(X_1, \dots, X_p) \quad .
    \end{equation*}
    Now, for large enough values of $x_1, \dots, x_p$,
    this means that 
    $f\left(
        a_1^{\omega X_1 + r_1} \cdots a_n ^{\omega X_n + r_n}\right)$
    coincides with an \kl{integer binomial polynomial},
    and we can therefore conclude selecting different values of $r$ and
    setting some subsets of variables to be assigned to the constant $0$.
\end{proofof}






\section{Proofs of section \ref{sec:ccl}}

\begin{proofof}[pre-compose-growth-commut:lemma]
    If $f \in \ZPoly[k]$, then 
    for every \kl{polyregular function} $g \in \Poly[\ell]$,
    $f \circ g \in \ZPoly[k \times \ell]$
    \cite{CDTL23}.

    Conversely, it was proven in \cite[Theorem III.3]{CDTL23}
    that $f \in \ZPoly[k]$ if and only if
    for all $\alpha_0, \cdots, \alpha_k$,
    for all $w_1, \dots, w_k$,
    we have 
    \begin{equation}
        \label{k-pumpable:eqn}
        f\left(
            \alpha_0 \prod_{i = 1}^k w_i^{X_i} \alpha_i
        \right)
        = \bigO( (X_1 + \cdots X_k)^{k} )
        \quad .
    \end{equation}
    Now, the \kl{commutative} \kl{star-free polyregular function} 
    $h \colon \set{1, \dots, k}^* \to \Sigma^*$ that maps
    a word $u$
    to $\alpha \prod_{i = 1}^k w_i^{\card[i]{u}}$
    has \kl{growth rate} $1$.
    Hence, 
    $f \circ h \in \ZPoly[k]$, 
    and we indeed conclude that 
    \cref{k-pumpable:eqn}
    holds, i.e. that $f \in \ZPoly[k]$.
\end{proofof}

\begin{proofof}[pre-compose-sf-commut:lemma]
    If $f \in \ZSF$, then for all $h \in \SF$,
    $f \circ h \in \ZSF$, and \emph{a fortiori}
    for \kl{commutative} functions $h$.

    Conversely, assume that $f \circ h \in \ZSF$
    for all \kl{commutative} functions $h \in \SF$.
    Using \cite[Theorem V.13]{CDTL23},
    to conclude that $f \in \ZSF$,
    it suffices to prove that
    for all $\alpha_0, \cdots, \alpha_k \in \Sigma^*$,
    for all $w_1, \dots, w_k \in \Sigma^*$,
    there exists a polynomial $P \in \Rel[X_1, \dots,X_k]$
    and a constant $N_0 \in \Nat$,
    such that if $X_1, \dots, X_k \geq N_0$:
    \begin{equation}
        \label{ultimate-polynomial:eqn}
        f\left(
            \alpha_0 \prod_{i = 1}^k w_i^{X_i} \alpha_i
        \right)
        = P(X_1, \dots, X_k)
        \quad .
    \end{equation}
    Let us consider
    the \kl{commutative} \kl{star-free polyregular function}
    $h \colon \set{1, \dots, k}^* \to \Sigma^*$ that maps
    a word $u$
    to $\alpha \prod_{i = 1}^k w_i^{\card[i]{u}}$.
    We know that
    $f \circ h \in \ZSF$, hence, that 
    there exists $Q \in \Rel[X_1, \dots, X_n]$
    and $M_0 \in \Nat$,
    such that for all $X_1, \dots, X_n \geq N_0$,
    $f \circ h( \prod_{i = 1}^k \underline{i}^{X_i}) = Q(X_1, \dots, X_n)$.
    In particular,
    we can take $N_0 = M_0$, and $P = Q$ to conclude that
    \cref{ultimate-polynomial:eqn} holds, hence, that
    $f \in \ZSF$.
\end{proofof}
