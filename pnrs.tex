%! TeX program = xelatex
\documentclass[a4paper,11pt]{article}

\usepackage[left=3cm,right=3cm]{geometry}


% INPUTS
%% PACKAGES %%

\usepackage[utf8]{inputenc}

% Macro making packages
\usepackage{xparse}
\usepackage{xpatch}
\usepackage{tokcycle}

% figures and subfigures
\usepackage{caption}
\usepackage{subcaption}


% micro typography
\usepackage{microtype}


% Standalone compilation
\usepackage[obeyclassoptions,mode=tex]{standalone}

% Math typesetting 
%\usepackage{amsmath}
%\usepackage{amsthm}
%\usepackage{thmtools}
\usepackage{upgreek}

% References and knowledge management
%\usepackage{hyperref}
\usepackage[capitalise,noabbrev,nameinlink]{cleveref}
\usepackage[composition,hyperref,xcolor,cleveref]{knowledge}
\knowledgeconfigure{notion}

% BIBTEX / BIBLATEX
%\usepackage[
%	%style=numeric-comp,
%    %citestyle=authortitle-icomp,
%	citestyle=numeric-comp,
%	%bibstyle=authoryear,
%	bibstyle=numeric,
%	%sorting=none,
%    sorting=nyt,
%	%sortcites=true,
%	%autocite=footnote,
%    maxnames=99,
%    %backend=biber, % Compile the bibliography with biber
%    backend=bibtex,
%    %refsegment=chapter, % split references ?
%	hyperref=true,
%	backref=true,
%	citecounter=true,
%	pagetracker=true,
%	citetracker=true,
%	ibidtracker=context,
%	autopunct=true,
%	autocite=plain,
%    doi=true,
%]{biblatex}

%\usepackage{amssymb}
%\usepackage{stmaryrd}


% Table typesetting
\usepackage{booktabs}
\usepackage{varwidth}

% Proofs typesetting
\usepackage{bussproofs}

% Drawing
\usepackage{tikz}
\usetikzlibrary{backgrounds}
\usetikzlibrary{shapes.geometric}
\usetikzlibrary{decorations.markings}
\usetikzlibrary{positioning}
\usetikzlibrary{automata}
\usetikzlibrary{tikzmark}
\usetikzlibrary{patterns}
 \usetikzlibrary{calc}
\usetikzlibrary{arrows}
\usepackage{tikz-cd}
\tikzset{every state/.style={minimum size=1pt}}

% algorithms
\usepackage[ruled,linesnumbered]{algorithm2e}

% Fonts
%\usepackage[fira]{fontsetup}

% ENS PARIS SACLAY COLORSCHEME
\definecolor{Prune}{RGB}{99,0,60}
\definecolor{A1}{HTML}{000000}
\definecolor{B1}{RGB}{49,62,72}
\definecolor{C1}{RGB}{124,135,143}
\definecolor{D1}{RGB}{213,218,223}
\definecolor{A2}{RGB}{198,11,70}
\definecolor{B2}{RGB}{237,20,91}
\definecolor{C2}{RGB}{238,52,35}
\definecolor{D2}{RGB}{243,115,32}
\definecolor{A3}{RGB}{124,42,144}
\definecolor{B3}{RGB}{125,106,175}
\definecolor{C3}{RGB}{198,103,29}
\definecolor{D3}{RGB}{254,188,24}
\definecolor{A4}{RGB}{0,78,125}
\definecolor{B4}{RGB}{14,135,201}
\definecolor{C4}{RGB}{0,148,181}
\definecolor{D4}{RGB}{70,195,210}
\definecolor{A5}{RGB}{0,128,122}
\definecolor{B5}{RGB}{64,183,105}
\definecolor{C5}{RGB}{140,198,62}
\definecolor{D5}{RGB}{213,223,61}


\foreach \name in {A,B,C,D} {
    \foreach \hue in {1,2,3,4,5} {
        \foreach \shade/\intensity in {hint/20,bg/50} {
            \xglobal\colorlet{\name\hue\shade}{\name\hue!\intensity!white}
        }
    }
}

\newcommand{\tableofcolors}{
    \begin{tikzpicture}
        \foreach \letter/\x in {A/0,B/1,C/2,D/3} {
            \foreach \y/\variant in {0/1,1/2,2/3,3/4,4/5} {
                \node[color=\letter\variant] (\letter\variant) at (\x,\y) {\letter\variant};
                \node[color=\letter\variant bg]
                    (BG\letter\variant) at ({\x - 0.2}, {\y - 0.2}) {\letter\variant};
                \node[color=\letter\variant hint] 
                    (HT\letter\variant) at ({\x - 0.4}, {\y - 0.4}) {\letter\variant};
            }
        }
        \begin{scope}[xshift=2cm]
            \foreach \name/\x/\y in {
                Prune/3/4
            } {
                \node[color=\name] (\name) at (\x,\y) {\name};
            }

        \end{scope}
    \end{tikzpicture}
}



% CONFIG
%% CUSTOM MACROS, META MACROS, AND CONFIG %%


% knowledge management
\knowledgestyle{intro notion}{color={A5}, emphasize}
\knowledgestyle{notion}{color={A4}}
\knowledgeconfigure{anchor point color={A2},
                    anchor point shape=corner}
\knowledgestyle{intro unknown}{color={D3}, emphasize}
\knowledgestyle{intro unknown cont}{color={C3}, emphasize}
\knowledgestyle{kl unknown}{color={D2}}
\knowledgestyle{kl unknown cont}{color={C2}}

\hypersetup{
    colorlinks=true,
    anchorcolor=A2,
    citecolor=A4,
    linkcolor=A4,
    urlcolor=A3,
    filecolor=A3,
    runcolor=D2,
    menucolor=D2,
}

\NewDocumentCommand{\klscope}{ o m }{
    \withkl{\kl[#1]}{#2}
}

% Common theorem styles
\theoremstyle{plain}
\newtheorem{theorem}{Theorem}[section]
\newtheorem{lemma}[theorem]{Lemma}
\newtheorem{proposition}[theorem]{Proposition}
\newtheorem{corollary}[theorem]{Corollary}
\newtheorem{fact}[theorem]{Fact}
\theoremstyle{definition}
\newtheorem{definition}[theorem]{Definition}
\newtheorem{example}[theorem]{Example}
\theoremstyle{remark}
\newtheorem{remark}[theorem]{Remark}
\newtheorem{exercise}[theorem]{Exercise}
\newtheorem{claim}[theorem]{Claim}
\newtheorem{tool}[theorem]{Tool}
\newtheorem{conjecture}[theorem]{Conjecture}


% Upgreek letters
\makeatletter
\newcommand\mathgr[1]{\tokcycle
  {\addcytoks{##1}}
  {\processtoks{##1}}
  {\ifcsname up\expandafter\@gobble\string##1\endcsname
   \addcytoks[1]{\csname up\expandafter\@gobble\string##1\endcsname}%
    \else\addcytoks{##1}\fi}
  {\addcytoks{##1}}{#1}%
  \expandafter\mathrm\expandafter{\the\cytoks}%
}
\makeatother

% Comment on a definition/text
\NewDocumentCommand{\comment}{m}{%
    \footnote{#1}%
}

% Citation configuration
\AtEveryBibitem{
    % Removes issn, isbn, and
    % unwanted items from the bibliography
	\clearfield{issn}
	\clearfield{isbn}
	\clearfield{archivePrefix}
	\clearfield{arxivId}
	\clearfield{pmid}
	\clearfield{eprint}
}

% Cite a knowledge
\NewDocumentCommand{\citek}{ o m }{
    \IfNoValueTF{#1}{
        \cite{m}
    }{
        \cite[{\kl(#1)[#2]}]{#1}
    }
}

%
% Here are the definitions
% of mathematical macros.
%
%

\newcommand{\defined}{\mathrel{:=}}
\newcommand{\eqdef}{\stackrel{\mathsf{def}}{=}}
\newcommand{\seqof}[2]{(#1)_{#2}}
\newcommand{\setof}[2]{\mathopen{\{} #1 \mid #2 \mathclose{\}}}
\newcommand{\set}[1]{\mathopen{\{} #1 \mathclose{\}}}

\newcommand{\ppcm}{\mathop{\mathrm{lcm}}}

\NewDocumentCommand{\card}{ O{} m }{ \mathopen{|} #2  \mathclose{|}_{#1}}

\newcommand{\Nat}{\mathbb{N}}
\newcommand{\Rel}{\mathbb{Z}}
\newcommand{\Rat}{\mathbb{Q}}
\newcommand{\Real}{\mathbb{R}}

\newcommand{\Pfin}{\mathcal{P}_{\mathsf{fin}}}

\newcommand{\MSO}{\mathsf{\kl[\MSO]{MSO}}}
\knowledge{\MSO}{notion}

\newcommand{\FO}{\mathsf{\kl[\FO]{FO}}}
\knowledge{\FO}{notion}


\newcommand{\ind}[1]{\mathbf{1}_{#1}}
\newcommand{\bigO}{\mathcal{O}}

\NewDocumentCommand{\Span}{m}{\mathop{\kl[\Span]{\mathsf{Span}_{#1}}}}
\knowledge{\Span}{notion}

\NewDocumentCommand{\const}{ m }{\mathop{\mathsf{const}_{#1}}}


\newcommand{\topartial}{\rightharpoonup}
\newcommand{\tosurj}{\twoheadrightarrow}
% functions

\newcommand{\dom}{\operatorname{dom}}

\NewDocumentCommand{\vcount}{ m }{ {{\#}#1} }


% GOOD POLYNOMIALS
\newcommand{\CoveredPoly}{\kl[\CoveredPoly]{\mathsf{PolyNNeg}}}
\newcommand{\CorrectPoly}{\kl[\CorrectPoly]{\mathsf{PolyRec}}}
\knowledge{\CoveredPoly}{notion}
\knowledge{\CorrectPoly}{notion}

\NewDocumentCommand{\Monomials}{o}{\mathop{\kl[\Monomials]{\IfNoValueTF{#1}{\mathsf{Mon}}{\mathsf{Mon}[#1]}}}}
\NewDocumentCommand{\MaximalMonomials}{}{\mathop{\kl[\MaximalMonomials]{\mathsf{maxmon}}}}
\knowledge{\Monomials}{notion}
\knowledge{\MaximalMonomials}{notion}



% Restrictions
%\newcommand{\restr}[2]{{#1}_{\kl[\restr]{| #2}}}
\newcommand{\restr}[2]{\withkl{\kl[\restr]}{\cmdkl{[}{#1}\cmdkl{]}_{#2}}}
\knowledge{\restr}{notion}

\NewDocumentCommand{\Poly}{O{}}{\kl[\Poly]{\mathsf{Poly}_{#1}}}
\NewDocumentCommand{\SF}{O{}}{\kl[\SF]{\mathsf{SF}_{#1}}}

% Add commands for Mealy / Sequential / Rational / Regular
\NewDocumentCommand{\Mealy}{O{}}{\kl[\Mealy]{\mathsf{Mealy}_{#1}}}
\NewDocumentCommand{\Sequential}{O{}}{\kl[\Sequential]{\mathsf{Seq}_{#1}}}
\NewDocumentCommand{\Rational}{O{}}{\kl[\Rational]{\mathsf{Rat}_{#1}}}
\NewDocumentCommand{\Regular}{O{}}{\kl[\Regular]{\mathsf{Reg}_{#1}}}
\knowledge{\Mealy}{notion}
\knowledge{\Sequential}{notion}
\knowledge{\Rational}{notion}
\knowledge{\Regular}{notion}

% Do the same for the aperiodic variants of Mealy, Sequential, Rational and Regular
\NewDocumentCommand{\AMealy}{O{}}{\kl[\AMealy]{\mathsf{AMealy}_{#1}}}
\NewDocumentCommand{\ASequential}{O{}}{\kl[\ASequential]{\mathsf{ASeq}_{#1}}}
\NewDocumentCommand{\ARational}{O{}}{\kl[\ARational]{\mathsf{ARat}_{#1}}}
\NewDocumentCommand{\ARegular}{O{}}{\kl[\ARegular]{\mathsf{AReg}_{#1}}}
\knowledge{\AMealy}{notion}
\knowledge{\ASequential}{notion}
\knowledge{\ARational}{notion}
\knowledge{\ARegular}{notion}





\NewDocumentCommand{\NPoly}{O{}}{\kl[\NPoly]{\mathbb{N}\mathsf{Poly}_{#1}}}
\NewDocumentCommand{\ZPoly}{O{}}{\kl[\ZPoly]{\mathbb{Z}\mathsf{Poly}_{#1}}}
\NewDocumentCommand{\NSF}{O{}}{\kl[\NSF]{\mathbb{N}\mathsf{SF}_{#1}}}
\NewDocumentCommand{\ZSF}{O{}}{\kl[\ZSF]{\mathbb{Z}\mathsf{SF}_{#1}}}
\NewDocumentCommand{\ZRat}{}{\kl[\ZRat]{\mathbb{Z}\mathsf{Series}}}
\NewDocumentCommand{\NRat}{}{\kl[\NRat]{\mathbb{N}\mathsf{Series}}}
\NewDocumentCommand{\ZCommut}{}{\kl[\ZCommut]{\mathsf{Commut}}}

\knowledge{\Poly}{notion}
\knowledge{\SF}{notion}
\knowledge{\NPoly}{notion}
\knowledge{\ZPoly}{notion}
\knowledge{\NSF}{notion}
\knowledge{\ZSF}{notion}
\knowledge{\ZRat}{notion}
\knowledge{\NRat}{notion}
\knowledge{\ZCommut}{notion}

\NewDocumentCommand{\commute}{O{\cdot}}{\{\!\{ #1 \}\!\}}





\NewDocumentCommand{\hoareleq}{}{\leq_{\mathcal{H}}}
\NewDocumentCommand{\hoarele}{}{<_{\mathcal{H}}}

%\newcommand{\growth}[1]{\mathsf{growth}_{#1}}

\NewDocumentCommand{\BadPoly}{}{\kl[\BadPoly]{\mathsf{P}_{\mathsf{bad}}}}
\knowledge{\BadPoly}{notion}

\NewDocumentCommand{\translate}{m}{\mathop{\kl[\translate]{\tau_{#1}}}}
\knowledge{\translate}{notion}

% Differences of polynomials
\NewDocumentCommand{\Diff}{m m}{ \mathop{\kl[\Diff]{\Delta_{#1}}}(#2) }
\knowledge{\Diff}{notion}

\NewDocumentCommand{\polysort}{}{\mathsf{sort}}
\NewDocumentCommand{\polysum}{}{\mathsf{sum}}


\NewDocumentCommand{\npolyleq}{ O{} }{\mathrel{\kl[\npolyleq]{\preceq_{\Nat #1}}}}
\knowledge{\npolyleq}{notion}
\NewDocumentCommand{\zpolyequiv}{ O{} }{\mathrel{\kl[\zpolyequiv]{\equiv_{\Rel #1}}}}
\knowledge{\zpolyequiv}{notion}

\NewDocumentCommand{\app}{ m m }{\mathop{{#2} \mathrel{\kl[\app]{\triangleright}} {#1}}}
\knowledge{\app}{notion}

\NewDocumentCommand{\Res}{}{ \mathop{\kl[\Res]{\mathsf{Res}}}}
\knowledge{\Res}{notion}


\NewDocumentCommand{\resleq}{ m m }{\mathrel{\kl[\resleq]{\leq_{{#1},{#2}}}}}
\knowledge{\resleq}{notion}
\NewDocumentCommand{\resleqsf}{ m m }{\mathrel{\kl[\resleqsf]{\leq_{{#1},{#2}}^{\mathsf{sf}}}}}
\knowledge{\resleqsf}{notion}

\NewDocumentCommand{\prefleq}{}{\mathrel{\kl[\prefleq]{\sqsubseteq_{\mathsf{pref}}}}}
\knowledge{\prefleq}{notion}
\NewDocumentCommand{\prefle}{}{\sqsubset_{\mathsf{pref}}}
\NewDocumentCommand{\preforth}{}{\bot_{\mathsf{pref}}}


\NewDocumentCommand{\aTransd}{}{\mathcal{A}}


\NewDocumentCommand{\ModuloTypes}{O{}}{\kl[\ModuloTypes]{{#1}\mathsf{Types}}}
\knowledge{\ModuloTypes}{notion}
\NewDocumentCommand{\moduloType}{O{}}{\kl[\moduloType]{{#1}\mathsf{type}}}
\knowledge{\moduloType}{notion}


\NewDocumentCommand{\floor}{m}{\lfloor #1 \rfloor}


% Name of examples


\NewDocumentCommand{\BadExOk}{}{\mathsf{f}}
\NewDocumentCommand{\BadExKo}{}{\mathsf{g}}

\NewDocumentCommand{\pbinom}{ m m }{\binom{#1}{#2}_{\kl[\pbinom]{p}}}
\knowledge{\pbinom}{notion}


\input{globals/knowledges.kl}
\title{Title of the document}
\author{Author of the document}
\date{\today}
\bibliography{globals/papers}

\newcommand{\acknowledge}{
    Acknowledge
}

\newcommand{\makeabstract}{
    \begin{abstract}
        Abstract of the paper.
    \end{abstract}
}


\usepackage{comment}
%\includecomment{forshort}
%\excludecomment{proof}



% DOCUMENT
\begin{document}

% FRONT MATTER
\maketitle
\makeabstract
\acknowledge


% MAIN MATTER
\section{Introduction}

We will start by formally stating the result of \citeauthor{KARH77}
in terms of polynomials and commutative $\Nat$-rational series, and
then introduce the necessary definitions. The characterization of
commutative $\Nat$-rational series computing polynomial functions
is described in terms of a class $\CoveredPoly[\vec{X}]$ of
polynomials with multiple indeterminate.

\begin{definition}[{\cite[Section 3, page 3]{KARH77}}]
    Let $\vec{X}$ be a finite tuple of indeterminate.
    The class $\CoveredPoly[\vec{X}]$
    is the class of polynomials $P \in \Rel[\vec{X}]$
    that are \kl{non-negative}
    and such that every \kl{maximal monomial} has a \kl{positive coefficient}.
    When the indeterminate are clear from the context, we write
    this class $\CoveredPoly$.
\end{definition}

We use the term \kl{$\Nat$-rational polynomial}, that will formally be
stated later on, to denote polynomials that can be computed via a
$\Nat$-rational series. In the upcoming theorem, this property
which is \emph{a priori} difficult to decide is reduced to 
the positivity of the polynomial, and the positivity of \emph{some}
coefficients. 


\begin{theorem}[{\cite[Theorem 3.3, page 4]{KARH77}}] 
    \label{karh:thm}
    Let $\Sigma$ be a finite alphabet.
    The set of \kl{$\Nat$-rational polynomials} over $\Sigma$
    coincides with $\CoveredPoly[\seqof{X_a}{a \in \Sigma}]$.
\end{theorem}

It is not clearly stated in \cite{KARH77}, but \cref{karh:thm} implies the
decidability of \kl{$\Nat$-rational polynomial} inside polynomials with $\Rel$
coefficients. We will not formalize this statement yet, as \cref{karh:thm}
does not hold in general.


\paragraph*{Context.} There has been recent interest in $\Rel$ and
$\Nat$-rational series exhibiting \emph{polynomial growth}
\cite{doueneau2021pebble,bojanczyk2022transducers}. This computational model,
in the counting case, can be traced back to a model introduced by
\textcite{schutzenberger1965finite}. It has been recently proven that one can
decide the growth rate of such functions, and decide whether they are aperiodic
in the case of $\Rel$-rational series \cite{LOPEZ23b}. It was conjectured in
the PhD thesis of Gaëtan Douéneau-Tabot, that the membership problem of
$\Nat$-rational series among $\Rel$-rational series was decidable under the
assumption of polynomial growth. This conjecture was based on the previously
introduced result of \textcite{KARH77} (\cref{karh:thm}), which we will
disprove in \cref{sec:c-example}. We then provide in \cref{sec:proof} a
corrected version of the theorem, together with a valid proof. As a
consequence, it does not invalidate the stated conjecture.

\begin{itemize}
    \item commutative input and output
    \item commutative output
    \item no commutativity ?
\end{itemize}

\AP \paragraph*{Rational series and regular languages.} This paper uses notions
of regularity and results about \kl{$\Nat$-rational series}, however most of
the technicalities will stem from the multiple indeterminate of polynomials,
and all there is to know about regularity in this document will be confined to
the following facts. For a survey about (non-commutative) rational series, we
refer the reader to \cite{berstel2011noncommutative}. 

\begin{definition}[Rational series]
    The following are equivalent
    \begin{itemize}
        \item weighted automata
        \item rational expressions
        \item 
    \end{itemize}
\end{definition}

\begin{definition}[Commutative rational series]
    A $\Rel$-rational series
    $f \colon \Sigma \to \Rel$ is \intro{commutative} whenever $f(uv) = f(vu)$ for
    all $u,v \in \Sigma^*$.
\end{definition}

\begin{fact}[Folklore about regular languages]
    \label{regular:fact}
    The language $\setof{ w \in \Sigma}{ \card[a]{w} = \card[b]{w}}$
    is not regular, whenever $a,b \in \Sigma$.
    Regular languages are closed under intersection.
\end{fact}

\begin{fact}
    \label{pre-image-regular:fact}
    The pre-image of a regular language by a $\Nat$-rational series
    is a regular language. In particular,
    for all $\Nat$-rational series $S$, $S^{-1}(\set{0})$ is a regular
    language.
\end{fact}



\AP \paragraph*{Polynomials.} All polynomials considered in this paper have
coefficients in $\Rel$ unless explicitly stated otherwise. A polynomial $P \in
\Rel[\seqof{X_a}{a \in \Sigma}]$ is a \intro{$\Rel$-rational polynomial} if
there exists a \reintro{commutative} $\Rel$-rational series $f$ such that $f(w)
= P(\seqof{ \card[a]{w} }{a \in \Sigma})$. Similarly, $P$ is a
\intro{$\Nat$-rational polynomial} if the \reintro{commutative} $\Rel$-rational
series is in fact $\Nat$-rational. Beware that we only apply the polynomial
to \emph{non-negative} values.

\begin{example}
    \label{negative-not-nrat:ex}
    With $\Sigma \defined \set{1}$,
    the polynomials $X_1$, and $X_1^2 + 3$ are \kl{$\Nat$-rational polynomial},
    but $- X_1$ is a \kl{$\Rel$-rational polynomial} that is 
    not a \kl{$\Nat$-rational polynomial}.
\end{example}
\begin{proof}
    The function $w \mapsto |w|$ is a \kl{polyregular function} with unary
    output of \kl{polynomial growth}, hence  can be computed by a
    \kl{$\Nat$-rational series}. Hence, $P(X_1) \defined X_1$ is
    a \kl{$\Nat$-rational polynomial}. Similarly,
    $w \mapsto |w|^2 + 3$ is a \kl{$\Nat$-rational series},
    showing that $Q(X_1) \defined X_1^2 + 3$
    is a \kl{$\Nat$-rational polynomial}.
    Finally, 
    $T(X_1) \defined - X_1$ cannot be $\Nat$-rational as $\Nat$-rational series
    have non-negative output.
\end{proof}

We saw in \cref{negative-not-nrat:ex}
a simple criterion to check whether a polynomial is $\Nat$-rational,
but being non-negative is not enough, as the following example
illustrates. Note that the proof scheme
of this example will be at the core of \cref{thm:counter-example}.

\begin{example}
    Over a binary alphabet $\Sigma \defined \set{a,b}$,
    the polynomial $P(X_a, X_b) \defined (X_a - X_b)^2$
    is a \kl{$\Rel$-rational polynomial} that is \kl{non-negative},
    but is
    not a \kl{$\Nat$-rational polynomial}.
\end{example}
\begin{proof}
    Assume by contradiction that
    $S$ is a $\Nat$-rational series computing $P$.
    Then, $S^{-1}(\set{0})$ should be a regular language
    (\cref{pre-image-regular:fact}),
    but $S^{-1}(\set{0}) = \setof{ w \in \Sigma }{ \card[a]{w} = \card[b]{w} }$
    is not a regular language (\cref{regular:fact}).
\end{proof}


\AP Let us now give some vocabulary on polynomials with multiple indeterminate
over $\Rel$. A \intro{monomial} is a product of variables and integers. For
instance, $XY$ is a \kl{monomial}, $3 X$ is a \kl{monomial}, but $X + Y$ or
$2X^2 + XY$ are not.

\AP A polynomial $P \in \Rel[X_1, \dots, X_k]$ is \intro{non-negative} if
$P(n_1, \dots, n_k) \geq 0$ for all $n_1, \dots, n_k \geq 0$. Beware that we do
not consider negative values, as the numbers $n_i$ will ultimately count the
number of occurrences of a letter in a word. As an example, the polynomial $(X
- Y)^2$ is \kl{non-negative}, and so is the polynomial $X^3$, but the
polynomial $X^2 - 2X$ is not.

\AP
In a polynomial $P \in \Rel[X_1, \dots, X_k]$, a monomial 
is a \intro{maximal monomial} if it is a maximal element
for the visibility ordering of polynomials \emph{in $\Rat$}
among monomials with non-zero coefficients in $P$.\footnote{
    One could equivalently define maximal elements by removing the (non-zero)
    multiplicative constant of the monomials, and then
    considering the usual divisibility relation of $\Rel$ polynomials.
}
In the polynomial 
$P(X,Y) \defined X^2 - 2XY + X^2 + X + Y$, the \reintro{maximal monomials}
are $X^2$, $-2 XY$, and $X^2$.
A \kl{monomial} has \intro{positive coefficient} in a polynomial $P$ if its
multiplicative constant is positive. In the polynomial $P(X,Y) \defined (X -
Y)^2$, the \kl{positive monomials} are $X^2$ and $Y^2$.


\paragraph*{Outline of the paper.}
todo.

\section{The Counter Example}
\label{sec:c-example}

Let us now introduce the counter example that was found trying to generalise
the original proof to non-commutative \kl{$\Rel$-rational series} of
\kl{polynomial growth}. The counter example will use three indeterminate, and
we will prove in \cref{sec:proof} that \citeauthor{KARH77}’s theorem holds on
polynomial with at most two indeterminate.

\begin{definition}[Counter Example Polynomial]
    \label{def:bad-polynomial}
    We define $P(X,Y,Z) \defined Z (X + Y)^2 + 2 (X - Y)^2$.
\end{definition}

\begin{theorem}
    \label{thm:counter-example}
    Over $\Sigma \defined \set{a,b,c}$,
    there exists a polynomial $P \in \CoveredPoly$ that is not
    a \kl{$\Nat$-rational polynomial}. Namely,
    $P$ is the polynomial of \cref{def:bad-polynomial}.
\end{theorem}
\begin{proof}
    Let us write $X,Y,Z$ instead of $X_a, X_b, X_c$ to improve
    readability.
    It is clear that $P$ is \kl{non-negative}. We can develop
    the expression of $P$ to 
    obtain
    $P = ZX^2 + ZY^2 + 2ZXY + 2X^2 -4XY + 2Y^2$.
    The \kl{maximal monomials} of $P$
    are $ZX^2$, $ZY^2$, and $2ZXY$, all of which have
    \kl{positive coefficients}.

    Assume by contradiction that $P$ is a \kl{$\Nat$-rational polynomial}.
    There exists a \kl{commutative}
    $\Nat$-rational series $f \colon \Sigma \to \Nat$
    such that $P(\card[a]{w}, \card[b]{w}, \card[c]{w}) = f(w)$.
    Remark that $P(X,Y,Z) = 0$
    if and only if $Z(X+Y)^2 = -2 (X-Y)^2$. Hence,
    $P(X,Y,Z)=0$ if and only if $Z = 0$ and $X = Y$, or 
    $Z \neq 0$, and $X = Y = 0$.

    Now, let us consider the language $L \defined \setof{w}{ f(w) = 0}$. By the
    above computation, we conclude that $L = \setof{ w \in \set{a,b}^* }{
    \card[a]{w} = \card[b]{w} } \cup \set{ c }^*$.
    Because $L \cap \set{a,b}^*$ is notoriously not a regular language
    (\cref{regular:fact}), we
    conclude that $L$ is not a regular language.
    However, $L = f^{-1}(\set{0})$, which is a regular language
    because $f$ is a $\Nat$-rational series
    (\cref{pre-image-regular:fact}).
\end{proof}

\begin{corollary}
    The result stated in \cite[Theorem 3.3]{KARH77}, restated
    in \cref{karh:thm}, is false
    for all polynomials with at least $3$ indeterminate.
\end{corollary}

As we will see in the next section, the characterization
of \kl{$\Nat$-rational polynomials} using the set $\CoveredPoly$
holds when the polynomials have at most $2$ indeterminate. This shows
that the examples in \cite{KARH77} are not invalidated, as they
all use at most two indeterminate.

\section{The Corrected Theorem}
\label{sec:proof}

Let us now provide another class of polynomials, $\CorrectPoly$, that restricts
$\CoveredPoly$ to a subset that is ``closed under fixing indeterminate", namely,
polynomials $P(X_1, \dots, X_n) \in \CoveredPoly$ such that partially evaluated
polynomials remain in $\CoveredPoly$. We use the following notation
to fix the value of some indeterminate
$\intro*\restr{P(X,Y)}{X = 1}$ is the polynomial $P(1,Y)$. More generally, if $\nu$ is
a partial function from $\vec{X}$ to $\Nat$, the restriction
$\restr{P(\vec{X})}{\nu}$ is the polynomial with indeterminate $\vec{Y}
\defined \vec{X} - \dom(\nu)$ obtained by fixing the variables of the domain of
$\nu$.


\begin{definition}
    Let $\vec{X}$ be a finite tuple of indeterminate.
    The class $\CorrectPoly[\vec{X}]$ is the collection of
    polynomials $P \in \Rel[\vec{X}]$ such that
    $P$ is 
    such that, for every partial function $\nu \colon \vec{X} \to \Nat$,
    every \kl{maximal monomial} of
    $\restr{P}{\nu}$ has a \kl{positive coefficient}.
\end{definition}

First, let us remark that $\CorrectPoly \subseteq \CoveredPoly$, because
polynomials in $\CorrectPoly$ are \kl{non-negative}. Let us also check that the
counter example provided in \cref{thm:counter-example} is not in
$\CorrectPoly$. For that, notice that for $P$ defined in
\cref{def:bad-polynomial}, $P(X,Y,1) = 3X^2 + 3Y^2 - 2XY$, which has a negative
coefficient for a \kl{maximal monomial}, namely $-2XY$. Let us also assert that
polynomials that have all coefficients in $\Nat$ are \kl{$\Nat$-rational
polynomials}, which will be a base case in an upcoming induction.

\begin{fact}
    \label{fact:n-poly-n-poly}
    Let $P \in \Nat[\vec{X}]$. Then, $P$
    is a \kl{$\Nat$-rational polynomial}.
\end{fact}

Let us now prove that \kl{$\Nat$-rational polynomials} are in $\CorrectPoly$.
To simplify the reasoning, we will use results on $\Nat$-polyregular functions,
also known as, polyregular functions with unary output
\cite{doueneau2021pebble,bojanczyk2019string}. Note that the connection between
rational series and polyregular functions is not specific to $\Nat$, as
$\Rel$-rational series of polynomial growth are computed as polyregular
functions with output in $\set{+1,-1}$ post-composed with computing the sum of
outputted letters \cite{LOPEZ23b}. Because we only consider functions with
output in $\Rel$, and that a commutative output monoid simplifies the definitions,
let us only define $\Rel$-polyregular functions.


\begin{definition}[$\Rel$-polyregular functions, {\cite[see, e.g.]{LOPEZ23b}}]
    Let $M$ be a finite monoid, $\mu \colon \Sigma^* \to M$
    be a morphism, $k \in \Nat$, and 
    $\pi \colon M^k \to \Rel$ be a production function.
    The \intro{$\Rel$-polyregular function}
    $\pi^\dagger \colon \Sigma^* \to \Rel$
    is computed as follows:
    \begin{equation*}
        f(w) \defined
        \sum_{w = u_1 \cdots u_k} \pi(\mu(u_1), \dots, \mu(u_k))
        \quad .
    \end{equation*}

    When $\pi$ has co-domain $\Nat$, the function $\pi^\dagger$
    is called \intro{$\Nat$-polyregular}.
\end{definition}

\begin{fact}[see e.g. ...]
    \label{polynomial-rational-polyreg:fact}
    Let $\Sigma$ be a finite alphabet, and $f \colon \Sigma^* \to \Nat$
    be a function. The following are equivalent:
    \begin{enumerate}
        \item $f$ can be computed as a \k{$\Nat$-rational series}
            of \kl{polynomial growth}.
        \item $f$ is a \kl{$\Nat$-polyregular function}.
    \end{enumerate}
\end{fact}

\begin{lemma}
    \label{n-poly-combinatorics:lem}
    Let $f$ be a  \kl{$\Nat$-polyregular} function. 
    There exists $\omega \in \Nat$
    such that for all $p \in \Nat$,
    for all $\alpha_0, \dots, \alpha_p \in \Sigma^*$,
    for all $u_1, \dots, u_p \in \Sigma^*$,
    there exists a polynomial $P \in \Rel[X_1, \dots, X_p]$
    whose \kl{maximal monomials} have \kl{positive coefficients},
    and such that:
    \begin{equation*}
        f(\alpha_0 \prod_{i = 1}^p (u_i^{\omega \times X_i} \alpha_i))
        = P(X_1, \dots, X_p) \quad .
    \end{equation*}
\end{lemma}
\begin{proof}[Proof Sketch]
    Let $\omega$ be an idempotent power for the finite monoid $M$.
    We define an equivalence between partitions of a word 
    $w \defined \alpha_0 \prod_{i = 1}^p (u_i^{\omega \times X_i} \alpha_i)$
    as follows: two partitions into $k$ words in $\Sigma^*$ are equivalent
    when normalizing them with the rules
    $u_i^{\omega} u_i^{\omega} \to u_i^{\omega}$ for all $1 \leq i \leq p$.
    This equivalence relation has finite index, and all partitions in the same
    equivalence class yield the same value through $\pi$, because
    $\mu(u_i^\omega) = \mu(u_i)^\omega$ is an idempotent element of the monoid $M$.

    Now, let us remark that the number of partitions
    in a certain equivalence class is proportional to
    the number of nanania, that is, is of the form
    $(X_i - K_i)^{n_i}$ for some $K$. We conclude that
    maximal monomials indeed have positive coefficients.
    We even conclude that
    the polynomial is in $\Nat[\vec{X}]$
    if we shift the variables with a large enough $K$. 
\end{proof}


\begin{lemma}
    The \kl{$\Nat$-rational polynomials} are in $\CorrectPoly$.
\end{lemma}
\begin{proof}[Proof Idea]
    Let $P$ be a \kl{$\Nat$-rational polynomial}.
    Let $\Sigma$  be the finite alphabet such that
    $P \in \Rel[\seqof{X_a}{a \in \Sigma}]$,
    and $f \colon \Sigma^* \to \Nat$ be the \kl{commutative}
    $\Nat$-rational series computing $P$.
    Because $f$ has polynomial growth, and is a $\Nat$-rational series,
    it is a $\Nat$-polyregular function 
    (\cref{polynomial-rational-polyreg:fact}).

    Using \cref{n-poly-combinatorics:lem},
    we conclude that maximal monomials of $P$ have
    positive coefficients. Furthermore, when fixing variables
    in $P$, we are computing a new \kl{$\Nat$-polyregular function},
    hence we also conclude that the maximal monomials have
    positive coefficients.
    We have proven that $P \in \CorrectPoly$.
\end{proof}



The core of the upcoming proof of \cref{lem:correct-to-n-rat} relies on the
following observation: if a polynomial $P$ is written $X^p + Q$, where all
terms of $Q$ have degree less than $p$, then one can find some large enough $K$
so that $X^p - (X - K)^p + Q$ also has a positive maximal coefficient.
It works because $X^p - (X - K)^p = K X^{p-1} + \cdots$, hence
the maximal coefficient can be chosen arbitrarily large by increasing
$K$.
This
trick will work similarly in multivariate polynomials, and the overall
proof scheme works as follows: for a bounded number of possible
values of $X$, we can fix the variable, and have one less item to work on.
For large enough values of $X$ (typically greater than some $K$), we can guarantee that
$X^p - (X - K)^p + Q$ has positive maximal coefficients, hence
can be computed by a $\Nat$-rational series, while
$(X - K)^p$ is clearly computable by a $\Nat$-rational series,
hence $(X^p - (X - K)^p + Q) + (X - K)^p$ is computable
as the sum of two $\Nat$-rational series, and we have
concluded that our original polynomial was a
\kl{$\Nat$-rational polynomial}. There are a number of problems in generalizing
this proof scheme:
\begin{itemize}
    \item We will have to guarantee that every polynomial considered
        is non-negative
    \item We do not only need to check that the \kl{maximal monomials}
        have \kl{positive coefficient}, but that it is the case
        when fixing arbitrarily many variables! (this
        covers the first point too)
\end{itemize}

\begin{lemma}
    Let $\vec{X}$ be a tuple of indeterminate, $Y$ be a fresh indeterminate,
    $\vec{p} \colon \vec{X}Y \to \Nat$,
    and $M \defined \times \prod_{X \in \vec{X}} X^{p(X)}$.
    Then, for all $K \geq 0$,
    $\Delta_{Y,K} \defined Y^p M - (Y - K)^p M$ has \kl{maximal monomials} with
    \kl{positive coefficients} of weight at least $K$, regardless of
    the variables fixed, as long as the (potential)
    value of $Y$ is greater
    than $K$.
\end{lemma}
\begin{proof}
    Let us first remark that
    \begin{equation*}
        \Delta_{Y,K}
        = KY^{p-1} M + Q M
        \quad .
    \end{equation*}
    In particular, there is a unique \kl{maximal monomial} for $\Delta_{Y,K}$,
    which is precisely $K Y^{p-1} M$, because all monomials
    in $Q M$ divide (strictly) $Y^{p-1} M$.
    Let us fix a variable,
    $X$ to any (non-zero) value $p > 0$. Then
    The \kl{maximal monomial} of $\restr{\Delta_{Y,K}}{X = p}$
    is exactly $\restr{KY^{p-1}M}{X = p}$, which remains
    with a \kl{positive coefficient}, that is proportional to $K$.
    Let us fix $Y$ to some value $K + l$, with $l > 0$,
    then
    $\restr{\Delta_{Y,K}}{Y = K + l}
    = (K + l)^p M - l^p M
    = ((K + l) - l) (\sum_{i = 1}^{p-1} (K + l)^i l^{p-i}) M
    = K \alpha M
    $
    which remains a monomial with a \kl{positive coefficient},
    that is proportional to $K$.
\end{proof}


We can refine the above lemma to use it in the case where
we do not have a single
monomial.

\begin{lemma}
    \label{lem:delta-cool}
    Let $\vec{X}$ be a tuple of indeterminate, and $Y$ be a fresh indeterminate.
    Let $P \in \Rel[\vec{X}Y]$ be in $\CorrectPoly$, and $\alpha Y^p M$ be a
    \kl{maximal monomial} of $P$, with $\alpha > 0$.
    Then, there exists $K \geq 0$
    such that $\Delta_{K} \defined P - \alpha (Y - K)^p M$
    has \kl{positive coefficients} for \kl{maximal monomials}
    in any partial evaluation that satisfies $Y > K$ (if it
    gives a value to $Y$).
\end{lemma}
\begin{proof}
    Let $N$ be greater than the sum of the absolute values of all coefficients
    appearing in $P$, and let $K = 2 N$.

    Let us remark that
    \begin{equation*}
        \Delta_K = \underbrace{(P - \alpha Y^p M)}_{\text{ left hand side }}
        + \underbrace{\alpha (Y^p M - (Y - K)^p M)}_{\text{ right hand side }}
        \quad .
    \end{equation*}
    an that for every partial evaluation $\nu$ that sends $Y$ to a value
    greater than $K$,
    we have that $\restr{(Y^p M - (Y - K)^p M)}{\nu}$ has
    \kl{maximal monomials} with \kl{positive coefficients}
    proportional to $K$ (note that here ``proportional"
    implies that they are greater than $K$, because we multiply by positive
    integers).
    
    To conclude, it is just a game of describing how
    \kl{maximal monomials} of $\restr{\Delta_Y}{\nu}$
    can appear. If the \kl{maximal monomial} is obtained
    by summing an element of the left hand side with an element
    of the right hand side, it needs to be a \kl{maximal monomial}
    of the right hand side, and the large value of $K$ guarantees
    that the sum of coefficients is positive.
    If it only appears on the right hand side, it is even better.
    If it only appears on the left hand side, then
    in particular it means that the monomial does not
    divide $\restr{Y^p M}{\nu}$, hence has the same value
    in $\restr{\Delta_K}{\nu}$ or in $\restr{P}{\nu}$,
    which is guaranteed to be positive because $P \in \CorrectPoly$.

    In particular, we conclude from this that
    $\Delta_Y \geq 0$, on all tuples such that $Y > K$.
\end{proof}

Note that we cannot conclude in the above lemma that the result is in
$\CorrectPoly$, because we have a restriction on the range where the variable
$Y$ can be taken.

\begin{lemma}
    \label{lem:correct-to-n-rat}
    Polynomials in $\CorrectPoly$ are
    \kl{$\Nat$-rational polynomials}.
\end{lemma}
\begin{proof}[Proof Sketch]
    We prove by induction on the number $d$ of variables
    and the ``higman ordering of polynomials" (we look at the multiset
    of tuples of things ...) that for every selection
    $K_1, \dots, K_n$ of lower bounds for the different variables
    (that can be $0$),
    functions that are in $\CorrectPoly$ \emph{over tuples of size
    at least nanania} can be computed by a $\Nat$-rational series
    that outputs $0$ outside the domain and the correct value
    inside.

    For $d = 0$,
    the result is true because
    a constant function is a \kl{$\Nat$-rational polynomial}
    if and only if it is \kl{non-negative}.

    Let us now assume that $d > 0$, and in particular
    there is a variable $Y$.
    Consider $P \in \CorrectPoly[\vec{X} Y]$.
    If $Y$ appears at least one in $P$, then it must appear
    in one of the \kl{maximal monomials} of $P$. Otherwise,
    $P \in \CorrectPoly[\vec{X}]$ and we conclude using the
    induction hypothesis.
    Let $\alpha Y^p M$ be a \kl{maximal monomial} in $P$.
    Using \cref{lem:delta-cool},
    we know that there exists $K$ such that
    $P - \alpha (Y - K)^p M$ almost belongs to $\CorrectPoly$,
    and has a strictly smaller "degree multiset".
    By induction hypothesis,
    we conclude that
    the following function is a $\Nat$-rational series
    (beware that we have to consider the lower bounds to apply
    the induction hypothesis):
    \begin{equation*}
        A \defined
        \begin{cases}
            0 & \text{ if some variable is bounded by its } K_i \\
            0 & \text{ otherwise if  } Y \leq K \\
            P - \alpha(Y - K)^pM & \text{ otherwise }
        \end{cases}
        \quad .
    \end{equation*}

    Because $\alpha \geq 0$, it is an easy check
    that the following function is also a $\Nat$-rational series
    \begin{equation*}
        B \defined
        \begin{cases}
            0 & \text{ if some variable is bounded by its } K_i \\
            0 & \text{ otherwise if } Y \leq K \\
            \alpha(Y - K)^pM & \text{ otherwise }
        \end{cases}
    \end{equation*}

    Remark that by fixing some of the variables
    to values 
    below the bounds $K_i$ (and potentially $K$ for $Y$)
    we obtain a \emph{finite} number of polynomials with strictly fewer
    indeterminate.
    As a consequence, the following function is computable
    as a $\Nat$-rational series:
    \begin{equation*}
        C \defined
        \begin{cases}
            P & \text{ if some variable is bounded by its } K_i \\
            P & \text{ otherwise if } Y \leq K \\
            0 & \text{ otherwise }
        \end{cases}
        \quad .
    \end{equation*}

    We conclude because
    $P = A + B + C$, all of which are $\Nat$-rational series.
\end{proof}

\begin{theorem}
    \label{corrected-version:thm}
    The \kl{$\Nat$-rational polynomials} are exactly
    the polynomials in $\CorrectPoly$.
\end{theorem}

\begin{remark}
    In the proof, we actually conclude that
    the \kl{$\Nat$-rational polynomials}
    are exactly those that can be computed 
    by \kl{commutative}
    \emph{star-free} $\Nat$-polyregular functions.
\end{remark}

\begin{lemma}
    \label{lem:correct-covered-2}
    $\CorrectPoly[X,Y] = \CoveredPoly[X,Y]$.
\end{lemma}
\begin{proof}
    It is clear that $\CorrectPoly[X,Y] \subseteq \CoveredPoly[X,Y]$,
    by considering the empty valuation $\nu \colon \set{X,Y} \to \Nat$.
    For the converse inclusion, let us consider $P(X,Y)$
    that is \kl{non-negative}, such that the \kl{maximal monomials}
    have \kl{positive coefficients}.
   

    If we fix none of the variables, then the \kl{maximal monomials}
    have \kl{positive coefficients} by assumption. If we fix one of the
    variables, we can assume without loss of generality that we 
    fix $X = k$ for some $k \in \Nat$.
    Then $P(k,Y)$ is a \kl{non-negative} \emph{univariate} polynomial, 
    and therefore must have a positive leading coefficient
    (which is the unique \kl{maximal monomial} in this case)
    or be constant equal to 0. In both cases, the \kl{maximal monomials}
    have \kl{positive coefficients}.
    The same reasoning applies \emph{a fortiori} in the case where
    we fix the two indeterminate, leading to a constant polynomial.
\end{proof}

\section{Deciding commutative $\Nat$-polyregular functions}
\label{sec:deciding}

The goal of this section is to go beyond polynomials and characterize all
$\Rel$-polyregular functions that are $\Nat$-polyregular, under the assumption
that they are \kl{commutative}.

\begin{theorem}
    \label{decidable-n-poly:thm}
    Given a \kl{commutative}
    \kl{$\Rel$-polyregular function} $f \colon \Sigma^* \to \Rel$,
    one can decide if $f$ can be computed by a \kl{$\Nat$-polyregular function},
    and effectively compute a representation of $f$.
\end{theorem}

\begin{lemma}
    \label{decide-correct-poly:lem}
    One can decide if a polynomial is in $\CorrectPoly$.
\end{lemma}
\begin{proof}
    It is easy to check that maximal monomials have positive coefficients.
    Assuming that it is the case, remark that 
    every partial valuation $\nu$
    such that $\restr{P}{\nu}$ has a negative maximal coefficient
    must have values that are bounded by the sum of absolute values of
    coefficients appearing in $P$.
    In particular, one can check all possible choices for $\nu$
    within these bounds.
\end{proof}

To prove the theorem, let us first provide a nice characterization of
\kl{commutative} \kl{$\Rel$-polyregular functions} in terms of polynomials.
This is actually a refinement of \cref{n-poly-combinatorics:lem} in the
commutative case.

\begin{lemma}
    \label{decompose-polynomial:lem}
    Let $f \colon \Sigma^* \to \Rel$ be a \kl{commutative}
    \kl{$\Rel$-polyregular function}. There exists $\omega \in \Nat$,
    and polynomials $\seqof{P_{\seqof{n_a}{a \in \Sigma}}}{ 0 \leq n_a < \omega}$,
    such that for all $w \in \Sigma^*$,
    \begin{equation*}
        f(w) = P_{\seqof{\card[a]{w} [\omega]}{a \in \Sigma}}
        (\seqof{\card[a]{w} / \omega}{a \in \Sigma}) \quad .
    \end{equation*}
    Which informally means that for every choice of rest modulo $\omega$
    for every variable, there exists a polynomial describing the
    behavior of $f$ when adding multiples of $\omega$ letters at a time.

    Furthermore, the polynomials are computable from $f$.
\end{lemma}
\begin{proof}
    equivalent partitions of the word, but commutativity?.q
\end{proof}


As a consequence, we conclude 
\begin{proof}[Proof of \cref{decidable-n-poly:thm}]
    We use \cref{decompose-polynomial:lem}
    to effectively compute
    a polynomial representation of $f$. Remark that
    $f$ is a \kl{$\Nat$-polyregular function} if and only if
    all these polynomials are \kl{$\Nat$-rational polynomials},
    which is decidable thanks to
    \cref{decide-correct-poly:lem,corrected-version:thm}.
\end{proof}

\section{Conclusion}
\label{sec:ccl}

\begin{itemize}
    \item Actually, we have an algorithm that allows
        us to \emph{decide} if a polynomial is $\Nat$-rational.
    \item Actually, we obtain a characterisation of
        commutative 
        $\Nat$-rational series of polynomial growth
        among $\Rel$-rational series.
    \item The question of $\Nat$-polyregular functions
        among $\Rel$-polyregular functions remains open
        because of the non-commutativity of the input.
    \item The question of $\Nat$-rational series (commutative)
        among $\Rel$-rational series remains open.
\end{itemize}


\begin{corollary}
    The following problem is decidable: given $P \in \Rel[\vec{X}]$,
    decide whether $P$ is a \kl{$\Nat$-rational polynomial}.
\end{corollary}
\begin{proof}
    Thanks to \cref{karh:thm}, one only has to decide whether
    $P \in \CoveredPoly$. The procedure is as follows.
    \begin{enumerate}
        \item Check if all \kl{maximal monomials} have
            \kl{positive coefficients} in $P$, if not, answer \texttt{false}.
        \item Consider $N \in \Nat$ large enough so that
              $P \geq 0$ when all indeterminate are chosen greater
              than $N$. This is possible because all \kl{maximal
              monomials} have \kl{positive coefficients}, and such an $N$
              is computable.
        \item Compute $P$
    \end{enumerate}

    Now, if all \kl{maximal monomials}
    have \kl{positive coefficients}, then there exists a computable
    $N \in \Nat$ such that 
    In particular, 


\end{proof}


\begin{conjecture}
    $\CorrectPoly = \setof{P}{ P \geq 0 \wedge \exists K \in \Nat, P \circ (+K) \in \Nat[\vec{X}]}$.
\end{conjecture}
\begin{proof}[Proof Sketch]
    If $S \defined P \circ (+K) \in \Nat[\vec{X}]$, then for every partial application,
    maximal coefficients of $S$ are positive. As maximal coefficients of 
    (partial applications of) $S$ are the same as the ones of $P$
    \emph{except for the constant terms} (full application),
    where $P$ is already guaranteed to be non-negative, we conclude
    that $P \in \CorrectPoly$.

    Conversely, we prove by induction on the degree of $P \in \CorrectPoly$,
    that there exists some $K$ such that $P \circ (+K) \in \Nat[\vec{X}]$.
    First, notice that $\CorrectPoly$ is closed under pre-composition with
    $(+K)$. Then, remark that $P(X - K)$ has the same maximal coefficients as
    $P$ and new maximal coefficients that are multiples of $K$. Hence, for a
    large enough $K$, we conclude that $P(X - K)$ is the sum of its maximal
    coefficients, plus a polynomial that lies in $\CorrectPoly$. We conclude by
    induction hypothesis. 
\end{proof}

\begin{conjecture}
    $\CorrectPoly$ is precisely composed of polynomials that
    have non-negative coefficients in a \emph{binomial basis},
    i.e., where polynomials are obtained
    as sums and products of $\binom{X_i}{p_i}$.
\end{conjecture}

\begin{conjecture}
    star-free $\Nat$-rational is star-free and $\Nat$-rational?
\end{conjecture}

% BACKMATTER
\printbibliography

\appendix

\end{document}
