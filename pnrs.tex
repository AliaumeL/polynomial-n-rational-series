%! TeX program = xelatex
\documentclass[a4paper,11pt]{article}

\usepackage[left=3cm,right=3cm]{geometry}


% INPUTS
%% PACKAGES %%

% Macro making packages
\usepackage{xparse}
\usepackage{xpatch}
\usepackage{tokcycle}


% Standalone compilation
\usepackage[obeyclassoptions,mode=tex]{standalone}

% Math typesetting 
\usepackage{amsmath}
\usepackage{amsthm}
\usepackage{thmtools}
\usepackage{upgreek}
%\usepackage{amssymb}
\usepackage{stmaryrd}

% References and knowledge management
\usepackage{hyperref}
\usepackage[capitalise,noabbrev,nameinlink]{cleveref}
\usepackage[electronic,hyperref,xcolor,cleveref]{knowledge}
\knowledgeconfigure{notion}

%% BIBTEX / BIBLATEX
\usepackage{natbib}

% Table typesetting
\usepackage{booktabs}
\usepackage{varwidth}

% Proofs typesetting
\usepackage{bussproofs}

% Drawing
\usepackage{tikz}
\usetikzlibrary{backgrounds}
\usetikzlibrary{shapes.geometric}
\usetikzlibrary{positioning}
\usetikzlibrary{automata}
\usetikzlibrary{tikzmark}
\usetikzlibrary{patterns}
\usetikzlibrary{arrows}
\usepackage{tikz-cd}
\tikzset{every state/.style={minimum size=1pt}}

% Fonts
%\usepackage[fira]{fontsetup}

\input{globals/colorscheme}

% CONFIG
%% CUSTOM MACROS, META MACROS, AND CONFIG %%

% knowledge management
% TODO: put an if paper mode.
%\knowledgestyle{intro notion}{color={A5}, emphasize}
%\knowledgestyle{notion}{color={A4}}
%\knowledgeconfigure{anchor point color={A2},
%                    anchor point shape=corner}
%\knowledgestyle{intro unknown}{color={D3}, emphasize}
%\knowledgestyle{intro unknown cont}{color={C3}, emphasize}
%\knowledgestyle{kl unknown}{color={D2}}
%\knowledgestyle{kl unknown cont}{color={C2}}

\hypersetup{
    %colorlinks=true,
    anchorcolor=A2,
    citecolor=A4,
    linkcolor=A4,
    urlcolor=A3,
    filecolor=A3,
    runcolor=D2,
    menucolor=D2,
}

\NewDocumentCommand{\klscope}{ o m }{
    \withkl{\kl[#1]}{#2}
}

% Common theorem styles
%\theoremstyle{plain}
%\newtheorem{theorem}{Theorem}[section]
\newtheorem{faketheorem}[theorem]{Flawed Theorem}
\newtheorem{problem}[theorem]{Problem}
%\newtheorem{lemma}[theorem]{Lemma}
%\newtheorem{proposition}[theorem]{Proposition}
%\newtheorem{corollary}[theorem]{Corollary}
\newtheorem{fact}[theorem]{Fact}
%\theoremstyle{definition}
%\newtheorem{definition}[theorem]{Definition}
%\newtheorem{example}[theorem]{Example}
%\theoremstyle{remark}
%\newtheorem{remark}[theorem]{Remark}
%\newtheorem{exercise}[theorem]{Exercise}
%\newtheorem{claim}[theorem]{Claim}
%\newtheorem{tool}[theorem]{Tool}
%\newtheorem{conjecture}[theorem]{Conjecture}


% Upgreek letters
\makeatletter
\newcommand\mathgr[1]{\tokcycle
  {\addcytoks{##1}}
  {\processtoks{##1}}
  {\ifcsname up\expandafter\@gobble\string##1\endcsname
   \addcytoks[1]{\csname up\expandafter\@gobble\string##1\endcsname}%
    \else\addcytoks{##1}\fi}
  {\addcytoks{##1}}{#1}%
  \expandafter\mathrm\expandafter{\the\cytoks}%
}
\makeatother

\NewDocumentCommand{\rightmarginnote}{m}{%
    \checkoddpage%
    \ifoddpage%
        \marginpar{#1}%
    \else%
        \reversemarginpar%
        \marginpar{#1}%
    \fi
}

% Create a proof environment for results in appendix,
% that take a label of a theorem in the main text, and
% produce a proof
% \begin{proof}[Proof of \cref{the-label}]
% \phantomsection\label{the-label:proof}
% content
% \end{proof}
\NewDocumentEnvironment{proofof}{o}{%
    \IfValueTF{#1}{%
        % checks if \#1 is a defined command in which case
        % we call it
        \def\insideRestate{1}
        \ifcsname #1\endcsname
            \csname #1\endcsname*
        \fi
        \begin{proof}[Proof of \cref{#1} on page \pageref{#1}]
        \phantomsection\label{#1:proof}
    }{%
        \begin{proof}
    }
    \let\oldqedsymbol\qedsymbol
    \renewcommand\qedsymbol{\hyperref[#1]{\oldqedsymbol}}
}{%
    \end{proof}
    \renewcommand\qedsymbol{\oldqedsymbol}
}

% Refers to the proof of a result
\NewDocumentCommand{\proofref}{m}{%
    \ifdefined\insideRestate%
        \rightmarginnote{\vspace{0.6em}\ttfamily\small\hyperref[#1]{Go to \cref{#1} p.\pageref{#1:proof}}}%
    \else%
    % if appendices are present, then refer to the appendix
    % otherwise do nothing.
    % \hyperref[#1:proof]{Proof of \cref{#1}}
    \IfRefUndefinedExpandable{#1:proof}{}{%
        %\hfill\hyperref[#1:proof]{(Proof p.\pageref{#1:proof})}
        \rightmarginnote{\vspace{0.6em}\ttfamily\small\hyperref[#1:proof]{Go to proof of \cref{#1} p.\pageref{#1:proof}}}%
    }%
    \fi%
}

%
% Here are the definitions
% of mathematical macros.
%
%

\newcommand{\defined}{\mathrel{:=}}
\newcommand{\seqof}[2]{(#1)_{#2}}
\newcommand{\setof}[2]{\mathopen{\{} #1 \mid #2 \mathclose{\}}}
\newcommand{\set}[1]{\mathopen{\{} #1 \mathclose{\}}}

\newcommand{\Pfin}{\mathcal{P}_{\mathsf{fin}}}

\newcommand{\MSO}{\mathsf{MSO}}
\newcommand{\FO}{\mathsf{FO}}

\newcommand{\ind}[1]{\mathbf{1}_{#1}}
\newcommand{\bigO}{\mathcal{O}}


\newcommand{\topartial}{\rightharpoonup}
% functions

\newcommand{\dom}{\operatorname{dom}}

\NewDocumentCommand{\vcount}{ m }{ {{\#}#1} }
% sets

\NewDocumentCommand{\card}{ O{} m }{ \mathopen{|} #2  \mathclose{|}_{#1}}
\newcommand{\Nat}{\mathbb{N}}
\newcommand{\Rel}{\mathbb{Z}}
\newcommand{\Rat}{\mathbb{Q}}

% GOOD POLYNOMIALS
\newcommand{\CoveredPoly}{\mathsf{PolyKu}}
\newcommand{\CorrectPoly}{\mathsf{PolyRec}}

\NewDocumentCommand{\Monomials}{o}{\IfNoValueTF{#1}{\mathsf{Mon}}{\mathsf{Mon}[#1]}}
\NewDocumentCommand{\MaximalMonomials}{}{\mathop{\mathsf{maxmon}}}

% Differences of polynomials
\NewDocumentCommand{\Diff}{m m}{ \Delta_{#1}(#2) }


% Restrictions
%\newcommand{\restr}[2]{{#1}_{\kl[\restr]{| #2}}}
\newcommand{\restr}[2]{\kl[\restr]{[}{#1}\kl[\restr]{]}_{#2}}
\knowledge{\restr}{notion}

\NewDocumentCommand{\Poly}{O{}}{\mathsf{Poly}_{#1}}
\NewDocumentCommand{\SF}{O{}}{\mathsf{SF}_{#1}}
\NewDocumentCommand{\NPoly}{O{}}{\mathbb{N}\mathsf{Poly}_{#1}}
\NewDocumentCommand{\ZPoly}{O{}}{\mathbb{Z}\mathsf{Poly}_{#1}}
\NewDocumentCommand{\NSF}{O{}}{\mathbb{N}\mathsf{SF}_{#1}}
\NewDocumentCommand{\ZSF}{O{}}{\mathbb{Z}\mathsf{SF}_{#1}}
\NewDocumentCommand{\ZRat}{}{\mathbb{Z}\mathsf{Series}}
\NewDocumentCommand{\NRat}{}{\mathbb{N}\mathsf{Series}}
\NewDocumentCommand{\ZCommut}{}{\mathsf{Commut}}

\NewDocumentCommand{\commute}{O{\cdot}}{\{\!\{ #1 \}\!\}}





\NewDocumentCommand{\hoareleq}{}{\leq_{\mathcal{H}}}
\NewDocumentCommand{\hoarele}{}{<_{\mathcal{H}}}

\newcommand{\growth}[1]{\mathsf{growth}_{#1}}

\NewDocumentCommand{\BadPoly}{}{P_{\mathsf{bad}}}

\NewDocumentCommand{\translate}{m}{\tau_{#1}}

\NewDocumentCommand{\polysort}{}{\mathsf{sort}}

\input{globals/knowledges.kl}
\title{
    Star-free commutative $\Nat$-polyregular functions:
    correcting a result of Karhumäki.
}
\author{
    Lopez Aliaume
}
\date{\today}
\bibliography{globals/papers}

\newcommand{\acknowledge}{
    I would like to thank Gaëtan Douéneau-Tabot
    for mentionning this problem, checking 
    my counter-example, and supporting me in publishing
    this paper.
}

\newcommand{\makeabstract}{
    \begin{abstract}
        Deciding whether a $\Rel$-rational series
        can be computed as a $\Nat$-rational series
        is an open problem that was solved by Karhumäki in the
        specific case of $\Rel$-rational series that are commutative
        and compute a polynomial.
        We propose a short counter-example to this
        previous result on commutative $\Nat$-rational series,
        and prove a corrected version of the theorem,
        working of tdd
        We then discuss how the syntactic criterion could
        pave the way towards a decision procedure when the commutativity
        of the $\Rel$-rational series
        is not assumed.

        \begin{itemize}
            \item Deciding $\Nat$ inside $\Rel$.
            \item Counter example.
            \item Characterization of $\Nat$-polyregular.
            \item Characterization of star free $\Nat$-polyregular.
        \end{itemize}
    \end{abstract}
}



% DOCUMENT
\begin{document}

% FRONT MATTER
\maketitle
\makeabstract
\acknowledge


% MAIN MATTER
\section{Introduction}

Let us first state the result 
of \citeauthor{KARH77}, in terms of its
class $\CoveredPoly$ of polynomials defined as follows:
\begin{definition}[{\cite[Section 3, page 3]{KARH77}}]
    Let $\vec{X}$ be a finite tuple of indeterminate.
    The class $\CoveredPoly[\vec{X}]$
    is the class of polynomials $P \in \Rel[\vec{X}]$
    that are \kl{non-negative}
    and such that every \kl{maximal monomial} has a \kl{positive coefficient}.
    When the indeterminate are clear from the context, we write
    this class $\CoveredPoly$.
\end{definition}

\begin{theorem}[{\cite[Theorem 3.3, page 4]{KARH77}}] 
    \label{karh:thm}
    Let $\Sigma$ be a finite alphabet.
    The set of \kl{$\Nat$-rational polynomials} over $\Sigma$
    coincides with $\CoveredPoly[\seqof{X_a}{a \in \Sigma}]$.
\end{theorem}


\paragraph*{Context.}
There has been recent interest in $\Rel$ and $\Nat$-rational series exhibiting
\emph{polynomial growth} \cite{doueneau2021pebble,bojanczyk2022transducers}.
This computational model, in the counting case, can be traced back to a model
introduced by \textcite{schutzenberger1965finite}. It has been recently proven
that one can decide the growth rate of such functions, and decide whether they
are aperiodic in the case of $\Rel$-rational series \cite{LOPEZ23b}. It was
conjectured in the PhD thesis of one of the authors, that the membership
problem of $\Nat$-rational series among $\Rel$-rational series was decidable
under the assumption of polynomial growth. This conjecture was based on a
result of \textcite{KARH77}, which we will disprove in \cref{sec:c-example}. We
then provide in \cref{sec:proof} a corrected version of the theorem, together
with a valid proof. As a consequence, it does not invalidate the stated
conjecture.

\AP
\paragraph*{Rational series and regular languages.}
\begin{fact}
    The language $\setof{ w \in \Sigma}{ \card[a]{w} = \card[b]{w}}$
    is not regular.
\end{fact}

\begin{fact}
    regular languages are closed under intersections.
\end{fact}

TODO.

clasical results from \cite{berstel2011noncommutative}.
\begin{fact}
    \label{pre-image-regular:fact}
    The pre-image of a regular language by a $\Nat$-rational series
    is a regular language.
\end{fact}

\begin{fact}
    \label{polynomial-rational-polyreg:fact}
    polynomial + rational = unary polyregular.
\end{fact}


\AP
\paragraph*{Polynomials.}
Before stating the result, let us introduce the necessary vocabulary on
rational series, together with concrete examples so that no doubt is allowed as
to their meaning. A $\Rel$-rational series $f \colon \Sigma \to \Rel$ is
\intro{commutative} whenever $f(uv) = f(vu)$ for all $u,v \in \Sigma^*$. A
polynomial $P \in \Rel[\seqof{X_a}{a \in \Sigma}]$ is a \intro{$\Rel$-rational
polynomial} if there exists a \reintro{commutative} $\Rel$-rational series $f$
such that $f(w) = P(\seqof{ \card[a]{w} }{a \in \Sigma})$. Similarly, a
\intro{$\Nat$-rational polynomial} if the \reintro{commutative} $\Rel$-rational
series is in fact $\Nat$-rational.

\begin{example}
    \label{negative-not-nrat:ex}
    With $\Sigma \defined \set{1}$,
    the polynomials $X_1$, and $X_1^2 + 3$ are \kl{$\Nat$-rational polynomial},
    but $- X_1$ is a \kl{$\Rel$-rational polynomial} that is 
    not a \kl{$\Nat$-rational polynomial}.
\end{example}
\begin{proof}
    TODO:
    The polynomial $X_1$ is computed by the rational series,
    the polynomial $X_1^2 + 3$ is computed by,
    and the polynomial
    $- X_1$ cannot be $\Nat$-rational as $\Nat$-rational series
    have non-negative output.
\end{proof}

We saw in \cref{negative-not-nrat:ex}
a simple criterion to check whether a polynomial is $\Nat$-rational,
but being non-negative is not enough, as the following example
illustrates. Note that the proof scheme
of this example will be at the core of \cref{thm:counter-example}.

\begin{example}
    Over a binary alphabet $\Sigma \defined \set{a,b}$,
    the polynomial $P(X_a, X_b) \defined (X_a - X_b)^2$
    is a \kl{$\Rel$-rational polynomial} that is \kl{non-negative},
    but is
    not a \kl{$\Nat$-rational polynomial}.
\end{example}
\begin{proof}
    Assume by contradiction that
    $S$ is a $\Nat$-rational series computing $P$.
    Then, $S^{-1}(\set{0})$ should be a regular language
    (\cref{pre-image-regular:fact}),
    but $S^{-1}(\set{0}) = \setof{ w \in \Sigma }{ \card[a]{w} = \card[b]{w} }$
    is not a regular languge.
\end{proof}


\AP Let us now give some vocabulary on polynomials with multiple indeterminate
over $\Rel$. 
A \intro{monomial} is a product of variables and integers.
\begin{example}
    $XY$ is a \kl{monomial}, $3 X$ is a \kl{monomial}, 
    but $X + Y$ or $2X^2 + XY$ are not.
\end{example}
A polynomial $P \in \Rel[X_1, \dots, X_k]$ is \intro{non-negative}
if $P(n_1, \dots, n_k) \geq 0$ for all $n_1, \dots, n_k \geq 0$. Beware that we
do not consider negative values, as the numbers $n_i$ will ultimately count the
number of occurrences of a letter in a word.
\begin{example}
    The polynomial $(X - Y)^2$ is \kl{non-negative},
    and so is the polynomial $X^3$, but the polynomial
    $X^2 - 2X$ is not.
\end{example}
In a polynomial $P \in \Rel[X_1, \dots, X_k]$, a monomial 
is a \intro{maximal monomial} if it is a maximal element
for the visibility ordering of polynomials \emph{in $\Rat$}
among monomials with non-zero coefficients in $P$.\footnote{
    One could equivalently define maximal elements by removing the (non-zero)
    multiplicative constant of the monomials, and then
    considering the usual divisibility relation of $\Rel$ polynomials.
}
\begin{example}
    In the polynomial 
    $P(X,Y) \defined X^2 - 2XY + X^2 + X + Y$, the \reintro{maximal monomials}
    are $X^2$, $-2 XY$, and $X^2$.
\end{example}
A \kl{monomial} has \intro{positive coefficient} in a polynomial $P$
if its multiplicative constant is positive.
\begin{example}
    In the polynomial $P(X,Y) \defined (X - Y)^2$, 
    the \kl{positive monomials} are
    $X^2$ and $Y^2$.
\end{example}


\section{The Counter Example}
\label{sec:c-example}

Let us now introduce the counter example that was found trying to
replicate the original proof. Notice that the counter example
will use three indeterminate, because we will prove
in \cref{sec:proof} that \citeauthor{KARH77}’s theorem holds
on polynomial with at most two indeterminate.

\begin{definition}[Counter Example Polynomial]
    \label{def:bad-polynomial}
    We define $P(X,Y,Z) \defined Z (X + Y)^2 + 2 (X - Y)^2$.
\end{definition}

\begin{theorem}
    \label{thm:counter-example}
    Over $\Sigma \defined \set{a,b,c}$,
    there exists a polynomial $P \in \CoveredPoly$ that is not
    a \kl{$\Nat$-rational polynomial}. Namely,
    $P$ is the polynomial of \cref{def:bad-polynomial}.
\end{theorem}
\begin{proof}
    Let us write $X,Y,Z$ instead of $X_a, X_b, X_c$ to improve
    readability.
    It is clear that $P$ is \kl{non-negative}. We can develop
    the expression of $P$ to 
    obtain
    $P = ZX^2 + ZY^2 + 2ZXY + 2X^2 -4XY + 2Y^2$.
    The \kl{maximal monomials} of $P$
    are $ZX^2$, $ZY^2$, and $2ZXY$, all of which have
    \kl{positive coefficients}.

    Assume by contradiction that $P$ is a \kl{$\Nat$-rational polynomial}.
    There exists a \kl{commutative}
    $\Nat$-rational series $f \colon \Sigma \to \Nat$
    such that $P(\card[a]{w}, \card[b]{w}, \card[c]{w}) = f(w)$.
    Remark that $P(X,Y,Z) = 0$
    if and only if $Z(X+Y)^2 = -2 (X-Y)^2$. Hence,
    $P(X,Y,Z)=0$ if and only if $Z = 0$ and $X = Y$, or 
    $Z \neq 0$, and $X = Y = 0$.

    Now, let us consider the language $L \defined \setof{w}{ f(w) = 0}$. By the
    above computation, we conclude that $L = \setof{ w \in \set{a,b}^* }{
    \card[a]{w} = \card[b]{w} } \cup \set{ c }^*$.
    Because $L \cap \set{a,b}^*$ is notoriously not a regular language, we
    conclude that $L$ is not a regular language.
    However, $L = f^{-1}(\set{0})$, which is a regular language
    because $f$ is a $\Nat$-rational series
    (\cref{pre-image-regular:fact}).
\end{proof}

\begin{corollary}
    The result stated in \cite[Theorem 3.3]{KARH77}, restated
    in \cref{karh:thm}, is false
    for all polynomials with at least $3$ indeterminate.
\end{corollary}

As we will see in the next section, the characterization
of \kl{$\Nat$-rational polynomials} using the set $\CoveredPoly$
holds when the polynomials have at most $2$ indeterminate. This shows
that the examples in \cite{KARH77} are not invalidated, as they
all use at most two indeterminate.

\section{The Corrected Theorem}
\label{sec:proof}

Let us now provide another class of polynomials, $\CorrectPoly$. The definition
will involve fixing the value of indeterminate, for which we use the following
notation: $\restr{P(X,Y)}{X = 1}$ is the polynomial $P(1,Y)$. More generally,
if $\nu$ is a partial function from $\vec{X}$ to $\Nat$, the restriction
$\restr{P(\vec{X})}{\nu}$ is the polynomial with indeterminate $\vec{Y}
\defined \vec{X} - \dom(\nu)$ obtained by fixing the variables of the
domain of $\nu$.


\begin{definition}
    Let $\vec{X}$ be a finite tuple of indeterminate.
    The class $\CorrectPoly[\vec{X}]$ is the collection of
    polynomials $P \in \Rel[\vec{X}]$ such that
    $P$ is 
    such that, for every partial function $\nu \colon \vec{X} \to \Nat$,
    every \kl{maximal monomial} of
    $\restr{P}{\nu}$ has a \kl{positive coefficient}.
\end{definition}

First, let us remark that $\CorrectPoly \subseteq \CoveredPoly$, because
polynomials in $\CorrectPoly$ are \kl{non-negative}.

Before proving that $\CorrectPoly$ captures \kl{$\Nat$-rational polynomials},
let us first check that the counter example provided in
\cref{thm:counter-example} is not in $\CorrectPoly$. For that, notice that for
$P$ defined in \cref{def:bad-polynomial}, $P(X,Y,1) = 3X^2 + 3Y^2 - 2XY$, which
has a negative coefficient for a \kl{maximal monomial}, namely $-2XY$.

Let us now prove that \kl{$\Nat$-rational polynomials} are
in $\CorrectPoly$.
To simplify the reasoning, we will use results on $\Nat$-polyregular
functions, also known as, polyregular functions with unary output.

\begin{lemma}
    The \kl{$\Nat$-rational polynomials} are in $\CorrectPoly$.
\end{lemma}
\begin{proof}[Proof Idea]
    Let $P$ be a \kl{$\Nat$-rational polynomial}.
    Let $\Sigma$  be the finite alphabet such that
    $P \in \Rel[\seqof{X_a}{a \in \Sigma}]$,
    and $f \colon \Sigma^* \to \Nat$ be the \kl{commutative}
    $\Nat$-rational series computing $P$.
    Because $f$ has polynomial growth, and is a $\Nat$-rational series,
    it is a $\Nat$-polyregular function 
    (\cref{polynomial-rational-polyreg:fact}).
    In particular, 
    there exists a tuple $\vec{x}$ of free variables
    $\MSO$ formulas $\psi_1(\vec{x}), \dots, \psi_n(\vec{x})$,
    and positive coefficients $\alpha_1, \dots, \alpha_n \in \Nat$,
    such that
    \begin{equation*}
        f(w) = \left(\sum_{i = 1}^n \alpha_i \vcount{\psi_i}\right)(w) \quad .
    \end{equation*}

    \textbf{TODO: only look at words of the form $\prod_{a \in \Sigma} a^{k_a}$,
    because this is enough}.
    \textbf{TODO: this has positive maximal coefficients}.
    Furthermore, this remains true when restricting the language.
\end{proof}

\begin{fact}
    \label{fact:n-poly-n-poly}
    Let $P \in \Nat[\vec{X}]$. Then, $P$
    is a \kl{$\Nat$-rational polynomial}.
\end{fact}


The core of the upcoming proof of \cref{lem:correct-to-n-rat} relies on the
following observation: if a polynomial $P$ is written $X^p + Q$, where all
terms of $Q$ have degree less than $p$, then one can find some large enough $K$
so that $X^p - (X - K)^p + Q$ also has a positive maximal coefficient.
It works because $X^p - (X - K)^p = K X^{p-1} + \cdots$, hence
the maximal coefficient can be chosen arbitrarily large by increasing
$K$.
This
trick will work similarly in multivariate polynomials, and the overall
proof scheme works as follows: for a bounded number of possible
values of $X$, we can fix the variable, and have one less item to work on.
For large enough values of $X$ (typically greater than some $K$), we can guarantee that
$X^p - (X - K)^p + Q$ has positive maximal coefficients, hence
can be computed by a $\Nat$-rational series, while
$(X - K)^p$ is clearly computable by a $\Nat$-rational series,
hence $(X^p - (X - K)^p + Q) + (X - K)^p$ is computable
as the sum of two $\Nat$-rational series, and we have
concluded that our original polynomial was a
\kl{$\Nat$-rational polynomial}. There are a number of problems in generalizing
this proof scheme:
\begin{itemize}
    \item We will have to guarantee that every polynomial considered
        is non-negative
    \item We do not only need to check that the \kl{maximal monomials}
        have \kl{positive coefficient}, but that it is the case
        when fixing arbitrarily many variables! (this
        covers the first point too)
\end{itemize}

\begin{lemma}
    Let $\vec{X}$ be a tuple of indeterminate, $Y$ be a fresh indeterminate,
    $\vec{p} \colon \vec{X}Y \to \Nat$,
    and $M \defined \times \prod_{X \in \vec{X}} X^{p(X)}$.
    Then, for all $K \geq 0$,
    $\Delta_{Y,K} \defined Y^p M - (Y - K)^p M$ has \kl{maximal monomials} with
    \kl{positive coefficients} of weight at least $K$, regardless of
    the variables fixed, as long as the (potential)
    value of $Y$ is greater
    than $K$.
\end{lemma}
\begin{proof}
    Let us first remark that
    \begin{equation*}
        \Delta_{Y,K}
        = KY^{p-1} M + Q M
        \quad .
    \end{equation*}
    In particular, there is a unique \kl{maximal monomial} for $\Delta_{Y,K}$,
    which is precisely $K Y^{p-1} M$, because all monomials
    in $Q M$ divide (strictly) $Y^{p-1} M$.
    Let us fix a variable,
    $X$ to any (non-zero) value $p > 0$. Then
    The \kl{maximal monomial} of $\restr{\Delta_{Y,K}}{X = p}$
    is exactly $\restr{KY^{p-1}M}{X = p}$, which remains
    with a \kl{positive coefficient}, that is proportional to $K$.
    Let us fix $Y$ to some value $K + l$, with $l > 0$,
    then
    $\restr{\Delta_{Y,K}}{Y = K + l}
    = (K + l)^p M - l^p M
    = ((K + l) - l) (\sum_{i = 1}^{p-1} (K + l)^i l^{p-i}) M
    = K \alpha M
    $
    which remains a monomial with a \kl{positive coefficient},
    that is proportional to $K$.
\end{proof}


We can refine the above lemma to use it in the case where
we do not have a single
monomial.

\begin{lemma}
    \label{lem:delta-cool}
    Let $\vec{X}$ be a tuple of indeterminate, and $Y$ be a fresh indeterminate.
    Let $P \in \Rel[\vec{X}Y]$ be in $\CorrectPoly$, and $\alpha Y^p M$ be a
    \kl{maximal monomial} of $P$, with $\alpha > 0$.
    Then, there exists $K \geq 0$
    such that $\Delta_{K} \defined P - \alpha (Y - K)^p M$
    has \kl{positive coefficients} for \kl{maximal monomials}
    in any partial evaluation that satisfies $Y > K$ (if it
    gives a value to $Y$).
\end{lemma}
\begin{proof}
    Let $N$ be greater than the sum of the absolute values of all coefficients
    appearing in $P$, and let $K = 2 N$.

    Let us remark that
    \begin{equation*}
        \Delta_K = \underbrace{(P - \alpha Y^p M)}_{\text{ left hand side }}
        + \underbrace{\alpha (Y^p M - (Y - K)^p M)}_{\text{ right hand side }}
        \quad .
    \end{equation*}
    an that for every partial evaluation $\nu$ that sends $Y$ to a value
    greater than $K$,
    we have that $\restr{(Y^p M - (Y - K)^p M)}{\nu}$ has
    \kl{maximal monomials} with \kl{positive coefficients}
    proportional to $K$ (note that here ``proportional"
    implies that they are greater than $K$, because we multiply by positive
    integers).
    
    To conclude, it is just a game of describing how
    \kl{maximal monomials} of $\restr{\Delta_Y}{\nu}$
    can appear. If the \kl{maximal monomial} is obtained
    by summing an element of the left hand side with an element
    of the right hand side, it needs to be a \kl{maximal monomial}
    of the right hand side, and the large value of $K$ guarantees
    that the sum of coefficients is positive.
    If it only appears on the right hand side, it is even better.
    If it only appears on the left hand side, then
    in particular it means that the monomial does not
    divide $\restr{Y^p M}{\nu}$, hence has the same value
    in $\restr{\Delta_K}{\nu}$ or in $\restr{P}{\nu}$,
    which is guaranteed to be positive because $P \in \CorrectPoly$.

    In particular, we conclude from this that
    $\Delta_Y \geq 0$, on all tuples such that $Y > K$.
\end{proof}

Note that we cannot conclude in the above lemma that the result is in
$\CorrectPoly$, because we have a restriction on the range where the variable
$Y$ can be taken.

\begin{lemma}
    \label{lem:correct-to-n-rat}
    Polynomials in $\CorrectPoly$ are
    \kl{$\Nat$-rational polynomials}.
\end{lemma}
\begin{proof}[Proof Sketch]
    We prove by induction on the number $d$ of variables
    and the ``higman ordering of polynomials" (we look at the multiset
    of tuples of things ...) that for every selection
    $K_1, \dots, K_n$ of lower bounds for the different variables
    (that can be $0$),
    functions that are in $\CorrectPoly$ \emph{over tuples of size
    at least nanania} can be computed by a $\Nat$-rational series
    that outputs $0$ outside the domain and the correct value
    inside.

    For $d = 0$,
    the result is true because
    a constant function is a \kl{$\Nat$-rational polynomial}
    if and only if it is \kl{non-negative}.

    Let us now assume that $d > 0$, and in particular
    there is a variable $Y$.
    Consider $P \in \CorrectPoly[\vec{X} Y]$.
    If $Y$ appears at least one in $P$, then it must appear
    in one of the \kl{maximal monomials} of $P$. Otherwise,
    $P \in \CorrectPoly[\vec{X}]$ and we conclude using the
    induction hypothesis.
    Let $\alpha Y^p M$ be a \kl{maximal monomial} in $P$.
    Using \cref{lem:delta-cool},
    we know that there exists $K$ such that
    $P - \alpha (Y - K)^p M$ almost belongs to $\CorrectPoly$,
    and has a strictly smaller "degree multiset".
    By induction hypothesis,
    we conclude that
    the following function is a $\Nat$-rational series
    (beware that we have to consider the lower bounds to apply
    the induction hypothesis):
    \begin{equation*}
        A \defined
        \begin{cases}
            0 & \text{ if some variable is bounded by its } K_i \\
            0 & \text{ otherwise if  } Y \leq K \\
            P - \alpha(Y - K)^pM & \text{ otherwise }
        \end{cases}
        \quad .
    \end{equation*}

    Because $\alpha \geq 0$, it is an easy check
    that the following function is also a $\Nat$-rational series
    \begin{equation*}
        B \defined
        \begin{cases}
            0 & \text{ if some variable is bounded by its } K_i \\
            0 & \text{ otherwise if } Y \leq K \\
            \alpha(Y - K)^pM & \text{ otherwise }
        \end{cases}
    \end{equation*}

    Remark that by fixing some of the variables
    to values 
    below the bounds $K_i$ (and potentially $K$ for $Y$)
    we obtain a \emph{finite} number of polynomials with strictly fewer
    indeterminate.
    As a consequence, the following function is computable
    as a $\Nat$-rational series:
    \begin{equation*}
        C \defined
        \begin{cases}
            P & \text{ if some variable is bounded by its } K_i \\
            P & \text{ otherwise if } Y \leq K \\
            0 & \text{ otherwise }
        \end{cases}
        \quad .
    \end{equation*}

    We conclude because
    $P = A + B + C$, all of which are $\Nat$-rational series.
\end{proof}

\begin{theorem}
    The \kl{$\Nat$-rational polynomials} are exactly
    the polynomials in $\CorrectPoly$.
\end{theorem}

\begin{remark}
    In the proof, we actually conclude that
    the \kl{$\Nat$-rational polynomials}
    are exactly those that can be computed 
    by \kl{commutative}
    \emph{star-free} $\Nat$-polyregular functions.
\end{remark}

\begin{lemma}
    \label{lem:correct-covered-2}
    $\CorrectPoly[X,Y] = \CoveredPoly[X,Y]$.
\end{lemma}
\begin{proof}
    It is clear that $\CorrectPoly[X,Y] \subseteq \CoveredPoly[X,Y]$,
    by considering the empty valuation $\nu \colon \set{X,Y} \to \Nat$.
    For the converse inclusion, let us consider $P(X,Y)$
    that is \kl{non-negative}, such that the \kl{maximal monomials}
    have \kl{positive coefficients}.
   

    If we fix none of the variables, then the \kl{maximal monomials}
    have \kl{positive coefficients} by assumption. If we fix one of the
    variables, we can assume without loss of generality that we 
    fix $X = k$ for some $k \in \Nat$.
    Then $P(k,Y)$ is a \kl{non-negative} \emph{univariate} polynomial, 
    and therefore must have a positive leading coefficient
    (which is the unique \kl{maximal monomial} in this case)
    or be constant equal to 0. In both cases, the \kl{maximal monomials}
    have \kl{positive coefficients}.
    The same reasoning applies \emph{a fortiori} in the case where
    we fix the two indeterminate, leading to a constant polynomial.
\end{proof}

\section{Conclusion}
\label{sec:ccl}

\begin{itemize}
    \item Actually, we have an algorithm that allows
        us to \emph{decide} if a polynomial is $\Nat$-rational.
    \item Actually, we obtain a characterisation of
        commutative 
        $\Nat$-rational series of polynomial growth
        among $\Rel$-rational series.
    \item The question of $\Nat$-polyregular functions
        among $\Rel$-polyregular functions remains open
        because of the non-commutativity of the input.
    \item The question of $\Nat$-rational series (commutative)
        among $\Rel$-rational series remains open.
\end{itemize}

% BACKMATTER
\printbibliography

\appendix

\end{document}
