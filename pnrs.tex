%! TEX program = pdflatex
% WARNING: this is a generated file.
%
% Please do not edit this file directly. 
% - If you want to update the medatata of the paper (title, authors, abstract), please
%   edit the `paper-meta.yaml` file in the root of the repository.
% - If you want to update the content of the paper, please edit the latex files
%   in the `src` directory.
% - If you want to update the template itself (e.g., change the layout), please
%   edit the `templates/plain-article.tex` file instead.
\documentclass[9pt,a4paper,twosided]{article}

% we setup a custom geometry because the default one is too narrow
\usepackage{geometry}
\geometry{margin=3.5cm}

% utf-8 for old systems
\usepackage[utf8]{inputenc}
\usepackage[T1]{fontenc}

% babel for language settings
\usepackage[english]{babel}

% microtype for better typography
\usepackage{microtype}

\usepackage{todonotes}
\usepackage{lineno}


% math packages
\usepackage{amsmath,amsthm,amssymb,stmaryrd,thmtools,upgreek}

% configure some theorems
\newtheorem{theorem}{Theorem}
\newtheorem{lemma}[theorem]{Lemma}
\newtheorem{corollary}[theorem]{Corollary}
\newtheorem{proposition}[theorem]{Proposition}
\newtheorem{conjecture}[theorem]{Conjecture}
\newtheorem{assumption}[theorem]{Assumption}
\theoremstyle{definition}
\newtheorem{definition}[theorem]{Definition}
\newtheorem{remark}[theorem]{Remark}
\newtheorem{example}[theorem]{Example}



% graphics packages
\usepackage{graphicx}
\usepackage[obeyclassoptions,mode=tex]{standalone}
\usepackage{tikz}
\usetikzlibrary{backgrounds}
\usetikzlibrary{shapes.geometric}
\usetikzlibrary{decorations.markings}
\usetikzlibrary{positioning}
\usetikzlibrary{automata}
\usetikzlibrary{tikzmark}
\usetikzlibrary{patterns}
\usetikzlibrary{arrows}

\tikzset{every state/.style={minimum size=1pt}}
\usepackage{tikz-cd}

% ornaments
\usepackage{pgfornament}


% links inside the document
\usepackage{hyperref}
\usepackage[capitalise,noabbrev,nameinlink]{cleveref}
\Crefname{assumption}{Assumption}{Assumptions}
\usepackage[composition,hyperref,xcolor,cleveref]{knowledge}
\knowledgeconfigure{notion}



% Tables 
\usepackage{booktabs}
\usepackage{varwidth}

% Algorithms
\usepackage{algorithm2e}
\Crefname{algocfline}{Algorithm}{Algorithms}
\crefname{algocfline}{Algorithm}{Algorithms}
\crefname{algocf}{Algorithm}{Algorithms}
\Crefname{algocf}{Algorithm}{Algorithms}

% Packages for macro definitions
\usepackage{xparse}
\usepackage{xpatch}
\usepackage{tokcycle}
\usepackage{ifthen}

% Proofs
\usepackage{bussproofs}

% Colors 
\usepackage{ensps-colorscheme}
% parametrize knowledge colors and style
% intro => A4 + emph 
% kl    => normal color, normal size not emph
\knowledgestyle{intro notion}{color={A2}, emphasize}
\knowledgestyle{notion}{color={B1}}
\hypersetup{
    colorlinks=true,
    breaklinks=true,
    linkcolor=A4,
    citecolor=A4,
    urlcolor=A4,
    filecolor=A4,
}

% Knowledge logo
\newcommand{\klogo}{%
\begin{tikzpicture}[scale=0.2,line/.style={draw, line width=0.2pt, line cap=round, line join=round}]
\coordinate (A00) at (0,0);
\coordinate (A01) at (0,1);
\coordinate (A10) at (1,0);
\coordinate (B10) at (1,0.2);
\coordinate (B01) at (0.2,1);

\coordinate (C01) at (0.4,0.7);
\coordinate (C10) at (0.7,0.4);
\coordinate (C12) at (0.4,1.2);
\coordinate (C21) at (1.2, 0.4);
\coordinate (C22) at (1.2, 1.2);

\coordinate (D00) at (C10);
\coordinate (D01) at (0.8,0.5);
\coordinate (D10) at (0.8,0.3);

\coordinate (E01) at (0.3,0.7);
\coordinate (E10) at (0.5,0.7);

\draw[line] (B01) -- (A01) -- (A00) -- (A10) -- (B10);
\draw[line] (C01) -- (C12) -- (C22) -- (C21) -- (C10);

\draw[line] (D01) -- (D00) -- (D10);
\draw[line] (E01) -- (E10);

\end{tikzpicture}%
}



% we include whatever the user wants to include in the header

% we include libraries (tex files) usually written in the `lib` directory
%
% Here are the definitions
% of mathematical macros.
%
%

\newcommand{\defined}{\mathrel{:=}}
\newcommand{\eqdef}{\stackrel{\mathsf{def}}{=}}
\newcommand{\seqof}[2]{(#1)_{#2}}
\newcommand{\setof}[2]{\mathopen{\{} #1 \mid #2 \mathclose{\}}}
\newcommand{\set}[1]{\mathopen{\{} #1 \mathclose{\}}}

\newcommand{\ppcm}{\mathop{\mathrm{lcm}}}

\NewDocumentCommand{\card}{ O{} m }{ \mathopen{|} #2  \mathclose{|}_{#1}}

\newcommand{\Nat}{\mathbb{N}}
\newcommand{\Rel}{\mathbb{Z}}
\newcommand{\Rat}{\mathbb{Q}}
\newcommand{\Real}{\mathbb{R}}

\newcommand{\Pfin}{\mathcal{P}_{\mathsf{fin}}}

\newcommand{\MSO}{\mathsf{\kl[\MSO]{MSO}}}
\knowledge{\MSO}{notion}

\newcommand{\FO}{\mathsf{\kl[\FO]{FO}}}
\knowledge{\FO}{notion}


\newcommand{\ind}[1]{\mathbf{1}_{#1}}
\newcommand{\bigO}{\mathcal{O}}

\NewDocumentCommand{\Span}{m}{\mathop{\kl[\Span]{\mathsf{Span}_{#1}}}}
\knowledge{\Span}{notion}

\NewDocumentCommand{\const}{ m }{\mathop{\mathsf{const}_{#1}}}


\newcommand{\topartial}{\rightharpoonup}
\newcommand{\tosurj}{\twoheadrightarrow}
% functions

\newcommand{\dom}{\operatorname{dom}}

\NewDocumentCommand{\vcount}{ m }{ {{\#}#1} }


% GOOD POLYNOMIALS
\newcommand{\CoveredPoly}{\kl[\CoveredPoly]{\mathsf{PolyNNeg}}}
\newcommand{\CorrectPoly}{\kl[\CorrectPoly]{\mathsf{PolyRec}}}
\knowledge{\CoveredPoly}{notion}
\knowledge{\CorrectPoly}{notion}

\NewDocumentCommand{\Monomials}{o}{\mathop{\kl[\Monomials]{\IfNoValueTF{#1}{\mathsf{Mon}}{\mathsf{Mon}[#1]}}}}
\NewDocumentCommand{\MaximalMonomials}{}{\mathop{\kl[\MaximalMonomials]{\mathsf{maxmon}}}}
\knowledge{\Monomials}{notion}
\knowledge{\MaximalMonomials}{notion}



% Restrictions
%\newcommand{\restr}[2]{{#1}_{\kl[\restr]{| #2}}}
\newcommand{\restr}[2]{\withkl{\kl[\restr]}{\cmdkl{[}{#1}\cmdkl{]}_{#2}}}
\knowledge{\restr}{notion}

\NewDocumentCommand{\Poly}{O{}}{\kl[\Poly]{\mathsf{Poly}_{#1}}}
\NewDocumentCommand{\SF}{O{}}{\kl[\SF]{\mathsf{SF}_{#1}}}

% Add commands for Mealy / Sequential / Rational / Regular
\NewDocumentCommand{\Mealy}{O{}}{\kl[\Mealy]{\mathsf{Mealy}_{#1}}}
\NewDocumentCommand{\Sequential}{O{}}{\kl[\Sequential]{\mathsf{Seq}_{#1}}}
\NewDocumentCommand{\Rational}{O{}}{\kl[\Rational]{\mathsf{Rat}_{#1}}}
\NewDocumentCommand{\Regular}{O{}}{\kl[\Regular]{\mathsf{Reg}_{#1}}}
\knowledge{\Mealy}{notion}
\knowledge{\Sequential}{notion}
\knowledge{\Rational}{notion}
\knowledge{\Regular}{notion}

% Do the same for the aperiodic variants of Mealy, Sequential, Rational and Regular
\NewDocumentCommand{\AMealy}{O{}}{\kl[\AMealy]{\mathsf{AMealy}_{#1}}}
\NewDocumentCommand{\ASequential}{O{}}{\kl[\ASequential]{\mathsf{ASeq}_{#1}}}
\NewDocumentCommand{\ARational}{O{}}{\kl[\ARational]{\mathsf{ARat}_{#1}}}
\NewDocumentCommand{\ARegular}{O{}}{\kl[\ARegular]{\mathsf{AReg}_{#1}}}
\knowledge{\AMealy}{notion}
\knowledge{\ASequential}{notion}
\knowledge{\ARational}{notion}
\knowledge{\ARegular}{notion}





\NewDocumentCommand{\NPoly}{O{}}{\kl[\NPoly]{\mathbb{N}\mathsf{Poly}_{#1}}}
\NewDocumentCommand{\ZPoly}{O{}}{\kl[\ZPoly]{\mathbb{Z}\mathsf{Poly}_{#1}}}
\NewDocumentCommand{\NSF}{O{}}{\kl[\NSF]{\mathbb{N}\mathsf{SF}_{#1}}}
\NewDocumentCommand{\ZSF}{O{}}{\kl[\ZSF]{\mathbb{Z}\mathsf{SF}_{#1}}}
\NewDocumentCommand{\ZRat}{}{\kl[\ZRat]{\mathbb{Z}\mathsf{Series}}}
\NewDocumentCommand{\NRat}{}{\kl[\NRat]{\mathbb{N}\mathsf{Series}}}
\NewDocumentCommand{\ZCommut}{}{\kl[\ZCommut]{\mathsf{Commut}}}

\knowledge{\Poly}{notion}
\knowledge{\SF}{notion}
\knowledge{\NPoly}{notion}
\knowledge{\ZPoly}{notion}
\knowledge{\NSF}{notion}
\knowledge{\ZSF}{notion}
\knowledge{\ZRat}{notion}
\knowledge{\NRat}{notion}
\knowledge{\ZCommut}{notion}

\NewDocumentCommand{\commute}{O{\cdot}}{\{\!\{ #1 \}\!\}}





\NewDocumentCommand{\hoareleq}{}{\leq_{\mathcal{H}}}
\NewDocumentCommand{\hoarele}{}{<_{\mathcal{H}}}

%\newcommand{\growth}[1]{\mathsf{growth}_{#1}}

\NewDocumentCommand{\BadPoly}{}{\kl[\BadPoly]{\mathsf{P}_{\mathsf{bad}}}}
\knowledge{\BadPoly}{notion}

\NewDocumentCommand{\translate}{m}{\mathop{\kl[\translate]{\tau_{#1}}}}
\knowledge{\translate}{notion}

% Differences of polynomials
\NewDocumentCommand{\Diff}{m m}{ \mathop{\kl[\Diff]{\Delta_{#1}}}(#2) }
\knowledge{\Diff}{notion}

\NewDocumentCommand{\polysort}{}{\mathsf{sort}}
\NewDocumentCommand{\polysum}{}{\mathsf{sum}}


\NewDocumentCommand{\npolyleq}{ O{} }{\mathrel{\kl[\npolyleq]{\preceq_{\Nat #1}}}}
\knowledge{\npolyleq}{notion}
\NewDocumentCommand{\zpolyequiv}{ O{} }{\mathrel{\kl[\zpolyequiv]{\equiv_{\Rel #1}}}}
\knowledge{\zpolyequiv}{notion}

\NewDocumentCommand{\app}{ m m }{\mathop{{#2} \mathrel{\kl[\app]{\triangleright}} {#1}}}
\knowledge{\app}{notion}

\NewDocumentCommand{\Res}{}{ \mathop{\kl[\Res]{\mathsf{Res}}}}
\knowledge{\Res}{notion}


\NewDocumentCommand{\resleq}{ m m }{\mathrel{\kl[\resleq]{\leq_{{#1},{#2}}}}}
\knowledge{\resleq}{notion}
\NewDocumentCommand{\resleqsf}{ m m }{\mathrel{\kl[\resleqsf]{\leq_{{#1},{#2}}^{\mathsf{sf}}}}}
\knowledge{\resleqsf}{notion}

\NewDocumentCommand{\prefleq}{}{\mathrel{\kl[\prefleq]{\sqsubseteq_{\mathsf{pref}}}}}
\knowledge{\prefleq}{notion}
\NewDocumentCommand{\prefle}{}{\sqsubset_{\mathsf{pref}}}
\NewDocumentCommand{\preforth}{}{\bot_{\mathsf{pref}}}


\NewDocumentCommand{\aTransd}{}{\mathcal{A}}


\NewDocumentCommand{\ModuloTypes}{O{}}{\kl[\ModuloTypes]{{#1}\mathsf{Types}}}
\knowledge{\ModuloTypes}{notion}
\NewDocumentCommand{\moduloType}{O{}}{\kl[\moduloType]{{#1}\mathsf{type}}}
\knowledge{\moduloType}{notion}


\NewDocumentCommand{\floor}{m}{\lfloor #1 \rfloor}


% Name of examples


\NewDocumentCommand{\BadExOk}{}{\mathsf{f}}
\NewDocumentCommand{\BadExKo}{}{\mathsf{g}}

\NewDocumentCommand{\pbinom}{ m m }{\binom{#1}{#2}_{\kl[\pbinom]{p}}}
\knowledge{\pbinom}{notion}


\input{globals/knowledges.kl}
%% CUSTOM MACROS, META MACROS, AND CONFIG %%


% knowledge management
\knowledgestyle{intro notion}{color={A5}, emphasize}
\knowledgestyle{notion}{color={A4}}
\knowledgeconfigure{anchor point color={A2},
                    anchor point shape=corner}
\knowledgestyle{intro unknown}{color={D3}, emphasize}
\knowledgestyle{intro unknown cont}{color={C3}, emphasize}
\knowledgestyle{kl unknown}{color={D2}}
\knowledgestyle{kl unknown cont}{color={C2}}

\hypersetup{
    colorlinks=true,
    anchorcolor=A2,
    citecolor=A4,
    linkcolor=A4,
    urlcolor=A3,
    filecolor=A3,
    runcolor=D2,
    menucolor=D2,
}

\NewDocumentCommand{\klscope}{ o m }{
    \withkl{\kl[#1]}{#2}
}

% Common theorem styles
\theoremstyle{plain}
\newtheorem{theorem}{Theorem}[section]
\newtheorem{lemma}[theorem]{Lemma}
\newtheorem{proposition}[theorem]{Proposition}
\newtheorem{corollary}[theorem]{Corollary}
\newtheorem{fact}[theorem]{Fact}
\theoremstyle{definition}
\newtheorem{definition}[theorem]{Definition}
\newtheorem{example}[theorem]{Example}
\theoremstyle{remark}
\newtheorem{remark}[theorem]{Remark}
\newtheorem{exercise}[theorem]{Exercise}
\newtheorem{claim}[theorem]{Claim}
\newtheorem{tool}[theorem]{Tool}
\newtheorem{conjecture}[theorem]{Conjecture}


% Upgreek letters
\makeatletter
\newcommand\mathgr[1]{\tokcycle
  {\addcytoks{##1}}
  {\processtoks{##1}}
  {\ifcsname up\expandafter\@gobble\string##1\endcsname
   \addcytoks[1]{\csname up\expandafter\@gobble\string##1\endcsname}%
    \else\addcytoks{##1}\fi}
  {\addcytoks{##1}}{#1}%
  \expandafter\mathrm\expandafter{\the\cytoks}%
}
\makeatother

% Comment on a definition/text
\NewDocumentCommand{\comment}{m}{%
    \footnote{#1}%
}

% Citation configuration
\AtEveryBibitem{
    % Removes issn, isbn, and
    % unwanted items from the bibliography
	\clearfield{issn}
	\clearfield{isbn}
	\clearfield{archivePrefix}
	\clearfield{arxivId}
	\clearfield{pmid}
	\clearfield{eprint}
}

% Cite a knowledge
\NewDocumentCommand{\citek}{ o m }{
    \IfNoValueTF{#1}{
        \cite{m}
    }{
        \cite[{\kl(#1)[#2]}]{#1}
    }
}


% We include the title and author information based on the 
% `paper-meta.yaml` file.
 
\title{Commutative \(\Nat\)-rational Series of Polynomial Growth}

\author{
Aliaume Lopez\thanks{University of Warsaw}
}

% For the date, we first check if the user has provided a date,
% and otherwise use the git meta inforamtion (if available).
\date{2025-09-22 10:31:27
+0200\footnote{00d33fe682fe4f73d3af284af9d9a8d732af14d2 -- branch theoretics at git@github.com:AliaumeL/polynomial-n-rational-series.git}}

\newcommand{\repositoryUrl}{\url{https://github.com/AliaumeL/polynomial-n-rational-series}}

\knowledge{notion}
 | kl-usage

% Now, we create the document itself.
\begin{document}
% Generate the title page
\maketitle
% Print the abstract
\begin{abstract}
    This paper studies which functions computed by \(\Rel\)-weighted
    automata can be realised by \(\Nat\)-weighted automata, under two
    extra assumptions: commutativity (the order of letters in the input
    does not matter) and polynomial growth (the output of the function
    is bounded by a polynomial in the size of the input). We leverage
    this effective characterization to decide whether a function
    computed by a commutative \(\Nat\)-weighted automaton of polynomial
    growth is star-free, a notion borrowed from the theory of regular
    languages that has been the subject of many investigations in the
    context of string-to-string functions
    \paragraph{Keywords:}
    Rational series, Weighted automata, Polyregular
function, Commutative function
\paragraph{Repository:} \repositoryUrl
\paragraph{conference} \hyperlink{https://doi.org/10.4230/LIPIcs.STACS.2025.67}{STACS'25} \cite{lopez:LIPIcs.STACS.2025.67}
\end{abstract}

\klogo\ This document uses \href{https://ctan.org/pkg/knowledge}{knowledge}:
\kl[kl-usage]{notion} points to its \intro[kl-usage]{definition}. 

% Include the content of the paper
\lowcotwo~This is a low-co2 research paper:
\lowcotwourl[\lowcotwoversion]. This research was developed, written,
submitted and presented without the use of air travel.

%! TeX program = xelatex
%! lang = en-US
% MAIN MATTER
\section{Introduction}
\label{introduction:sec}

\AP Given a semiring $\mathbb{S}$, and a finite alphabet $\Sigma$, the class of
\kl{(noncommutative) $\mathbb{S}$-rational series} is defined as functions from
$\Sigma^*$ to $\mathbb{S}$ that are computed by \kl{$\mathbb{S}$-weighted
automata} \cite{BERE10}. This computational model is a generalization of the
classical notion of non-deterministic finite automata to the weighted setting,
where transitions are labeled with elements of $\mathbb{S}$. The semantics of
\kl{$\mathbb{S}$-weighted automata} on a given word $w$ is defined by the sum
over all accepting runs reading $w$, of the product of the weights of the
transitions taken along this run. In this paper, we are interested in the case
where $\mathbb{S}$ equals $\Nat$ or $\Rel$, hence, in $\Nat$-rational series
($\intro*\NRat$) and $\Rel$-rational series ($\intro*\ZRat$). It is clear that
$\NRat$ is a proper subclass of $\ZRat$, and a longstanding open problem is to
provide an algorithm that decides whether a given $\ZRat$ is in $\NRat$
\cite{KARH77}. 

\begin{problem}
    \label{n-in-z-rat:problem}
    Input: A $\ZRat$ $f$. Output: Is $f$ in $\NRat$?
\end{problem}

\Cref{n-in-z-rat:problem} recently received attention in the context of
\kl{polyregular functions}, a computational model that aims to
generalize the theory of regular languages to the setting of string-to-string
functions \cite{BOJA18}. In the case of regular languages, \emph{star-free
languages} form a robust subclass of regular languages described equivalently
in terms of first order logic \cite{MNPA71}, counter-free automata
\cite{MNPA71}, or aperiodic monoids \cite{SCHU65}. Analogously, there exists a
\emph{star-free} fragment of \kl{polyregular functions} called \kl{star-free
polyregular functions} \cite{BOJA18}. One open question in this area is to
decide whether a given \kl{polyregular function} is \kl{star-free}.

\begin{problem}
    \label{sf-polyregular:problem}
    Input: A \kl{polyregular function} $f$. Output: Is $f$ \kl{star-free}?
\end{problem}

\AP In order to approach decision problems on \kl{polyregular functions},
restricting the output alphabet to a single letter has proven to be a fruitful
method \cite{DOUE21,DOUE22}. Because words over a unary alphabet are
canonically identified with natural numbers, unary output \kl{polyregular
functions} are often called \kl{$\Nat$-polyregular functions} ($\NPoly$), and
their \emph{star-free} counterpart \kl{star-free $\Nat$-polyregular functions}
($\NSF$). Coincidentally, \kl{polyregular functions} with unary output form
subclass of \kl{$\Nat$-rational series}, namely the classf of
\kl{$\Nat$-rational series} of \emph{polynomial growth}, i.e. the output of the
function is bounded by a polynomial in the size of the input. In \cite{CDTL23},
the authors introduced the class of \kl{$\Rel$-polyregular functions}
($\ZPoly$) as a subclass of \kl{$\Rel$-rational series} that generalizes
\kl{$\Nat$-polyregular functions} by allowing negative outputs, and showed that
membership in the \emph{star-free} subclass $\ZSF$ inside $\ZPoly$ is decidable
\cite[Theorem V.8]{CDTL23}. Although this could not be immediately leveraged to
decide $\NSF$ inside $\NPoly$, it was conjectured that $\NPoly \cap \ZSF =
\NSF$ \cite[Conjecture 7.61]{DOUE23}. It was believed that understanding the
membership problem of $\NPoly$ inside $\ZPoly$, that is, a restricted version
of \cref{n-in-z-rat:problem}, would be a key step towards proving $\NPoly \cap
\ZSF = \NSF$, which itself would give hope in designing an algorithm for
\cref{sf-polyregular:problem}. We illustrate in
\cref{previously-known-inclusions:fig} the known inclusions and related open
problems between the discussed classes of functions.

\begin{figure}
    \centering
    \includestandalone[height=5cm]{tikz/class-inclusions}
    \caption{
        Decidability and inclusions of classes of functions,
        arranged along two axes. The first one is the complexity
        of the output alphabet ($\Rel$, $\Nat$, $\Sigma$). The second
        one is the allowed computational power
        (\kl{star-free polyregular functions}, \kl{polyregular functions}, 
        \kl{rational series}).
        Arrows denote strict inclusions,
        and effectiveness (both in terms of decidability and of effective
        representation) is represented by thick double arrows. Inclusions that are
        suspected to be effective are represented using a dashed arrow together with a
        question mark.
    }
    \label{previously-known-inclusions:fig}
\end{figure}


\subparagraph*{Contributions.} We prove that $\NSF = \NPoly \cap \ZSF$
\cite[Conjecture 7.61]{DOUE23} and design an algorithm that decides whether a
function in $\ZPoly$ is in $\NPoly$ \cite[Open question 5.55]{DOUE23}, both
under the extra assumption of \emph{commutativity}, that is, assuming that the
function is invariant under the permutation of its input. As a consequence, the
upper left square of \cref{previously-known-inclusions:fig} has all of its
arrows decidable and with effective conversion procedures under this extra
assumption. Because \kl{$\Rel$-rational series} with \emph{polynomial growth}
are exactly \kl{$\Rel$-polyregular functions} \cite{CDTL23}, this can be seen
as decision procedure for \cref{n-in-z-rat:problem} under the extra assumption
of \emph{commutativity} and \emph{polynomial growth}. Similarly, our results
provide an algorithm for \cref{sf-polyregular:problem} under the extra
assumption of \emph{commutativity} and \emph{unary output alphabet}.

As an intermediate step, we provide a complete and decidable characterization
of polynomials in $\Rat[\vec{X}]$ that can be computed using $\NRat$ (resp.
$\ZRat)$. These characterizations uncover a fatal flaw in the proof of a former
characterization of such polynomials \cite[Theorem 3.3, page 4]{KARH77}.
Furthermore, these characterizations provide effective descriptions of
polynomials that can be expressed in $\ZRat$ as those obtained using integer
combinations of products of \emph{binomial coefficients} (called \kl{integer
binomial polynomials}, defined page \kpageref{integer binomial polynomial}) and
similarly for $\NRat$ by introducing the notion of \kl{strongly natural
binomial polynomials} (defined page \kpageref{strongly natural binomial
polynomial}), which we believe has its own interest. Finally, these
characterizations demonstrate that polynomials expressible by $\ZRat$ (resp.
$\NRat$) are exactly those expressible by $\ZSF$ (resp. $\NSF$), that is,
polynomials are inherently \emph{star free} functions.

\subparagraph*{Outline of the paper.} In \cref{preliminaries:sec}, we provide a
combinatorial definition of \kl{$\Nat$-polyregular functions} (resp.
\kl{$\Rel$-polyregular functions}), show that one can decide if a function $f
\in \ZPoly$ is \kl{commutative} (\cref{decidable-commutative-poly:lemma}). In
\cref{polynomials:sec}, we provide a counterexample to the flawed result of
\cite[Theorem 3.3, page 4]{KARH77} (\cref{thm:counter-example}), and correct it
by providing effective characterizations of polynomials computed by $\ZRat$
(\cref{integer-binomial-polynomials:cor}) and $\NRat$
(\cref{decide-rat-poly-npoly:cor}). Finally, in \cref{beyond-polynomials:sec},
answer positively to \cite[Open question 5.55]{DOUE23}
(\cref{decidable-n-poly:thm}) and to \cite[Conjecture 7.61]{DOUE23}
(\cref{zsf-npoly-nsf:thm}), both under the extra assumption of
\emph{commutativity}.

%! TeX program = xelatex
%! lang = en-US
\section{Preliminaries}
\label{preliminaries:sec}

\AP In the rest of this paper, $\Rel$ (resp. $\Nat$) denotes the set of
integers (resp. non-negative integers). The capital letters $\Sigma,\Gamma$
denotes fixed alphabets, i.e. a finite set of letters, and $\Sigma^*, \Gamma^*$
(resp. $\Sigma^+, \Gamma^+$) are the set of words (resp. non-empty words) over
$\Sigma, \Gamma$. The empty word is written $\varepsilon \in \Sigma^*$. When $w
\in \Sigma^*$ and $a \in \Sigma$, we let $\card{w} \in \Nat$ be the length of
$w$, and $\card[a]{w}$ be the number of occurrences of $a$ in $w$. We assume
that the reader is familiar with the basics of automata theory, in particular
the notions of monoid morphisms, idempotents in monoids, monadic second-order
($\MSO$) logic and first-order ($\FO$) logic over finite words (see e.g.
\cite{THOM97}). Let us recall that a monoid $M$ is \intro(monoid){aperiodic}
whenever for all $x \in M$, there exists $n \in \Nat$ such that $x^{n+1} =
x^n$.

\AP We also assume the reader to be familiar with rational series and weighted
automata, and refer to \cite{BERE10} for a comprehensive introduction. Let us
recall that one of the equivalent definitions of a $K$-rational series (where
$K$ is a ring) is given by weighted automata, which are non-deterministic
finite automata with transitions labelled by elements of $K$, and whose output
on a given word is the sum over all accepting paths of the products of the
weights along the path.

\AP We use the notation $\commute \colon \Sigma^* \to \Nat^\Sigma$ for the map
that counts occurrences of every letter in the input word (that is, computes
the Parikh vector) namely: $ \commute[w] \defined (a \mapsto \card[a]{w})$.
Given a set $X$, a function $f \colon \Sigma^* \to X$ is \intro{commutative}
whenever for all $u \in \Sigma^*$, for all permutations $\sigma$ of $\set{1,
\dots, \card{w}}$, $f(\sigma(u)) = f(u)$. Equivalently, it is
\reintro{commutative} whenever there exists a map $g \colon \Nat^\Sigma \to X$
such that $g \circ \commute = f$. A function $g \colon \Nat^n \to X$ is
\intro{represented} by a function $f \colon \Sigma^* \to X$ if there exists a
map $\eta \colon \set{1, \dots, n} \to \Sigma$ such that $g \circ (v \colon
\Nat^\Sigma \mapsto \seqof{v(\eta(k))}{1 \leq k \leq n} \colon \Nat^n) \circ
\commute = f$. This notion will be useful to formally state that a polynomial
``is'' a \kl{polyregular function}.

\subsection{Polyregular Functions}

\AP Because the functions of interest in this paper have output in $\Nat$ or
$\Rel$, we will only provide the definition of \kl{polyregular functions} for
these two output semigroups, and we refer the reader to \cite{BOKL19} for the
general definition of \kl{polyregular functions}. The following
\cref{nat-rel-poly:def} is one of the equivalent definitions of \cite{CDTL23},
and is similar in shape to the \emph{finite counting automata} introduced by
\cite{SCHU62}. 

\begin{definition}[$\Rel$-polyregular functions {\cite{CDTL23}}]
    \label{nat-rel-poly:def}
    Let $M$ be a finite monoid, $\mu \colon \Sigma^* \to M$
    be a morphism, $k \in \Nat$, and 
    $\pi \colon M^k \to \Rel$ be a function.
    The \intro{$\Rel$-polyregular function}
    $\pi^\dagger \colon \Sigma^* \to \Rel$
    is computed as follows:
    \begin{equation*}
        \pi^\dagger (w) \defined
        \sum_{w = u_1 \cdots u_k} \pi(\mu(u_1), \dots, \mu(u_k))
        \quad .
    \end{equation*}
    When $\pi$ has co-domain $\Nat$, the function $\pi^\dagger$
    is called \intro{$\Nat$-polyregular}.
    We call $\pi$ the \emph{production function} of $\pi^\dagger$.
    When the monoid $M$ is \kl(monoid){aperiodic}
    then
    the function $f$ is \intro{star-free $\Rel$-polyregular}
    (resp. \intro{star-free $\Nat$-polyregular}).
\end{definition}


\begin{example}
    \label{size-of-word-nsf:ex}
    The map $f \colon w \mapsto \card{w} + 1$
    belongs to $\NSF$.
\end{example}
\begin{proof}
    Let us define $M \defined (\set{1}, \times)$ which is 
    a finite \kl{aperiodic monoid}, $\mu \colon \Sigma^* \to M$
    defined by $\mu(w) \defined 1$, and
    $\pi \colon M^2 \to \Nat$
    that is the constant function equal to $1$.
    We check that for all $w \in \Sigma^*$:
    \begin{equation*}
        \pi^\dagger(w)
        =
        \sum_{uv = w} 1
        =
        \card{w} + 1
        = f(w)
        \quad . 
        \qedhere
    \end{equation*}
\end{proof}

\AP The \intro{growth rate} of a function $f \colon \Sigma^* \to \Rel$ is
defined as the minimal $k$ such that $\card{f(w)} =
\bigO\left(\card{w}^k\right)$. If such a $k$ exists, we say that the function
$f$ has \intro{polynomial growth}. We write $\reintro*\ZPoly[k]$ (resp. $\reintro*\NPoly[k]$) for
the functions in $\ZPoly$ (resp. $\NPoly$) that have growth rate at most $k$.
It is clear that a function $f \in \ZPoly$ defined using a production function
$\pi \colon M^k \to \Rel$ has \kl{growth rate} at most $k-1$, and more
generally it is known that \kl{polyregular functions} have \kl{polynomial
growth} \cite{BOJA22}. Not all \kl{$\Nat$-rational series} have \kl{polynomial
growth}, a typical example being the function $a^n \mapsto 2^n$.

\AP Let us briefly recall in the following lemmas alternative characterizations
of $\NPoly$ (resp. $\ZPoly$), some of which will be helpful in the upcoming
analysis of their \kl{commutative} counterpart, and others that provide a nicer
syntax to construct examples. These characterizations were previously known,
and can be found for instance in \cite{DOUE23,CDTL23}. For every language $L
\subseteq \Sigma^*$ we define $\ind{L} \colon \Sigma^* \to \set{0,1}$ as the
indicator function of $L$. Given an $\MSO$ (resp. $\FO$) formula with
first-order free variables $\varphi(\vec{x})$, we define
$\vcount{\varphi(\vec{x})}$ as the function from $\Sigma^*$ to $\Nat$ that maps
a word $w$ to the number of assignments $\nu \colon \vec{x} \to w$, such that
$w, \nu \models \varphi(\vec{x})$. If $X$ is a set of functions from $\Sigma^*$
to $\Rel$, and a subset $\mathbb{S}$ of $\Rel$, we define
$\intro*\Span{\mathbb{S}}(X)$ as the smallest subset of functions from
$\Sigma^* \to \Rel$ containing $X$ and closed under (finite) sums and
multiplication with elements $s \in \mathbb{S}$.

\begin{lemma}
    \label{polynomial-rational-polyreg:fact}
    Let $f \in \NRat$ (resp. $f \in \ZRat$), then
    the following are equivalent:
    \begin{enumerate}
        \item $f \in \NPoly$ (resp. $\ZPoly$);
        \item $f$ is a \kl{polyregular function} with output
            in $\set{1}^*$,
            post-composed with $\polysum$
            (resp. with output in $\set{-1,+1}^*$);
        \item $f$ has \kl{polynomial growth};
        \item \label{npoly-counting-mso:item} $f$ belongs to
            $\Span{\mathbb{S}}(\setof{\vcount{\varphi}}{\varphi(\vec{x}) \in \MSO})$
            with $\mathbb{S} = \Nat$
            (resp. $\mathbb{S} = \Rel$).
    \end{enumerate}
\end{lemma}

Similar characterizations can be obtained for $\NSF$ and $\ZSF$, which correctly
generalizes aperiodicity to functions as \cref{regular-language:ex} illustrates.

\begin{lemma}
    Let $f \in \NPoly$ (resp. $f \in \ZPoly$), then
    the following are equivalent:
    \begin{enumerate}
        \item $f \in \NSF$ (resp. $\ZSF$);
        \item $f$ is a \kl{star-free polyregular function} with output
            in $\set{1}^*$,
            post-composed with $\polysum$
            (resp. with output in $\set{-1,+1}^*$);
        \item $f$ belongs to
            $\Span{\mathbb{S}}(\setof{\vcount{\varphi}}{\varphi(\vec{x}) \in \FO})$
            with $\mathbb{S} = \Nat$
            (resp. $\mathbb{S} = \Rel$).
    \end{enumerate}
\end{lemma}

\begin{example}
    \label{regular-language:ex}
    Let $\Sigma$ be a finite alphabet, and
    $L \subseteq \Sigma^*$. Then,
    $L$ is a regular language if and only if
    $\ind{L} \colon \Sigma^* \to \set{0,1}$ is a
    \kl{$\Nat$-polyregular function}.
    Furthermore, $L$ is a star-free regular language
    if and only if $\ind{L}$ is a
    \kl{star-free $\Nat$-polyregular function}.
\end{example}

It has been proven in \cite{CDTL23,BOJA22} that the classes $\ZPoly[k]$ (resp.
$\NPoly[k]$, $\ZSF[k]$, $\NSF[k]$) are also characterized by the maximal number
of first order free variables used to construct the function as in
\cref{npoly-counting-mso:item} of \cref{polynomial-rational-polyreg:fact}.

Let us briefly state that \kl{commutativity} is a decidable property of
\kl{$\Rel$-rational series}. The proof is straightforward in the case of
\kl{$\Rel$-polyregular functions}, and slightly more involved for general
\kl{$\Rel$-rational series}.

\begin{lemma}
    \label{decidable-commutative-poly:lemma}
    \label{decidable-commutative-rat:lemma}
    Let $f \in \ZRat$. One can decide if 
    $f$
    is \kl{commutative}.
\end{lemma}

\subsection{Polynomials} \AP All polynomials considered in this paper have
coefficients in $\Rel$ unless explicitly stated otherwise. A polynomial $P \in
\Rel[X_1, \dots, X_n]$ is an \intro{$\Nat$-rational polynomial} if it is
\kl{represented} by a \kl{$\Nat$-polyregular function}. The analogue notion of
\intro{$\Rel$-rational polynomials} is not of particular interest, since every
polynomial $P$ is \kl{represented} by a \kl{$\Rel$-polyregular function}.

\begin{example}
    \label{negative-not-nrat:ex}
    The polynomials $X$, and $X^2 + 3$ are \kl{$\Nat$-rational polynomials},
    but $- X$ is a \kl{$\Rel$-rational polynomial} that is 
    not an \kl{$\Nat$-rational polynomial}.
\end{example}

\AP A polynomial $P \in \Rel[X_1, \dots, X_k]$ is \intro{non-negative} when for
all non-negative integer inputs $n_1, \dots, n_k \geq 0$, the output  $P(n_1,
\dots, n_k)$ of the polynomial is non-negative. Beware that we do not consider
negative values as input, as the numbers $n_i$ will ultimately count the number
of occurrences of a letter in a word. As an example, the polynomial $(X - Y)^2$
is \kl{non-negative}, and so is the polynomial $X^3$, but the polynomial $X^2 -
2X$ is not.

All \kl{$\Nat$-rational polynomials} are \kl{non-negative}, but the converse
does not hold, as the following example illustrates. Note that the proof scheme
of this example will be at the core of the counter-example to \cite[Theoerm
3.3]{KARH77} in \cref{thm:counter-example}, and leverages a key property of
$\NRat$ recalled hereafter.

\begin{fact}
    \label{pre-image-regular:fact}
    The pre-image of a regular language by an \kl{$\Nat$-rational series}
    is a regular language. 
\end{fact}

\begin{example}
    Let $P(X, Y) \defined (X - Y)^2$.
    Then $P$ is
    is \kl{non-negative}, but is
    not an \kl{$\Nat$-rational polynomial}.
\end{example}
\begin{proof}
    Assume by contradiction that
    $f \in \NPoly$ \kl{represents} $f$ over the alphabet $\Sigma \defined \set{a,b}$.
    Then, $f^{-1}(\set{0})$ is a regular language
    (\cref{pre-image-regular:fact}),
    but $S^{-1}(\set{0}) = \setof{ w \in \Sigma }{ \card[a]{w} = \card[b]{w} }$
    is not.
\end{proof}


\AP Let us now give some vocabulary on polynomials with multiple indeterminate
over $\Rel$. A \intro{monomial} is a product of indeterminates and integers.
For instance, $XY$ is a \kl{monomial}, $3 X$ is a \kl{monomial}, $-Y$ is a
\kl{monomial}, but $X + Y$ or $2X^2 + XY$ are not. We write $\intro*\Monomials[X_1,
\dots, X_n]$ for the set of \kl{monomials} over these indeterminates.
Every polynomial $P \in \Rel[X_1, \dots, X_n]$ decomposes uniquely
into a sum of \kl{monomials}.

\AP A \kl{monomial} $S$ \intro{divides} a \kl{monomial} $T$, when $S$ divides
$T$ seen as polynomials in $\Rat$. For instance, $2X$ \kl{divides} $XY$, $-YZ$
\kl{divides} $X^2 Y Z^3$, and $Y$ does not \kl{divide} $X$. In the
decomposition of $P \in \Rel[X_1, \dots, X_k]$, a \kl{monomial} is a
\intro{maximal monomial} if it is a maximal element for the \kl{divisibility
ordering} of \kl{monomials}. In the polynomial $P(X,Y) \defined X^2 - 2XY + X^2
+ X + Y$, the set of \reintro{maximal monomials}, written
$\intro*\MaximalMonomials(P)$, is $\set{X^2,  -2 XY,  X^2}$.  In the polynomial
$P(X,Y) \defined (X - Y)^2$, the \kl{non-negative} \kl{monomials} are $X^2$ and
$Y^2$.

%! TeX program = xelatex
%! lang = en-US
\section{$\Nat$-rational Polynomials}
\label{polynomials:sec}

\AP In this section, we will completely characterize which polynomials in
$\Rat[\vec{X}]$ that are \kl{represented} by \kl{$\Nat$-polyregular functions}
(resp. \kl{$\Rel$-polyregular functions}). To that end, we start by
characterizing these classes for polynomials in $\Rel[\vec{X}]$.
We say that a polynomial $P \in \Rel[X_1, \dots, X_n]$ is an
\intro{$\Nat$-rational polynomial} if it is \kl{represented} by a
\kl{$\Nat$-rational series}, in which case it is also \kl{represented} by a
\kl{$\Nat$-polyregular function}, because it has \kl{polynomial growth}
(\cref{polyregular-polynomial-growth:lemma}).
The analogue notion of \intro{$\Rel$-rational polynomials} is not
of particular interest, since every polynomial $P \in \Rel[X_1, \dots, X_n]$ is
\kl{represented} by a \kl{$\Rel$-polyregular function}. 
It is an easy check that
polynomials with coefficients in $\Nat$ are \kl{$\Nat$-rational polynomials}
(\cref{n-poly-n-poly:example}). However, \cref{negative-but-npoly:ex} provides
a polynomial with negative coefficients that is an \kl{$\Nat$-rational
polynomial}. 
The problem of characterizing \kl{$\Nat$-rational polynomials} was claimed to
be solved in \cite{KARH77}, using the \cref{karh:def}
to  characterize
\kl{$\Nat$-rational polynomials}, as restated in \cref{karh:thm}.

\begin{lemma}
    \label{n-poly-n-poly:example}
    \proofref{n-poly-n-poly:example}
    Let $P \in \Nat[\vec{X}]$. Then, $P$
    is an \kl{$\Nat$-rational polynomial}.
\end{lemma}
\begin{proof}
    The polynomials of $\Nat[\vec{X}]$
    are obtained from basic functions (constant in $\Nat$,
    $X_i$ for some $1 \leq i \leq n$)
    by products and sums (\cref{stability-polyregular:lemma}). Because the basic functions are
    \kl{represented} by \kl{star-free $\Nat$-polyregular} functions (see
    \cref{size-of-word-nsf:ex}),
    that are closed under these operations, we conclude.
\end{proof}

\begin{example}
    \label{negative-but-npoly:ex}
    \proofref{negative-but-npoly:ex}
    The polynomials $X$, $X^2 + 3$,
    and $X^2 - 2X + 2$
    are \kl{$\Nat$-rational polynomials},
    but $- X$ is a \kl{$\Rel$-rational polynomial} that is 
    not an \kl{$\Nat$-rational polynomial}.
\end{example}



\begin{definition}[{\cite[Section 3, page 3]{KARH77}}]
    \label{karh:def}
    The class $\intro*\CoveredPoly[\vec{X}]$
    is the class of polynomials $P \in \Rel[\vec{X}]$
    that are \kl{non-negative}
    and such that every \kl{maximal monomial} is \kl{non-negative}.
    When the indeterminates are clear from the context, we write
    this class $\reintro*\CoveredPoly$.
\end{definition}

\begin{faketheorem}[{\cite[Theorem 3.3, page 4]{KARH77}}] 
    \label{karh:thm}
    Let $P \in \Rel[\vec{X}]$ be a polynomial. Then,
    $P$ is an \kl{$\Nat$-rational polynomial}
    if and only if 
    $P \in \CoveredPoly$.
\end{faketheorem}

Before giving a counterexample to the above statement, let us first provide via
\cref{non-neg-not-nrat:ex} simple proof that there exists some
\kl{non-negative} polynomial that is not an \kl{$\Nat$-rational polynomial}.
While the example will not be in $\CoveredPoly$, it illustrates the key
difference between \kl{non-negative} polynomials and \kl{$\Nat$-rational
polynomials}. In order to derive this example, we will need the following
fundamental
result about the pre-image of regular languages by \kl{polyregular functions}.\footnote{
    In this particular case, one could have
    considered more generally \kl{$\Nat$-rational series},
    and replaced regular languages over a unary alphabet
    by semi-linear sets.
}

\begin{theorem}[{\cite[Theorem 1.7]{BOJA18}}]
    \label{pre-image-regular:fact}
    The pre-image of a regular language by a \kl{polyregular function}
    is a regular language.
\end{theorem}

\begin{example}
    \label{non-neg-not-nrat:ex}
    Let $P(X, Y) \defined (X - Y)^2$.
    Then $P$ is \kl{non-negative}, but is
    not an \kl{$\Nat$-rational polynomial}.
    Indeed, assume by contradiction that
    $f \in \NPoly$ \kl{represents} $f$ over the alphabet $\Sigma \defined \set{a,b}$.
    Then, $f^{-1}(\set{0})$ is a regular language
    (\cref{pre-image-regular:fact}),
    but $f^{-1}(\set{0}) = \setof{ w \in \Sigma }{ \card[a]{w} = \card[b]{w} }$
    is not.
\end{example}


\AP
Let us now design a counterexample to \cref{karh:thm} by suitably tweaking
\cref{non-neg-not-nrat:ex} to ensure that the polynomial not only is
\kl{non-negative}, but also belongs to $\CoveredPoly$.
\label{def:bad-polynomial}
We define $\intro*\BadPoly(X,Y,Z) \defined Z (X + Y)^2 + 2 (X - Y)^2$.

\begin{lemma}
    \label{thm:counter-example}
    The polynomial $\BadPoly$ belongs to $\CoveredPoly$,
    but is not an \kl{$\Nat$-rational polynomial}.
    As a corollary, 
    the result stated in \cite[Theorem 3.3]{KARH77}, restated
    in \cref{karh:thm}, is false
    when allowing at least $3$ indeterminates.
\end{lemma}
\begin{proof}
    It is clear that $\BadPoly$ is \kl{non-negative}. We can expand
    the expression of $\BadPoly$ to 
    obtain
    $\BadPoly = ZX^2 + ZY^2 + 2ZXY + 2X^2 -4XY + 2Y^2$.
    The \kl{maximal monomials} of $P$
    are $ZX^2$, $ZY^2$, and $2ZXY$, all of which are
    \kl{non-negative}.

    Assume by contradiction that $\BadPoly$ is an \kl{$\Nat$-rational polynomial}.
    Let $\Sigma \defined \set{a,b,c}$ be a finite alphabet.
    There exists a \kl{commutative}
    \kl{$\Nat$-polyregular function} $f \colon \Sigma^* \to \Nat$
    such that for all $w \in \Sigma^*$,
    $\BadPoly(\card[a]{w}, \card[b]{w}, \card[c]{w}) = f(w)$.

    Remark that for all $x,y,z \geq 0$, $\BadPoly(x,y,z) = 0$
    if and only if $z(x+y)^2 = -2 (x-y)^2$. Hence,
    $\BadPoly(x,y,z)=0$ if and only if $z = 0$ and $x = Y$, or 
    $z \neq 0$, and $x = y = 0$.

    Now, let us consider the language $L \defined \setof{w}{ f(w) = 0}$. By the
    above computation, we conclude that $L = \setof{ w \in \set{a,b}^* }{
    \card[a]{w} = \card[b]{w} } \cup \set{ c }^*$.
    Because $L \cap \set{a,b}^*$ not a regular language,
    we
    conclude that $L$ is not a regular language.
    However, $L = f^{-1}(\set{0})$ is a regular language
    (\cref{pre-image-regular:fact}). 
\end{proof}

%! TeX program = xelatex
%! lang = en-US
\subsection{The Corrected Theorem}
\label{sec:proof}

The counter example provided by \cref{def:bad-polynomial} relies on the fact
that $\CoveredPoly$ is not stable under fixing indeterminates, while
\kl{$\Nat$-polyregular functions} are. In this section, we prove that closing
$\CorrectPoly$ under assignments of variable is enough to recover from
\cref{karh:thm}.

\AP We use the following notation to fix the value of some indeterminate, if
$P(X,Y)$ is a polynomial in $\Rel[X,Y]$, then $\intro*\restr{P(X,Y)}{X = 1}$ is
the polynomial $P(1,Y) \in \Rel[Y]$. More generally, if $\nu$ is a partial
function from $\vec{X}$ to $\Nat$, written $\nu \colon \vec{X} \topartial
\Nat$, the restriction $\restr{P(\vec{X})}{\nu}$ is the polynomial with
indeterminates $\vec{Y} \defined \vec{X} - \dom(\nu)$ obtained by fixing the
variables of the domain of $\nu$.


\begin{definition}
    Let $\vec{X}$ be a finite tuple of indeterminates.
    The class $\intro*\CorrectPoly[\vec{X}]$ is the collection of
    polynomials $P \in \Rel[\vec{X}]$ such that
    $P$ is 
    such that, for every partial function $\nu \colon \vec{X} \topartial \Nat$,
    every \kl{maximal monomial} of
    $\restr{P}{\nu}$ is \kl{non-negative}.
\end{definition}

First, let us remark that $\CorrectPoly \subseteq \CoveredPoly$, because
polynomials in $\CorrectPoly$ are \kl{non-negative}. Let us also check that the
counter example provided in \cref{thm:counter-example} is not in
$\CorrectPoly$. For that, notice that for $\BadPoly$ introduced in
\cref{def:bad-polynomial}, $\BadPoly(X,Y,1) = 3X^2 + 3Y^2 - 2XY$, which has a negative
coefficient for a \kl{maximal monomial}, namely $-2XY$. 

Let us now prove that \kl{$\Nat$-rational polynomials} are in $\CorrectPoly$.
This follows from the correct implication in the statement of \cref{karh:thm},
but we provide a self-contained proof using a refinement of the
classical combinatorial arguments for $\ZPoly$ \cite[Lemma 4.16]{CDTL23} and
$\NPoly$ \cite[Lemma 5.37]{DOUE23}.

\begin{lemma}
    \label{n-poly-combinatorics:lem}
    Let $f$ be a  \kl{$\Nat$-polyregular} function. 
    There exists a computable $\omega \in \Nat$
    such that for all $p \in \Nat$,
    for all $\alpha_0, \dots, \alpha_p \in \Sigma^*$,
    for all $u_1, \dots, u_p \in \Sigma^*$,
    there exists a polynomial $P \in \Rel[X_1, \dots, X_p]$
    whose \kl{maximal monomials} are \kl{non-negative},
    and such that for all $X_1, \dots, X_p \geq \omega$:
    \begin{equation*}
        f\left(
            \alpha_0 \prod_{i = 1}^p u_i^{\omega X_i} \alpha_i
        \right)
        = P(X_1, \dots, X_p) \quad .
    \end{equation*}
\end{lemma}


\begin{corollary}
    \label{n-rat-correct:lem}
    Let $P \in \Rel[X_1, \dots, X_p]$ be \kl{represented}
    by a \kl{$\Nat$-rational series}. Then,
    $P \in \CorrectPoly$.
\end{corollary}

\AP The core of the upcoming \cref{lem:correct-to-n-rat} leverages a notion of
\intro{discrete derivative} to perform an induction on the \kl{maximal monomials}.
This notion of derivation is built by translating the domain of the polynomial.
To that end, let us write $\intro*\translate{K}$ for the \intro{translation function}
that maps a polynomial $P \in \Rel[X_1, \dots, X_n]$ to the polynomial $P(X_1 +
K, \dots, X_n + K)$.

\begin{definition}
    \label{discrete-derivative:def}
    Let $\vec{X}$ be a tuple of indeterminates,
    $K \in \Nat$,
    and 
    $P \in \Rel[\vec{X}]$ be a polynomial.
    \begin{equation*}
        \intro*\Diff{K}{P} \defined 
        \translate{K}(P) - P
        \quad .
    \end{equation*}
\end{definition}

As one would expect, the \kl{discrete derivatives} are linear operations on
polynomials, that commutes with \kl{translation operators}. However, the
\kl{translation operators} do not perfectly commute with partial applications
$\restr{\cdot}{\nu}$.

\begin{remark}
    \label{discrete-deriv-linear:fact}
    For all tuples $\vec{X}$ of indeterminates,
    for all $K \in \Nat$, $L \in \Rel$, for all partial functions
    $\nu \colon \vec{X} \topartial \Nat$:
    \begin{enumerate}
        \item The maps $\translate{K} \colon \Rel[\vec{X}] \to \Rel[\vec{X}]$,
        $\Diff{K}{ \cdot } \colon \Rel[\vec{X}] \to \Rel[\vec{X}]$,
        and
        $\restr{\cdot}{\nu} \colon \Rel[\vec{X}] \to \Rel[\vec{X}]$
            are linear operators,
        \item $\Diff{K}{ \cdot } \circ \translate{L}
            = \translate{L} \circ \Diff{K}{\cdot}$,
        \item $\restr{\cdot}{\nu} \circ \translate{L}
            = \translate{L} \circ \restr{\cdot}{\translate{L}(\nu)}$.
    \end{enumerate}
\end{remark}

In the actual statement of \cref{corrected-version:thm}, the representation of
polynomials in $\CorrectPoly$ is strengthened from \kl{$\Nat$-polyregular
functions} to \kl{star-free $\Nat$-polyregular functions}. This is not
surprising, given the conjecture that \kl{ultimately polynomial}
\kl{$\Nat$-polyregular functions} (the formal definition will be given in
\cref{star-free:sec}) were conjectured to be star free
(\cref{zsf-nsf:conjecture}), and that polynomials are always \kl{ultimately
polynomial}.

\AP When multiple indeterminates are involved, the notion of degree must be
adapted, and to ensure that the decision procedure is effective, one must
obtain explicit bounds on the \kl{translations} that are needed. To that end,
let us now introduce the ordering over which the induction of
\cref{lem:correct-to-n-rat} is built. Recall that $\MaximalMonomials(P)$ is the
set of \kl{maximal monomials} of $P$, hence belongs to $\Pfin(\Monomials)$. We
order \kl{monomials} with the \kl{divisibility ordering}, making $\Monomials$ a
\kl{well-quasi-ordering} that is isomorphic to $\Nat^k$ with the product
ordering \cite[Dickson’s Lemma]{SCSC12}. We endow $\Pfin(\Monomials)$ with the
\intro{Hoare ordering}, that is, $S_1 \hoareleq S_2$ whenever for all
\kl{monomials} $M_1 \in S_1$, there exists a monomial $M_2 \in S_2$, such that
$M_1$ \kl{divides} $M_2$. The set $(\Pfin(\Monomials), \hoareleq)$ remains a
\kl{well-quasi-ordering}, and is in particular well-founded \cite[Hoare
quasi-ordering]{SCSC12}.


Let us now illustrate how the \kl{discrete differentiation} operator interacts
with \kl{maximal monomials} with respect to the \kl{Hoare ordering}: it
extracts \emph{sub-maximal} monomials from a given polynomial.

\begin{fact}
    \label{derivation-monomials:fact}
    Let $K > 0$,
    and let $M,T$ be two \kl{monomials}, such that
    $T$ \kl{strictly divides} $M$.
    Then,
    every \kl{monomial} of $\Diff{K}{M}$ \kl{strictly divides} $M$,
    and 
    $T$ \kl{divides} some \kl{maximal monomial}
    of $\Diff{K}{M}$, which has a coefficient
    that is a multiple of $K$.
\end{fact}

\begin{lemma}
    \label{derivation-simplifies:lemma}
    For all $P \in \Rel[\vec{X}]$ that are non-constant,
    for all $K \in \Nat$,
    $\MaximalMonomials(\Diff{K}{P}) \hoarele
    \MaximalMonomials(P)$.
\end{lemma}

The following \cref{derivation-stabilises-correct:lem} is the main
combinatorial argument of this section. It leverages the positivity of the
\kl{maximal monomials} to compute a shift large enough to make lower degree
coefficients positive. This lemma will be lifted to a given polynomial $P$ by
noticing that if $K \in \Nat$, and $P = P_1 + P_2$, then $\translate{K}(P) =
P_1 + \Diff{K}{P_1} + \translate{K}(P_2)$.

\begin{lemma}
    \label{derivation-stabilises-correct:lem}
    Let $P \in \CorrectPoly$,
    $P_1$ be the sum of \kl{maximal monomials} of $P$,
    and $P_2 \defined P - P_1$ be the sum of
    non-maximal monomials of $P$.
    There exists a computable $K$,
    such that
    $Q \defined (\Diff{K}{P_1} + \translate{K}(P_2)) \in \CorrectPoly$.
\end{lemma}


\begin{lemma}
    \label{lem:correct-to-n-rat}
    Let $\vec{X}$ be a tuple of indeterminates,
    and let $P \in \Rel[\vec{X}]$.
    If $P \in \CorrectPoly$, then $P$ is \kl{represented}
    by a \kl{star-free $\Nat$-polyregular function},
    which can be explicitly constructed from $P$.
\end{lemma}
\begin{proof}
    We prove the result by induction on $\MaximalMonomials(P)$. 
    In the proof, we write $\vec{X}$ for the indeterminates appearing in $P$,
    that is, we assume without loss of generality that all indeterminates are used.

    \textbf{Base case:} If the (unique) \kl{maximal monomial} of $P$ is a
    constant term. Since $P \in \CorrectPoly$, $P = n \in \Nat$, and therefore
    $P$ is \kl{represented} by a constant \kl{star-free $\Nat$-polyregular
    function}.

    \textbf{Induction:} Assume that $P$ is not a constant polynomial, and let
    us write $P = P_1 + P_2$ where $P_1$ is the sum of the \kl{maximal
    monomials} of $P$. We compute a bound $K$ such that $Q \defined
    (\Diff{K}{P_1} + \translate{K}(P_2)) \in \CorrectPoly$ using
    \cref{derivation-stabilises-correct:lem}. Thanks to
    \cref{derivation-simplifies:lemma}, we also know that $\MaximalMonomials(Q)
    \hoarele \MaximalMonomials(P)$. By induction hypothesis, $Q$ is
    \kl{represented} by a \kl{star-free $\Nat$-polyregular function} which is
    effectively computable.

    Let us now remark that $P_1 \in \Nat[\vec{X}]$, and is therefore
    (effectively) \kl{represented} by a \kl{star-free $\Nat$-polyregular
    function} (\cref{n-poly-n-poly:example}). As a consequence,
    $\translate{K}(P) = P_1 + Q$ is (effectively) \kl{represented} by a
    function $f_\Delta$.

    For all partial valuations $\nu \colon \vec{X} \topartial \set{0, \dots,
    K}$ fixing at least one indeterminate,  we know that
    $\MaximalMonomials(\restr{P}{\nu}) \hoarele \MaximalMonomials{P}$. Because
    $\restr{P}{\nu} \in \CorrectPoly$, one can use the induction hypothesis to
    compute a \kl{star-free $\Nat$-polyregular function} $f_\nu$ that
    \kl{represents} $\restr{P}{\nu}$.


    Let us assume that the alphabet over which the (\kl{commutative}) functions
    $f_\Delta$ and $f_\nu$ are defined is $\set{a_1, \dots, a_n}$, with $a_i$
    representing the indeterminate $X_i$ of the polynomials. Now, let us define
    by case analysis the following \kl{commutative} \kl{star-free
    $\Nat$-polyregular function}, defined on words $w$ of the form $w \defined
    a_1^{X_1} \cdots a_n^{X_n}$, with $X_1, \dots, X_n \geq 0$.

    \begin{equation*}
        f(w) \defined
        \begin{cases}
            f_\nu(w) & \text{ if } \exists i \in \set{1, \dots, n}, X_i \leq K \\
            f_\Delta(a_1^{X_1 - K} \cdots a_n^{X_n - K}) & \text{ otherwise }
        \end{cases}
        \quad .
    \end{equation*}
    It is clear that
    $f$ is a \kl{commutative} \kl{star-free $\Nat$-polyregular function},
    and that
    $f$ \kl{represents} $P$.
\end{proof}


While \cref{lem:correct-to-n-rat} provides effective conversion, it does not
explicitly state that the membership is decidable to keep the proof clearer. A
similar proof scheme can be followed to conclude that membership is decidable,
and even show that elements in $\CorrectPoly$ are, up to suitable translations,
polynomials in $\Nat[\vec{X}]$. Beware that partial applications are still
needed in this characterization, as \cref{bad-poly-translate:ex} illustrates.

\begin{lemma}
    \label{derivation-translation:lem}
    Let $P \in \Rel[\vec{X}]$, 
    there exists a computable $K \in \Nat$
    such that the following are equivalent:
    \begin{enumerate}
        \item \label{d-t-correct:item} $P \in \CorrectPoly$,
        \item \label{d-t-transl:item}
            for 
            all partial functions $\nu \colon \vec{X} \topartial \Nat$,
            $\translate{K}(\restr{P}{\nu}) \in \Nat[\vec{X}]$,
        \item \label{d-t-transl-fin:item}
            for all partial functions
            $\nu \colon \vec{X} \topartial \set{0, \dots, K}$,
            $\translate{K}(\restr{P}{\nu}) \in \Nat[\vec{X}]$.
    \end{enumerate}
    Furthermore, the membership is decidable.
\end{lemma}


\begin{example}
    \label{bad-poly-translate:ex}
    The polynomial $\BadPoly$ is not a 
    \kl{$\Nat$-rational polynomial},
    but is \kl{non-negative} and satisfies
    $\translate{10}(\BadPoly) \in \Nat[\vec{X}]$.
\end{example}

We are now ready to state the corrected and generalized version of
\cref{karh:thm}, which is the main technical contribution of the paper.

\begin{theorem}
    \label{corrected-version:thm}
    Let $P \in \Rel[\vec{X}]$ be a polynomial.
    The following are equivalent:
    \begin{enumerate}
        \item \label{corrected-1:item} $P \in \CorrectPoly$,
        \item \label{corrected-2:item} $P$ is \kl{represented} by a \kl{$\Nat$-rational series},
        \item \label{corrected-3:item} $P$ is \kl{represented} by a \kl{$\Nat$-polyregular function},
        \item \label{corrected-4:item} $P$ is \kl{represented} by a \kl{star-free $\Nat$-polyregular function},
    \end{enumerate}
    Furthermore, the membership is decidable, and effective conversion
    procedures exist between all the representations.
\end{theorem}
\begin{proof}
    The implications 
    \cref{corrected-4:item} $\implies$
    \cref{corrected-3:item} $\implies$
    \cref{corrected-2:item} are obvious.
    \cref{lem:correct-to-n-rat} proves
    \cref{corrected-1:item} $\implies$ \cref{corrected-4:item},
    while \cref{n-rat-correct:lem}
    proves 
    \cref{corrected-2:item} $\implies$ \cref{corrected-1:item}.
    Note that the lemmas provide effective conversion procedures,
    and that \cref{lem:correct-to-n-rat} also provides a decision
    procedure.
\end{proof}

Let us now state, for the sake of completeness, the equality of $\CorrectPoly$
and $\CoveredPoly$ in the case of two indeterminates. This is relevant because
it shows that examples given in \cite{KARH77} are actually correct, even though
based on an invalid result, and it may be the case for other works based on
\cite{KARH77}.

\begin{lemma}
    \label{lem:correct-covered-2}
    $\CorrectPoly[X,Y] = \CoveredPoly[X,Y]$.
\end{lemma}
\begin{proof}
    It is clear that $\CorrectPoly[X,Y] \subseteq \CoveredPoly[X,Y]$,
    by considering the empty valuation $\nu \colon \set{X,Y} \topartial \Nat$.
    For the converse inclusion, let us consider $P(X,Y)$
    that is \kl{non-negative}, such that the \kl{maximal monomials}
    are  \kl{non-negative}.
   

    If we fix none of the variables, then the \kl{maximal monomials}
    are \kl{non-negative} by assumption. If we fix one of the
    variables, we can assume without loss of generality that we 
    fix $X = k$ for some $k \in \Nat$.
    Then $P(k,Y)$ is a \kl{non-negative} \emph{univariate} polynomial, 
    and therefore must have a positive leading coefficient
    (which is the unique \kl{maximal monomial} in this case)
    or be constant equal to 0. In both cases, the \kl{maximal monomials}
    have positive coefficient.
    The same reasoning applies \emph{a fortiori} in the case where
    we fix the two indeterminates, leading to a constant polynomial.
\end{proof}


%! TeX program = xelatex
%! lang = en-US
\section{Beyond Polynomials}
\label{beyond-polynomials:sec}
\label{star-free:sec}

The goal of this section is to go beyond polynomials and consider the case of
\kl{commutative} \kl{$\Rel$-polyregular functions}. The hypothesis of
\kl{commutativity} reduces $\ZPoly$ to a combination of a finite number of
polynomials, arranged by counting modulo some numbers. This characterization
will be proven in \cref{decompose-polynomial:lem}, and will be the key
ingredient in the decision procedures of $\NPoly$ inside $\ZPoly$, and $\NSF$
inside $\NPoly$. Note that \cref{decompose-polynomial:lem} is actually a
refinement of \cref{n-poly-combinatorics:lem} under the extra assumption of
\kl{commutativity}. 




\AP To actually state the lemma, we will need some notations that will
formalize the intuitive idea of counting tuples of integers modulo some
constant. Let $\omega \in \Nat$, we define the set $\intro*\ModuloTypes[\omega]
\defined \set{0, \dots, \omega^2} \times \set{0, \dots, \omega - 1}$ of
\intro{types modulo $\omega$}. To a number $n \in \Nat$, we associate its
\reintro{$\omega$-type} written $\intro*\moduloType[\omega](n)$ which is
defined as the pair $(t, r)$ where $t = \min (n, \omega^2)$ and $r$ is the rest
of the Euclidian division of $n$ by $\omega$. This notion of type is lifted to
vectors in $\Nat^p$ for any $p \in \Nat$ by pointwise application.


\begin{lemma}
    \label{decompose-polynomial:lem}
    Let $f \colon \Sigma^* \to \Rel$ be a \kl{commutative}
    \kl{$\Rel$-polyregular function},
    where we fix the alphabet $\Sigma = \set{a_1, \dots, a_n}$.
    There exists a computable
    $\omega \in \Nat$,
    and computable 
    polynomials $\seqof{P_{t}}{t \in \ModuloTypes[\omega]^n}$
    such that for all $w \in \Sigma^*$,
    \begin{equation*}
        f\left(a_1^{X_1} \cdots a_n ^{X_n}\right) 
        = P_{\moduloType[\omega](X_1, \dots, X_n)}
        \left(
            \floor{X_1/\omega}, \dots, \floor{X_n/\omega}
        \right)
        \quad .
    \end{equation*}
\end{lemma}

In the spirit of previous characterizations of subclasses of $\ZPoly$ in terms
of \emph{polynomial pumping arguments}
\cite{DOUE21,DOUE22,CDTL23}, we will provide a
semantic property in the form of \cref{k-combinatorial:def}, that is conjectured
to characterize $\NPoly$ inside $\ZPoly$, even in the non-commutative setting.

\begin{definition}
    \label{k-combinatorial:def}
    Let $k \in \Nat$, and $f \colon \Sigma^* \to \Rel$
    be a \kl{$\Rel$-polyregular function}. The function $f$ is 
    \intro{$(k,\Nat)$-combinatorial} if there exists $\omega \in \Nat$,
    such that
    for all
    $\alpha_0, \dots, \alpha_k \in \Sigma^*$,
    for all $u_1, \dots, u_k \in \Sigma^*$,
    there exists a polynomial $P \in \CorrectPoly$,  
    satisfying for all $X_1, \dots, X_k \geq \omega$:
    \begin{equation*}
        f
        \left(
            \alpha_0 \prod_{i = 1}^k u_i^{\omega X_i} \alpha_i
        \right)
        = 
        P(X_1, \dots, X_k) \quad .
    \end{equation*}
    We say that $f$ is \reintro{combinatorial}
    if it is \reintro{$(k,\Nat)$-combinatorial} for all $k \in \Nat$.
\end{definition}

\begin{theorem}
    \label{decidable-n-poly:thm}
    Let $k \in \Nat$, and 
    let $f \in \ZPoly[k]$ be a \kl{commutative} \kl{$\Rel$-polyregular function}.
    The following are equivalent:
    \begin{enumerate}
        \item \label{f-combinatorial:item} $f$ is \kl{$(k,\Nat)$-combinatorial},
        \item \label{f-npoly-combi:item} $f \in \NPoly[k]$,
    \end{enumerate}
    Furthermore, the properties are decidable,
    and conversions effective.
\end{theorem}

Let us now prove that the above characterizations of \kl{commutative}
\kl{$\Nat$-polyregular functions} can be combined with the recent advances in
the study of \kl{$\Rel$-polyregular functions} \cite{CDTL23} to decide
aperiodicity. The key ingredient of this study is the use of a semantic
characterization of \kl{star-free $\Rel$-polyregular functions} among
\kl{$\Rel$-rational series} that generalizes the aperiodicity of languages to
functions by the means of polynomial behaviors (see
\cref{aperidic-ultimately-polynomial:ex}).

\begin{definition}[Ultimately polynomial]
    \label{ultimately-polynomial:def}
    Let $\Sigma$ be a finite alphabet. 
    A function $f \colon \Sigma^* \to \Rel$
    is \intro{ultimately polynomial}
    whenever there exists $N_0 \in \Nat$ such that
    for all $n \in \Nat$
    and for all words $\alpha_0, w_1, \alpha_1, \cdots, \alpha_{n-1}, w_n, \alpha_n
    \in \Sigma^*$, there exists a polynomial $P \in \Rel[X_1, \dots, X_n]$
    such that
    \begin{equation*}
        f\left(
            \alpha_0 \prod_{i = 1}^{n} w_i^{X_i} \alpha_i
        \right)
        = 
        P(X_1, \dots, X_n)
        \quad 
        \forall X_1, \dots, X_n \geq N_0
        \quad .
    \end{equation*}
\end{definition}

\begin{example}
    \label{aperidic-ultimately-polynomial:ex}
    A language $L$ is aperiodic if and only if 
    $\ind{L}$ is \kl{ultimately polynomial}.
\end{example}

\AP The decidability of aperiodicity for \kl{$\Rel$-polyregular functions}
relies on the construction of a canonical object called the \emph{residual
transducer}, the latter being essentially based on differences between
functions, crucially leveraging negative outputs \cite{CDTL23}. Although the
proof method does not carry from $\Rel$-output functions to $\Nat$-output
functions, it was conjectured that the semantic property of being
\kl{ultimately polynomial} would also characterize $\NSF$ inside $\NPoly$, a
conjecture that is restated hereafter.

\begin{conjecture}[{\cite[Conjecture 7.61]{DOUE23}}]
    \label{zsf-nsf:conjecture}
    Let $k \in \Nat$.
    A function $f \in \NPoly[k]$
    belongs to $\NSF[k]$ if and only if
    it is \kl{ultimately polynomial}.
    In particular,
    $\NSF[k] = \ZSF[k] \cap \NPoly$.
\end{conjecture}

We answer positively to \cref{zsf-nsf:conjecture} in the commutative case, by
leveraging the semantic characterizations respectively of $\ZSF$ inside
$\ZPoly$ (\kl{ultimately polynomial}) and $\NPoly$ inside $\ZPoly$
(\kl{$(k,\Nat)$-combinatorial}), which is possible thanks to the decomposition
obtained in \cref{decompose-polynomial:lem}. Decidability then follows
immediately from the effective conversions, and the previous decidability
result for $\ZSF$.

\begin{theorem}
    \label{zsf-npoly-nsf:thm}
    Let $\Sigma$ be a finite alphabet, 
    and $f \colon \Sigma^* \to \Rel$ be a \kl{commutative}
    \kl{$\Nat$-polyregular function}.
    Then, the following are equivalent:
    \begin{enumerate}
        \item $f$ is \kl{ultimately polynomial},
        \item $f \in \ZSF$,
        \item $f \in \NSF$.
    \end{enumerate}
    Furthermore, membership is decidable and conversions are effective.
\end{theorem}



%! TeX program = xelatex
%! lang = en-US
\section{Outlook}
\label{sec:ccl}

Let us conclude with a more general remark on the status of commutative input
in the study of unary output polyregular functions. Notice that semantic
characterizations of subclasses of $\ZPoly$ (\kl{combinatorial}, \kl{ultimately
polynomial}) arise from pre-composition with a \kl{commutative}
\kl{star-free polyregular function}, as formally stated in
\cref{pre-compose-growth-commut:lemma,pre-compose-sf-commut:lemma}. Following
this idea, we conjecture that $\ZPoly$ and $\NPoly$ can be described as
combinations of commutative functions, leading to
\cref{z-poly-commutative-encoding:conj}, which would be a non-commutative
version of \cref{decompose-polynomial:lem}. For the subclass of 
\emph{polyblind functions} \cite{LENP21,DOUE22}, parts of the conjecture holds,
because they essentially behave as commutative functions \cite[Theorem
6.12]{DOUE23}.

%Indeed, a suitable application of Parikh's Theorem \cite{PARI66}
%guarantees the existence of the functions $\mathsf{enc}$ and $\mathsf{dec}$,
%although they may fail to be  \kl{polyregular functions}.

\begin{lemma}
    \label{pre-compose-growth-commut:lemma}
    Let $f \in \ZRat$, and $k \in \Nat$. Then,
    $f \in \ZPoly[k]$ if and only if 
    for every \kl{commutative} \kl{star-free polyregular function} $h$
            of \kl{growth rate} $l \in \Nat$,
            $(f \circ h) \in \ZPoly[k\times l]$.
    \proofref{pre-compose-growth-commut:lemma}
\end{lemma}


\begin{lemma}
    \label{pre-compose-sf-commut:lemma}
    Let $f \in \ZPoly$. Then, $f \in \ZSF$,
    if and only if for every \kl{commutative} \kl{star-free polyregular function} $h$,
            $(f \circ h) \in \ZSF$.
    \proofref{pre-compose-sf-commut:lemma}
\end{lemma}

\begin{conjecture}
    \label{z-poly-commutative-encoding:conj}
    Let $k \in \Nat$, and let $f \in \ZPoly[k]$ (resp. $\ZSF[k]$).
    There exists a finite set $Q$, 
    parameters $l_q \in \Nat$ for all $q \in Q$,
    \kl{commutative} polyregular functions (resp. \kl{star-free polyregular})
    $(g_q)_{q \in Q}$
    from $\Nat^{l_q}$ to $\Rel$,
    a surjective polyregular function (resp. \kl{star-free polyregular})
    $\mathsf{dec} \colon \Sigma^* \tosurj \sum_{q \in Q} \Nat^{l_q}$,
    and a polyregular function (resp. \kl{star-free polyregular})
    $\mathsf{enc} \colon \sum_{q \in Q} \Nat^{l_q} \to \Sigma^*$,
    such that
    $f = (\sum_{q \in Q} g_q) \circ \mathsf{dec}$
    and $\mathsf{dec} \circ \mathsf{enc} = \mathsf{id}$.
\end{conjecture}


\begin{lemma}
    \label{z-poly-commutative-encoding:remark}
    The \cref{z-poly-commutative-encoding:conj} holds for the
    subclass of polyblind functions when dropping the assumption that
    $\mathsf{enc}$ and $\mathsf{dec}$ are \kl{polyregular functions}
    (resp. \kl{star-free polyregular functions}).
    \proofref{z-poly-commutative-encoding:remark}
\end{lemma}


% Include the bibliography
\bibliographystyle{plainurl}
\bibliography{short.bib}

% If there are any appendices, we include them here.


\end{document}
