%! TeX program = xelatex
%! lang = en-US
\documentclass[sigconf,natbib=false,screen, review,anonymous]{acmart}



% INPUTS
%% PACKAGES %%

% Macro making packages
\usepackage{xparse}
\usepackage{xpatch}
\usepackage{tokcycle}


% Standalone compilation
\usepackage[obeyclassoptions,mode=tex]{standalone}

% Math typesetting 
\usepackage{amsmath}
\usepackage{amsthm}
\usepackage{thmtools}
\usepackage{upgreek}
%\usepackage{amssymb}
\usepackage{stmaryrd}

% References and knowledge management
\usepackage{hyperref}
\usepackage[capitalise,noabbrev,nameinlink]{cleveref}
\usepackage[electronic,hyperref,xcolor,cleveref]{knowledge}
\knowledgeconfigure{notion}

%% BIBTEX / BIBLATEX
\usepackage{natbib}

% Table typesetting
\usepackage{booktabs}
\usepackage{varwidth}

% Proofs typesetting
\usepackage{bussproofs}

% Drawing
\usepackage{tikz}
\usetikzlibrary{backgrounds}
\usetikzlibrary{shapes.geometric}
\usetikzlibrary{positioning}
\usetikzlibrary{automata}
\usetikzlibrary{tikzmark}
\usetikzlibrary{patterns}
\usetikzlibrary{arrows}
\usepackage{tikz-cd}
\tikzset{every state/.style={minimum size=1pt}}

% Fonts
%\usepackage[fira]{fontsetup}

\input{globals/colorscheme}

% CONFIG
%% CUSTOM MACROS, META MACROS, AND CONFIG %%

% knowledge management
% TODO: put an if paper mode.
%\knowledgestyle{intro notion}{color={A5}, emphasize}
%\knowledgestyle{notion}{color={A4}}
%\knowledgeconfigure{anchor point color={A2},
%                    anchor point shape=corner}
%\knowledgestyle{intro unknown}{color={D3}, emphasize}
%\knowledgestyle{intro unknown cont}{color={C3}, emphasize}
%\knowledgestyle{kl unknown}{color={D2}}
%\knowledgestyle{kl unknown cont}{color={C2}}

\hypersetup{
    %colorlinks=true,
    anchorcolor=A2,
    citecolor=A4,
    linkcolor=A4,
    urlcolor=A3,
    filecolor=A3,
    runcolor=D2,
    menucolor=D2,
}

\NewDocumentCommand{\klscope}{ o m }{
    \withkl{\kl[#1]}{#2}
}

% Common theorem styles
%\theoremstyle{plain}
%\newtheorem{theorem}{Theorem}[section]
\newtheorem{faketheorem}[theorem]{Flawed Theorem}
\newtheorem{problem}[theorem]{Problem}
%\newtheorem{lemma}[theorem]{Lemma}
%\newtheorem{proposition}[theorem]{Proposition}
%\newtheorem{corollary}[theorem]{Corollary}
\newtheorem{fact}[theorem]{Fact}
%\theoremstyle{definition}
%\newtheorem{definition}[theorem]{Definition}
%\newtheorem{example}[theorem]{Example}
%\theoremstyle{remark}
%\newtheorem{remark}[theorem]{Remark}
%\newtheorem{exercise}[theorem]{Exercise}
%\newtheorem{claim}[theorem]{Claim}
%\newtheorem{tool}[theorem]{Tool}
%\newtheorem{conjecture}[theorem]{Conjecture}


% Upgreek letters
\makeatletter
\newcommand\mathgr[1]{\tokcycle
  {\addcytoks{##1}}
  {\processtoks{##1}}
  {\ifcsname up\expandafter\@gobble\string##1\endcsname
   \addcytoks[1]{\csname up\expandafter\@gobble\string##1\endcsname}%
    \else\addcytoks{##1}\fi}
  {\addcytoks{##1}}{#1}%
  \expandafter\mathrm\expandafter{\the\cytoks}%
}
\makeatother

\NewDocumentCommand{\rightmarginnote}{m}{%
    \checkoddpage%
    \ifoddpage%
        \marginpar{#1}%
    \else%
        \reversemarginpar%
        \marginpar{#1}%
    \fi
}

% Create a proof environment for results in appendix,
% that take a label of a theorem in the main text, and
% produce a proof
% \begin{proof}[Proof of \cref{the-label}]
% \phantomsection\label{the-label:proof}
% content
% \end{proof}
\NewDocumentEnvironment{proofof}{o}{%
    \IfValueTF{#1}{%
        % checks if \#1 is a defined command in which case
        % we call it
        \def\insideRestate{1}
        \ifcsname #1\endcsname
            \csname #1\endcsname*
        \fi
        \begin{proof}[Proof of \cref{#1} on page \pageref{#1}]
        \phantomsection\label{#1:proof}
    }{%
        \begin{proof}
    }
    \let\oldqedsymbol\qedsymbol
    \renewcommand\qedsymbol{\hyperref[#1]{\oldqedsymbol}}
}{%
    \end{proof}
    \renewcommand\qedsymbol{\oldqedsymbol}
}

% Refers to the proof of a result
\NewDocumentCommand{\proofref}{m}{%
    \ifdefined\insideRestate%
        \rightmarginnote{\vspace{0.6em}\ttfamily\small\hyperref[#1]{Go to \cref{#1} p.\pageref{#1:proof}}}%
    \else%
    % if appendices are present, then refer to the appendix
    % otherwise do nothing.
    % \hyperref[#1:proof]{Proof of \cref{#1}}
    \IfRefUndefinedExpandable{#1:proof}{}{%
        %\hfill\hyperref[#1:proof]{(Proof p.\pageref{#1:proof})}
        \rightmarginnote{\vspace{0.6em}\ttfamily\small\hyperref[#1:proof]{Go to proof of \cref{#1} p.\pageref{#1:proof}}}%
    }%
    \fi%
}

%
% Here are the definitions
% of mathematical macros.
%
%

\newcommand{\defined}{\mathrel{:=}}
\newcommand{\seqof}[2]{(#1)_{#2}}
\newcommand{\setof}[2]{\mathopen{\{} #1 \mid #2 \mathclose{\}}}
\newcommand{\set}[1]{\mathopen{\{} #1 \mathclose{\}}}

\newcommand{\Pfin}{\mathcal{P}_{\mathsf{fin}}}

\newcommand{\MSO}{\mathsf{MSO}}
\newcommand{\FO}{\mathsf{FO}}

\newcommand{\ind}[1]{\mathbf{1}_{#1}}
\newcommand{\bigO}{\mathcal{O}}


\newcommand{\topartial}{\rightharpoonup}
% functions

\newcommand{\dom}{\operatorname{dom}}

\NewDocumentCommand{\vcount}{ m }{ {{\#}#1} }
% sets

\NewDocumentCommand{\card}{ O{} m }{ \mathopen{|} #2  \mathclose{|}_{#1}}
\newcommand{\Nat}{\mathbb{N}}
\newcommand{\Rel}{\mathbb{Z}}
\newcommand{\Rat}{\mathbb{Q}}

% GOOD POLYNOMIALS
\newcommand{\CoveredPoly}{\mathsf{PolyKu}}
\newcommand{\CorrectPoly}{\mathsf{PolyRec}}

\NewDocumentCommand{\Monomials}{o}{\IfNoValueTF{#1}{\mathsf{Mon}}{\mathsf{Mon}[#1]}}
\NewDocumentCommand{\MaximalMonomials}{}{\mathop{\mathsf{maxmon}}}

% Differences of polynomials
\NewDocumentCommand{\Diff}{m m}{ \Delta_{#1}(#2) }


% Restrictions
%\newcommand{\restr}[2]{{#1}_{\kl[\restr]{| #2}}}
\newcommand{\restr}[2]{\kl[\restr]{[}{#1}\kl[\restr]{]}_{#2}}
\knowledge{\restr}{notion}

\NewDocumentCommand{\Poly}{O{}}{\mathsf{Poly}_{#1}}
\NewDocumentCommand{\SF}{O{}}{\mathsf{SF}_{#1}}
\NewDocumentCommand{\NPoly}{O{}}{\mathbb{N}\mathsf{Poly}_{#1}}
\NewDocumentCommand{\ZPoly}{O{}}{\mathbb{Z}\mathsf{Poly}_{#1}}
\NewDocumentCommand{\NSF}{O{}}{\mathbb{N}\mathsf{SF}_{#1}}
\NewDocumentCommand{\ZSF}{O{}}{\mathbb{Z}\mathsf{SF}_{#1}}
\NewDocumentCommand{\ZRat}{}{\mathbb{Z}\mathsf{Series}}
\NewDocumentCommand{\NRat}{}{\mathbb{N}\mathsf{Series}}
\NewDocumentCommand{\ZCommut}{}{\mathsf{Commut}}

\NewDocumentCommand{\commute}{O{\cdot}}{\{\!\{ #1 \}\!\}}





\NewDocumentCommand{\hoareleq}{}{\leq_{\mathcal{H}}}
\NewDocumentCommand{\hoarele}{}{<_{\mathcal{H}}}

\newcommand{\growth}[1]{\mathsf{growth}_{#1}}

\NewDocumentCommand{\BadPoly}{}{P_{\mathsf{bad}}}

\NewDocumentCommand{\translate}{m}{\tau_{#1}}

\NewDocumentCommand{\polysort}{}{\mathsf{sort}}

\input{globals/knowledges.kl}
\title{
    Star-free commutative $\Nat$-polyregular functions:
    correcting a result of Karhumäki.
}
\author{
    Lopez Aliaume
}
\date{\today}
\bibliography{globals/papers}

\newcommand{\acknowledge}{
    I would like to thank Gaëtan Douéneau-Tabot
    for mentionning this problem, checking 
    my counter-example, and supporting me in publishing
    this paper.
}

\newcommand{\makeabstract}{
    \begin{abstract}
        Deciding whether a $\Rel$-rational series
        can be computed as a $\Nat$-rational series
        is an open problem that was solved by Karhumäki in the
        specific case of $\Rel$-rational series that are commutative
        and compute a polynomial.
        We propose a short counter-example to this
        previous result on commutative $\Nat$-rational series,
        and prove a corrected version of the theorem,
        working of tdd
        We then discuss how the syntactic criterion could
        pave the way towards a decision procedure when the commutativity
        of the $\Rel$-rational series
        is not assumed.

        \begin{itemize}
            \item Deciding $\Nat$ inside $\Rel$.
            \item Counter example.
            \item Characterization of $\Nat$-polyregular.
            \item Characterization of star free $\Nat$-polyregular.
        \end{itemize}
    \end{abstract}
}


\usepackage{comment}
%\includecomment{forshort}
%\excludecomment{proof}



\makeabstract
% DOCUMENT
\begin{document}

% FRONT MATTER
\maketitle
\acknowledge

\section{TODOs}

\begin{itemize}
    \item Use $\Nat$-polyregular functions all the time, and just 
        in the introduction talk about $\Nat$-rational series.
    \item Prove that $\Nat$-rational series of polynomial 
        growth are $\Nat$-polyregular functions, maybe using
        the results about ambiguity in automata?
        Theorem 5.22
    \item Define commutativity by a map 
        $\commute \colon \Sigma^* \topartial \Nat^{\Sigma}$.
    \item Define "commutative functions" and its symbol.
    \item Define "computed by" (like, recognised by) for commutative 
        functions.
    \item Change \cref{corrected-version:thm} to use directly
        star-free!
    \item Clear up the computability problems.
    \item Polynomial + Npolyregular => star free?
    \item Polynomial + N rational => N polyregular?
    \item cite the paper of Quantitative Monadic Second-Order Logic
        about why it is interesting class.
    \item Example of polyregular functions
    \item Example of non-polyregular functions
    \item Expand a bit on the notion of "derivative"
        and divisibility of monomials.
    \item What about matrices ? They are not minimisable... so we do not know
        about their eigenvalues ?
    \item What about the residual transducer ?
    \item What about weigthed automata? Does it mean something?
        -> we have unary input, so the automata really is an automata.
        -> we output numbers in $\Nat$
        -> residuals are what? totally ordered?
        -> recursive calls?
        -> if the function computes polynomials in $\Nat$, we can build
        the usual thing? and then we add "a few steps before"?

    \item Is the proof simpler by assuming
        $P(X + K) \in \Nat[X]$? This is way better.
    
    \item Prove decidability of the property.

    \item Show the equivalence with binomial

    \item What about 
\end{itemize}

%! TeX program = xelatex
%! lang = en-US
% MAIN MATTER
\section{Introduction}
\label{introduction:sec}

Traditional results from automata theory state the equivalence of languages
defined in terms of deterministic finite automata and regular expressions
\cite{kleene1956representation}, Monadic Second Order Logic ($\MSO$)
\cite{buchi1960weak}, or finite monoids \cite{schutzenberger1961definition}.
Among regular languages, there exists a robust subclass sharing similar
characterizations, namely the \emph{star-free} regular languages. These are
obtained via counter-free automata \cite{mcnaughton1971counter}, represented by
star-free regular expressions \cite{schutzenberger1965finite}, definable in
First Order Logic ($\FO$) \cite{perrin1986first}, and recognized by aperiodic
monoids \cite{schutzenberger1965finite}. It is decidable whether a regular
language is aperiodic, thanks to the characterizations in terms of minimal
automaton and/or syntactic monoid.

There have been recurrent attempts to generalize the notions of regularity from
languages to functions, leading to classes of various expressiveness
(sequential, rational, regular, and \kl{polyregular functions}) \cite[see,
e.g.][]{bojanczyk2019string}. In this paper, we will focus on \kl{polyregular
functions}, that shares with regular languages the characterizations in terms
of logic (via $\MSO$ interpretations), automata (via pebble automata or SST
\cite{}), and adds new bridges with programming languages (lambda-terms)
for-transducers. Although the class of "star-free" polyregular functions is
defined and similarly characterized using first-order logic.

Classical problems, including the decidability of aperiodicity, become
difficult in the case of \kl{polyregular functions}, due to the lack of
canonical objects akin to the syntactic monoid or the minimal automaton.
Indeed, for sequential functions that admit a canonical object, aperiodicity is
decidable \cite{choffrut03}, and more recently, it aperiodicity has been proven
decidable for rational functions
\cite{filiot2016aperiodicity,filiot2018canonical}. The problem remains open for
regular functions, although it has been proven decidable when changing the
semantics of function equality to account for the \emph{origins} in the output
\cite{bojanczyk14}.

Another way to approach the problem of deciding star-free for polyregular
functions is to not restrict the computational power, nor change the semantics,
but consider restrictions of the output alphabet. In particular, there have
been successful results proven on \emph{unary polyregular functions}, also
known as \kl{$\Nat$-polyregular functions}
\cite{doueneau2021pebble,doueneau2022hiding}. Let us note that results on
\emph{unary output functions} often generalize to the slightly more general
class of functions with \emph{commutative output}, and that algorithms leverage
the absence of combinatorial information in the output. Furthermore,
\kl{$\Nat$-polyregular functions} benefit from theorems proven in the field of
rational series \cite{berstel2011noncommutative}, where they coincide with
weighted automata of polynomial ambiguity, rational series of polynomial
growth, and a notion of quantitative $\MSO$
\cite{berstel2011noncommutative,kreutzer2013,schutzenberger1962}. Although
promising, the restriction to unary output did not yield effective procedure to
decide the aperiodicity of polyregular functions.
Notice that unary outputs in $\Rel$ and $\Nat$
are therefore special kind of weighted automata,
and that questions arising from the weighted automata
theory (good first order logic?) are also connected.
\cite{droste2019aperiodic,reutenauer_series_1980}.

Recently, the introduction of $\Rel$-output polyregular functions, called
\kl{$\Rel$-polyregular functions} (and more generally, commutative invertible
outputs) allowed to solve the general problem \cite{LOPEZ23b}. Note that
$\Rel$-output generalizes $\Nat$-output, but somehow has a combinatorial
description that is farther away from traditional \kl{polyregular functions}
because of the invertible outputs. The proof relies on the construction of a
canonical object essentially based on differences between functions, which is
not applicable to unary output. Note that already for rational series, the
problem is open.


We write $\Poly$ for the class of \kl{polyregular functions},
$\SF$ for the class of \kl{star-free polyregular functions},
$\NRat$ for the class of \kl{$\Nat$-rational series},
and $\ZRat$ for the class of \kl{$\Rel$-rational series}.
Arrows denote strict inclusions, and effectiveness
(both in terms of decidability and of effective representation)
is represented by thick double arrows.
Inclusions that are suspected to be effective have a question mark.
It is also conjectured that $\NPoly \cap \ZSF = \NSF$.

\begin{figure}
    \includestandalone[width=\linewidth]{tikz/class-inclusions}
    \caption{
        Decidability and inclusions of classes of functions. 
    }
\end{figure}

It was conjectured in a thesis that some particular form of polynomials
would characterize star-free functions.This paper follows the line of
research on \kl{polyregular functions} that focuses on semantic
characterizations of the asymptotic behavior, rather than the construction
of canonical objects of computation
\cite{doueneau2021pebble,doueneau2022hiding,LOPEZ23b}. The goal was to
decide arrows X,Y and Z, namely, to understand precisely the relationship
between star-free and $\Nat$ output. In the case of \kl{$\Nat$-rational
series}, a result of \citeauthor{KARH77} characterizes \kl{commutative}
\kl{$\Nat$-rational series} among \kl{$\Rel$-rational series}
\cite{KARH77}.
This conjecture was based on the previously introduced result of
\textcite[Theorem 3.3]{KARH77}, which we will disprove in \cref{sec:c-example}.
We then provide a corrected version of the theorem, and prove the conjecture of
Douéneau in the case of \emph{commutative} $\Nat$-rational series of polynomial
growth.

\hrule

\paragraph{Contributions.} 
\begin{itemize}
    \item We provide a candidate semantic characterization
    \item We prove that it is correct for commutative inputs
    \item We provide a counter-example to the theorem of nanania.
    \item We show that effective procedures are possible for all arrows 
        and answer positively to the conjecture of XX
        in the case of commutative inputs.
\end{itemize}


\paragraph*{Outline of the paper.}
\begin{itemize}
    \item In a first section, we recall the preliminary definitions.
    \item In a second section, we provide a counter example to the
        theorem of XX.
    \item In a third section, we provide an alternative result.
    \item In a fourth section, we make the results effective,
        and compute everyting.
\end{itemize}

%! TeX program = xelatex
%! lang = en-US
\section{Preliminaries}
\label{preliminaries:sec}

\AP In the rest of this paper, we use the symbol $\Rel$ to denote the set of
integers, $\Rat$ to denote the set of rational numbers, and $\Nat$ to denote
the set of non-negative integers. The capital letters $\Sigma,\Gamma$ denote
fixed alphabets, i.e. finite set of letters, and $\Sigma^*, \Gamma^*$ (resp.
$\Sigma^+, \Gamma^+$) are the set of words (resp. non-empty words) over
$\Sigma, \Gamma$. The empty word is written $\varepsilon \in \Sigma^*$. When $w
\in \Sigma^*$ and $a \in \Sigma$, we let $\card{w} \in \Nat$ be the length of
$w$, and $\card[a]{w}$ be the number of occurrences of $a$ in $w$. 

\AP
We assume that the reader is familiar with the basics of automata theory, in
particular the notions of monoid morphisms, idempotents in monoids, monadic
second-order ($\MSO$) logic and first-order ($\FO$) logic over finite words
(see e.g. \cite{THOM97}). As aperiodicity will be a central notion of this
paper, let us recall that a monoid $M$ is \intro(monoid){aperiodic} whenever
for all $x \in M$, there exists $n \in \Nat$ such that $x^{n+1} = x^n$.

\AP Let us briefly recall that one of the equivalent definitions of a
\intro{$K$-rational series} (where $K$ is a ring), and refer to \cite{BERE10}
for a comprehensive introduction. The \kl{$K$-rational series} from $\Sigma^*$
to $K$ are the functions computed by \intro{$K$-weighted automata} which are
non-deterministic finite automata with transitions labelled by elements of $K$,
and whose output on a given word is the sum over all accepting paths of the
products of the weights along the path. For instance, the function $w \mapsto
2^{\card{w}}$ is computed by an \kl{$\Nat$-weighted automaton} with a single
state, and a loop labelled by $2$, and is therefore an \kl{$\Nat$-rational
series}.

\AP We use the notation $\commute \colon \Sigma^* \to \Nat^\Sigma$ for the map
that counts occurrences of every letter in the input word (that is, computes
the Parikh vector) namely: $ \commute[w] \defined \seqof{a \mapsto
\card[a]{w}}{a \in \Sigma}$. Given a set $X$, a function $f \colon \Sigma^* \to
X$ is \intro{commutative} whenever for all $u \in \Sigma^*$, for all
permutations $\sigma$ of $\set{1, \dots, \card{w}}$, $f(\sigma(u)) = f(u)$.
Equivalently, it is \reintro{commutative} whenever there exists a map $g \colon
\Nat^\Sigma \to X$ such that $g \circ \commute = f$.

\AP Let $n \in \Nat$, and let $\Sigma$ be a finite alphabet. Given a function
$\eta \colon \set{1, \dots, n} \to \Sigma$, we define the $\eta^\dagger \colon
\Nat^n \to \Sigma^*$ as $\eta^\dagger(\vec{x}) \defined \eta(1)^{x_1} \dots
\eta(n)^{x_n}$. A function $f \colon \Nat^n \to X$ is \intro{represented} by a
\kl{commutative} function $g \colon \Sigma^* \to X$ if there exists a map $\eta
\colon \set{1, \dots, n} \to \Sigma$ such that $g \circ \eta^\dagger = f$. This
notion will be useful to formally state that a polynomial ``is'' a
\kl{commutative} \kl{polyregular function}. For instance, the polynomial
function $P(X,Y) = X \times Y$ is \kl{represented} by the \kl{commutative}
function $g \colon \set{a,b}^* \to \Rel$ defined by $g(w) \defined \card[a]{w}
\times \card[b]{w}$.

\subsection{Polyregular Functions}

\AP Because the functions of interest in this paper have output in $\Nat$ or
$\Rel$, we will only provide the definition of \kl{polyregular functions} for
these two output semigroups, and we refer the reader to \cite{BOKL19} for the
general definition of \kl{polyregular functions}. As a model of computation, we
will use the following definition of transducers, that is based on the marble
transducers \cite{EHB99}, and follows the general pattern of transducers
introduced in \cite{CDTL23}. Note that we only consider $\Rel$-valued (resp.
$\Nat$-valued) functions, but could consider other commutative semigroups.

\begin{definition}
    Let $\mathcal{H}$ be a family of functions
    from $\Sigma^*$ to $\Rel$.
    An \intro{$\mathcal{H}$-transducer} $\aTransd$ is
    a tuple $(Q, q_0, \delta, \lambda, F)$ where
         $Q$ is a finite set of states,
         $q_0 \in Q$ is called the initial state of the transducer,
         $\delta \colon Q \times \Sigma \to Q$
            is called the deterministic transition function of the transducer,
         $\lambda \colon Q \times \Sigma \to \mathcal{H}$
            is called the correction function of the transducer,
         and $F \colon Q \to \Rel$ is called the output function of the transducer.
\end{definition}

Based on this syntax, let us now define the semantics of a transducer, in terms
of the function from $\Sigma^*$ to $\Rel$ that it computes. The intuition is
that it suffices to apply the correction function $\lambda$ on every transition
of the underlying automaton, sum the results, and add to this number the output
associated to the final state of the computation.

\begin{definition}
    \label{transducer-sem:def}
    Let $\aTransd \defined (Q,q_0, \delta, \lambda, F)$ be an \kl{$\mathcal{H}$-transducer}.
    The function
    $\aTransd \colon Q \times \Sigma^* \to \Rel$
    is defined inductively as follows:
        $\aTransd(q, \varepsilon) \defined F(q)$, and
        $\aTransd(q, a u) \defined \aTransd(\delta(q,a), u)
            + \lambda(q,a)(u)$.
    The function \intro{computed} by a transducer $\aTransd$
    is $w \mapsto \aTransd(q_0, w)$.
\end{definition}

\begin{example}
    \label{simple-transd:ex}
    Let $\mathsf{Const}$ be the set of constant functions from $\Sigma^*$ to $\Rel$,
    and let us write $\const{n}$ for the constant function equal to $n \in \Rel$.
    In 
    \cref{simple-transd:fig}, we provide a graphical depiction of a
    \kl{$\mathsf{Const}$-transducer} that \kl{computes} the function
    $f \colon \set{a,b}^* \to \Rel$ defined by
    $f(\varepsilon) = 1$,
    $f(aw) = 3 \times \card{w}$,
    and $f(bw) = 3 \times \card{w} + 1$.
\end{example}

\begin{figure}
    \centering
    \begin{tikzpicture}[
        etat/.style={minimum size=2em}
        ]
        \node[etat,state,initial,
            accepting by arrow,
            accepting text={$1$},
            accepting where=below,
            ] (A) at (0,0) {$q_0$};
        \node[etat,state,
            accepting by arrow,
            accepting text={$0$},
            accepting where=below,
            ] (B) at (2.5,0) {$q_1$};

        \draw[->] (A) to[bend right=20] node[midway, below] {$a \mid \const{0}$} (B);
        \draw[->] (A) to[bend left=20] node[midway, above] {$b \mid \const{1}$} (B);


        \draw[->] (B) edge[loop right] node[midway, right]
            {$a,b \mid \const{3}$} (B);
    \end{tikzpicture}
    \caption{A \kl{$\mathsf{Const}$-transducer}, where $\mathsf{Const}$ is the set of constant functions
        from $\set{a,b}^*$ to $\Rel$,
        with a set of states $Q \defined \set{ q_0, q_1 }$,
        an initial state $q_0$,
        an output function $F \colon Q \to \Rel$ such that $F(q_0) = 1$ and $F(q_1) = 0$,
        and correction functions $\lambda(q,a) = \const{0}$, $\lambda(q,b) = \const{1}$,
        and $\lambda(q_1,a) = \lambda(q_1,b) = \const{3}$.
    }
    \label{simple-transd:fig}
\end{figure}

\AP Let us recall that a \intro{counter} in an automaton is a pair $u,w$ of
words together with an integer $n > 1$, such that $\delta(q_0, uw^n) =
\delta(q_0, u)$ and $\delta(q_0, u w^k) \neq \delta(q_0, u)$ for all $k < n$;
where we used the extension of $\delta \colon Q \times \Sigma \to Q$ to words
by defining $\delta(q, \varepsilon) = q$, and $\delta(q,au) =
\delta(\delta(q,a), u)$. An automaton is \reintro{counter-free} if it contains
no \reintro{counters}. This notion can be lifted to
\kl{$\mathcal{H}$-transducers} by considering counters of their underlying
automata. 

\AP The notion of \kl{$\mathcal{H}$-transducer} can be used to inductively
define the classes of
\intro{$\Rel $-polyregular functions}
$\intro*\ZPoly[k]$,
\intro{$\Nat $-polyregular functions}
$\intro*\NPoly[k]$,
\intro{star-free $\Rel $-polyregular functions}
$\intro*\ZSF[k]$, and
\intro{star-free $\Nat $-polyregular functions}
$\intro*\NSF[k]$, 
as follows \cite[Theorems 5.15 and 7.10]{DOUE23}: in the base
case, $\reintro*\NPoly[-1] = \reintro*\NSF[-1]  = \reintro*\ZSF[-1] =
\reintro*\ZPoly[-1]$ is the singleton containing the constant function equal to
$0$; and for $k \geq 0$, $\reintro*\ZPoly[k]$ is the set of functions
\kl{computed} by a \kl{$\reintro*\ZPoly[k-1]$-transducer}, and
$\reintro*\NPoly[k]$ is the set of functions \kl{computed} by an
\kl{$\reintro*\NPoly[k-1]$-transducer} \emph{with an $\Nat$-valued output
function $F$}, and similarly for $\reintro*\NSF[k]$ and $\reintro*\ZSF[k]$,
where the transducer is furthermore required to be \kl{counter-free}. Then,
$\reintro*\NPoly = \bigcup_{k \in \Nat} \reintro*\NPoly[k]$, and classes
$\reintro*\NSF$, $\reintro*\ZPoly$ and $\reintro*\ZSF$ are defined analogously.
In the rest of the paper, we will always assume that
\kl{$\reintro*\NPoly[k]$-transducers} and \kl{$\reintro*\NSF[k]$-transducers}
have an $\Nat$-valued output functions, making a clean separation between
$\NPoly$ and $\ZPoly$.

\begin{example}
    \label{non-canonical-transd:ex}
    Let $\BadExOk \colon \set{a}^* \to \Nat$
    be defined as
    $\BadExOk(\varepsilon) = 1$,
    $\BadExOk(a) = 0$,
    and $\BadExOk(a^n) = n-1$ for all $n \geq 2$.
    The function $\BadExOk$ is \kl{computed} by the two \kl{$\NPoly[0]$-transducers}
    depicted in \cref{non-canonical-transd:fig}.
    As a consequence, $f \in \NPoly[1]$.
    Furthermore, because the automaton of 
        \cref{non-canonical-transd:fig:counter-free}
        has no \kl{counter},
    and the correction function only uses functions in $\NSF[0]$,
    $f \in \NSF[1]$.
\end{example}
\begin{proofof}[non-canonical-transd:ex]
    Let us prove that the transducer of
    \cref{non-canonical-transd:fig:counter-free}
    computes $f$.
    We prove by induction on $w$
    that $A(q_0, w) = f(w)$,
    and that $A(q_1, w) = f(aw)$.
    Let us first remark that $f(au) = \card{u}$.
    
    When $w = \varepsilon$, $A(q_0, \varepsilon) = F(\varepsilon) = 1 = f(\varepsilon)$.
    Similarly, $A(q_1, \varepsilon) = F(q_1) = 0 = f(a)$.

    Assume that $w = au$. Then:
    \begin{align*}
        A(q_0, w) = A(q_0, au) &= \lambda(q_0, a)(u) + A(q_1, u) \\ 
                               &= 0 + A(q_1,u) \\
                               &= f(au) & \text{by induction hypothesis} \\
                               &= f(w) 
    \end{align*}
    Similarly,
    \begin{align*}
        A(q_1, w) = A(q_1, au) &= \lambda(q_1, a)(u) + A(q_1, u) \\ 
                               &= 1 + \ind{\card{u} = 0} + A(q_1,u) \\
                               &= 1 + \ind{\card{u} = 0} + f(au) & \text{by induction hypothesis} \\
                               &= \begin{cases}
                                    2 + f(a) & \text{if $u = \varepsilon$} \\
                                    1 + f(au) & \text{otherwise}
                               \end{cases}
                               \\
                               &= \begin{cases}
                                    2 & \text{if $u = \varepsilon$} \\
                                    1 + \card{u} & \text{otherwise}
                               \end{cases}
                               \\
                               &= \begin{cases}
                                   f(aa) & \text{if $u = \varepsilon$} \\
                                   f(
                               \end{cases}
    \end{align*}
\end{proofof}


\begin{figure}
    \centering
    \begin{subfigure}[b]{0.49\linewidth}
        \begin{tikzpicture}[
            etat/.style={minimum size=2em}
            ]
            \node[etat,state,initial,
                accepting by arrow,
                accepting text={$1$},
                accepting where=below,
                ] (A) at (0,0) {$\varepsilon$};
            \node[etat,state,
                accepting by arrow,
                accepting text={$0$},
                accepting where=below,
                ] (B) at (2,0) {$a$};

            \draw[->] (A) to node[midway, below] {$a \mid 0$} (B);

            \draw[->] (B) to[bend right=45] node[midway, above] {$a \mid 1 + \ind{\card{w} \geq 1}$} (A);
        \end{tikzpicture}
        \caption{This automaton has a \kl{counter}.}
    \end{subfigure}
    \begin{subfigure}[b]{0.49\linewidth}
        \begin{tikzpicture}[
            etat/.style={minimum size=2em}
            ]
            \node[etat,state,initial,
                accepting by arrow,
                accepting text={$1$},
                accepting where=below,
                ] (A) at (0,0) {$\varepsilon$};
            \node[etat,state,
                accepting by arrow,
                accepting text={$0$},
                accepting where=below,
                ] (B) at (2,0) {$a$};

            \draw[->] (A) to node[midway, below] {$a \mid 0$} (B);

            \draw[->] (B) edge[loop right] node[midway, right]
                {$a \mid 1$} (B);
        \end{tikzpicture}
        \caption{This automaton is \kl{counter-free}.}
        \label{non-canonical-transd:fig:counter-free}
    \end{subfigure}
    \caption{Two \kl{$\NPoly[0]$-transducers}
    \kl{computing} the function $\BadExOk$ of \cref{non-canonical-transd:ex},
    that maps $a^n$ to $n$ for all $n \geq 2$, $a$ to $0$, and $\varepsilon$ to $1$.
    }
    \label{non-canonical-transd:fig}
\end{figure}


\AP One of the appeals of $\NPoly$ and $\ZPoly$ are the numerous
characterizations of these classes in terms of logic, weighted automata, and
the larger class of \kl{polyregular functions} \cite{CDTL23,DOUE23}. In this
paper, the main focus will be the connection to weighted automata, which is
based on the notion of \emph{growth rate}. The \intro{growth rate} of a
function $f \colon \Sigma^* \to \Rel$ is defined as the minimal $k$ such that
$\card{f(w)} = \bigO\left(\card{w}^k\right)$. If such a $k$ exists, we say that
the function $f$ has \intro{polynomial growth}. It turns out that for all $k
\in \Nat$, $\ZPoly[k]$ (resp. $\NPoly[k]$) are precisely functions in $\ZRat$
(resp. in $\NRat)$ that have growth rate at most $k$. Not all
\kl{$\Nat$-rational series} have \kl{polynomial growth}, a typical example
being the function $a^n \mapsto 2^n$.

Let us briefly state that \kl{commutativity} is a decidable property of
\kl{$\Rel$-rational series}, hence of \kl{$\Rel$-polyregular functions}.

\begin{lemma}
    \label{decidable-commutative-poly:lemma}
    \label{decidable-commutative-rat:lemma}
    Let $f \in \ZRat$. One can decide if 
    $f$
    is \kl{commutative}.
\end{lemma}

\subsection{Polynomials} 
\AP All polynomials considered in this paper have
coefficients in $\Rel$ unless explicitly stated otherwise. 
A polynomial $P \in \Rel[X_1, \dots, X_k]$ is \intro{non-negative} when for
all non-negative integer inputs $n_1, \dots, n_k \geq 0$, the output  $P(n_1,
\dots, n_k)$ of the polynomial is non-negative. Beware that we do not consider
negative values as input, as the numbers $n_i$ will ultimately count the number
of occurrences of a letter in a word. As an example, the polynomial $(X - Y)^2$
is \kl{non-negative}, and so is the polynomial $X^3$, but the polynomial $X^2 -
2X$ is not.

\AP Let us now give some vocabulary on polynomials with multiple indeterminates
over $\Rel$. A \intro{monomial} is a product of indeterminates and integers.
For instance, $XY$ is a \kl{monomial}, $3 X$ is a \kl{monomial}, $-Y$ is a
\kl{monomial}, but $X + Y$ or $2X^2 + XY$ are not. We write $\intro*\Monomials[X_1,
\dots, X_n]$ for the set of \kl{monomials} over these indeterminates.
Every polynomial $P \in \Rel[X_1, \dots, X_n]$ decomposes uniquely
into a sum of \kl{monomials}.

\AP A \kl{monomial} $S$ \intro{divides} a \kl{monomial} $T$, when $S$ divides
$T$ seen as polynomials in $\Rat$. For instance, $2X$ \kl{divides} $XY$, $-YZ$
\kl{divides} $X^2 Y Z^3$, and $Y$ does not \kl{divide} $X$. In the
decomposition of $P \in \Rel[X_1, \dots, X_k]$, a \kl{monomial} is a
\intro{maximal monomial} if it is a maximal element for the \kl{divisibility
ordering} of \kl{monomials}. In the polynomial $P(X,Y) \defined X^2 - 2XY + Y^2
+ X + Y$, the set of \reintro{maximal monomials}, written
$\intro*\MaximalMonomials(P)$, is $\set{X^2,  -2 XY,  Y^2}$.  In the polynomial
$P(X,Y) \defined (X - Y)^2$, the \kl{non-negative} \kl{monomials} are $X^2$ and
$Y^2$.

%! TeX program = xelatex
%! lang = en-US
\section{$\Nat$-rational Polynomials}
\label{polynomials:sec}

The goal of this section is to provide semantic characterizations of
\kl{$\Nat$-rational polynomials}, decide if a polynomial $P$ is an
\kl{$\Nat$-rational polynomial}, and if so, explicitly compute an
\kl{$\Nat$-polyregular function} $f$ that \kl{represents} $P$. It is an easy
check that polynomials with coefficients in $\Nat$ are \kl{$\Nat$-rational
polynomials} (\cref{n-poly-n-poly:example}). However,
\cref{negative-but-npoly:ex} provides a polynomial with negative coefficients
that is an \kl{$\Nat$-rational polynomial}. As a consequence, the class of
interest strictly contains $\Nat[\vec{X}]$.

\begin{lemma}
    \label{n-poly-n-poly:example}
    Let $P \in \Nat[\vec{X}]$. Then, $P$
    is an \kl{$\Nat$-rational polynomial}.
    \proofref{n-poly-n-poly:example}
\end{lemma}

\begin{example}
    \label{negative-but-npoly:ex}
    Let $P \defined X^2 -2X + 2$. Then $P$
    is an \kl{$\Nat$-rational polynomial}.
    \proofref{negative-but-npoly:ex}
\end{example}

Let us now introduce the characterization of \kl{$\Nat$-rational polynomials}
given by \cite{KARH77}, which is stated using a class
$\CoveredPoly[\vec{X}]$ of polynomials with multiple indeterminate.

\begin{definition}[{\cite[Section 3, page 3]{KARH77}}]
    Let $\vec{X}$ be a finite tuple of indeterminate.
    The class $\intro*\CoveredPoly[\vec{X}]$
    is the class of polynomials $P \in \Rel[\vec{X}]$
    that are \kl{non-negative}
    and such that every \kl{maximal monomial} is \kl{non-negative}.
    When the indeterminate are clear from the context, we write
    this class $\reintro*\CoveredPoly$.
\end{definition}

\begin{faketheorem}[{\cite[Theorem 3.3, page 4]{KARH77}}] 
    \label{karh:thm}
    Let $\Sigma$ be a finite alphabet
    and $P$ be a polynomial. Then,
    $P$ is an \kl{$\Nat$-rational polynomial}
    if and only if 
    $P \in \CoveredPoly$.
\end{faketheorem}


\subsection{The counter example}
\label{sec:c-example}

Let us now provide a counter example to \cref{karh:thm}. The counter example
will use three indeterminates, and we will prove in \cref{sec:proof} that
\cref{karh:thm} holds on polynomial with at most two indeterminate. In
particular, the examples appearing in \cite{KARH77} are not invalidated, as
they all use at most two indeterminates. Furthermore, the converse implication,
that is, that \kl{$\Nat$-rational polynomials} belong to $\CoveredPoly$ is
correct, and can even be strengthened as we will see in
\cref{corrected-version:thm}.

\begin{definition}
    \label{def:bad-polynomial}
    We define $\intro*\BadPoly(X,Y,Z) \defined Z (X + Y)^2 + 2 (X - Y)^2$.
\end{definition}

\begin{lemma}
    \label{thm:counter-example}
    The polynomial $\BadPoly$ belongs to $\CoveredPoly$,
    but is not an \kl{$\Nat$-rational polynomial}.
\end{lemma}
\begin{proof}
    It is clear that $\BadPoly$ is \kl{non-negative}. We can develop
    the expression of $\BadPoly$ to 
    obtain
    $\BadPoly = ZX^2 + ZY^2 + 2ZXY + 2X^2 -4XY + 2Y^2$.
    The \kl{maximal monomials} of $P$
    are $ZX^2$, $ZY^2$, and $2ZXY$, all of which are
    \kl{non-negative}.

    Assume by contradiction that $\BadPoly$ is an \kl{$\Nat$-rational polynomial}.
    Let $\Sigma \defined \set{a,b,c}$ be a finite alphabet.
    There exists a \kl{commutative}
    \kl{$\Nat$-polyregular function} $f \colon \Sigma^* \to \Nat$
    such that for all $w \in \Sigma^*$,
    $\BadPoly(\card[a]{w}, \card[b]{w}, \card[c]{w}) = f(w)$.

    Remark that $\BadPoly(X,Y,Z) = 0$
    if and only if $Z(X+Y)^2 = -2 (X-Y)^2$. Hence,
    $\BadPoly(X,Y,Z)=0$ if and only if $Z = 0$ and $X = Y$, or 
    $Z \neq 0$, and $X = Y = 0$.

    Now, let us consider the language $L \defined \setof{w}{ f(w) = 0}$. By the
    above computation, we conclude that $L = \setof{ w \in \set{a,b}^* }{
    \card[a]{w} = \card[b]{w} } \cup \set{ c }^*$.
    Because $L \cap \set{a,b}^*$ not a regular language,
    we
    conclude that $L$ is not a regular language.
    However, $L = f^{-1}(\set{0})$ is a regular language
    (\cref{pre-image-regular:fact}). 
\end{proof}

\begin{corollary}
    The result stated in \cite[Theorem 3.3]{KARH77}, restated
    in \cref{karh:thm}, is false
    when allowing at least $3$ indeterminates.
\end{corollary}

%! TeX program = xelatex
%! lang = en-US
\subsection{The Corrected Theorem}
\label{sec:proof}

\AP
The counter example provided by \cref{def:bad-polynomial} relies on the fact
that $\CoveredPoly$ is not stable under fixing indeterminates, while
\kl{$\Nat$-polyregular functions} are. In this section, we prove that closing
$\CorrectPoly$ under assignments of variable is enough to recover from
\cref{karh:thm}.
We use the following notation to fix the value of some indeterminate, if
$P(X,Y)$ is a polynomial in $\Rel[X,Y]$, then $\intro*\restr{P(X,Y)}{X = 1}$ is
the polynomial $P(1,Y) \in \Rel[Y]$. More generally, if $\nu$ is a partial
function from $\vec{X}$ to $\Nat$, written $\nu \colon \vec{X} \topartial
\Nat$, the restriction $\restr{P(\vec{X})}{\nu}$ is the polynomial with
indeterminates $\vec{Y} \defined \vec{X} - \dom(\nu)$ obtained by fixing the
variables of the domain of $\nu$.


\begin{definition}
    Let $\vec{X}$ be a finite tuple of indeterminates.
    The class $\intro*\CorrectPoly[\vec{X}]$ is the collection of
    polynomials $P \in \Rel[\vec{X}]$ such that
    $P$ is 
    such that, for every partial function $\nu \colon \vec{X} \topartial \Nat$,
    every \kl{maximal monomial} of
    $\restr{P}{\nu}$ is \kl{non-negative}.
\end{definition}

First, let us remark that $\CorrectPoly \subseteq \CoveredPoly$, because
polynomials in $\CorrectPoly$ are \kl{non-negative}. Let us also check that the
counter example provided in \cref{thm:counter-example} is not in
$\CorrectPoly$. For that, notice that for $\BadPoly$ introduced in
\cref{def:bad-polynomial}, $\BadPoly(X,Y,1) = 3X^2 + 3Y^2 - 2XY$, which has a negative
coefficient for a \kl{maximal monomial}, namely $-2XY$. 

Let us now prove that \kl{$\Nat$-rational polynomials} are in $\CorrectPoly$.
This follows from the correct implication in the statement of \cref{karh:thm},
but we provide a self-contained proof using a refinement of the
classical combinatorial arguments for $\ZPoly$ \cite[Lemma 4.16]{CDTL23} and
$\NPoly$ \cite[Lemma 5.37]{DOUE23}.

\begin{lemma}
    \label{n-poly-combinatorics:lem}
    Let $f$ be an \kl{$\Nat$-polyregular} function. 
    There exists a computable $\omega \in \Nat$
    such that for all $p \in \Nat$,
    for all $\alpha_0, \dots, \alpha_p \in \Sigma^*$,
    for all $u_1, \dots, u_p \in \Sigma^*$,
    there exists a polynomial $P \in \Rel[X_1, \dots, X_p]$
    whose \kl{maximal monomials} are \kl{non-negative},
    and such that for all $X_1, \dots, X_p \geq \omega$:
    \begin{equation*}
        f\left(
            \alpha_0 \prod_{i = 1}^p u_i^{\omega X_i} \alpha_i
        \right)
        = P(X_1, \dots, X_p) \quad .
    \end{equation*}
    \proofref{n-poly-combinatorics:lem}
\end{lemma}


\begin{corollary}
    \label{n-rat-correct:lem}
    Let $P \in \Rel[X_1, \dots, X_p]$ be an \kl{$\Nat$-rational polynomial}.
    Then,
    $P \in \CorrectPoly$.
    \proofref{n-rat-correct:lem}
\end{corollary}

\AP The core of the upcoming \cref{lem:correct-to-n-rat} leverages a notion of
\intro{discrete derivative} to perform an induction on the \kl{maximal monomials}.
This notion of derivation is built by translating the domain of the polynomial.
To that end, let us write $\intro*\translate{K}$ for the \intro{translation function}
that maps a polynomial $P \in \Rel[X_1, \dots, X_n]$ to the polynomial $P(X_1 +
K, \dots, X_n + K)$.

\begin{definition}
    \label{discrete-derivative:def}
    Let
    $K \in \Nat$,
    and 
    $P \in \Rel[\vec{X}]$ be a polynomial,
    then 
    %\begin{equation*}
    $
        \intro*\Diff{K}{P} \defined 
        \translate{K}(P) - P
    $.
    %    \quad .
    %\end{equation*}
\end{definition}


\AP Let us now introduce the ordering over which the induction of
\cref{lem:correct-to-n-rat} is built. Recall that $\MaximalMonomials(P)$ is the
set of \kl{maximal monomials} of $P$, hence belongs to $\Pfin(\Monomials)$. We
order \kl{monomials} with the \kl{divisibility ordering}, making $\Monomials$ a
\kl{well-quasi-ordering} that is isomorphic to $\Nat^k$ with the product
ordering \cite[Dickson’s Lemma]{SCSC12}. We endow $\Pfin(\Monomials)$ with the
\intro{Hoare ordering}, that is, $S_1 \hoareleq S_2$ whenever for all
\kl{monomials} $M_1 \in S_1$, there exists a monomial $M_2 \in S_2$, such that
$M_1$ \kl{divides} $M_2$. The set $(\Pfin(\Monomials), \hoareleq)$ remains a
\kl{well-quasi-ordering}, and is in particular well-founded \cite[Hoare
quasi-ordering]{SCSC12}.

Let us now illustrate how the \kl{discrete differentiation} operator interacts
with \kl{maximal monomials} with respect to the \kl{Hoare ordering}: it
extracts \emph{sub-maximal} monomials from a given polynomial.

\begin{lemma}
    \label{derivation-simplifies:lemma}
    For all $P \in \Rel[\vec{X}]$ that are non-constant,
    for all $K \in \Nat$,
    $\MaximalMonomials(\Diff{K}{P}) \hoarele
    \MaximalMonomials(P)$.
    \proofref{derivation-simplifies:lemma}
\end{lemma}

The following \cref{derivation-stabilises-correct:lem} is the main
combinatorial argument of this section. It leverages the positivity of the
\kl{maximal monomials} to compute a shift large enough to make lower degree
coefficients positive. This lemma will be lifted to a given polynomial $P$ by
noticing that if $K \in \Nat$, and $P = P_1 + P_2$, then $\translate{K}(P) =
P_1 + \Diff{K}{P_1} + \translate{K}(P_2)$.

\begin{lemma}
    \label{derivation-stabilises-correct:lem}
    Let $P \in \CorrectPoly$,
    $P_1$ be the sum of \kl{maximal monomials} of $P$,
    and $P_2 \defined P - P_1$ be the sum of
    non-maximal monomials of $P$.
    There exists a computable $K$,
    such that
    $Q \defined (\Diff{K}{P_1} + \translate{K}(P_2)) \in \CorrectPoly$.
    \proofref{derivation-stabilises-correct:lem}
\end{lemma}

%In the actual statement of \cref{corrected-version:thm}, the representation of
%polynomials in $\CorrectPoly$ is strengthened from \kl{$\Nat$-polyregular
%functions} to \kl{star-free $\Nat$-polyregular functions}. This is not
%surprising, given the conjecture that \kl{ultimately polynomial}
%\kl{$\Nat$-polyregular functions} (the formal definition will be given in
%\cref{star-free:sec}) were conjectured to be star free
%(\cref{zsf-nsf:conjecture}), and that polynomials are always \kl{ultimately
%polynomial}.

\begin{lemma}
    \label{lem:correct-to-n-rat}
    Let $\vec{X}$ be a tuple of indeterminates,
    and let $P \in \Rel[\vec{X}]$.
    If $P \in \CorrectPoly$, then $P$ is \kl{represented}
    by a \kl{star-free $\Nat$-polyregular function},
    which can be explicitly constructed from $P$.
\end{lemma}


While \cref{lem:correct-to-n-rat} provides effective conversion, it does not
explicitly state that the membership is decidable to keep the proof clearer. A
similar proof scheme can be followed to conclude that membership is decidable,
and even show that elements in $\CorrectPoly$ are, up to suitable translations,
polynomials in $\Nat[\vec{X}]$. Beware that partial applications are still
needed in this characterization, as \cref{bad-poly-translate:ex} illustrates.

\begin{lemma}
    \label{derivation-translation:lem}
    Let $P \in \Rel[\vec{X}]$, 
    there exists a computable $K \in \Nat$
    such that the following are equivalent:
    \begin{enumerate}
        \item \label{d-t-correct:item} $P \in \CorrectPoly$,
        \item \label{d-t-transl:item}
            for 
            all partial functions $\nu \colon \vec{X} \topartial \Nat$,
            $\translate{K}(\restr{P}{\nu}) \in \Nat[\vec{X}]$,
        \item \label{d-t-transl-fin:item}
            for all partial functions
            $\nu \colon \vec{X} \topartial \set{0, \dots, K}$,
            $\translate{K}(\restr{P}{\nu}) \in \Nat[\vec{X}]$.
    \end{enumerate}
    Furthermore, the membership is decidable.
    \proofref{derivation-translation:lem}
\end{lemma}


\begin{example}
    \label{bad-poly-translate:ex}
    The polynomial $\BadPoly$ is not a 
    \kl{$\Nat$-rational polynomial},
    but is \kl{non-negative} and satisfies
    $\translate{10}(\BadPoly) \in \Nat[\vec{X}]$.
\end{example}

We are now ready to state the corrected and generalized version of
\cref{karh:thm}, which is the main technical contribution of the paper.

\begin{theorem}
    \label{corrected-version:thm}
    Let $P \in \Rel[\vec{X}]$ be a polynomial.
    The following are equivalent:
    \begin{enumerate}
        \item \label{corrected-1:item} $P \in \CorrectPoly$,
        \item \label{corrected-2:item} $P$ is \kl{represented} by a \kl{$\Nat$-rational series},
        \item \label{corrected-3:item} $P$ is \kl{represented} by a \kl{$\Nat$-polyregular function},
        \item \label{corrected-4:item} $P$ is \kl{represented} by a \kl{star-free $\Nat$-polyregular function},
    \end{enumerate}
    Furthermore, the properties are decidable, and conversions effective.
\end{theorem}


The counter example of \cref{thm:counter-example} used three indeterminates,
and this is not a coincidence: in the particular case of at most two
indeterminates, the classes $\CorrectPoly$ and $\CoveredPoly$ coincide. In
particular, the examples appearing in \cite{KARH77} are not invalidated, as
they all use at most two indeterminates. This is because, in the
univariate case, being non-negative and having non-negative maximal coefficient
implies being an \kl{$\Nat$-rational polynomial}.

\begin{lemma}
    \label{lem:correct-covered-2}
    $\CorrectPoly[X,Y] = \CoveredPoly[X,Y]$.
    \proofref{lem:correct-covered-2}
\end{lemma}


%! TeX program = xelatex
%! lang = en-US
\section{Beyond Polynomials}
\label{beyond-polynomials:sec}
\label{star-free:sec}

The goal of this section is to go beyond polynomials and consider the case of
\kl{commutative} \kl{$\Rel$-polyregular functions}. The hypothesis of
\kl{commutativity} reduces $\ZPoly$ to a combination of a finite number of
polynomials, arranged by counting modulo some numbers. This characterization
will be proven in \cref{decompose-polynomial:lem}, and will be the key
ingredient in the decision procedures of $\NPoly$ inside $\ZPoly$, and $\NSF$
inside $\NPoly$. Note that \cref{decompose-polynomial:lem} is actually a
refinement of \cref{n-poly-combinatorics:lem} under the extra assumption of
\kl{commutativity}. 

\AP To actually state the lemma, we will need some notations that will
formalize the intuitive idea of counting tuples of integers modulo some
constant. Let $\omega \in \Nat$, we define the set $\intro*\ModuloTypes[\omega]
\defined \set{0, \dots, \omega^2} \times \set{0, \dots, \omega - 1}$ of
\intro{types modulo $\omega$}. To a number $n \in \Nat$, we associate its
\reintro{$\omega$-type} written $\intro*\moduloType[\omega](n)$ which is
defined as the pair $(t, r)$ where $t = \min (n, \omega^2)$ and $r$ is the rest
of the Euclidian division of $n$ by $\omega$. This notion of type is lifted to
vectors in $\Nat^p$ for any $p \in \Nat$ by pointwise application.


\begin{lemma}
    \label{decompose-polynomial:lem}
    Let $f \colon \Sigma^* \to \Rel$ be a \kl{commutative}
    \kl{$\Rel$-polyregular function},
    where we fix the alphabet $\Sigma = \set{a_1, \dots, a_n}$.
    There exists a computable
    $\omega \in \Nat$,
    and computable 
    polynomials $\seqof{P_{t}}{t \in \ModuloTypes[\omega]^n}$
    such that for all $w \in \Sigma^*$,
    \begin{equation*}
        f\left(a_1^{X_1} \cdots a_n ^{X_n}\right) 
        = P_{\moduloType[\omega](X_1, \dots, X_n)}
        \left(
            \floor{X_1/\omega}, \dots, \floor{X_n/\omega}
        \right)
        \quad .
    \end{equation*}
\end{lemma}

In the spirit of previous characterizations of subclasses of $\ZPoly$ in terms
of \emph{polynomial pumping arguments}
\cite{doueneau2021pebble,doueneau2022hiding,LOPEZ23b}, we will provide a
semantic property in the form of \cref{k-combinatorial:def}, that is conjecture
to characterize $\NPoly$ inside $\ZPoly$, even in the non-commutative setting.

\begin{definition}
    \label{k-combinatorial:def}
    Let $k \in \Nat$, and $f \colon \Sigma^* \to \Rel$
    be a \kl{$\Rel$-polyregular function}. The function $f$ is 
    \intro{$(k,\Nat)$-combinatorial} if there exists $\omega \in \Nat$,
    such that
    for all
    $\alpha_0, \dots, \alpha_k \in \Sigma^*$,
    for all $u_1, \dots, u_k \in \Sigma^*$,
    there exists a polynomial $P \in \CorrectPoly$,  
    satisfying for all $X_1, \dots, X_k \geq \omega$:
    \begin{equation*}
        f
        \left(
            \alpha_0 \prod_{i = 1}^k u_i^{\omega X_i} \alpha_i
        \right)
        = 
        P(X_1, \dots, X_k) \quad .
    \end{equation*}
    We say that $f$ is \reintro{combinatorial}
    if it is \reintro{$(k,\Nat)$-combinatorial} for all $k \in \Nat$.
\end{definition}

\begin{theorem}
    \label{decidable-n-poly:thm}
    Let $k \in \Nat$, and 
    let $f \in \ZPoly[k]$ be a \kl{commutative} \kl{$\Rel$-polyregular function}.
    The following are equivalent:
    \begin{enumerate}
        \item \label{f-combinatorial:item} $f$ is \kl{$(k,\Nat)$-combinatorial},
        \item \label{f-npoly-combi:item} $f \in \NPoly[k]$,
    \end{enumerate}
    Furthermore, the properties are decidable,
    and conversions effective.
\end{theorem}

Let us now prove that the above characterizations of \kl{commutative}
\kl{$\Nat$-polyregular functions} can be combined with the recent advances in
the study of \kl{$\Rel$-polyregular functions} \cite{LOPEZ23b} to decide
aperiodicity. The key ingredient of this study is the use of a semantic
characterization of \kl{star-free $\Rel$-polyregular functions} among
\kl{$\Rel$-rational series} that generalizes the aperiodicity of languages to
functions by the means of polynomial behaviors (see
\cref{aperidic-ultimately-polynomial:ex}).

\begin{definition}[Ultimately polynomial]
    \label{ultimately-polynomial:def}
    Let $\Sigma$ be a finite alphabet. 
    A function $f \colon \Sigma^* \to \Rel$
    is \intro{ultimately polynomial}
    whenever there exists $N_0 \in \Nat$ such that
    for all $n \in \Nat$
    and for all words $\alpha_0, w_1, \alpha_1, \cdots, \alpha_{n-1}, w_n, \alpha_n
    \in \Sigma^*$, there exists a polynomial $P \in \Rel[X_1, \dots, X_n]$
    such that
    \begin{equation*}
        f\left(
            \alpha_0 \prod_{i = 1}^{n} w_i^{X_i} \alpha_i
        \right)
        = 
        P(X_1, \dots, X_n)
        \quad 
        \forall X_1, \dots, X_n \geq N_0
        \quad .
    \end{equation*}
\end{definition}

\begin{example}
    \label{aperidic-ultimately-polynomial:ex}
    A language $L$ is aperiodic if and only if 
    $\ind{L}$ is \kl{ultimately polynomial}.
\end{example}

\begin{theorem}
    \label{zsf-npoly-nsf:thm}
    Let $\Sigma$ be a finite alphabet, 
    and $f \colon \Sigma^* \to \Rel$ be a \kl{commutative}
    \kl{$\Nat$-polyregular function}.
    Then, the following are equivalent:
    \begin{enumerate}
        \item $f$ is \kl{ultimately polynomial},
        \item $f \in \ZSF$,
        \item $f \in \NSF$.
    \end{enumerate}
    Furthermore, membership is decidable and conversions are effective.
\end{theorem}



%! TeX program = xelatex
%! lang = en-US
\section{Outlook}
\label{sec:ccl}

Let us conclude with a more general remark on the status of commutative input
in the study of unary output polyregular functions. Notice that semantic
characterizations of subclasses of $\ZPoly$ (\kl{combinatorial}, \kl{ultimately
polynomial}) arise from pre-composition with a \kl{commutative}
\kl{star-free polyregular function}, as formally stated in
\cref{pre-compose-growth-commut:lemma,pre-compose-sf-commut:lemma}.

%Indeed, a suitable application of Parikh's Theorem \cite{PARI66}
%guarantees the existence of the functions $\mathsf{enc}$ and $\mathsf{dec}$,
%although they may fail to be  \kl{polyregular functions}.

\begin{lemma}
    \label{pre-compose-growth-commut:lemma}
    Let $f \in \ZRat$, and $k \in \Nat$. Then,
    $f \in \ZPoly[k]$ if and only if 
    for every \kl{commutative} \kl{star-free polyregular function} $h$
            of \kl{growth rate} $l \in \Nat$,
            $(f \circ h) \in \ZPoly[k\times l]$.
    \proofref{pre-compose-growth-commut:lemma}
\end{lemma}

\begin{lemma}
    \label{pre-compose-npoly:lemma}
    Let $f \in \ZRat$, and $k \in \Nat$. Then,
    $f \in \NPoly$ if and only if
    for every \kl{commutative} \kl{star-free polyregular function} $h$
            $(f \circ h) \in \NPoly$.
\end{lemma}

\begin{lemma}
    \label{pre-compose-sf-commut:lemma}
    Let $f \in \ZPoly$. Then, $f \in \ZSF$,
    if and only if for every \kl{commutative} \kl{star-free polyregular function} $h$,
            $(f \circ h) \in \ZSF$.
    \proofref{pre-compose-sf-commut:lemma}
\end{lemma}

Remark that if those two lemmas were to hold for \kl{$\Nat$-polyregular
functions}, then the decidability of $\NPoly$ inside $\ZPoly$ would follow
immediately, and $\NSF$ inside $\NPoly$ would immediately follow. On the one
hand, one can guess a candidate function and check for equivalence, on the
other hand, one can guess a \kl{commutative} \kl{star-free polyregular
function} and check membership (which is decidable thanks to this paper).




% BACKMATTER
\printbibliography

\appendix

\end{document}
