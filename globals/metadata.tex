\date{\today}

\title{Commutative $\Nat$-polyregular functions}

\author{Aliaume Lopez}%
       {University of Warsaw}%
       {ad.lopez@uw.edu.pl}%
       {https://orcid.org/0000-0002-1825-0097}%
       {}

\authorrunning{A. Lopez}

%\titlerunning{Commutative $\Nat$-polyregular functions}
\date{\today}

\Copyright{Aliaume Lopez}

%TODO mandatory; please add comma-separated list of keywords
\keywords{Rational series, weighted automata, polyregular function, commutative} 
\category{} %optional, e.g. invited paper
\relatedversion{} %optional, e.g. full version hosted on arXiv, HAL, or other respository/website
\relatedversiondetails[linktext={arXiv:2404.02232}, cite={lopez2024com:arxiv}]{Full Version}{https://arxiv.org/abs/2404.02232}

\ccsdesc[500]{Theory of Computation~Models of Computation}
\ccsdesc[500]{Theory of computation~Automata extensions}


\newcommand{\acknowledge}{
}

\newcommand{\makeabstract}{
    \begin{abstract}
This paper studies which functions computed by $\Rel$-weighted automata can be
realised by $\Nat$-weighted automata, under two extra assumptions:
commutativity (the order of letters in the input does not matter) and
polynomial growth (the ouptut of the function is bounded by a polynomial in the
size of the input). This continues the work of Karhumäki, who studied the same
question under the assumption that the computed function was a polynomial,
which is more restrictive. Answering this question starts by exhibiting a
counterexample to the theorem of Karhumäki, proposing a new and valid
characterization in the case of polynomials, that lifts to the more general
setting of polynomial growth and commutativity. Surprisingly, we manage to
leverage this effective characterization to decide whether a function computed
by a commutative $\Nat$-weighted automaton of polynomial growth is star-free, a
notion borrowed from the theory of regular languages that has been the subject
of many investigations in the context of string-to-string functions
during the last decade. 
    \end{abstract}
}

%Editor-only macros:: begin (do not touch as author)%%%%%%%%%%%%%%%%%%%%%%%%%%%%%%%%%%
\EventEditors{John Q. Open and Joan R. Access}
\EventNoEds{2}
\EventLongTitle{42nd Conference on Very Important Topics (CVIT 2016)}
\EventShortTitle{Event Short Title}
\EventAcronym{Event Acronym}
\EventYear{2025}
\EventDate{December 24--27, 2016}
\EventLocation{Little Whinging, United Kingdom}
\EventLogo{}
\SeriesVolume{42}
\ArticleNo{285}
%%%%%%%%%%%%%%%%%%%%%%%%%%%%%%%%%%%%%%%%%%%%%%%%%%%%%%
