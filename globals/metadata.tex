\date{\today}

\title{Commutative $\Nat$-rational Series of Polynomial Growth}

\author{Aliaume Lopez}%
       {University of Warsaw}%
       {ad.lopez@uw.edu.pl}%
       {https://orcid.org/0000-0002-4205-327X}%
       {Aliaume Lopez was supported by the Polish National Science Centre (NCN) grant “Polynomial
finite state computation” (2022/46/A/ST6/00072).}

\authorrunning{A. Lopez}

%\titlerunning{Commutative $\Nat$-polyregular functions}
\date{\today}

\Copyright{Aliaume Lopez}

%TODO mandatory; please add comma-separated list of keywords
\keywords{Rational series, weighted automata, polyregular function, commutative} 
\category{} %optional, e.g. invited paper
\relatedversion{} %optional, e.g. full version hosted on arXiv, HAL, or other respository/website
\relatedversiondetails[linktext={arXiv:2404.02232}, cite={LOPEZ24a}]{Full Version}{https://arxiv.org/abs/2404.02232}

\ccsdesc{Theory of computation~Quantitative automata}
\ccsdesc{Theory of computation~Transducers}


\newcommand{\acknowledge}{
}

\newcommand{\makeabstract}{
    \begin{abstract}
This paper studies which functions computed by $\Rel$-weighted automata can be
realised by $\Nat$-weighted automata, under two extra assumptions:
commutativity (the order of letters in the input does not matter) and
polynomial growth (the output of the function is bounded by a polynomial in the
size of the input). We
leverage this effective characterization to decide whether a function computed
by a commutative $\Nat$-weighted automaton of polynomial growth is star-free, a
notion borrowed from the theory of regular languages that has been the subject
of many investigations in the context of string-to-string functions
during the last decade. 
    \end{abstract}
}

%Editor-only macros:: begin (do not touch as author)%%%%%%%%%%%%%%%%%%%%%%%%%%%%%%%%%%
\EventEditors{Olaf Beyersdorff, Micha\l{} Pilipczuk, Elaine Pimentel, and Nguyen Kim Thang}
\EventNoEds{4}
\EventLongTitle{42nd International Symposium on Theoretical Aspects of Computer Science (STACS 2025)}
\EventShortTitle{STACS 2025}
\EventAcronym{STACS}
\EventYear{2025}
\EventDate{March 4--7, 2025}
\EventLocation{Jena, Germany}
\EventLogo{}
\SeriesVolume{327}
\ArticleNo{58}
%%%%%%%%%%%%%%%%%%%%%%%%%%%%%%%%%%%%%%%%%%%%%%%%%%%%%%
