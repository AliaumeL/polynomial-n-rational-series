%
% Here are the definitions
% of mathematical macros.
%
%

\newcommand{\defined}{\mathrel{:=}}
\newcommand{\seqof}[2]{(#1)_{#2}}
\newcommand{\setof}[2]{\mathopen{\{} #1 \mid #2 \mathclose{\}}}
\newcommand{\set}[1]{\mathopen{\{} #1 \mathclose{\}}}

\newcommand{\Pfin}{\mathcal{P}_{\mathsf{fin}}}

\newcommand{\MSO}{\mathsf{MSO}}
\newcommand{\FO}{\mathsf{FO}}

\newcommand{\ind}[1]{\mathbf{1}_{#1}}
\newcommand{\bigO}{\mathcal{O}}

\NewDocumentCommand{\Span}{m}{\mathop{\mathsf{Span}_{#1}}}


\newcommand{\topartial}{\rightharpoonup}
% functions

\newcommand{\dom}{\operatorname{dom}}

\NewDocumentCommand{\vcount}{ m }{ {{\#}#1} }
% sets

\NewDocumentCommand{\card}{ O{} m }{ \mathopen{|} #2  \mathclose{|}_{#1}}
\newcommand{\Nat}{\mathbb{N}}
\newcommand{\Rel}{\mathbb{Z}}
\newcommand{\Rat}{\mathbb{Q}}

% GOOD POLYNOMIALS
\newcommand{\CoveredPoly}{\mathsf{PolyKu}}
\newcommand{\CorrectPoly}{\mathsf{PolyRec}}

\NewDocumentCommand{\Monomials}{o}{\IfNoValueTF{#1}{\mathsf{Mon}}{\mathsf{Mon}[#1]}}
\NewDocumentCommand{\MaximalMonomials}{}{\mathop{\mathsf{maxmon}}}

% Differences of polynomials
\NewDocumentCommand{\Diff}{m m}{ \Delta_{#1}(#2) }


% Restrictions
%\newcommand{\restr}[2]{{#1}_{\kl[\restr]{| #2}}}
\newcommand{\restr}[2]{\kl[\restr]{[}{#1}\kl[\restr]{]}_{#2}}
\knowledge{\restr}{notion}

\NewDocumentCommand{\Poly}{O{}}{\mathsf{Poly}_{#1}}
\NewDocumentCommand{\SF}{O{}}{\mathsf{SF}_{#1}}
\NewDocumentCommand{\NPoly}{O{}}{\mathbb{N}\mathsf{Poly}_{#1}}
\NewDocumentCommand{\ZPoly}{O{}}{\mathbb{Z}\mathsf{Poly}_{#1}}
\NewDocumentCommand{\NSF}{O{}}{\mathbb{N}\mathsf{SF}_{#1}}
\NewDocumentCommand{\ZSF}{O{}}{\mathbb{Z}\mathsf{SF}_{#1}}
\NewDocumentCommand{\ZRat}{}{\mathbb{Z}\mathsf{Series}}
\NewDocumentCommand{\NRat}{}{\mathbb{N}\mathsf{Series}}
\NewDocumentCommand{\ZCommut}{}{\mathsf{Commut}}

\NewDocumentCommand{\commute}{O{\cdot}}{\{\!\{ #1 \}\!\}}





\NewDocumentCommand{\hoareleq}{}{\leq_{\mathcal{H}}}
\NewDocumentCommand{\hoarele}{}{<_{\mathcal{H}}}

\newcommand{\growth}[1]{\mathsf{growth}_{#1}}

\NewDocumentCommand{\BadPoly}{}{P_{\mathsf{bad}}}

\NewDocumentCommand{\translate}{m}{\tau_{#1}}


\NewDocumentCommand{\cauchy}{}{\mathrel{\otimes}}


\NewDocumentCommand{\polysort}{}{\mathsf{sort}}
\NewDocumentCommand{\polysum}{}{\mathsf{sum}}


\NewDocumentCommand{\npolyleq}{ O{} }{\mathrel{\preceq_{\Nat #1}}}
\NewDocumentCommand{\zpolyequiv}{ O{} }{\mathrel{\equiv}_{\Rel #1}}

\NewDocumentCommand{\app}{ m m }{\mathop{{#2} \triangleright {#1}}}
\NewDocumentCommand{\Res}{}{ \mathsf{Res}}


\NewDocumentCommand{\divides}{}{\mathrel{|}}


\NewDocumentCommand{\resleq}{ m m }{\mathrel{\leq_{{#1},{#2}}}}
\NewDocumentCommand{\prefleq}{}{\sqsubseteq_{\mathsf{pref}}}
\NewDocumentCommand{\prefle}{}{\sqsubset_{\mathsf{pref}}}
\NewDocumentCommand{\preforth}{}{\bot_{\mathsf{pref}}}


\NewDocumentCommand{\aTransd}{}{\mathcal{A}}


\NewDocumentCommand{\ModuloTypes}{O{}}{{#1}\mathsf{Types}}
\NewDocumentCommand{\moduloType}{O{}}{{#1}\mathsf{type}}


\NewDocumentCommand{\floor}{m}{\lfloor #1 \rfloor}
