%
% Here are the definitions
% of mathematical macros.
%
%

\newcommand{\defined}{\mathrel{:=}}
\newcommand{\seqof}[2]{(#1)_{#2}}
\newcommand{\setof}[2]{\mathopen{\{} #1 \mid #2 \mathclose{\}}}
\newcommand{\set}[1]{\mathopen{\{} #1 \mathclose{\}}}

\newcommand{\Pfin}{\mathcal{P}_{\mathsf{fin}}}

\newcommand{\MSO}{\mathsf{MSO}}

% functions

\newcommand{\dom}{\operatorname{dom}}

\NewDocumentCommand{\vcount}{ m }{ {{\#}#1} }
% sets

\NewDocumentCommand{\card}{ O{} m }{ \mathopen{|} #2  \mathclose{|}_{#1}}
\newcommand{\Nat}{\mathbb{N}}
\newcommand{\Rel}{\mathbb{Z}}
\newcommand{\Rat}{\mathbb{Q}}

% GOOD POLYNOMIALS
\newcommand{\CoveredPoly}{\mathcal{P}}
\newcommand{\CorrectPoly}{\mathcal{Q}}

\NewDocumentCommand{\Monomials}{o}{\IfNoValueTF{#1}{\mathsf{Monom}}{\mathsf{Monom}[#1]}}
\NewDocumentCommand{\MaximalMonomials}{}{\mathop{\mathsf{maxmonom}}}

% Differences of polynomials
\NewDocumentCommand{\Diff}{m}{ \Delta_{#1} }


% Restrictions
\newcommand{\restr}[2]{{#1}_{\kl[\restr]{| #2}}}
\knowledge{\restr}{notion}

\NewDocumentCommand{\NPoly}{O{}}{\mathbb{N}\mathsf{Poly}_{#1}}
\NewDocumentCommand{\ZPoly}{O{}}{\mathbb{Z}\mathsf{Poly}_{#1}}
\NewDocumentCommand{\NSF}{O{}}{\mathbb{N}\mathsf{SF}_{#1}}
\NewDocumentCommand{\ZSF}{O{}}{\mathbb{Z}\mathsf{SF}_{#1}}
\NewDocumentCommand{\ZRat}{}{\mathbb{Z}\mathsf{Series}}
\NewDocumentCommand{\NRat}{}{\mathbb{N}\mathsf{Series}}
\NewDocumentCommand{\ZCommut}{}{\mathsf{Commutative}}

\NewDocumentCommand{\commute}{O{\cdot}}{\llbracket #1 \rrbracket}





\NewDocumentCommand{\hoareleq}{}{\leq_{\mathcal{H}}}
\NewDocumentCommand{\hoarele}{}{<_{\mathcal{H}}}
